A scripted trade comprises

\begin{itemize}
\item external data that parametrises the payoff script
\item the payoff script
\end{itemize}

In the following example, a plain vanilla equity call or put option is described by providing the expiry and settlement
date, the strike, a put / call indicator and the underlying index as external data in the \verb+Data+ node, and the
payoff script is referenced by the \verb+ScriptName+ node:

\begin{minted}[fontsize=\footnotesize]{xml}
<Trade id="VanillaOption">
  <TradeType>ScriptedTrade</TradeType>
  <Envelope/>
  <ScriptedTradeData>
    <ScriptName>EuropeanOption</ScriptName>
    <Data>
      <Event>
        <Name>Expiry</Name>
        <Value>2020-02-09</Value>
      </Event>
      <Event>
        <Name>Settlement</Name>
        <Value>2020-02-15</Value>
      </Event>
      <Number>
        <Name>Strike</Name>
        <Value>1200</Value>
      </Number>
      <Number>
        <Name>PutCall</Name>
        <Value>1</Value>
      </Number>
      <Index>
        <Name>Underlying</Name>
        <Value>EQ-RIC:.SPX</Value>
      </Index>
      <Currency>
        <Name>PayCcy</Name>
        <Value>USD</Value>
      </Currency>
    </Data>
  </ScriptedTradeData>
</Trade>
\end{minted}

The script itself is defined in a script library, which is part of
the static data setup of the application like bond or credit index static data, as follows

\begin{minted}[fontsize=\footnotesize]{xml}
<ScriptLibrary>
  <Script>
    <Name>EuropeanOption</Name>
    <ProductTag>SingleAssetOption({AssetClass})</ProductTag>
    <Script>
      <Code><![CDATA[
          Option = PAY(max( PutCall * (Underlying(Expiry) - Strike), 0 ),
                                                  Expiry, Settlement, PayCcy);
        ]]></Code>
      <NPV>Option</NPV>
      <Results>
        <Result>ExerciseProbability</Result>
      </Results>
    </Script>
  </Script>
  <Script>
  ...
  </Script>
  ...
</ScriptLibrary>
\end{minted}

Many scripted trades can thus share the same script, and the payoff scripts can be managed and maintained
centrally. Each scipt defines an exotic product type. And adding a product type to the library is a configuration rather
than a code release change.

\ifdefined\STModuleDoc

For the ORE CLI the script library can be loaded by specifying the (optional) scriptLibrary parameter:

\begin{minted}[fontsize=\footnotesize]{xml}
<ORE>
  <Setup>
  <Parameter name="asofDate">2016-02-05</Parameter>
  ...
  <Parameter name="scriptLibrary">scriptlibrary.xml</Parameter>
  </Setup>
  ...
</ORE>
\end{minted}

In the pricer\_app the script library can be loaded by specifying the (optional) script\_library parameter, e.g. in config.txt

\begin{minted}[fontsize=\footnotesize]{text}
asof=2018-12-31
base_currency=USD
trade=scripted_fx_onetouch_option.xml
script_library=scriptlibrary.xml
\end{minted}

Alternatively, the script can be inlined in the trade representation:

\begin{minted}[fontsize=\footnotesize]{xml}
<Trade id="VanillaOption">
  <TradeType>ScriptedTrade</TradeType>
  <Envelope/>
  <ScriptedTradeData>
    <ProductTag>SingleAssetOption({AssetClass})</ProductTag>
    <Script>
      <Code><![CDATA[
          Option = PAY(max( PutCall * (Underlying(Expiry) - Strike), 0 ),
                                                  Expiry, Settlement, PayCcy);
      ]]></Code>
      <NPV>Option</NPV>
      <Results>
          <Result>ExerciseProbability</Result>
      <Results>
    </Script>
    <Data>
      ...
    </Data>
  </ScriptedTradeData>
</Trade>
\end{minted}

The product tag node is optional both for scripts in the library and inlined script and used for the assignment of a
pricing model and its parameters, see \ref{producttags_engineconfig}

\fi