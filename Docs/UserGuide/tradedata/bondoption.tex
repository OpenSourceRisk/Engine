\subsubsection{Bond Option}
\label{ss:bondoption}

The structure of a trade node representing a \emph{BondOption}  is shown in
listing \ref{lst:bondoption_data}: 
\begin{itemize}
\item The \lstinline!BondOptionData!  node is the trade data container for
the option part of a bond option trade type. Vanilla bond 
options are supported, the exercise style must be \emph{European}. 
The \lstinline!BondOptionData!  node includes one and 
only one \lstinline!OptionData! trade component sub-node plus elements
specific to the bond option. 
\item The latter also includes the underlying Bond description in the \lstinline!BondData!
  node, see section \ref{ss:bond}, listing \ref{lst:bonddata} for details
\end{itemize}

\begin{listing}[H]
%\hrule\medskip
\begin{minted}[fontsize=\footnotesize]{xml}
  <Trade id="...">
    <TradeType>BondOption</TradeType>
    <Envelope>
        ...
    </Envelope>
    <BondOptionData>
      <OptionData>
          ...
      </OptionData>
      <StrikeData>
        <StrikePrice>
          <Value>11809123.56</Value>
          <Currency>EUR</Currency>
        </StrikePrice>
      </StrikeData>
      <Redemption>100.00</Redemption>
      <PriceType>Dirty</PriceType>
      <KnocksOut>false</KnocksOut>
      <BondData>
         <VolatilityCurveId>YieldVols-EUR</VolatilityCurveId>
          ...
      <BondData>
    </BondOptionData>
  </Trade>
\end{minted}
\caption{Bond Option data}
\label{lst:bondoption_data}
\end{listing}

The meanings and allowable values of the elements in the \lstinline!BondOptionData!  node follow below.

\begin{itemize}
	\item OptionData: This is a trade component sub-node outlined in section \ref{ss:option_data} Option Data. Note 
	that the bond option type allows for \emph{European} option style only.
  \item StrikeData: A node containing the strike information.
	Allowable values: Supports \lstinline!StrikePrice! and \lstinline!StrikeYield! as described in Section \ref{ss:strikedata}.

  \item Redemption: Redemption ratio in percent
	\item PriceType: This node defines which strike should be used for the pricing. If the node takes the value Dirty, 
	the strike price should be set equal to the value of the Strike node. If the node takes the value Clean, 
	the strike price should be set equal to the value of the Strike node plus accrued interest at the expiration date of the option.\\
	Allowable values: Dirty or Clean.
  \item KnocksOut: If true the option knocks out if the underlying defaults before the option expiry, if false the
    option is written on the recovery value in case of a default of the bond before the option expiry
\end{itemize}

The meanings and allowable values of the elements in the \lstinline!BondData! are:

\begin{itemize}
  \item VolatilityCurveId: The yield volatility curve to use for the valuation of this bond option.
\end{itemize}
