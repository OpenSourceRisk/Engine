\subsubsection{Callable Bond}
\label{ss:callable_bond}

A callable bond is a bond with issuer call and / or investor put rights. Typically, the call style is American while the
put is Bermudan, but we support any combination of styles. Listing \ref{lst:callablebonddata1} shows an example trade
xml. The meanings and allowable values of the elements in the {\tt CallableBondData} block are as follows:

\begin{itemize}
  \item SecurityId: The underlying security identifier\\
      Allowable values:  Typically the ISIN of the underlying bond, with the ISIN: prefix.
  \item BondNotional: The notional of the underlying bond expressed in the currency of the bond.\\
      Allowable values:  Any positive real number.
    \item CreditRisk [Optional] Boolean flag indicating whether to show Credit Risk on the Bond product. \\
      Allowable Values: \emph{true} or \emph{false} Defaults to \emph{true} if left blank or omitted.
\end{itemize}

\begin{listing}[H]
\begin{minted}[fontsize=\footnotesize]{xml}
  <Trade id="CallableBond">
    <TradeType>CallableBond</TradeType>
    <Envelope>...</Envelope>
    <CallableBondData>
      <BondData>
        <SecurityId>ISIN:XS0123456789</SecurityId>
        <BondNotional>1000000.00</BondNotional>
      </BondData>
    </CallableBondData>
  </Trade>
\end{minted}
\caption{Callable bond set up using reference data}
\label{lst:callablebonddata1}
\end{listing}

The bond terms of the trade in \ref{lst:callablebonddata1} is set up in reference data, see \ref{lst:callablebonddata2}
for an example. The fields in the reference data have the following meaning:

\begin{itemize}
  \item BondData: The vanilla part of the bond, see \ref{ss:bond}.
  \item CallData: The call terms of the bond, as described below. Optional, if not given, no calls are present.
  \item PutData: The put terms of the bond, as described below. Optional, if not given, no puts are present.
\end{itemize}

\begin{listing}[H]
\begin{minted}[fontsize=\footnotesize]{xml}
  <ReferenceDatum id="ISIN:XS0123456789">
    <Type>CallableBond</Type>
    <CallableBondReferenceData>
      <BondData> ... </BondData>
      <CallData> ... </CallData>
      <PutData> ... </PutData>
    </CallableBondReferenceData>
  </ReferenceDatum>
\end{minted}
\caption{Callable bond reference data}
\label{lst:callablebonddata2}
\end{listing}

\underline{Specification of CallData / PutData:}

All lists specified in subnodes (except the date list itself of course) can be specified as either an explicit list of
values corresponding to the schedule dates list or using the attribute \verb+startDate+. An explicit value list can be
shorter than the list of dates, in which case the last value from the list is associated to the remaining dates.

See listings \ref{lst:callablebonddata_callputdata_1},\ref{lst:callablebonddata_callputdata_2},\ref{lst:callablebonddata_callputdata_3}
for examples of exercise schedules.

\begin{itemize}
\item Styles: A list of the exercise styles. Notice that Bermudan is used to define European exercises as well, namely
as a Bermudan exercise with a single exercise date. The attribute \verb+startDate+ can be used to specify the list. \\
Allowable values: American, Bermudan

\item ScheduleData: A schedule of exercise dates (for Bermudan exercises) or start / end dates (for American exercises) \\
  Allowable values: see \ref{ss:schedule_data}.

\item Prices: A list of exercise prices in relative terms, i.e. if the price is $1.02$ then the amount paid on the
  exercise is this price times the current notional of the bond (plus accrued interest, if the price type is clean, see
  below). The attribute \verb+startDate+ can be used to specify the list.\\
  Allowable values: Any positive real number.

\item PriceType: A list of the flavour in which the exercise prices are given. The attribute \verb+startDate+ can be
  used to specify the list.\\
  Allowable values: Clean, Dirty.

\item IncludeAccrual: A list of flags specifying whether accruals have to be paid on exercise. This is independent of
  the quoting style of the exercise prices (PriceType).\\
  Allowable values: true, false
\end{itemize}

\begin{listing}[H]
\begin{minted}[fontsize=\footnotesize]{xml}
  <!-- Bermudan issuer call on three dates at a clean price of 100, 100, 102
       accruals are paid on exercise -->
  <CallData>
    <Styles>
      <Style>Bermudan</Style>
    </Styles>
    <ScheduleData>
      <Dates>
        <Dates>
          <Date>2016-08-03</Date>
          <Date>2017-08-03</Date>
          <Date>2018-08-03</Date>
        </Dates>
      </Dates>
    </ScheduleData>
    <Prices>
      <Price>1.00</Price>
      <Price>1.00</Price>
      <Price>1.02</Price>
    </Prices>
    <PriceTypes>
      <PriceType>Clean</PriceType>
    </PriceTypes>
    <IncludeAccruals>
      <IncludeAccrual>true</IncludeAccrual>
    </IncludeAccruals>
    <TriggerRatios/>
  </CallData>
\end{minted}
\caption{Callable bond call data example 1}
\label{lst:callablebonddata_callputdata_1}
\end{listing}

\begin{listing}[H]
\begin{minted}[fontsize=\footnotesize]{xml}
  <!-- American issuer call between 2016-08-03 and 2018-08-03
       at a clean price of 100 -->
  <CallData>
    <Styles>
      <Style>American</Style>
    </Styles>
    <ScheduleData>
      <Dates>
        <Dates>
          <Date>2016-08-03</Date>
          <Date>2018-08-03</Date>
        </Dates>
      </Dates>
    </ScheduleData>
    <Prices>
      <Price>1.00</Price>
    </Prices>
    <PriceTypes>
      <PriceType>Clean</PriceType>
    </PriceTypes>
    <IncludeAccruals>
      <IncludeAccrual>true</IncludeAccrual>
    </IncludeAccruals>
  </CallData>
\end{minted}
\caption{Callable bond call data example 2}
\label{lst:callablebonddata_callputdata_2}
\end{listing}

\begin{listing}[H]
\begin{minted}[fontsize=\footnotesize]{xml}
  <!-- Bermudan puts calls at 100, 101, 102 at 3 dates from 2016 to 2018 -->
  <PutData>
    <Styles>
      <Style>Bermudan</Style>
    </Styles>
    <ScheduleData>
      <Dates>
        <Dates>
          <Date>2016-08-03</Date>
          <Date>2017-08-03</Date>
          <Date>2018-08-03</Date>
        </Dates>
      </Dates>
    </ScheduleData>
    <Prices>
      <Price>1.00</Price>
      <Price>1.01</Price>
      <Price>1.02</Price>
    </Prices>
    <PriceTypes>
      <PriceType>Clean</PriceType>
    </PriceTypes>
    <IncludeAccruals>
      <IncludeAccrual>true</IncludeAccrual>
    </IncludeAccruals>
  </PutData>
\end{minted}
\caption{Callable bond put data example 3}
\label{lst:callablebonddata_callputdata_3}
\end{listing}
