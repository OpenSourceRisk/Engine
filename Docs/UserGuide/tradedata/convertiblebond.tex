\subsubsection{Convertible Bond}
\label{ss:convertible_bond}

\ifdefined\IncludePayoff{{\bf Payoff}

A convertible bond is a bond, that can be converted to a prespecified number of shares. The shares are usually from the
bond issuer, but it is also possible that the shares are from a different issuer (exchangeables). In addition, the share
currency can be different from the bond currency in both cases (cross-currency convertibles).

The bond might be callable by the issuer (typically in American style) and / or puttable by the investor (typcially in
Bermudan style). The issuer calls can be ``hard calls'', which are call rights in the traditional sense, as opposed to
``soft calls'' which can only the exercised if the equity price observed on (and possibly during a period before) the
exercise date is above a prespecified threshold. If a soft call is exercised, the investor has the right to convert the
bond into shares instead of accepting the payment from the issuer call (``forced conversion'').

For a detachable or stripped convertible bond the optionality can be traded separately from the bond. We set the NPV for
a detachable convertible bond to the difference of the convertible bond npv and the bond floor npv, where the bond floor
denotes the underyling vanilla bond stripped of any optionality.

Additional features of convertible bonds include dividend protection, contingent conversion, mandatory conversion,
conversion ratio resets, make-whole calls, dividend-forfeit clauses, copay clauses.

Refer to \cite{Spiegeleer_convertible_handbook} and \cite{Bloomberg_OVCV} for a deeper dive into the convertible bond
universe.

In the following we use the notation from table \ref{tab:convertiblebond_notation}.

\begin{table}[h]
  \begin{tabular}{p{2cm}|p{13cm}}
    Variable & Description \\
    \hline
    $N_0$ & Bond Face Value at $t=0$, the bond reference data in ORE is typically set up such that $N_0 = 1$, even if
            the true face value of the bond quoted in the market is $100$ or $100,000$. \\ \hline
    $N_t$ & The outstanding notional at time $t$. For bullet bonds, $N_t = N_0$. For amortising bonds $N_t \leq N_0$ and it
            is assumed that a conversion payout is scaled down by $N_t / N_0$ in this case. \\ \hline
    $S_t$ & The share price at time $t$ \\ \hline
    $C^R_t$ & The conversion ratio at time $t$. The conversion ratio is the number
              of shares $S_t$ received for one bond (i.e. for the bond face value $N_0$) upon conversion.
              The conversion ratio can vary with time, typically due to the presence of dividend protection or
              reset features. It is also possible to set up a deterministic schedule of varying conversion rates in the
              trade term sheet, but this is less common. \\   \hline
    $C^V_t$ & The conversion value or ``parity'', i.e. the payout from a conversion $C^V_t = C^R_t S_t$.
              Notice that for amortising bonds an
              additional factor $N_t/N_0$ is applied to the actual payout, but $C^V$ is defined without this factor.
              For cross-currency trades the parity is expressed in bond ccy, i.e. $C^V_t = C^R_t S_t X_t$ where
              $X_t$ convertes on equity ccy unit to bond ccy. \\ \hline
    $C^P_t$ & The conversion price, $C^P_t = N_0 / C^R_t$, i.e. the price of the share that upon conversion would give a
              payout of $N_0$. For cross-currency trades the conversion price is usually quoted in equity ccy, i.e.
              $C^P_t = N_0 / (C^R_t X_t)$ where $X_t$ converts one equity ccy unit to bond ccy.
              \\ \hline
  \end{tabular}
  \caption{Notation for convertible bonds}
  \label{tab:convertiblebond_notation}
\end{table}

\underline{Vanilla bond underlying:}

The underlying bond will pay coupons and the final redemption $N_0$ (or $N_T$ for amortising bonds) at its maturity
$T$. The coupons are usually fixed rate coupons for convertibles, but floating rate coupons are also possible. For
amortising bonds, there will be additional intermediate redemption payments. We denote all cashflows of the underlying
bond by

\begin{equation}
c_i,\, i = 1, \ldots, n
\end{equation}

i.e. thess include the all coupon and intermediate and final redemption payments.

\underline{Calls and Puts, forced conversion:}

The bond may be callable or puttable or both. Usually, issuer calls are American style, while investor puts are Bermudan
style. For a hard issuer call at $t$, the issuer has the option to pay

\begin{equation}\label{formula:convertible_bond_callpayout}
  P N_t + a
\end{equation}

to the investor and terminate the bond transaction. Here, $P$ is the call price in clean terms, e.g. $1.00$ for a call
at par or $1.02$ if the bond can be called at $102\%$. $a$ are the accruals to be paid for the current bond coupon
period. There might be a clause in the term sheet excluding the payment of the accruals $a$. Call prices can be
specified both in clean and dirty terms for convenience, but this does not control the payout
\ref{formula:convertible_bond_callpayout}, i.e. whether $a$ is paid or not is specified separately from the quotation
style of the call price.

For a soft issuer call, the call can only be exercised if the stock price observed on the exercise date exceeds a
barrier

\begin{equation}
  S_t > C^P_t T
\end{equation}

defined by a trigger level $T$ and the current conversion price $C^P_t$. If the trigger is for example $T=1.3$ and the
current conversion ratio is $C^R_t = 0.05$ (and $N_0=1$), then the conversion price is $C^P_t = 20$ and the soft call
can only be exercised if the current stock price $S_t$ is above $1.3\cdot 20 = 26$. Notice that for cross currency
trades it does not matter if we compare the share price in equity ccy with $C^P_t T$ with $C^P_t$ expressed in equity
currency, or both expressed in bond currency, i.e. converted to the bond currency by multiplying with $X_t$.

For both hard and soft calls, an issuer call will give the investor the option to instead convert the bond on the
exercise date at the current conversion ratio $C^R_t$, i.e. to receive $C^R_t\cdot S_t$ instead of the call price payout
\ref{formula:convertible_bond_callpayout} (``forced conversion''). In reality, there is usually a notice period that can
be several months after the exercise date and the investor can decide to convert during this whole period, a feature
that we ignore in the pricing though. This type of conversion is called forced conversion as opposed to voluntary
conversion described below.

Sometimes a call triggers an extra payout for the investor. One variant is that in case of an issuer call the conversion
ratio is adjusted upwards, so that in case of a forced conversion the investore receives a higher payout. The adjustment
of the conversion ratio is in general a function of the exercise time and current stock price on the conversion date.

If there is a investor put option, the investor has the option to receive

\begin{equation}
P N_t + a
\end{equation}

on the exercise date from the issuer and terminate the bond transaction. Again, the payment of the accruals $a$ can be
excluded in the trade terms.

More features, that are currently {\em not} supported, include:

\begin{itemize}
\item There might be an additional fee payable by the issuer on a call that makes a call less attractive to the issuer
  (``Make whole'' payment). This payment can be computed in several ways, e.g. as a fixed amount or as the NPV of
  outstanding coupon payment in the underlying bond discounted on the treasury curve.

\item Soft call execution can be conditioned on a barrier condition $S_t > B$ that is met on $N$ out of $M$ days before
  the exercise date (typcially $20$ out of $30$).
\end{itemize}

\underline{Voluntary conversion}

The (voluntary) conversion right is held by the investor and can be Bermudan or American style. On a date where a
conversion is possible, the investor has the right to terminate the bond transaction and receive

\begin{equation}
  C^R_t S_t + a
\end{equation}

i.e. $C^R_t$ shares per bond face value $N_0$ from the issuer and where $a$ are unpaid accruals. For cross currency
trades think of $C^R_tS_t$ as the amount converted to the bond currency, i.e. multiplied by $X_t$. For an amortising
bond, this amount is scaled by $N_t/N_0$.  There might be a cash-out option for the usser, i.e. the rigt to settle the
payment in cash instead of delivering shares, a feature that can be of dramatic importance for the investor, but which
does not impact the valuation.

\underline{Contingent conversion}

The conversion right may be contingent on a condition

\begin{equation}
  C^R_t S_t > B
\end{equation}

i.e. the conversion is only possible if the current ``Parity'' $C^R_t S_t$ exceeds a given barrier $B$. The observation
of the trigger condition can be made on the conversion date itself (Spot) or on the start date of the current conversion
period (StartOfPeriod, for American style conversion rights only). For cross currency trades the parity is expressed in
bond ccy, i.e. as $C^R S_t X_t$, where $X_t$ converts one unit in equity ccy to bond ccy.

\underline{Mandatory conversion}

A mandatory conversion date is a date (obviously placed after all other, voluntary conversion dates, if any) where a
conversion is enforced, i.e. it is neither an option of the investor nor the issuer. There are several types of payoffs
possible, we support

\begin{itemize}
  \item PEPS (Preferred Redeemable Increased Dividend Securities), where the payoff is defined as

\begin{eqnarray*}
  \begin{cases}
    C^{R1} \cdot S_t & \text{for } S_t < S_1 \\
    N_t / S_t \cdot S_t &  \text{for }S_1 \leq S_t \leq S_2 \\
    C^{R2} \cdot S_t &  \text{for } S_t > S_2
  \end{cases}
\end{eqnarray*}

with barrier levels $S_1$ and $S_2$ and associated prespecified conversion ratios $C^{R1}$ and $C^{R2}$. As for the
usual voluntary conversion the payoff for $S_t<S_1$ and $S_t>S_2$ is multiplied by $N_t/N_0$ for amortising bonds. For
cross-currency trades the barrier levels $S_1$ and $S_2$ are expressed in equity currency and the payoff for $S_1 \leq
S_t \leq S_2$ is defined as $N_t /(X_t S_t) \cdot X_t S_t$, which of course is just $N_t$ (paid in bond ccy) again.

\end{itemize}

\underline{Conversion ratio reset}

The conversion ratio can be dynamically adjuted during the lifetime of a convertible bond. For this a reset schedule is
defined. On each date of the reset schedule the conversion ratio is adjusted as follows

\begin{equation}\label{formula:convertible_bond_reset1}
C^R \rightarrow \max \left\{ C^R, \min\left( \frac{N_0}{g\cdot S_t}, \frac{N_0}{f\cdot C^P_0}, \frac{N_0}{F\cdot
C^P_0} \right) \right\}
\end{equation}

if the stock price on the reset date falls below a threshold $T$

\begin{equation}\label{formula:convertible_bond_reset2}
S_t < T C^P_0
\end{equation}

with a gearing $g$, a floor $f$ and a global floor $F$. Here, the reference for the trigger price and the floor is the
initial conversion price $C^P_0$. Alternatively, the reference can also be chosen to be the current conversion price
$C^P_t$ (i.e. the one from the most recent reset), i.e. the trigger condition is

\begin{equation}
S_t < T C^P_t
\end{equation}

For cross-currency trades, $C^P_0$ and $C^P_t$ are expressed in equity currency, i.e. $C^P_0 = N_0 / (C^R_0 X_t)$ and
$C^P_t = N_0 / (C^R_t X_t)$. Notice that we use $X_t$ to convert $C^P_0$, so that the trigger conditions can be both
expressed with a conversion price quoted in equity or bond currency and the same trigger level $T$. The adjustment
formula \ref{formula:convertible_bond_reset1} changes to

\begin{equation}\label{formula:convertible_bond_reset1_ccy}
C^R \rightarrow \max \left\{ C^R, \min\left( \frac{N_0}{g\cdot X_t S_t}, \frac{N_0}{f\cdot X_t C^P_0}, \frac{N_0}{X_t F} \right) \right\}
\end{equation}

and likewise the corresponding formula when using the current conversion price $C^P_t$ instead of $C^P_0$, both of which
are expressed in equity ccy here. Notice that in particular the global floor is expressed in equity ccy.

As a (single-currency) example consider the case $C^P_0 = 20$, $T = 0.9$: If the stock price falls below $90\%$ of the
initial conversion price, i.e. below $20\cdot 0.9 = 18$, the conversion ratio is adjusted upwards. Let's assume the
current stock price is $S_t=16$ and $g=0.8$, $f=0.7$, $F=15$. The adjustment by the gearing would result in a new
conversion price of $0.8\cdot 16=12.8$. However, this value is floored by $0.7\cdot 20 = 14$ (the . In addition, a
global floor at $F=15$ is defined, so that the final adjustment of the conversion price is $20 \rightarrow 15$. In other
words the conversion ratio is adjusted from $0.05$ to $0.0667$.

\underline{Dividend protection}

A dividend protection schedule is a series of dates on which either dividend amounts are passed through to the investor
as an additional payment or on which the conversion ratio is adjusted. The first date in the schedule defines the start
of the dividend protection phase, i.e. dividends that are paid from this date on are accounted for in the conversion
ratio adjustments or passthroughs, which take place on the second, third, ..., last date of the given schedule.

Let's denote the absolute dividend amount that is paid between the last and the current dividend protection date by
$D$. There is a threshold $H$ defined, such that the difference $D-H$ is passed through to the investor (in case $D<H$
the investor pays the difference $H-D$ to the issuer) or $\max(D-H,0)$ is passed through to the investor (in case $D<H$
no payment is done). The former form is called pass-through dividend protection with up-and-down compensation, while the
latter is a pass-through dividend protection with up-only compensation. The dividend amount per share determined in this
way is mulplied by the current conversion ratio before passed on to the investor.

The variant where the conversion ratio is adjusted works as follows. If the adjustment is agreed upon on an absolute
basis, the conversion rate is adjusted as follows

\begin{equation}
  C^R \rightarrow C^R \frac{S_t}{S_t-C}
\end{equation}

where

\begin{eqnarray*}
C=\begin{cases}
  D-H & \text{up-and-down adjustment} \\
  \max(D-H, 0) & \text{up-only adjustment}
\end{cases}
\end{eqnarray*}

Notice that for cross-currency trades the threshold $H$ is expressed in equity ccy.

If the adjustment is made on a relative basis on the other hand, the conversion rate will be adjusted

\begin{equation}
  C^R \rightarrow C^R ( 1 + C)
\end{equation}

where

\begin{eqnarray*}
C=\begin{cases}
  D / S_t -H & \text{up-and-down adjustment} \\
  \max(D / S_t -H, 0) & \text{up-only adjustment}
\end{cases}
\end{eqnarray*}

In all cases, $S_t$ is the stock price observed on the dividend protection date. The adjustment styles used in the trade
xml associated to these adjustments are ``CrUpDown'', ``CrUpOnly''.

There is an alternative adjustment formula

\begin{equation}
  C^R \rightarrow C^R \frac{S_t - H}{S_t - D}
\end{equation}

associated to adjustment style ``CrUpDown2''. The up-only variant of this adjustment

\begin{equation}
  C^R \rightarrow C^R \max\left(1, \frac{S_t - H}{S_t - D} \right)
\end{equation}

is associated to adjustment style ``CrUpOnly2''. The latter two styles are only defined for dividend type = absolute,
i.e. relative is not allowed.

\underline{Default of the issuer, secured and unsecured exchangables}

For non-exchangeables the investor has generally only a claim on the higher of the recovery value of the bond and the
conversion value of the issuer stock, i.e.

\begin{equation}
\max( \rho N_t, C^V )
\end{equation}

where $\rho$ is the recovery value.

For exchangeables, there are two variants:

\begin{itemize}
\item non-secured: the investor has a claim only on the bond recovery value $\rho N_t$, with $\rho$ the recovery rate,
  if the bond issuer defaults. If the equity issuer defaults, the investor has still a claim on the full notional of the
  bond.
\item secured with pledged shares: In case of the bond issuer default, the investor has a claim on the conversion value
  plus the recovery value on the bond notional less the conversion value (if positive), i.e.
  $C^V + \max ( \rho (N_t - C^V), 0 )$. In case of the equity issuer default, the invstor has a claim on the full
  notional of the bond.
\end{itemize}

\underline{Copay clause (currently not supported)}

If the bond price at the $i$th coupon is below a barrier $L$ or above a barrier $H$, the $(i+1)$th coupon will be adjusted
forward either as a percentage of $N_t$ or as a percentage of the bond price at the $i$th coupon payment date.

\underline{Dividend forfeit clause (currently not supported)}

Shares delivered by the issuer on conversion may be new shares, for which the investor does not receive a dividend
payment in the current fiscal year.

\underline{Variable conversion ratio (currently not supported)}

The conversion ratio is a function of the stock price at the conversion date.

{\bf Input}}\fi

A convertible bond is set up in ORE using a {\tt ConvertibleBondData} block as shown in listing
\ref{lst:convertiblebonddata1}. The bond details are read from reference data in this case. 

A convertible bond is a bond, that can be converted into a prespecified number of shares, given by:
$$
NumberOfShares = \frac{BondNotional}{ConversionRatio}
$$

Where the Conversion Ratio is specified in the underlying bond reference data.

The shares are usually from the bond issuer, but it is also possible that the shares are
from a different issuer (exchangeables). In addition, the share currency can be different
from the bond currency in both cases (cross-currency convertibles).


The bond might be callable by the issuer (typically in American style) and / or puttable
by the investor (typically in Bermudan style). The issuer calls can be “hard calls”,
which are call rights in the traditional sense, as opposed to “soft calls” which can only
the exercised if the equity price observed on the exercise date is above a prespecified
threshold given by TriggerRatios. If a soft call is exercised, the investor has the right to convert the bond into
shares instead of accepting the payment from the issuer call (“forced conversion”).


The meanings and allowable
values of the elements in the {\tt ConvertibleBondData} block are as follows:

\begin{itemize}
  \item SecurityId: The underlying security identifier\\
      Allowable values:  Typically the ISIN of the underlying bond, with the ISIN: prefix.
  \item BondNotional: The notional of the underlying bond expressed in the currency of the bond.\\
      Allowable values:  Any positive real number.
    \item CreditRisk [Optional] Boolean flag indicating whether to show Credit Risk on the Bond product. \\
      Allowable Values: \emph{true} or \emph{false} Defaults to \emph{true} if left blank or omitted.          
\end{itemize}

\begin{listing}[H]
\begin{minted}[fontsize=\footnotesize]{xml}
  <Trade id="ConvertibleBond">
    <TradeType>ConvertibleBond</TradeType>
    <Envelope>...</Envelope>
    <ConvertibleBondData>
      <BondData>
        <SecurityId>ISIN:XS0451905367</SecurityId>
        <BondNotional>1000000.00</BondNotional>
      </BondData>
    </ConvertibleBondData>
  </Trade>
\end{minted}
\caption{Convertible bond set up using reference data}
\label{lst:convertiblebonddata1}
\end{listing}

Alternatively the bond can be set up with further explicit details using the blocks as shown in listing
\ref{lst:convertiblebonddata2}. All fields that are not given in the trade XML are filled up with the information from
the reference data if available in the reference data. In other words, if reference data is given, the trade xml can
still be used to overwrite the information partially, if this seems appropriate. The meanings and allowable values of
the elements in the block are as follows:

\begin{itemize}
  \item BondData: The vanilla part of the bond, see \ref{ss:bond}.
  \item CallData: The call terms of the bond, as described below. Optional, if not given, no calls are present.
  \item PutData: The put terms of the bond, as described below. Optional, if not given, no puts are present.
  \item ConversionData: The conversion terms of the bond, as described below. This node must always be given, even if no
    conversion rights are present (in which case an empty conversion date list can be used).
  \item DividendProtectionData: The dividend protection terms of the bond, as described below. Optional, if not given,
    no dividend prtection is present.
  \item Detachable: If true, the trade represents the embedded optionality, i.e. the difference between the full
    convertible bond and the bond floor. Optional, defaults to false. \\
    Allowable values: true, false
\end{itemize}

The convertible bond trade type supports perpetual schedules, i.e. perpetual convertible bonds can be represented by
omitting the EndDate in the following schedules to indicate perpetual schedules. Only rule based schedules can be used
to indicate perpetual schedules.

\begin{itemize}
\item BondData / LegData: Omitting the EndDate in this schedule indicates that the underlying bond runs perpetually.
\item CallData: Omitting the EndDate in this schedule indicates perpetual call dates. For American call dates, where
  only two dates have to be specified (start and end date of the american call window), a rule based schedule with Tenor
  = 0D, Rule = Zero and without EndDate can be used to indicate an end date infinitely far away in the future.
\item PutData: Same as CallData.
\item ConversionData: Omitting the EndDate in this schedule indicates perpetual conversion rights. For American rights,
  the same comment as under CallData applies.
\item ConversionData / ConversionResets: Omitting the EndDate in this schedule indicates perpetual conversion resets.
\item DividendProtectionData: Omitting the EndDate in this schedule indicates a perpetual dividend protection schedule.
\end{itemize}

\begin{listing}[H]
\begin{minted}[fontsize=\footnotesize]{xml}
  <Trade id="ConvertibleBond">
    <TradeType>ConvertibleBond</TradeType>
    <Envelope>...</Envelope>
    <ConvertibleBondData>
      <BondData> ... </BondData>
      <CallData> ... </CallData>
      <PutData> ... </PutData>
      <ConversionData> ... </ConversionData>
      <DividendProtectionData> ... </DividendProtectionData>
      <Detachable>false</Detachable>
    </ConvertibleBondData>
  </Trade>
\end{minted}
\caption{Convertible bond set up using the detail blocks}
\label{lst:convertiblebonddata2}
\end{listing}

\underline{Specification of CallData / PutData:}

All lists specified in subnodes (except the date list itself of course) can be specified as either an explicit list of
values corresponding to the schedule dates list or using the attribute \verb+startDate+. An explicit value list can be
shorter than the list of dates, in which case the last value from the list is associated to the remaining dates.

See listings
\ref{lst:convertiblebonddata_callputdata_1},\ref{lst:convertiblebonddata_callputdata_2},\ref{lst:convertiblebonddata_callputdata_3},\ref{lst:convertiblebonddata_callputdata_4},\ref{lst:convertiblebonddata_callputdata_5},\ref{lst:convertiblebonddata_callputdata_6},\ref{lst:convertiblebonddata_callputdata_7}
for examples of exercise schedules.

\begin{itemize}

\item Styles: A list of the exercise styles. Notice that Bermudan is used to define European exercises as well, namely
as a Bermudan exercise with a single exercise date. The attribute \verb+startDate+ can be used to specify the list. \\
Allowable values: American, Bermudan

\item ScheduleData: A schedule of exercise dates (for Bermudan exercises) or start / end dates (for American exercises) \\
  Allowable values: see \ref{ss:schedule_data}.

\item Prices: A list of exercise prices in relative terms, i.e. if the price is $1.02$ then the amount paid on the
  exercise is this price times the current notional of the bond (plus accrued interest, if the price type is clean, see
  below). The attribute \verb+startDate+ can be used to specify the list.\\
  Allowable values: Any positive real number.

\item PriceType: A list of the flavour in which the exercise prices are given. The attribute \verb+startDate+ can be
  used to specify the list.\\
  Allowable values: Clean, Dirty.

\item IncludeAccrual: A list of flags specifying whether accruals have to be paid on exercise. This is independent of
  the quoting style of the exercise prices (PriceType).\\
  Allowable values: true, false

\item Soft: A list of flags specifying whether the call is soft (true) or hard (false). The attribute \verb+startDate+
  can be used to specify the list. Optional, defaults to false. Only applicable to Calls, not to Puts. Optional, if not
  given, false is assumed, i.e. hard calls. If soft calls are specified, at least one conversion exercise date with
  corresponding conversion rate must be defined under ConversionData. \\
  Allowable values: true, false

\item TriggerRatios: A list of trigger ratios $T$ for soft calls. A soft call can be executed only if the equity price
  on the exercise date is above the Conversion Price (defined below) times the trigger ratio, i.e. $S_t > C^P_tT$. Only applicable to
  Calls, not to Puts. Required for soft calls, can be omitted otherwise.\\
  
$$
Conversion Price, C^P_t = \frac{1}{ConversionRatio}
$$
  
For cross-currency trades the conversion price is usually quoted in equity ccy, i.e.  
  
$$
Conversion Price, C^P_t = \frac{1}{ConversionRatio \cdot X_t}
$$  
  
where $X_t$ converts one equity ccy unit to bond ccy  
  
  Allowable values: Any positive real number.

\item NOfMTriggers: A list of n-of-m trigger specifications for calls, i.e. the soft-call trigger defined by
  TriggerRatios must be observed on n of the m calendar days in the period before (and including) a call date. Only applicable
  to Calls, not to Puts. Optional, defaults to ``1-of-1'' \\
  Allowable values: x-of-y with x, y non-negative integers, ``1-of-1'' corresponds to a vanilla call specification

\item MakeWhole: A list of make whole conditions. Optional. Possible subnodes are:
  \begin{itemize}
    \item ConversionRatioIncrease: In case of a call exercise, the conversion ratio (applicable in case of a forced
      conversion) is adjusted upwards. The adjustment is additive, i.e. if the current conversion ratio is $CR$ the
      conversion ratio applicable in case of a forced conversion will be $CR+d$ where $d$ is interpolated from a matrix
      of effective dates (rows) and stock prices (columns). The conversion rate adjustment might be capped by a
      prespecified rate. If the exercise date / stock price lies outside the matrix, $d$ is zero, i.e. no adjustment is
      made. Notice that a soft call trigger is checked w.r.t. $CR$, i.e. the unadjusted conversion ratio.
      \begin{itemize}
      \item Cap: An upper bound for the adjusted conversion ratio. Optional, if not given, no cap will be applied.\\
        Allowable values: Any non-negative real number.
      \item StockPrices: A comma separated list of stock prices defining the interpolation grid's x values. At least two
        stock prices must be given.\\
        Allowable values: A list of non-negative real numbers.
      \item CrIncreases: A node that contains at least two subnodes CrIncrease. Each subnode must have an attribute
        startDate defining the effective date of the adjustment and a list of conversion ratio adjustments $d$. The
        number of adjustments must match the number of prices given in the StockPrices node. \\
        Allowable values: A list of non-negative real numbers.
      \end{itemize}
  \end{itemize}

\end{itemize}

\begin{listing}[H]
\begin{minted}[fontsize=\footnotesize]{xml}
  <!-- Bermudan issuer call on three dates at a clean price of 100 (hard calls),
       accruals are paid on exercise -->
  <CallData>
    <Styles>
      <Style>Bermudan</Style>
    </Styles>
    <ScheduleData>
      <Dates>
        <Dates>
          <Date>2016-08-03</Date>
          <Date>2017-08-03</Date>
          <Date>2018-08-03</Date>
        </Dates>
      </Dates>
    </ScheduleData>
    <Prices>
      <Price>1.00</Price>
    </Prices>
    <PriceTypes>
      <PriceType>Clean</PriceType>
    </PriceTypes>
    <IncludeAccruals>
      <IncludeAccrual>true</IncludeAccrual>
    </IncludeAccruals>
    <Soft>
      <Soft>false</Soft>
    </Soft>
    <TriggerRatios/>
    <NOfMTriggers>
      <NOfMTrigger>20-of-30</NOfMTrigger>
    </NOfMTriggers>
  </CallData>
\end{minted}
\caption{Convertible bond call data example 1}
\label{lst:convertiblebonddata_callputdata_1}
\end{listing}

\begin{listing}[H]
\begin{minted}[fontsize=\footnotesize]{xml}
  <!-- Bermudan issuer call on three dates at a clean price of 101, 102 and 103,
       soft calls with trigger ratio of 0.8, 0.85, 0.9,
       accrual are _not_ paid on exercise -->
  <CallData>
    <Styles>
      <Style>Bermudan</Style>
    </Styles>
    <ScheduleData>
      <Dates>
        <Dates>
          <Date>2016-08-03</Date>
          <Date>2017-08-03</Date>
          <Date>2018-08-03</Date>
        </Dates>
      </Dates>
    </ScheduleData>
    <Prices>
      <Price>1.01</Price>
      <Price>1.02</Price>
      <Price>1.03</Price>
    </Prices>
    <PriceTypes>
      <PriceType>Clean</PriceType>
    </PriceTypes>
    <IncludeAccruals>
      <IncludeAccrual>false</IncludeAccrual>
    </IncludeAccruals>
    <Soft>
      <Soft>true</Soft>
    </Soft>
    <TriggerRatios>
      <TriggerRatio>0.8</TriggerRatio>
      <TriggerRatio>0.85</TriggerRatio>
      <TriggerRatio>0.9</TriggerRatio>
    </TriggerRatios>
  </CallData>
\end{minted}
\caption{Convertible bond call data example 2}
\label{lst:convertiblebonddata_callputdata_2}
\end{listing}

\begin{listing}[H]
\begin{minted}[fontsize=\footnotesize]{xml}
  <!-- American issuer call between 2016-08-03 and 2018-08-03
       at a clean price of 100 (hard calls) -->
  <CallData>
    <Styles>
      <Style>American</Style>
    </Styles>
    <ScheduleData>
      <Dates>
        <Dates>
          <Date>2016-08-03</Date>
          <Date>2018-08-03</Date>
        </Dates>
      </Dates>
    </ScheduleData>
    <Prices>
      <Price>1.00</Price>
    </Prices>
    <PriceTypes>
      <PriceType>Clean</PriceType>
    </PriceTypes>
    <IncludeAccruals>
      <IncludeAccrual>true</IncludeAccrual>
    </IncludeAccruals>
    <Soft>
      <Soft>false</Soft>
    </Soft>
    <TriggerRatios/>
  </CallData>
\end{minted}
\caption{Convertible bond call data example 3}
\label{lst:convertiblebonddata_callputdata_3}
\end{listing}

\begin{listing}[H]
\begin{minted}[fontsize=\footnotesize]{xml}
  <!-- American issuer call between 2016-08-03 and 2020-08-03 (excl),
       hard calls at 100 between 2016-08-03 and 2018-08-03 (excl),
       soft calls at 102 between 2018-08-03 and 2019-08-03 (excl),
       soft calls at 103 between 2019-08-03 and 2020-08-03 -->
  <CallData>
    <Styles>
      <Style>American</Style>
    </Styles>
    <ScheduleData>
      <Dates>
        <Dates>
          <Date>2016-08-03</Date>
          <Date>2018-08-03</Date>
          <Date>2019-08-03</Date>
          <Date>2020-08-03</Date>
        </Dates>
      </Dates>
    </ScheduleData>
    <Prices>
      <Price>1.00</Price>
      <Price startDate="2018-08-03">1.02</Price>
      <Price startDate="2019-08-03">1.03</Price>
    </Prices>
    <PriceTypes>
      <PriceType>Clean</PriceType>
    </PriceTypes>
    <IncludeAccruals>
      <IncludeAccrual>true</IncludeAccrual>
    </IncludeAccruals>
    <Soft>
      <Soft>false</Soft>
      <Soft startDate="2018-03-03">true</Soft>
    </Soft>
    <TriggerRatios>
      <TriggerRatio>0.8</TriggerRatio>
      <TriggerRatio startDate="2019-08-03">0.9</TriggerRatio>
    </TriggerRatios>
  </CallData>
\end{minted}
\caption{Convertible bond call data example 4}
\label{lst:convertiblebonddata_callputdata_4}
\end{listing}

\begin{listing}[H]
\begin{minted}[fontsize=\footnotesize]{xml}
  <!-- Bermudan (hard) calls at 100 at 3 dates from 2016 to 2018,
       followed by American (soft) calls at 102 between 2018 and 2020 -->
  <CallData>
    <Styles>
      <Style>Bermudan</Style>
      <Style startDate="2018-08-03">American</Style>
    </Styles>
    <ScheduleData>
      <Dates>
        <Dates>
          <Date>2016-08-03</Date>
          <Date>2017-08-03</Date>
          <Date>2018-08-03</Date>
          <Date>2020-08-03</Date>
        </Dates>
      </Dates>
    </ScheduleData>
    <Prices>
      <Price>1.00</Price>
      <Price startDate="2018-08-03">1.02</Price>
    </Prices>
    <PriceTypes>
      <PriceType>Clean</PriceType>
    </PriceTypes>
    <IncludeAccruals>
      <IncludeAccrual>true</IncludeAccrual>
    </IncludeAccruals>
    <Soft>
      <Soft>false</Soft>
      <Soft startDate="2018-08-03">true</Soft>
    </Soft>
    <TriggerRatios>
      <TriggerRatio>0.8</TriggerRatio>
    </TriggerRatios>
  </CallData>
\end{minted}
\caption{Convertible bond call data example 5}
\label{lst:convertiblebonddata_callputdata_5}
\end{listing}

\begin{listing}[H]
\begin{minted}[fontsize=\footnotesize]{xml}
  <!-- Bermudan puts calls at 100, 101, 102 at 3 dates from 2016 to 2018 -->
  <PutData>
    <Styles>
      <Style>Bermudan</Style>
    </Styles>
    <ScheduleData>
      <Dates>
        <Dates>
          <Date>2016-08-03</Date>
          <Date>2017-08-03</Date>
          <Date>2018-08-03</Date>
        </Dates>
      </Dates>
    </ScheduleData>
    <Prices>
      <Price>1.00</Price>
      <Price>1.01</Price>
      <Price>1.02</Price>
    </Prices>
    <PriceTypes>
      <PriceType>Clean</PriceType>
    </PriceTypes>
    <IncludeAccruals>
      <IncludeAccrual>true</IncludeAccrual>
    </IncludeAccruals>
  </PutData>
\end{minted}
\caption{Convertible bond put data example 6}
\label{lst:convertiblebonddata_callputdata_6}
\end{listing}

\begin{listing}[H]
\begin{minted}[fontsize=\footnotesize]{xml}
<CallData>
...
   <MakeWhole>
     <ConversionRatioIncrease>
       <Cap>0.0740740</Cap>
       <StockPrices>13.50,15.00,16.20,18.00</StockPrices>
       <CrIncreases>
         <CrIncrease startDate="2020-06-25">0.0123456,0.0107487,0.0097173,0.0084567</CrIncrease>
         <CrIncrease startDate="2021-07-01">0.0123456,0.0096880,0.0086963,0.0075294</CrIncrease>
         <CrIncrease startDate="2022-07-01">0.0123456,0.0083927,0.0074222,0.0063383</CrIncrease>
         <CrIncrease startDate="2023-07-01">0.0123456,0.0069360,0.0058790,0.0048322</CrIncrease>
         <CrIncrease startDate="2024-07-01">0.0123456,0.0054453,0.0040025,0.0028833</CrIncrease>
         <CrIncrease startDate="2025-07-01">0.0123456,0.0049380,0.0000000,0.0000000</CrIncrease>
       </CrIncreases>
     </ConversionRatioIncrease>
   </MakeWhole>
</CallData>
\end{minted}
\caption{Convertible bond make whole data (conversion ratio increase)}
\label{lst:convertiblebonddata_callputdata_7}
\end{listing}

\underline{Specification of ConversionData:}

As in the case of the CallData, all lists can be specified as either an explicit list of values corresponding to the
schedule dates list or using the attribute \verb+startDate+. The ConversionRatios element is an expcetion, the given
start dates are interpreted independently of these schedule dates.

See listings \ref{lst:convertiblebonddata_conversion_1},
\ref{lst:convertiblebonddata_conversion_2},\ref{lst:convertiblebonddata_conversion_3},\ref{lst:convertiblebonddata_conversion_4},
\ref{lst:convertiblebonddata_conversion_5},\ref{lst:convertiblebonddata_conversion_6}
for examples of conversion schedules.

\begin{itemize}

\item Styles: The styles of the conversion rights. Notice that Bermudan is used to define European conversion rights as
  well, namely as a Bermudan conversion right with a single date. The attribute \verb+startDate+ can be used to
  specify the list. Can be omitted, if no conversion dates are given.\\
  Allwoable values: American, Bermudan

\item ScheduleData: The dates defining when the bond is convertible. For Bermudan exercises, the conversion can be
  executed on the single dates given in the list. For American exercises, the conversion can be executed between a given
  start and end date. Can be omitted, if no conversion rights are present.\\
  Allowable values: see \ref{ss:schedule_data}.

\item ConversionRatios: A list of conversion ratios $C^R$. The attribute \verb+startDate+ can be used to specify a date
  from which the ratio is valid. Notice that this date is always interpreted ``as is'', i.e. it is not mapped onto the
  next date in the defined schedule. If no startDate is given for a ratio, this ratio is interpreted as the initial
  ratio. \\
  Allowable values: Any non-negative real number.

\item FixedConversionAmounts: If this node is given, the conversion is specified to be conversion to fixed cash amounts
  instead of equity. If the cash amount currency is different from the bond currency, the FXIndex node must be
  given. See \ref{lst:convertiblebonddata_conversion_6} for an example. As for ConversionRatios the attribute
  \verb+startDate+ can be used to specify a date from which the amount is valid and this date is interpreted ``as is'',
  i.e. not mapped onto the next date in the defined schedule. The nodes
\begin{itemize}
  \item ConversionRatios
  \item ContingentConversion
  \item MandatoryConversion
  \item ConversionResets
  \item Underlying
  \item Exchangeable
\end{itemize}
must {\em not} be given, if this node is present. Furthermore, the following nodes from other sections are not
applicable if the conversion is specified to be fixed cash amounts, and must therefore not be given:
\begin{itemize}
\item CallData/Soft
\item CallData/TriggerRatios
\item CallData/NoMTriggers
\item CallData/MakeWhole
\item DividendProtectionData (including all subnodes)
\end{itemize}

\item ContingentConversion: This adds a condition $C^R_t S_t > B$ on the convertibility for the periods defined by the
  conversion dates. Optional.
  \begin{itemize}
  \item Observations: A list of observation modes. \\
    Allowable values: Spot (trigger is checked on the conversion date), StartOfPeriod (trigger is checked on the start
    of the conversion period defined by the dates list, for American style conversion only)
  \item Barriers: A list of barriers $B$ associated to the conversion dates. \\
    Allowable values: Positive real number or zero (conversion is not made contingent for this date).
  \end{itemize}

\item MandatoryConversion: This adds a mandatory conversion obligation at a date greater than all other conversion dates
  (if any). Optional.
  \begin{itemize}
  \item Date: The mandatory conversion date.\\
    Allowable values: Any date not earlier than the last otherwise specified conversion date.
  \item Type: The type of the mandatory conversion.\\
    Allowable values: PEPS
  \item PepsData: Details of mandatory conversion type PEPS.
    \begin{itemize}
      \item UpperBarrier: upper barrier for PEPS payoff.\\
        Allowable values: A real number.
      \item LowerBarrier: lower barrier for PEPS payoff.\\
        Allowable values:  A real number.
      \item UpperConversionRatio: conversion ratio for upper barrier in PEPS payoff.\\
        Allowable values: A real number.
      \item LowerConversionRatio: conversion ratio for lower barrier in PEPS payoff.\\
        Allowable values:  A real number.
    \end{itemize}
  \end{itemize}

\item ConversionResets: This adds a reset schedule for the conversion rate. If a reset feature is defined, only an
  initial ConversionRatio can be defined, the future conversion ratios are determined by the resets. The startDate
  attribute can be used to define references, thresholds, gearings, floors, global floors. Optional.
  \begin{itemize}
  \item ScheduleData: The conversion reset dates. \\
      Allowable values: see \ref{ss:schedule_data}.
  \item References: Whether the initial conversion price $C^P_0$ or the current conversion price $C^P_t$ is the reference for the reset.\\
    Allowable values: InitialConversionPrice, CurrentConversionPrice
  \item Thresholds: The threshold $T$ that triggers a reset ($S_t < TC^P_0$ or $S_t < TC^P_t$, depending on Reference)\\
    Allowable values: positive number or zero (disables the reset on this date effectively)
  \item Gearings: The gearings $g$ for the conversion rate adjustment. Option, defaults to $0$ (= no gearing applicable)\\
    Allowable values: positive number or zero (no gearing applicable on this date).
  \item Floors: The floors $f$ for the conversion rate adjustment. Optional, defaults to $0$ (= no floor applicable)\\
    Allowable values: positive number or zero (no floor applicable on this date)
  \item GlobalFloors: The global floors for the conversion rate adjustment. Option, defaults to $0$ (= no global floor applicable)\\
    Allowable values: positive number or zero (no global floor applicable on this date)
  \end{itemize}

\item Underlying: The equity underlying. \\
  Allwoable values: See \ref{ss:underlying}, the underlying type must be equity.

\item FXIndex: If equity ccy is different from bond ccy, an fx index for the two involved ccy is required. \\
  Allowable values:  The format of the FX Index is``FX-SOURCE-CCY1-CCY2'' as described in table \ref{tab:fxindex_data}.

\item Exchangeable: Node with data for exchangeables. Option, if omitted, the structure is considered non-exchangeable. Subnodes are:\\
  \begin{itemize}
  \item IsExchangeable: indicates whether the convertible bond is exchangeable\\
    Allowable values: true, false
  \item EquityCreditCurve: the credit curve modeling the equity issuer default, required if IsExchangeable is
    true. \\
    Allowable values: A valid credit curve identifier, e.g the ISIN of a reference bond with the ISIN: prefix:
    \verb+ISIN:XXNNNNNNNNNN+
  \item Secured: Indicates whether the convertible is secured with pledged shares or not. Optional, defaults to false.\\
    Allowable values: true, false.
  \end{itemize}
\end{itemize}

\begin{listing}[H]
\begin{minted}[fontsize=\footnotesize]{xml}
  <!-- Three conversion dates (Bermudan), conversion ratio is 0.5 -->
    <ConversionData>
      <Styles>
        <Style>Bermudan</Style>
      </Styles>
      <ScheduleData>
        <Dates>
          <Dates>
            <Date>2016-08-03</Date>
            <Date>2017-08-03</Date>
            <Date>2018-08-03</Date>
          </Dates>
        </Dates>
      </ScheduleData>
      <ConversionRatios>
        <ConversionRatio>0.05</ConversionRatio>
      </ConversionRatios>
      <Underlying>
        <Type>Equity</Type>
        <Name>RIC:.ABCD</Name>
      </Underlying>
      <FXIndex>FX-ECB-EUR-USD</FXIndex>
      <Exchangeable>
        <IsExchangeable>true</IsExchangeable>
        <EquityCreditCurve>ISIN:XS0982710740</EquityCreditCurve>
        <Secured>true</Secured>
      </Exchangeable>
    </ConversionData>
\end{minted}
\caption{Convertible bond conversion example 1}
\label{lst:convertiblebonddata_conversion_1}
\end{listing}

\begin{listing}[H]
\begin{minted}[fontsize=\footnotesize]{xml}
  <!-- American conversion between 2016-08-03 and 2020-08-03, with
       conversion ratio 0.5 for 2016-08-03 through 2018-08-03 (excl) and
       conversion ratio 0.6 for 2018-08-03 through 2020-08-03 -->
    <ConversionData>
      <Styles>
        <Style>American</Style>
      </Styles>
      <ScheduleData>
        <Dates>
          <Dates>
            <Date>2016-08-03</Date>
            <Date>2018-08-03</Date>
            <Date>2020-08-03</Date>
          </Dates>
        </Dates>
      </ScheduleData>
      <ConversionRatios>
        <ConversionRatio>0.05</ConversionRatio>
        <ConversionRatio startDate="2018-08-03">0.06</ConversionRatio>
      </ConversionRatios>
      <Underlying>
        <Type>Equity</Type>
        <Name>RIC:.ABCD</Name>
      </Underlying>
    </ConversionData>
\end{minted}
\caption{Convertible bond conversion example 2}
\label{lst:convertiblebonddata_conversion_2}
\end{listing}

\begin{listing}[H]
\begin{minted}[fontsize=\footnotesize]{xml}
  <!-- American conversion between 2016-08-03 and 2018-08-03, with conversion
       ratio 0.5, the conversion is contingent on the parity being above 1.3
       on 2016-08-03 for the conversion between 2016-08-03 and 2017-08-03 (excl)
       on 2017-08-03 for the conversion between 2017-08-03 and 2018-08-03 -->
    <ConversionData>
      <Styles>
        <Style>American</Style>
      </Styles>
      <ScheduleData>
        <Dates>
          <Dates>
            <Date>2016-08-03</Date>
            <Date>2017-08-03</Date>
            <Date>2018-08-03</Date>
          </Dates>
        </Dates>
      </ScheduleData>
      <ConversionRatios>
        <ConversionRatio>0.05</ConversionRatio>
      </ConversionRatios>
      <ContingentConversion>
        <Observations>
          <Observation>StartOfPeriod</Observation>
        </Observations>
        <Barriers>
          <Barrier>1.3</Barrier>
        </Barriers>
      </ContingentConversion>
      <Underlying>
        <Type>Equity</Type>
        <Name>RIC:.ABCD</Name>
      </Underlying>
    </ConversionData>
\end{minted}
\caption{Convertible bond conversion example 3}
\label{lst:convertiblebonddata_conversion_3}
\end{listing}

\begin{listing}[H]
\begin{minted}[fontsize=\footnotesize]{xml}
  <!-- American converion between 2016-08-03 and 2018-08-03 with CR 0.5.
       Mandatory conversion on 2020-08-03:
       LowerConversionRatio applies if stock price < LowerBarrier,
       UpperConversionRatio applies if stock price > UpperBarrier -->
    <ConversionData>
      <Styles>
        <Style>American</Style>
      </Styles>
      <ScheduleData>
        <Dates>
          <Dates>
            <Date>2016-08-03</Date>
            <Date>2018-08-03</Date>
          </Dates>
        </Dates>
      </ScheduleData>
      <ConversionRatios>
        <ConversionRatio>0.05</ConversionRatio>
      </ConversionRatios>
      <MandatoryConversion>
        <Date>2020-08-03</Date>
        <Type>PEPS</Type>
        <PepsData>
          <UpperBarrier>32.5</UpperBarrier>
          <LowerBarrier>20.5</LowerBarrier>
          <UpperConversionRatio>0.08</UpperConversionRatio>
          <LowerConversionRatio>0.03</LowerConversionRatio>
        </PepsData>
      </MandatoryConversion>
      <Underlying>
        <Type>Equity</Type>
        <Name>RIC:.ABCD</Name>
      </Underlying>
    </ConversionData>
\end{minted}
\caption{Convertible bond conversion example 4}
\label{lst:convertiblebonddata_conversion_4}
\end{listing}

\begin{listing}[H]
\begin{minted}[fontsize=\footnotesize]{xml}
  <!-- American conversion between 2016-08-03 and 2018-08-03 with CR 0.5.
       The conversion ratio is reset on 2016-11-03, 2017-02-03, 2018-05-03
       using T = 0.9, g = 0.8, f = 0.6, F = 0.6. -->
    <ConversionData>
      <Styles>
        <Style>American</Style>
      </Styles>
      <ScheduleData>
        <Dates>
          <Dates>
            <Date>2016-08-03</Date>
            <Date>2018-08-03</Date>
          </Dates>
        </Dates>
      </ScheduleData>
      <ConversionRatios>
        <ConversionRatio>0.05</ConversionRatio>
      </ConversionRatios>
      <ConversionResets>
        <ScheduleData>
          <Dates>
            <Dates>
              <Date>2016-11-03</Date>
              <Date>2017-02-03</Date>
              <Date>2018-05-03</Date>
            </Dates>
          </Dates>
        </ScheduleData>
        <References>
          <Reference>InitialConversionPrice</Reference>
        </References>
        <Thresholds>
          <Threshold>0.9</Threshold>
        </Thresholds>
        <Gearings>
          <Gearing>0.8</Gearing>
        </Gearings>
        <Floors>
          <Floor>0.7</Floor>
        </Floors>
        <GlobalFloors>
          <GlobalFloor>15</GlobalFloor>
        </GlobalFloors>
      </ConversionResets>
      <Underlying>
        <Type>Equity</Type>
        <Name>RIC:.ABCD</Name>
      </Underlying>
    </ConversionData>
\end{minted}
\caption{Convertible bond conversion example 5}
\label{lst:convertiblebonddata_conversion_5}
\end{listing}

\begin{listing}[H]
\begin{minted}[fontsize=\footnotesize]{xml}
  <!-- American conversion between 2024-08-24 and 2027-05-13, with
       conversion to 0.87 GBP cash for 2024-08-24 through 2024-11-23 (excl) and
       conversion to 0.75 GBP cash for 2024-11-23 through 2027-05-13 -->
    <ConversionData>
      <Styles>
        <Style>American</Style>
      </Styles>
      <ScheduleData>
        <Dates>
          <Dates>
            <Date>2024-08-24</Date>
            <Date>2024-11-23</Date>
            <Date>2027-05-13</Date>
          </Dates>
        </Dates>
      </ScheduleData>
      <FixedAmountConversion>
        <Currency>GBP</Currency>
        <Amounts>
          <Amount>0.87</Amount>
          <Amount startDate="2024-11-24">0.75</Amount>
        </Amounts>
      </FixedAmountConversion>
    </ConversionData>
\end{minted}
\caption{Convertible bond conversion example 6}
\label{lst:convertiblebonddata_conversion_6}
\end{listing}

\underline{Specification of DividendProtectionData:}

As for the CallData, all lists can be specified as either an explicit list of values corresponding to the schedule dates
list or using the attribute \verb+startDate+.

See listings \ref{lst:convertiblebonddata_divprot_1}, \ref{lst:convertiblebonddata_divprot_2}
for examples of dividend protection schedules.

\begin{itemize}
\item ScheduleData: The dates of the dividend protection schedule. The first date marks the date when the dividend
  protection becomes effective, i.e. dividend payments from this date on are taken into account in conversion ratio
  adjustments or passthroughs. The second date is then the first date on which the accumulated dividends between the
  first and second date trigger a conversion ratio reset or passthrough, and similar for all subsequent dates. The last
  given date is the last date with a conversion ratio reset or passthrough. \\ Allowable values: see
  \ref{ss:schedule_data}.
\item AdjustmentStyles: Whether the dividend excessing the threshold is passed through or the conversion ratio is
  adjusted. In both cases, the adjustment can be upwards only or up and down.\\
  Allwoable values: CrUpOnly, CrUpDown, CrUpOnly2, CrUpDown2, PassThroughUpOnly, PassThroughUpDown
\item DividendTypes: Whether the conversion ratio adjustment is calculated in terms of absolute or relative
  dividends. Does not have an effect for pass through dividends (should be set to Aboslute in this case).\\
  Allwoable values: Absolute, Relative
\item Thresholds: The threshold $H$. Notice that the threshold applies to each single period of the dividend protection
  schedule. If the threshold is e.g. provided on an annual basis in the terms of the convertible bond, but the dividend
  protection schedule is quarterly, then the threshold in the trade xml should be the annual threshold divided by
  $4$.\\
  Allwoable values: Any non-negativee number.
\end{itemize}

\begin{listing}[H]
\begin{minted}[fontsize=\footnotesize]{xml}
  <!-- Divdend protection based on aboslute dividend amounts via adjustment
       of the conversion rate, up-only adjustment. -->
    <DividendProtectionData>
      <ScheduleData>
        <Dates>
          <Dates>
            <Date>2016-08-03</Date>
            <Date>2017-08-03</Date>
            <Date>2018-08-03</Date>
            <Date>2019-08-03</Date>
          </Dates>
        </Dates>
      </ScheduleData>
      <AdjustmentStyles>
        <AdjustmentStyle>CrUpOnly</AdjustmentStyle>
      </AdjustmentStyles>
      <DividendTypes>
        <DividendType>Absolute</DividendType>
      </DividendTypes>
      <Thresholds>
        <Threshold>1.2</Threshold>
      </Thresholds>
    </DividendProtectionData>
\end{minted}
\caption{Convertible bond dividend protection example 1}
\label{lst:convertiblebonddata_divprot_1}
\end{listing}

\begin{listing}[H]
\begin{minted}[fontsize=\footnotesize]{xml}
  <!-- Dividend protection based on relative dividend amounts via adjustment
       of the conversion rate, up-only adjustment. -->
    <DividendProtectionData>
      <ScheduleData>
        <Dates>
          <Dates>
            <Date>2016-08-03</Date>
            <Date>2017-08-03</Date>
            <Date>2018-08-03</Date>
            <Date>2019-08-03</Date>
          </Dates>
        </Dates>
      </ScheduleData>
      <AdjustmentStyles>
        <AdjustmentStyle>CrUpOnly</AdjustmentStyle>
      </AdjustmentStyles>
      <DividendTypes>
        <DividendType>Relative</DividendType>
      </DividendTypes>
      <Thresholds>
        <Threshold>0.01</Threshold>
      </Thresholds>
    </DividendProtectionData>
\end{minted}
\caption{Convertible bond dividend protection example 2}
\label{lst:convertiblebonddata_divprot_2}
\end{listing}
