\subsubsection{Bond Total Return Swap}
\label{ss:TRS}

A vanilla Bond Total Return Swap is set up using a {\tt BondTRSData} block as shown in listing \ref{lst:bondtrsdata}. The block is comprised of three sub-blocks, which
are {\tt TotalReturnData}, {\tt FundingData}, {\tt BondData}.

\begin{itemize}
\item The {\tt BondData} block specifies the underlying bond. It is set up as for the ordinary bond, see Section~\ref{ss:bonddata}.
\item The {\tt FundingData} block specifies the funding leg, which is of either fixed or floating type. The {\tt FundingData} contains exactly one {\tt Leg}.
\item The {\tt TotalReturnData} block specifies the schedule of the total return payments and contains a "Payer" element, which determines if the counterparty receives or pays the returns of the underlying bond.
\end{itemize}


\begin{listing}[H]
%\hrule\medskip
\begin{minted}[fontsize=\footnotesize]{xml}
    <BondTRSData>
      <BondData>
      ...
      </BondData>
      <TotalReturnData>
         <Payer>false</Payer>
          <ScheduleData>
          ...
          </ScheduleData>
      </TotalReturnData>
      <FundingData>
         <LegData>
          ...
         </LegData>
      </FundingData>
    </BondTRSData>
\end{minted}
\caption{Bond Data}
\label{lst:bondtrsdata}
\end{listing}

The bond pricing requires a recovery rate that can be specified per SecurityId in the market data configuration.