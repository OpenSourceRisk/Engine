\subsubsection{Bondfuture}
\label{ss:bondfuture}

Akin to the forward bond, a bondfuture is a contract that establishes an agreement to buy or sell (determined by LongInForward)
an underlying bond at a future point in time (expiry) at an agreed price (strike amount). In contrast to the forward bond, the
underlying bond is selected by the contract seller picked from a list of eligible securities (DeliveryBasket). By its nature
this particular security is called the "cheapest to deliver" (CTD).

A bondfuture is set up using a {\tt BondFutureData} block as shown below and the trade type is \emph{BondFuture}, see \ref{lst:bondfuturetradedata}.

\begin{itemize}
  \item ContractName: This ID defines both: which bond future reference datum to take and security specific spread to be used for pricing.
  \item ContractNotional: The notional of the position, expressed in the currency of the bond.
  \item LongShort: A flag that determines whether the forward contract is entered in long (\emph{true}) or short (\emph{false}) position.
\end{itemize}

The corresponding reference data is shown in listing \ref{lst:bondfuturerefdata}.

\begin{itemize}
  \item Currency: The currency in which the future is denominated. Allowable values: See Table \ref{tab:currency}.
  \item DeliveryBasket: A list of eligible securities/bond identifiers.
  Allowable Values: A valid bond identifier, typically the ISIN of the reference bond with the ISIN: prefix
  \item DeliverableGrade: The deliverable graded restricting the deliverable underlyings. This is used for calculation of the conversion factor.
  Allowable values are: \emph{ZT, Z3N, ZF, ZN, TN, TWE, ZB, UB}.
  \item LastTradingDate: The expiry date of the future
  \item LastDeliveryDate: The settlement date of the future
  \item Settlement [Optional]: Cash or Physical. Optional, defaults to Physcial.
  \item DirtyQuotation [Optional]: Whether the market quote of the future price is dirty (true) or clean (false, default if not specified).
\end{itemize}

The latter two {\tt LastTradingDate} and {\tt LastDeliveryDate} can be derived from the following set of fields:

\begin{itemize}
  \item ContractMonth: specifies the delivery month. Allowable values are written English calendar month or its three letter abbreviation, e.g. "January" or "Jan".
  \item RootDate: used to calculate the day of the month.
  Allowable values are "first" for the beginning of the month, "end" for month end or nth weekday (e.g. "Monday,3" for third Monday of the month).
  \item ExpiryBasis: used to set the basis for the expiry derivation.
  Allowable values are "ROOT" for the above root date as start date, "SETTLEMENT" for the settlement date as a start date
  \item SettlementBasis: used to set the basis for the settlement date derivation.
  Allowable values are "ROOT" for the above root date as start date, "EXPIRY" for the expiry date as a start date
  \item ExpiryLag: Period (positive/negative) which will be added/subtracted from the ExpiryBasis, to arrive at the expiry date.
  Allowable values are any combination of integers and D for days, M for months or Y for years, e.g. 3D means a 3-day period.
  \item SettlementLag: Period (positive/negative) which will be added/subtracted from the SettlementBasis, to arrive at the settlement date.
  Allowable values are any combination of integers and D for days, M for months or Y for years, e.g. 3D means a 3-day period.
\end{itemize}

\begin{listing}[H]
  %\hrule\medskip
  \begin{minted}[fontsize=\footnotesize]{xml}
    <BondFutureData>
      <ContractName>with_ref</ContractName>
      <ContractNotional>1000000</ContractNotional>
      <LongShort>L</LongShort>
    </BondFutureData>
  \end{minted}
  \caption{BondFutureData}
  \label{lst:bondfuturetradedata}
\end{listing}


\begin{listing}[H]
  %\hrule\medskip
  \begin{minted}[fontsize=\footnotesize]{xml}
    <BondFutureReferenceData>
      <Currency>USD</Currency>
      <FairPrice>false</FairPrice>
      <ContractMonth>Mar</ContractMonth>
      <DeliverableGrade>ZN</DeliverableGrade>
      <RootDate>End</RootDate>
      <ExpiryBasis>Settlement</ExpiryBasis>
      <SettlementBasis>Root</SettlementBasis>
      <ExpiryLag>-7D</ExpiryLag>
      <SettlementLag>0D</SettlementLag>
      <DeliveryBasket>
        <SecurityId>ISIN:US91282CDJ71</SecurityId>
        <SecurityId>ISIN:US91282CEP23</SecurityId>
        <SecurityId>ISIN:US91282CLM19</SecurityId>
        <SecurityId>ISIN:US91282CLU35</SecurityId>
        <SecurityId>ISIN:US91282CMC28</SecurityId>
        <SecurityId>ISIN:US91282CMK44</SecurityId>
        <SecurityId>ISIN:US91282CMM00</SecurityId>
        <SecurityId>ISIN:US91282CMR96</SecurityId>
      </DeliveryBasket>
    </BondFutureReferenceData>
  \end{minted}
  \caption{BondFutureReferenceData}
  \label{lst:bondfuturerefdata}
\end{listing}

\subsubsection*{Derivation of the LastTradingDate and LastDeliveryDate}
The example with the reference data block above, i.e. listing \ref{lst:bondfuturerefdata}, shows how to set up a future referencing an USD 10-Year-T-Note.
The rules to derive last trading and last delivery date are taken from the CME Group primer "Understanding Treasury Futures". These are:
\begin{itemize}
  \item Last Delivery Day: Last business day of the delivery month
  \item Last Trading Day: Seventh business day preceding the last business day of the delivery month
\end{itemize}
The year is derived form the as-of date. Being the following year or the same depending whether the contract month has been passed this year or not.
Our starting point, i.e. the root date, is the 'Last business day of the delivery month'.
We achieved this by setting {\tt ContractMonth} = \emph{March} and {\tt RootDate} = \emph{End}.
From this root we can define the settlement date (last delivery) by {\tt SettlementBasis} = \emph{Root} in combination with {\tt SettlementLag} = \emph{0D}.
From the settlement, we can define the expiry date (last trading) by {\tt ExpiryBasis} = \emph{Settlement} and {\tt ExpiryLag} = \emph{-7D}.
From this we are getting the Last Trading Date to be the 20th of March and the Last Delivery Date to be the 31st of March.

\subsubsection*{CTD Selection}
The selection of the CTD bond is implemented in ORE as described in Hull's "Options, Futures and Other Derivatives":
Be \emph{sp} the quoted future settlement price, \emph{ai} accrued interest, \emph{cf} the bond specific conversion
factor and \emph{bp} the bond price.  The party with the short position receives

$$ (sp \cdot cf) + ai $$

and the cost of purchasing a bond is

$$ bp + ai $$

The cheapest-to-deliver bond is the one for which

$$ bp - (sp \cdot cf) $$

is least. The decision is taking place at future expiry.
