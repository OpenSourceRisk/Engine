\documentclass[12pt, a4paper]{article}

\usepackage{supertabular}
\usepackage{hyperref}
\usepackage{xcolor}
\usepackage{minted}
\usepackage{graphicx}
\usepackage{mathrsfs}
\usepackage{amsmath, amssymb, mathtools, empheq}
%\usepackage[disable]{todonotes}
\usepackage{todonotes}

% Turn off indentation but allow \indent command to still work.
\newlength\tindent
\setlength{\tindent}{\parindent}
\setlength{\parindent}{0pt}
\renewcommand{\indent}{\hspace*{\tindent}}

\addtolength{\textwidth}{0.8in}
\addtolength{\oddsidemargin}{-.4in}
\addtolength{\evensidemargin}{-.4in}
\addtolength{\textheight}{1.6in}
\addtolength{\topmargin}{-.8in}

\usepackage{longtable,supertabular}
\usepackage{listings}
\lstset{
  frame=top,frame=bottom,
  basicstyle=\ttfamily,
  language=XML,
  tabsize=2,
  belowskip=2\medskipamount
}

\usepackage{float}
\usepackage{tabu}
\tabulinesep=1.0mm
\restylefloat{table}

\usepackage{siunitx}

%\usepackage[colorlinks=true]{hyperref}


\usepackage{xcolor}
\definecolor{gray}{rgb}{0.4,0.4,0.4}
\definecolor{darkblue}{rgb}{0.0,0.0,0.6}
\definecolor{cyan}{rgb}{0.0,0.6,0.6}

\lstset{
  basicstyle=\ttfamily,
  columns=fullflexible,
  showstringspaces=false,
  commentstyle=\color{gray}\upshape
}

\lstdefinelanguage{XML}
{
  morestring=[b]",
  morestring=[s]{>}{<},
  morecomment=[s]{<?}{?>},
  stringstyle=\color{black},
  identifierstyle=\color{darkblue},
  keywordstyle=\color{cyan},
  morekeywords={xmlns,version,type}% list your attributes here
}


\usemintedstyle{manni}
\definecolor{mintedBg}{rgb}{0.98,0.98,0.70}

\renewcommand\P{\ensuremath{\mathbb{P}}}
\newcommand\E{\ensuremath{\mathbb{E}}}
\newcommand\Q{\ensuremath{\mathbb{Q}}}
\newcommand\I{\mathds{1}}
\newcommand\F{\ensuremath{\mathcal F}}
\newcommand\G{\ensuremath{\mathcal G}}
\newcommand\V{\mathds{V}}
\newcommand\YOY{{\rm YOY}}
\newcommand\Prob{\ensuremath{\mathbb{P}}}
\newcommand{\D}[1]{\mbox{d}#1}
\newcommand{\NPV}{\mathit{NPV}}
\newcommand{\CVA}{\mathit{CVA}}
\newcommand{\DVA}{\mathit{DVA}}
\newcommand{\FVA}{\mathit{FVA}}
\newcommand{\COLVA}{\mathit{COLVA}}
\newcommand{\FCA}{\mathit{FCA}}
\newcommand{\FBA}{\mathit{FBA}}
\newcommand{\KVA}{\mathit{KVA}}
\newcommand{\MVA}{\mathit{MVA}}
\newcommand{\PFE}{\mathit{PFE}}
\newcommand{\EE}{\mathit{EE}}
\newcommand{\EPE}{\mathit{EPE}}
\newcommand{\ENE}{\mathit{ENE}}
\newcommand{\PD}{\mathit{PD}}
\newcommand{\LGD}{\mathit{LGD}}

\begin{document}

\title{Open Risk Engine \\ User Guide}
\date{\today}
\maketitle

\newpage

%-------------------------------------------------------------------------------
\section*{Document History}

\begin{center}
\begin{supertabular}{|l|l|l|p{7cm}|}
\hline
Version & Date & Author & Comment \\
\hline
0.2 & 12 May & Roland Lichters & initial version\\
\hline
\end{supertabular}
\end{center}

\vspace{3cm}

\newpage

\tableofcontents
\newpage

\section{Introduction}

Open Risk Engine (ORE) \cite{ORE} is a free/open software project sponsored by Quaternion Risk Management \cite{QRM} and based on QuantLib, the free/open source library for quantitative finance \cite{QL}. 

ORE currently provides portfolio pricing, cash flow generation, sensitivity analysis and a range of contemporary derivative portfolio analytics. The latter are based on a Monte Carlo simulation framework which yields the evolution of various {\bf credit exposure} and {\bf market risk measures} 
\begin{itemize}
\item EE aka EPE (Expected Exposure or Expected Positive Exposure), 
\item ENE (Expected Negative Exposure, i.e. the counterparty's perspective), 
\item 'Basel' exposure measures relevant for regulatory capital charges under internal model methods 
\item PFE (Potential Future Exposure at some use defined quantile), 
\item VaR (Value at Risk),
\item ES (Expected Shortfall),
\end{itemize}
and {\bf derivative value adjustments}
\begin{itemize}
\item CVA (Credit Value Adjustment)
\item DVA (Debit Value Adjustment)
\item FVA (Funding Value Adjustment)
\item COLVA (Collateral Value Adjustment).
\end{itemize}
Moreover, ORE computes {\bf regulatory capital charges} for counterparty credit risk under the new standardized approach (SA-CCR). The Monte Carlo based market risk measures are complemented by parametric VaR, as well as {\bf ISDA's Standard Initial Margin}.

\medskip
The first release of ORE covers the simulation of interest rate and FX risk factors and portfolios of Interest Rate Swaps, FX Forwards, Cross Currency Swaps, FX Options and Swaptions. Subsequent releases will extend the derivative product and the risk factor range to Inflation, Credit, Equity and Commodity. With the introduction of credit risk factors, the scope will also be extended to cover cash products (loans and bonds) and related portfolio analytics.    

\medskip
Ultimately - after going through a series of releases, community feedback and contributions - the project aims at establishing a framework and platform for {\bf transparent pricing and risk analysis} that can be adapted for production purposes in financial institutions and used as a benchmarking and validation tool to test in-house and black box vendor solutions applied in the industry. 

\medskip
For details on the models applied in ORE's risk factor evolution we refer the reader to {\em Modern Derivatives Pricing and Credit Exposure Analysis} \cite{Lichters}: The IR/FX risk factor evolution is based on a cross currency model consisting of an arbitrage free combination of Linear Gauss Markov models for all interest rates and lognormal processes for the FX rates, calibrated to cross currency discounting and forward curves, Swaptions and FX Options.%, also summarised in \cite{XCCYLGM}.
 
\vspace{1em}

This document focuses on instructions how to use ORE to cover basic workflows from individual deal analysis to portfolio processing. Starting with a series of examples the guide addresses
\begin{itemize}
\setlength{\itemsep}{0pt}
\item Launching ORE using its command line application
\item Analytics results and standard reports
\item Interactive analysis using Jupyter, Calc and Excel 
\item ORE parametrisation
\item Market and trade data input
\end{itemize}

\newpage

%========================================================
\section{Installation}
%========================================================

\subsection{ORE for End Users}

\subsection{ORE for Developers}

\subsection{Dependencies}

\subsubsection{Gnuplot}

Gnuplot is required for running the examples in section \ref{sec:examples} under OS X and Linux.

\subsubsection*{Windows}

\subsubsection*{OSX}

Installation on OS X is easiest using Homebrew, the package manager for OS X:

\medskip
\centerline{\tt brew install gnuplot --with-aquaterm --with-cairo --with-pdflib-lite } 
\medskip

AquaTerm support will only be built into Gnuplot if the standard AquaTerm
package has already been installed onto your system.
If you subsequently remove AquaTerm, you will need to uninstall and then
reinstall Gnuplot.

\subsubsection*{Linux}

\subsubsection{Jupyter}

Jupyter is part of the Anaconda Open Data Science Analytics Platform \cite{Anaconda}.
Anaconda installation instructions for Windows, OS X and Linux are available on the anaconda site, with graphical installers for Windows and OS X.


\todo[inline]{Add installation instructions}

%========================================================
\section{Examples}\label{sec:examples}
%========================================================

The examples shown in the following are intended to help getting started with ORE, and to serve as plausibility checks for the simulation results generated with ORE.
All results can be produced with the {\tt run.sh} scripts in the ORE release's {\tt Examples} folder. In a nutshell, all scripts call ORE's command line application with a single input XML file

\medskip
\centerline{\tt ore[.exe] input.xml}
\medskip

The structure of the input file and of the portfolio, market and other configuration files referred to therein will be explained in section \ref{sec:configuration}. 

%--------------------------------------------------------
\subsection{Interest Swap Exposure}\label{sec:example1}
%--------------------------------------------------------

We start with a vanilla single currency swap (currency EUR, maturity 20y, notional 10m, receive fixed 2\% annual, pay 6M-Euribor flat). The market yield curves (for both discounting and forward projection are manipulated to be flat at 2\% for all maturities, i.e. the Swap is at the money initially and remains at the money on average throughout its life. Running ORE in directory {\tt Examples/Example\_1} with

\medskip
\centerline{\tt ./run.sh } 
\medskip

yields the exposure evolution in 

\medskip
\centerline{\tt Examples/Example\_1/Output/plot.pdf } 
\medskip

and shown in figure \ref{fig_1}. 
\begin{figure}[hbt]
\begin{center}
\includegraphics[scale=1.0]{example_swap_1.pdf}
\end{center}
\caption{Vanilla ATM swap expected exposure in a flat market environment from both parties' perspectives. The black symbols are European Swaption prices.}
\label{fig_1}
\end{figure}
The black symbols are prices of European Swaptions with expiry at the symbol's time and otherwise same underlying as the Swap considered here. Both Swap simulation and Swaption pricing are run with calls to the ORE executable, essentially 

\medskip
\centerline{\tt ore[.exe] ore.xml} 

\centerline{\tt ore[.exe] ore\_swaption.xml} 
\medskip

which are wrapped into the script {\tt Examples/Example\_1/run.sh} provided with the ORE release.
It is instructive to look into the input folder in Examples/Example\_1, the content of the main input file {\tt ore.xml}, together with the explanations in section \ref{sec:configuration}.

\medskip
Moving to {\tt Examples/Example\_4}, we see what changes when using a realistic (non-flat) market environment as of 26/02/2016. Running the example with

\medskip
\centerline{\tt ./run.sh } 
\medskip

yields the exposure evolution in 

\medskip
\centerline{\tt Examples/Example\_4/Output/plot.pdf } 
\medskip

shown in figure \ref{fig_2}.
\begin{figure}[hbt]
\begin{center}
\includegraphics[scale=1.0]{example_swap_3.pdf}
\end{center}
\caption{Vanilla ATM swap expected exposure in a realistic market environment as of 26/02/2016 from both parties' perspectives. The Swap is the same as in figure \ref{fig_1} but receiving fixed 1\%, at the money on 26/02/2016. The symbols are the prices of European payer and receiver Swaptions.}
\label{fig_2}
\end{figure}
In this case, where the curves (discount and forward) are upward sloping, the receiver swap is at the money at inception only and moves (on average) out of the money during its life. Similarly, the swap moves into the money from the counterparty's perspective. Hence the expected exposure evolutions from our perspective (EPE) and the counterparty's perspective (ENE) 'detach' here, while both can still be be reconciled with payer and receiver Swaption prices.

%--------------------------------------------------------
\subsection{European Swaption Exposure}\label{sec:european_swaption}
%--------------------------------------------------------

This demo case in folder {\tt Examples/Example\_7} shows the exposure evolution of European Swaptions with cash and physical delivery, respectively, see figure \ref{fig_3}.
\begin{figure}[hbt]
\begin{center}
\includegraphics[scale=1.0]{example_swaption.pdf}
\end{center}
\caption{European Swaption exposure evolution, expiry in 10 years, final maturity in 20 years, for cash and physical delivery.}
\label{fig_3}
\end{figure}
The delivery type (cash vs physical) yields significantly different valuations as of today due to the steepness of the relevant yield curves (EUR). The cash settled Swaption's exposure graph is truncated at the exercise date, whereas the physically settled Swaption exposure turns into a Swap-like exposure after expiry. For comparison, the example also provides the exposure evolution of the underlying forward starting Swap which yields a somewhat higher exposure after the forward start date than the physically settled Swaption. This is due to scenarios with negative swap NPV at expiry and positive NPVs thereafter.

%--------------------------------------------------------
\subsection{Bermudan Swaption Exposure}
%--------------------------------------------------------

This demo case in folder {\tt Examples/Example\_10} shows the exposure evolution of Bermudan rather than European Swaptions with cash and physical delivery, respectively, see figure \ref{fig_3b}.
\begin{figure}[hbt]
\begin{center}
\includegraphics[scale=1.0]{example_bermudan_swaption.pdf}
\end{center}
\caption{Bermudan Swaption exposure evolution, 5 annual exercise dates starting in 10 years, final maturity in 20 years, for cash and physical delivery.}
\label{fig_3b}
\end{figure}
The underlying Swap is the same as in the European Swaption example in section \ref{sec:european_swaption}. Note in particular the difference between the Bermudan and European Swaption exposures with cash settlement: The Bermudan shows the typical step-wise decrease due to the series of exercise dates. Also note that we are using the same Bermudan option pricing engines for both settlement types, in contrast to the European case, so that the Bermudan option cash and physical exposures are identical up to the first exercise date. identical . When running this example, you will notice the significant difference in computation time compared to the European case (ballpark 30 minutes here for 2 Swaptions, 1000 samples, 90 time steps). The Bermudan example is way slower because we use an LGM grid engine for pricing under scenarios in this case. In a realistic context one would more likely resort to American Monte Carlo simulation, feasible in ORE, but not provided in the first release. However, this implementation can be used to benchmark any faster / more sophisticated approach to Bermudan Swaption exposure simulation.

%--------------------------------------------------------
\subsection{Callable Swap Exposure}
%--------------------------------------------------------

This demo case in folder {\tt Examples/Example\_6} shows the exposure evolution of a European callable Swap, represented as two trades - the non-callable Swap and a Swaption with physical delivery. We have sold the call option, i.e. the Swaption is a right for the counterparty to enter into an offsetting Swap which economically terminates all future flows if exercised. The resulting exposure evolutions for the individual components (Swap, Swaption), as well as the callable Swap are shown in figure \ref{fig_4}. 
\begin{figure}[hbt]
\begin{center}
\includegraphics[scale=1.0]{example_callable_swap.pdf}
\end{center}
\caption{European callable Swap represented as a package consisiting of non-callable Swap and Swaption. The Swaption has physical delivery and offsets all future Swap cash flows if exercised. The exposure evolution of the package is shown here as 'EPE NettingSet' (green line). This is covered by the pink line, the exposure evolution of the same Swap but with maturity on the exercise date. The graphs match perfectly here, because the example Swap is deep in the money and exercise probability is close to one. }
\label{fig_4}
\end{figure}
The example is an extreme case where the underlying Swap is deep in the money (receiving fixed 5\%), and hence the call exercise probability is close to one. Modify the Swap and Swaption fixed rates closer to the money ($\approx$ 1\%) to see the deviation between net exposure of the callable Swap and the exposure of a 'short' Swap with maturity on exercise.  

%--------------------------------------------------------
\subsection{FX Forward Exposure}\label{sec:fxfwd}
%--------------------------------------------------------

The example in folder {\tt Examples/Example\_2} generates the exposure evolution for a EUR/USD FX Forward transaction with value date in 10Y. This is a particularly simple show case because of the single cash flow in 10Y. On the other hand it checks the cross currency model implementation by means of comparison to analytic limits - EPE and ENE at the trade's value date must match corresponding Vanilla FX Option prices, as shown in figure \ref{fig_5}.  
\begin{figure}[hbt]
\begin{center}
\includegraphics[scale=1.0]{example_fxforward.pdf}
\end{center}
\caption{EUR/USD FX Forward expected exposure in a realistic market environment as of 26/02/2016 from both parties' perspectives. Value date is obviously in 10Y. The flat lines are FX Option prices which coincide with EPE and ENE, respectively, on the value date.}
\label{fig_5}
\end{figure}

%--------------------------------------------------------
\subsection{Cross Currency Swap Exposure}
%--------------------------------------------------------

The case in {\tt Examples/Example\_8} is a vanilla cross currency swap. It shows the typical blend of and interest rate swap's saw tooth exposure evolution with an FX forward's exposure which increases monotonically to final maturity, see figure \ref{fig_6}. \todo[inline]{Add notional resetting feature and example}
\begin{figure}[hbt]
\begin{center}
\includegraphics[scale=1.0]{example_ccswap.pdf}
\end{center}
\caption{Cross Currency Swap exposure evolution without mark-to-market notional reset.}
\label{fig_6}
\end{figure}

%--------------------------------------------------------
\subsection{FX Option Exposure}
%--------------------------------------------------------

This example (in folder {\tt Examples/Example\_2}, as the FX Forward example) illustrates the exposure evolution for an FX Option, see figure \ref{fig_7}. 
\begin{figure}[hbt]
\begin{center}
\includegraphics[scale=1.0]{example_fxoption_fwdvariance_corrected.pdf}
\end{center}
\caption{EUR/USD FX Call and Put Option exposure evolution, same underlying and market data as in section \ref{sec:fxfwd}, compared to the call and put option price as of today (flat line).}
\label{fig_7}
\end{figure}
Recall that the FX Option value $NPV(t)$ as of time $0 \leq t \leq T$ satisfies
\begin{align*}
\frac{NPV(t)}{N(t)} &= \mbox{Nominal}\times\E_t\left[\frac{(X(T) - K)^+}{N(T)}\right]\\
NPV(0) &= \E\left[\frac{NPV(t)}{N(t)}\right] = \E\left[\frac{NPV^+(t)}{N(t)} \right]= \EPE(t) 
\end{align*}
One would therefore expect a flat exposure evolution up to option expiry. The deviation from this in ORE's simulation is due to the pricing approach chosen here under scenarios. A Black FX option pricer is used with simulated interest rate curves input and {\em deterministic} Black volatility derived from today's volatility structure (pushed or rolled forward, see section \ref{sec:sim_market}). The deviation is  removed by extending the volatility modelling, e.g. implying model consistent Black volatilities in each simulation step on each path.
\todo[inline]{Add exposure evolution graph with 'simulated' FX vol}
 
%--------------------------------------------------------
\subsection{Netting and Collateral}
%--------------------------------------------------------

In this example (see folder {\tt Examples/Example\_5}) we showcase a small netting set consisting of three swaps in different currencies, with different collateral choices
\begin{itemize}
\item no collateral - figure \ref{fig_8},
\item collateral with threshold (THR) 1m EUR, minimum transfer amount (MTA) 100k EUR, margin period of risk (MPoR) 2w - figure \ref{fig_9}
\item collateral with zero THR and MTA, and MPoR 2w - figure \ref{fig_10}
\end{itemize}
 
\begin{figure}[hbt]
\begin{center}
\includegraphics[scale=1.0]{example_nocollateral_epe.pdf}
\end{center}
\caption{Three swaps netting set, no collateral.}
\label{fig_8}
\end{figure}

\begin{figure}[hbt]
\begin{center}
\includegraphics[scale=1.0]{example_threshold_epe.pdf}
\end{center}
\caption{Three swaps netting set, THR=1m EUR, MTA=100k EUR, MPoR=2w.}
\label{fig_9}
\end{figure}

%\begin{figure}[hbt]
%\begin{center}
%\includegraphics[scale=1.0]{example_mta_epe.pdf}
%\end{center}
%\caption{Three swaps, threshold = 0, mta > 0.}
%\label{fig_7}
%\end{figure}

\begin{figure}[hbt]
\begin{center}
\includegraphics[scale=1.0]{example_mpor_epe.pdf}
\end{center}
\caption{Three swaps, THR=MTA=0, MPoR=2w.}
\label{fig_10}
\end{figure}

%--------------------------------------------------------
\subsection{CVA, DVA, FVA, COLVA and Collateral Floor}
%--------------------------------------------------------

We use one of the cases in {\tt Examples/Example\_5} to demonstrate the
XVA outputs, see folder {\tt Examples/Example\_5/Output/collateral\_threshold}.

\medskip
The summary of CVA, DVA, FVA, COLVA and Collateral Floor is provided in file {\tt xva.csv}. 
The file includes the allocated CVA and DVA introduced in the next section. The following table illustrates the file's layout,
omitting the three right-most columns containing allocated data. 

\begin{center}
\footnotesize
\begin{tabular}{|l|l|r|r|r|r|r|r|}
\hline
TradeId & NettingSetId & CVA & DVA & FBA & FCA & COLVA & CollateralFloor \\ %& AllocatedCVA & AllocatedDVA & AllocationMethod \\
\hline
 & CUST\_A & 32464 & 43083 & 40016 & 73582 & 3524 & 200651 \\ %& 32464.4 & 43082.6 & Marginal \\
70309 & CUST\_A & 110673 & 211303 & 134234 & 396001 & n/a & n/a \\ %& 20934 & 25703.8 & Marginal \\
938498 & CUST\_A & 73000 & 81002 & 133391 & 154382 & n/a & n/a \\ %& 6569.42 & 7063.29 & Marginal \\
919020 & CUST\_A & 74031 & 86618 & 119285 & 147321 & n/a & n/a \\%& 4961 & 10315.5 & Marginal \\
\hline
\end{tabular}
\end{center}

The line(s) with empty TradeId column contain values at netting set level, the others contain uncollateralised single-trade VAs.
Note that COLVA and Collateral Floor are only available at netting set level at which collateral is posted.

\medskip
Detailed output is written for COLVA and Collateral Floor to file {\tt colva\_nettingset\_*.csv} which shows the 
incremental contributions to these two VAs through time.


%--------------------------------------------------------
\subsection{Exposure \& XVA Allocation to Trades}
%--------------------------------------------------------

The same example considered in the previous section (see folder {\tt Examples/Example\_5}) also features the allocation of netting set exposure and XVA to the trade level as frequently required by finance functions.  We start again with the uncollateralised case in figure \ref{fig_8}, followed by the case with threshold 1m EUR in figure \ref{fig_9}.
\begin{figure}[hbt]
\begin{center}
\includegraphics[scale=1.0]{example_nocollateral_allocated_epe.pdf}
\end{center}
\caption{Exposure allocation without collateral.}
\label{fig_12}
\end{figure}
In both cases we apply the {\em marginal} (Euler) allocation method as published by Pykhtin and Rosen in 2010, hence we see the typical negative EPE for one of the trades at times when it reduces the netting set exposure. The case with collateral moreover shows the typical spikes in the allocated exposures.
\begin{figure}[hbt]
\begin{center}
\includegraphics[scale=1.0]{example_threshold_allocated_epe.pdf}
\end{center}
\caption{Exposure allocation with collateral and threshold 1m EUR.}
\label{fig_13}
\end{figure}
The analytics results also feature allocated XVAs in file {\tt xva.csv} which are derived from the allocated exposure profiles. Note that ORE also offers alternative allocation methods to the marginal method by Pykhtin/Rosen, which can be explored with {\tt Examples/Example\_5}.

%========================================================
\section{Visualisation}\label{sec:visualisation}
%========================================================

\subsection{Jupyter}\label{sec:jupyter}

\todo[inline]{Add Jupyter section}

\subsection{Calc}\label{sec:calc}

ORE comes with a simple LibreOffice Calc \cite{LO} sheet as an ORE launcher and basic result viewer. This is demonstrated on the example in section \ref{sec:example1}. It is currently based on the stable LibreOffice version 5.0.6 and tested on Mac OSX only. Launch Calc following the instructions in Example\_1's Readme: Open a terminal, change to directory {\tt Examples/Example\_1}, and run

\medskip
{\centerline{\tt ./launchCalc.sh} }

\medskip
This will show the blank sheet in figure \ref{fig_14}.
\begin{figure}[hbt]
\begin{center}
\includegraphics[scale=0.5]{demo_calc_1}
\end{center}
\caption{Calc sheet after launching.}
\label{fig_14}
\end{figure}
The user can choose a configuration (one of the ore*.xml files in Example\_1's subfolder Input) by hitting the 'Select' button. Initially Input/ore.xml is pre-selected. The ORE process is then kicked off by hitting 'Run'. Once completed, pre-selected results can be viewed by hitting 'View' which should show the result in figure \ref{fig_15}.
\begin{figure}[hbt]
\begin{center}
\includegraphics[scale=0.5]{demo_calc_2}
\end{center}
\caption{Calc sheet after hitting 'Run' and 'View'.}
\label{fig_15}
\end{figure}

\todo[inline]{Remove hard-coded file names from Python scripts}
\todo[inline]{Calc example on Windows and Linux} 
%\todo[inline]{Support both OpenOffice and LibreOffice Calc.}
\todo[inline]{Harmonise layout with Excel launcher} 

\subsection{Excel}

The Python module {\tt xlwings} \cite{xlwings} provides similar interaction between Python and Excel on both Windows and Mac OSX, as does the Python uno module for LibreOffice.

\todo[inline]{Add Excel section, using VBA macros only}


%========================================================
\section{Parametrisation}\label{sec:configuration}
%========================================================

ORE's batch version is kicked off with a single command line parameter 

\medskip
\centerline{\tt ore[.exe] ore.xml}
\medskip

which points to the 'master input file' referred to  as {\tt ore.xml} subsequently. 
This file is the starting point of the engine's configuration explained in the following sub section.

\subsection{Master Input File: {\tt ore.xml}}\label{sec:master_input}

The master input file contains general setup information (paths to configuration, trade data and market data), as well as the selection and configuration of  analytics to produce. The file has an opening and closing root element {\tt <OpenRiskEngine>}, {\tt </OpenRiskEngine>}, and within these three sections 
\begin{itemize}
\item Setup
\item Markets
\item Analytics
\end{itemize}
which we will explain in the following.

\subsubsection{Setup}

This subset of data is easiest explained using an example
{\footnotesize
\begin{lstlisting}[caption={ORE setup example},
 	label=lst:ore_setup]
<Setup>    
	<Parameter name="asofDate">2016-02-26</Parameter>
	<Parameter name="inputPath">Input</Parameter>
	<Parameter name="outputPath">Output</Parameter>
	<Parameter name="logFile">log.txt</Parameter>
	<Parameter name="marketDataFile">../../Input/market_20160226.txt</Parameter>
	<Parameter name="fixingDataFile">../../Input/fixings_20160226.txt</Parameter>
	<Parameter name="implyTodaysFixings">Y</Parameter>
	<Parameter name="curveConfigFile">../../Input/curveconfig.xml</Parameter>
	<Parameter name="conventionsFile">../../Input/conventions.xml</Parameter>
	<Parameter name="marketConfigFile">../../Input/todaysmarket.xml</Parameter>
	<Parameter name="pricingEnginesFile">../../Input/pricingengine.xml</Parameter>
	<Parameter name="portfolioFile">portfolio.xml</Parameter>
</Setup>
\end{lstlisting}
}

Parameter names are self explanatory: Input and output path are interpreted relative from the directory where the ORE executable is started, but can also be specified using absolute paths. All file names are then interpreted relative to the 'inputPath' and'outputPath', respectively. 

When ORE starts, it will initialise today's market, i.e. load market data and fixings, and build all term structures as specified in {\tt todaysmarket.xml}. 
Moreover, ORE will load the trades in {\tt portfolio.xml} and link them with pricing engines as specified in {\tt pricingengine.xml}. When parameter {\tt    
implyTodaysFixings} is set to Y, today's fixings would not be loaded but implied, relevant when pricing/bootstrapping off hypothetical market data as e.g. in scenario analysis and stress testing.

\subsubsection{Markets}\label{sec:master_input_markets}

The {\tt Markets} section is used to choose the two market configurations for simulation model calibration and the subsequent market simulation phase, respectively. These configurations have to defined in {\tt todaysmarket.xml} (see section \ref{sec:market} below). For example, the calibration of the simulation model's interest rate components requires local OIS discounting whereas the simulation phase requires cross currency adjusted discount curves to get FX product pricing right. So far, the market configurations are used only to distinguish discount curve sets, but the market configuration concept in ORE applies to all term structure types.

{\footnotesize
\begin{lstlisting}[caption={ORE markets},
 	label=lst:ore_markets]
<Markets>
	<Parameter name="lgmcalibration">collateral_inccy</Parameter>
	<Parameter name="fxcalibration">collateral_eur</Parameter>
	<Parameter name="pricing">collateral_eur</Parameter>
	<Parameter name="simulation">collateral_eur</Parameter>
</Markets>
\end{lstlisting}
}

\subsubsection{Analytics}

The {\tt Analytics} section lists all permissible analytics using tags {\tt <Analytic type="..."> ... </Analytic>} where type can be (so far) in
\begin{itemize}
\item npv
\item cashflow
\item curves
\item simulation
\item xva
\item initialMargin
\end{itemize}

Each {\tt Analytic} section contains a list of key/value pairs to parameterise the analysis of the form {\tt <Parameter name="key">value</Parameter>}. Each analysis
must have one key {\tt active} set to Y or N to activate/deactivate this analysis. 
The following listing shows the parametrisation of the first three basic analytics in the list above.

{\footnotesize
\begin{lstlisting}[caption={ORE analytics: npv, cashflow and curves},
 	label=lst:ore_analytics]
<Analytics>
    
	<Analytic type="npv">
		<Parameter name="active">Y</Parameter>
		<Parameter name="baseCurrency">EUR</Parameter>
		<Parameter name="outputFileName">npv.csv</Parameter>
	</Analytic>      

	<Analytic type="cashflow">
		<Parameter name="active">Y</Parameter>
		<Parameter name="outputFileName">flows.csv</Parameter>
	</Analytic>      

	<Analytic type="curves">
		<Parameter name="active">Y</Parameter>
		<Parameter name="configuration">default</Parameter>
		<Parameter name="grid">240,1M</Parameter>
		<Parameter name="outputFileName">curves.csv</Parameter>
	</Analytic>      

	<Analytic type="...">
		<!-- ... -->
	</Analytic>      

</Analytics>      
\end{lstlisting}
}

The curves analytic exports all yield curves that have been built according to the specification in {\tt todaysmarket.xml}. Key {\tt configuration} selects the curve set to be used (see explanation in the previous Markets section).  Key {\tt grid} defines the time grid on which the yield curves are evaluated, in the example above a grid of 240 monthly time steps from today. The discount factors for all curves with configuration default will be exported on this monthly grid into the csv file specified by key {\tt outputFileName}.

\medskip
The purpose of the {\tt simulation} 'analytics' is to run a Monte Carlo simulation which evolves the market as specified in the simulation config file. The primary result is an NPV cube file, i.e. valuations of all trades in the portfolio file (see section Setup), for all future points in time on the simulation grid and for all paths. Apart from the NPV cube, additional scenario data (such as simulated overnight rates etc) are stored in this process which are needed for subsequent XVA analytics.  

{\footnotesize
\begin{lstlisting}[caption={ORE analytic: simulation},
 	label=lst:ore_simulation]
<Analytics>
	<Analytic type="simulation">
		<Parameter name="active">Y</Parameter>
		<Parameter name="simulationConfigFile">simulation.xml</Parameter>
		<Parameter name="pricingEnginesFile">../../Input/pricingengine.xml</Parameter>
		<Parameter name="baseCurrency">EUR</Parameter>
		<Parameter name="storeScenarios">N</Parameter>
		<Parameter name="cubeFile">cube.dat</Parameter>
		<Parameter name="additionalScenarioDataFileName">scenariodata.dat</Parameter>
		<Parameter name="disableEvaluationDateObservation">N</Parameter>
	</Analytic>
</Analytics>      
\end{lstlisting}
}

The pricing engines file specifies how trades are priced under future scenarios which can differ from pricing as of today (specified in section Setup).
Key base currency determines into which currency all NPVs will be converted. Key store scenarios (Y or N) determines whether the market scenarios are written to a file for later reuse.
\todo[inline]{Implement the store scenarios option, add file name to parametrisation}
\todo[inline]{Expose all performance options (fixing and observer model)} 

\medskip
The xva analytic section offers CVA, DVA, FVA and COLVA calculations which can be selected/deselected here individually. All XVA calculations depend on a previously generated NPV cube (see above) which is referenced here via the {\tt cubeFile} parameter. This means one can re-run the xva analytics without regenerating the cube each time. The xva reports depend in particular on the settings in the {\tt csaFile} which determines CSA details such as margining frequency, collateral thresholds, minimum transfer amounts, margin period of risk. By splitting the processing in to pre-processing (cube generation) and post-processing (aggregation and XVA analysis) it is possible to vary these CSA details and analyse their impact on XVAs quickly without re-generating the NPV cube.  

{\footnotesize
\begin{lstlisting}[caption={ORE analytic: xva},
 	label=lst:ore_xva]
<Analytics>
	<Analytic type="xva">
		<Parameter name="active">Y</Parameter>
		<Parameter name="csaFile">netting.xml</Parameter>
		<Parameter name="cubeFile">cube.dat</Parameter>
		<Parameter name="scenarioFile">scenariodata.dat</Parameter>
		<Parameter name="baseCurrency">EUR</Parameter>
		<Parameter name="exposureProfiles">Y</Parameter>
		<Parameter name="quantile">0.95</Parameter>
		<Parameter name="calculationType">Symmetric</Parameter>      
		<Parameter name="allocationMethod">None</Parameter>    
		<Parameter name="marginalAllocationLimit">1.0</Parameter>
		<Parameter name="exerciseNextBreak">N</Parameter>
		<Parameter name="cva">Y</Parameter>
		<Parameter name="dva">N</Parameter>
		<Parameter name="dvaName">BANK</Parameter>
		<Parameter name="fva">N</Parameter>
		<Parameter name="fvaBorrowingCurve">BANK_EUR_BORROW</Parameter>
		<Parameter name="fvaLendingCurve">BANK_EUR_LEND</Parameter>
		<Parameter name="colva">Y</Parameter>
		<Parameter name="collateralSpread">0.0010</Parameter>
		<Parameter name="collateralFloor">Y</Parameter>
		<Parameter name="rawCubeOutputFile">rawcube.csv</Parameter>
		<Parameter name="netCubeOutputFile">netcube.csv</Parameter>     
	</Analytic>
</Analytics>
\end{lstlisting}
}

Further parameters:
\begin{itemize}
\item {\tt csaFile:} Netting set definitions file covering CSA details such as margining frequency, thresholds, minimum transfer amounts, margin period of risk 
\item {\tt cubeFile:} NPV cube file previously generated and to be post-processed here
\item {\tt scenarioFile:} Scenario data previously generated and used in the post-processor (simulated index fixings and FX rates) 
\item {\tt baseCurrency:} Expression currency for all NPVs, value adjustments, exposures 
\item {\tt eposureProfiles:} Flag to enable/disable exposure output  
\item {\tt quantile} Confidence level for Potential Future Exposure (PFE) 
reporting
\item {\tt calculationType} Determines the settlement of margin calls: Symmetric - margin for both counterparties settled after the margin period of risk; AsymmetricCVA - margin requested from the counterparty settles with delay, margin requested from us settles immediately; AsymmetricDVA - vice versa). \todo[inline]{Move calculationType into the {\tt csaFile}?}   
\item {\tt allocationMethod:} XVA allocation method, choices are {\em None, Marginal, RelativeXVA} 
\item {\tt marginalAllocationLimit:} The marginal allocation method ala Pykhtin/Rosen breaks down when the netting set value vanishes while the exposure does not. This parameter acts as a cutoff for the marginal allocation when the absolute netting set value falls below this limit and switches to equal distribution of the exposure in this case.  
\item {\tt exerciseNextBreak:} Flag to terminate all trades at their next break date before aggregation and the subsequent analytics
\item {\tt cva, dva, fva, colva, collateralFloor:} Flags to enable/disable these analytics. \todo[inline]{Add collateral rates floor to the collateral model file (netting.xml)}
\item {\tt dvaName:} Credit name to look up the own default probability curve and recovery rate for DVA calculation
\item {\tt fvaBorrowingCurve:} Identifier of the borrowing yield curve 
\item {\tt fvaLendingCurve:} Identifier of the lending yield curve
\item {\tt collateralSpread:} Deviation between collateral rate and overnight rate, expressed in absolute terms (one basis point is 0.0001) assuming the day count convention of the collateral rate. \todo[inline]{Move collateralSpread to the collateral model file (netting.xml)}
\item {\tt rawCubeOutputFile:} File name for the trade NPV cube in human readable csv file format (per trade, date, sample)
\item {\tt netCubeOutputFile:} File name for the aggregated NPV cube in human readable csv file format (per netting set, date, sample) {\em after} taking collateral into account
\end{itemize}

The latter two cube file outputs are provided for interactive analysis and visualisation purposes, see section \ref{sec:visualisation}.

%--------------------------------------------------------
\subsection{Market: {\tt todaysmarket.xml}}\label{sec:market}
%--------------------------------------------------------

This configuration file determines the subset of the 'market' universe which is going to be built by ORE. It is the user's responsibility to make sure that this subset is sufficient to cover the portfolio to be analysed. If it is not, the application will complain at run time and exit.

\medskip
We assume that the market configuration is provided in file {\tt todaysmarket.xml}, however, the file name can be chosen by the user. The file name needs to be entered into the master configuration file {\tt ore.xml}, see section \ref{sec:master_input}.

\medskip
The file starts and ends with the opening and closing tags {\tt <TodaysMarket>} and {\tt </TodaysMarket>}. The file then contains configuration blocks for 
\begin{itemize}
\item Discounting curves
\item Index curves (to project index fixings)
\item Swap index curves (to project swap rates)
\item FX spot rates
\item FX Volatility structures
\item Swaption volatility structures
\item Default curves
\end{itemize}

There can be alternative versions of each block each labeled with a unique identifier (e.g. Discount curve block with ID 'default', discount curve block with ID 'ois', another one with ID 'xois', etc). The purpose of these IDs will be explained at the end of this section. We now discuss each block's layout.

\subsubsection{Discounting Curves} 

We pick one discounting curve block as an example here (see {\tt Examples/Input/todaysmarket.xml}), the one with ID 'ois' 

{\footnotesize
\begin{lstlisting}[caption={Discount curve block with ID 'ois'}, 	label=lst:discountcurve_spec]
	<DiscountingCurves Id="ois">
		<DiscountingCurve Currency="EUR"> Yield/EUR/EUR1D </DiscountingCurve>
		<DiscountingCurve Currency="USD"> Yield/USD/USD1D </DiscountingCurve>
		<DiscountingCurve Currency="GBP"> Yield/GBP/GBP1D </DiscountingCurve>
		<DiscountingCurve Currency="CHF"> Yield/CHF/CHF6M </DiscountingCurve>
		<DiscountingCurve Currency="JPY"> Yield/JPY/JPY6M </DiscountingCurve>
		...
	</DiscountingCurves>
\end{lstlisting}
}

This block instructs ORE to build five discount curves for the indicated currencies. The string within the tags, e.g. Yield/EUR/EUR1D, uniquely identifies the curve to be built.  Curve Yield/EUR/EUR1D is defined in the curve configuration file explained in section \ref{sec:curveconfig} below. In this case ORE is instructed to build an Eonia Swap curve made of Overnight Deposit and Eonia Swap quotes. The right most token of the string Yield/EUR/EUR1D (EUR1D) is user defined, the first two tokens Yield/EUR have to be used to point to a yield curve in currency EUR.
 
\subsubsection{Index Curves} 

See an excerpt of the index curve block with ID 'default' from the same example file:

{\footnotesize
\begin{lstlisting}[caption={Index curve block with ID 'default'}, 	label=lst:indexcurve_spec]
	<IndexForwardingCurves Id="default">
		..
		<Index Name="EUR-EURIBOR-3M"> Yield/EUR/EUR3M </Index>
		<Index Name="EUR-EURIBOR-6M"> Yield/EUR/EUR6M </Index>
		<Index Name="EUR-EURIBOR-12M"> Yield/EUR/EUR6M </Index>
		<Index Name="EUR-EONIA"> Yield/EUR/EUR1D </Index>
		<Index Name="USD-LIBOR-3M"> Yield/USD/USD3M </Index>
		...
	</IndexForwardingCurves>
\end{lstlisting}
}

This block of curve specifications instructs ORE to build another set of yield curves, unique strings (e.g. Yield/EUR/EUR6M etc.) point to the {\tt curveconfig.xml} file where these curves are defined. Each curve is then associated with an index name (of format Ccy-IndexName-Tenor, e.g. EUR-EURIBOR-6M) so that ORE will project the respective index using the selected curve (e.g. Yield/EUR/EUR6M).

\subsubsection{Swap Index Curves}

The following is an excerpt of the swap index curve block with ID 'default' from the same example file:

{\footnotesize
\begin{lstlisting}[caption={Swap index curve block with ID 'default'}, 	label=lst:swapindexcurve_spec]
	<SwapIndexCurves Id="default">
		<SwapIndex Name="EUR-CMS-1Y">
			<Index>EUR-EURIBOR-6M</Index>
			<Discounting>EUR-EONIA</Discounting>
		</SwapIndex>
		<SwapIndex Name="EUR-CMS-30Y">
			<Index>EUR-EURIBOR-6M</Index>
			<Discounting>EUR-EONIA</Discounting>
		</SwapIndex>
		...
	</SwapIndexCurves>
\end{lstlisting}
}

These instructions do not build any additional curves. They only build the respective swap index objects and associate them with the required index forwarding and discounting curves already built above. This enables a swap index to project the fair rate of forward starting Swaps. Swap indices are also containers for conventions. Swaption volatility surfaces require two swap indices each available in the market object, a long term and a short term swap index. The curve configuration file below will show that in particular the required short term index has term 1Y, and the required long term index has 30Y term. This is why we build these two indices at this point.

\subsubsection{FX Spot}

The following is an excerpt of the FX spot block with ID 'default' from the same example file:

{\footnotesize
\begin{lstlisting}[caption={FX spot block with ID 'default'}, label=lst:fxspot_spec]
	<FxSpots Id="default">
		<FxSpot Pair="EURUSD">FX/EUR/USD</FxSpot>
		<FxSpot Pair="EURGBP">FX/EUR/GBP</FxSpot>
		<FxSpot Pair="EURCHF">FX/EUR/CHF</FxSpot>
		<FxSpot Pair="EURJPY">FX/EUR/JPY</FxSpot>
	</FxSpots>
\end{lstlisting}
}

This block instructs ORE to provide four FX quote objects in the market object, all quoted with target currency EUR so that foreign currency amounts can be converted into EUR via multiplication with that rate.
 
\subsubsection{FX Volatilities}

The following is an excerpt of the FX Volatilities block with ID 'default' from the same example file:

{\footnotesize
\begin{lstlisting}[caption={FX volatility block with ID 'default'}, label=lst:fxvol_spec]	
	<FxVolatilities Id="default">
		<FxVolatility Pair="EURUSD"> FXVolatility/EUR/USD/EURUSD </FxVolatility>
		<FxVolatility Pair="EURGBP"> FXVolatility/EUR/GBP/EURGBP </FxVolatility>
		<FxVolatility Pair="EURCHF"> FXVolatility/EUR/CHF/EURCHF </FxVolatility>
		<FxVolatility Pair="EURJPY"> FXVolatility/EUR/JPY/EURJPY </FxVolatility>
	</FxVolatilities>
\end{lstlisting}
}

This instructs ORE to build four FX volatility structures for all FX pairs with target currency EUR, see curve configuration file for the definition of the volatility structure.

\subsubsection{Swaption Volatilities}

The following is an excerpt of the Swaption Volatilities block with ID 'default' from the same example file:

{\footnotesize
\begin{lstlisting}[caption={Swaption volatility block with ID 'default'}, label=lst:swaptionvol_spec]	
	<SwaptionVolatilities Id="default">
		<SwaptionVolatility Currency="EUR"> SwaptionVolatility/EUR/EUR_SW_N </SwaptionVolatility>
		<SwaptionVolatility Currency="USD"> SwaptionVolatility/USD/USD_SW_N </SwaptionVolatility>
		<SwaptionVolatility Currency="GBP"> SwaptionVolatility/GBP/GBP_SW_N </SwaptionVolatility>
		<SwaptionVolatility Currency="CHF"> SwaptionVolatility/CHF/CHF_SW_N </SwaptionVolatility>
		<SwaptionVolatility Currency="JPY"> SwaptionVolatility/CHF/JPY_SW_N </SwaptionVolatility>
	</SwaptionVolatilities>
\end{lstlisting}
}

This instructs ORE to build five Swaption volatility structures, see the curve configuration file for the definition of the volatility structure. The latter token (e.g. EUR\_SW\_N) is user defined and will be found in the curve configuration's CurveId tag.

\subsubsection{Default Curves}

The following is an excerpt of the Default Curves block with ID 'default' from the same example file:

{\footnotesize
\begin{lstlisting}[caption={Default curves block with ID 'default'}, label=lst:defaultcurve_spec]	
	<DefaultCurves Id="default">
		<DefaultCurve Name="BANK"> Default/USD/BANK_SR_USD </DefaultCurve>
		<DefaultCurve Name="CUST_A"> Default/USD/CUST_A_SR_USD </DefaultCurve>
		...
	</DefaultCurves>
\end{lstlisting}
}

This instructs ORE to build a set of default probability curves, again defined in the curve configuration file. Each curve is then associated with a name (BANK, CUST\_A) for subsequent lookup.
As before, the last token (e.g. BANK\_SR\_USD) is user defined and will be found in the curve configuration's CurveId tag.  

\subsubsection{Market Configurations}

Finally, representatives of each type of block (Discount Curves, Index Curves etc) can be bundled into a market configuration. This is done by adding the following to the {\tt todaysmarket.xml} file:

{\footnotesize
\begin{lstlisting}[caption={Configuration block with ID 'default'}, label=lst:config_spec]	
	<Configuration Id="default">
		<DiscountingCurvesId> xois_eur </DiscountingCurvesId>
		<IndexForwardingCurvesId> default </IndexForwardingCurvesId>
		<SwapIndexCurvesId> default </SwapIndexCurvesId>
		<FxSpotsId> default </FxSpotsId>
		<FxVolatilitiesId> default </FxVolatilitiesId>
		<SwaptionVolatilitiesId> default </SwaptionVolatilitiesId>
		<DefaultCurvesId> default </DefaultCurvesId>
	</Configuration>

	<Configuration Id="collateral_inccy">
		<DiscountingCurvesId>ois</DiscountingCurvesId>
		...
	</Configuration>

	<Configuration Id="collateral_eur">
		<DiscountingCurvesId>xois_eur</DiscountingCurvesId>
		...
	</Configuration>

	<Configuration Id="libor">
		<DiscountingCurvesId>inccy_swap</DiscountingCurvesId>
		...
	</Configuration>
\end{lstlisting}
}

When ORE constructs the market object, all configurations will be provided side by side it passes the desired configuration ID (e.g. 'default') which causes the library to build the series of curves and volatility structures that are 'bundled' into that configuration ID.  This allows configuring a market setup for different alternative purposes side by side in the same {\tt todaysmarket.xml} file. Typical use cases are
\begin{itemize}
\item different discount curves needed for model calibration and risk factor evolution, respectively
\item different discount curves needed for collateralised and uncollateralised derivatives pricing.
\end{itemize}
The former is actually used throughout the {\tt Examples}�section. Each master input file has a Markets section (see \ref{sec:master_input}) where two market configuration IDs have to be provided, the one used for model calibration, and the one used for risk factor evolution.

\medskip
The configuration ID concept extends across all curve and volatility objects though currently used only to distinguish discounting. 
 
%--------------------------------------------------------
\subsection{Pricing Engines: {\tt pricingengine.xml}}
%--------------------------------------------------------

The pricing engine configuration file is provided to select pricing models and pricing engines by  
product type. The following excerpt of the Example section's {\tt pricingengine.xml} shows the selection for Bermudan Swaption pricing:

{\footnotesize
\begin{lstlisting}[caption={Pricing engine configuration}, label=lst:pricingengine_config]	
<PricingEngines>
	<Product type="BermudanSwaption">
		<Model>LGM</Model>
		<ModelParameters>
			<Parameter name="Calibration">Bootstrap</Parameter>
			<Parameter name="BermudanStrategy">CoterminalATM</Parameter>
			<Parameter name="Reversion">0.03</Parameter>
			<Parameter name="ReversionType">HullWhite</Parameter>
			<Parameter name="Volatility">0.01</Parameter>
			<Parameter name="VolatilityType">Hagan</Parameter>
			<Parameter name="Tolerance">0.0001</Parameter>
		</ModelParameters>
		<Engine>Grid</Engine>
		<EngineParameters>
			<Parameter name="sy">3.0</Parameter>
			<Parameter name="ny">10</Parameter>
			<Parameter name="sx">3.0</Parameter>
			<Parameter name="nx">10</Parameter>
		</EngineParameters>
	</Product>
</PricingEngines>
\end{lstlisting}
}

These settings will be taken into account when the engine factory is asked to build a Bermudan Swaption
pricing model, calibrate it and construct the pricing engine for it:

\begin{itemize}
\item The only model currently supported for Bermudan Swaption pricing is the LGM selected here. 

\item The first block of model parameters then provides initial values for the model (Reversion, Volatility) and chooses the parametrisation of the LGM model with ReversionType and VolatilityType choices {\em HullWhite} and {\em Hagan}. Calibration and BermudanStrategy can be set to {\em None} in order to skip model calibration. Alternatively, Calibration is set to {\em Bootstrap} and BermudanStrategy to {\em CoterminalATM} in order to calibrate to instrument-specific co-terminal ATM Swaptions, i.e. chosen to match the instruments first expiry and final maturity.

\item The second block of engine parameters specifies the Numerical Swaption engine parameters which determine the number of standard deviations covered in the probability density integrals (sy and sx), and the number of grid points used per standard deviation (ny and nx).
\end{itemize}

This file is relevant in particular for structured products which are in scope of future ORE releases. But it is also intended to allow the selection of optimised pricing engines for vanilla products such as Interest Rate Swaps.
 
%--------------------------------------------------------
\subsection{Simulation: {\tt simulation.xml}}
%--------------------------------------------------------

This file determines the behaviour of the risk factor simulation (scenario generation) module.
It is structured in three blocks of data.

{\footnotesize
\begin{lstlisting}[caption={Simulation configuration}, label=lst:simulation_configuration]
<Root>
	<Simulation>
		<Parameters> ... </Parameters>
		<CrossAssetModel> ... </CrossAssetModel>
		<Market> ... </Market>
	</Simulation>
</Root>
\end{lstlisting}
}

Each of the three blocks is sketched in the following.

\subsubsection{Parameters}\label{sec:sim_params}

Let us discuss this section using the following example

{\footnotesize
\begin{lstlisting}[caption={Simulation configuration}, label=lst:simulation_params_configuration]
	<Parameters>
		<Discretization>Exact</Discretization>
		<Grid>80,3M</Grid>
		<Calendar>EUR,USD,GBP,CHF</Calendar>
		<Sequence>SobolBrownianBridge</Sequence>
		<Scenario>Simple</Scenario>
		<Seed>42</Seed>
		<Samples>1000</Samples>
		<Fixings>
			<SimulateFixings>Y</SimulateFixings>
			<EstimationMethod>Backward</EstimationMethod>
			<ForwardHorizonDays>1</ForwardHorizonDays>
		</Fixings>
	</Parameters>
\end{lstlisting}
}

\begin{itemize}
\item {\tt Discretization:} Chooses between time discretization schemes for the risk factor evolution. {\em Exact} means exploiting the analytcal tractability of the model to avoid any time discretization error. {\em Euler} uses a naive time discretization scheme which has numerical error and requires small time steps fro accurate results (useful for testing purposes)
\item {\tt Grid: }�Specifies the simulation time grid, here 80 quarterly steps.
\item {\tt Calendar:} Calendar or combination of calendars used to adjust the dates of the grid. Date adjustment is required because the simulation must step over 'good' dates on which index fixings can be stored.
\item {\tt Scenario: } Choose between {\em Simple }�and {\em Complex } implementations, the latter optimized for more efficient memory usage. \todo[inline]{Remove Scenario choice}
\item {\tt Sequence:} Choose random sequence generator ({\em PseudoRandom, PseudoRandomAntithetic, Sobol, SobolBrownianBridge}).
\item {\tt Seed:} Random number generator seed
\item {\tt Samples:} Number of Monte Carlo paths to be produced
\item {\tt Fixings: } Choose whether fixings should be simulated or not, and if so which fixing simulation method to use ({\em Backward, Forward, BestOfForwardBackward, InterpolatedForwardBackward}), which number of forward horizon days to use if one of the {\em Forward } related methods is chosen.
\end{itemize}

\subsubsection{Model}\label{sec:sim_model}

The {\tt CrossAssetModel} section determines the cross asset model's number of currencies covered, composition, and each component's calibration. It is currently made of a sequence of LGM models for each currency (say $n$ currencies), $n-1$ FX models for each exchange rate to the base currency, and finally, a specification of the correlation structure between all components.

\medskip
The simulated currencies are specified as follows, with clearly identifying the domestic currency which is also the target currency for all FX models listed subsequently. If the portfolio requires more currencies to be simulated, this will lead to an exception at run time, so that it is the user's responsibility to make sure that the list of currencies here is sufficient. The list can be larger than actually required by the portfolio. This will not lead to any exceptions, but add to the run time of ORE.

{\footnotesize
\begin{lstlisting}[caption={Simulation model currencies configuration}, label=lst:simulation_model_currencies_configuration]
	<CrossAssetModel>
		<DomesticCcy>EUR</DomesticCcy>
		<Currencies>
			<Currency>EUR</Currency>
			<Currency>USD</Currency>
			<Currency>GBP</Currency>
			<Currency>CHF</Currency>
			<Currency>JPY</Currency>
		</Currencies>

		<BootstrapTolerance>0.0001</BootstrapTolerance>
      
		...
	</CrossAssetModel>
\end{lstlisting}
} 

Bootstrap tolerance is a global paramater that applies to the calibration of all model components. If the calibration error of any component exceeds this tolerance, this will trigger an exception at runtime, early in the ORE process.	

\medskip

Each interest rate model is specified by a block as follows

{\footnotesize
\begin{lstlisting}[caption={Simulation model IR configuration}, label=lst:simulation_model_ir_configuration]

	<CrossAssetModel>	
		...
		<InterestRateModels>
			<LGM ccy="default">
				<CalibrationType>Bootstrap</CalibrationType>
				<Volatility>
					<Calibrate>Y</Calibrate>
					<VolatilityType>Hagan</VolatilityType>
					<ParamType>Piecewise</ParamType>
					<TimeGrid>1.0,2.0,3.0,4.0,5.0,7.0,10.0</TimeGrid>
					<InitialValue>0.01,0.01,0.01,0.01,0.01,0.01,0.01,0.01<InitialValue>
				</Volatility>
				<Reversion>
					<Calibrate>N</Calibrate>
					<ReversionType>HullWhite</ReversionType>
					<ParamType>Constant</ParamType>
					<TimeGrid/>
					<InitialValue>0.03</InitialValue>
				</Reversion>
	  			<CalibrationSwaptions>
					<Expiries>1Y,2Y,4Y,6Y,8Y,10Y,12Y,14Y,16Y,18Y,19Y</Expiries>
					<Terms>19Y,18Y,16Y,14Y,12Y,10Y,8Y,6Y,4Y,2Y,1Y</Terms>
					<Strikes/>
				</CalibrationSwaptions>
				<ParameterTransformation>
					<ShiftHorizon>0.0</ShiftHorizon>
					<Scaling>1.0</Scaling>
				</ParameterTransformation>
			</LGM>
			<LGM ccy="EUR">
				...
			</LGM>
			<LGM ccy="USD">
				...
			</LGM>
		</InterestRateModels>	
		...		
	</CrossAssetModel>
\end{lstlisting}
} 

We have LGM sections by currency, but starting with a section for currency 'default'. As the name implies, this is used as default configuration for any currency in the currency list for which we do not provide an explicit parametrisation. Within each LGM section, the interpretation of elements is as follows:

\begin{itemize}
\item {\tt CalibrationType: } Choose between {\em Bootstrap} and {\em BestFit}, where Bootstrap is chosen when we expect to be able to achieve a perfect fit (as with calibration of piecewise volatility to a series of co-terminal Swaptions)
\item {\tt Volatility/Calibrate: } Flag to enable/disable calibration of this particular parameter
\item {\tt Volatility/VolatilityType: } Choose volatility parametrisation ala {\em HullWhite} or {\em Hagan} 
\item {\tt Volatility/ParamType: } Choose between {\em Constant} and {\em Piecewise}
\item {\tt Volatility/TimeGrid: } Initial time grid for this parameter, can be left empty if ParamType is Constant
\item {\tt Volatility/InitialValue: } Vector of initial values, matching number of entries in time, or single value if the time grid is empty
\item {\tt Reversion/Calibrate: } Flag to enable/disable calibration of this particular parameter
\item {\tt Reversion/VolatilityType: } Choose reversion parametrisation ala {\em HullWhite} or {\em Hagan} 
\item {\tt Reversion/ParamType: } Choose between {\em Constant} and {\em Piecewise}
\item {\tt Reversion/TimeGrid: } Initial time grid for this parameter, can be left empty if ParamType is Constant
\item {\tt Reversion/InitialValue: } Vector of initial values, matching number of entries in time, or single value if the time grid is empty
\item {\tt CalibrationSwaptions: } Choice of calibration instruments by expiry, underlying Swap term and strike
\item {\tt ParameterTransformation: } LGM model prices are invariant under scaling and shift transformations \cite{Lichters} with advantages for numerical convergence of results in long term simulations. These transformations can be chosen here. Default settings are shiftHorizon 0 and scaling 1.
\end{itemize}

\medskip

Each interest rate model is specified by a block as follows

{\footnotesize
\begin{lstlisting}[caption={Simulation model FX configuration}, label=lst:simulation_model_fx_configuration]
	<CrossAssetModel>	
		...
		<ForeignExchangeModels>
			<CrossCcyLGM foreignCcy="default">
				<DomesticCcy>EUR</DomesticCcy>
				<CalibrationType>Bootstrap</CalibrationType>
				<Sigma>
					<Calibrate>Y</Calibrate>
					<ParamType>Piecewise</ParamType>
					<TimeGrid>1.0,2.0,3.0,4.0,5.0,7.0,10.0</TimeGrid>
					<InitialValue>0.1,0.1,0.1,0.1,0.1,0.1,0.1,0.1</InitialValue>
				</Sigma>
				<CalibrationOptions>
					<Expiries>1Y,2Y,3Y,4Y,5Y,10Y</Expiries>
					<Strikes/>
				</CalibrationOptions>
			</CrossCcyLGM>
			<CrossCcyLGM foreignCcy="USD">
	  			...
			</CrossCcyLGM>
			<CrossCcyLGM foreignCcy="GBP">
				...
			</CrossCcyLGM>
			...
		</ForeignExchangeModels>
		...
	<CrossAssetModel>	
\end{lstlisting}
} 

CrossCcyLGM sections are defined by foreign currency, but we also support a default configuration as above for the IR model parametrisations.
Within each CrossCcyLGM section, the interpretation of elements is as follows:

\begin{itemize}
\item {\tt DomesticCcy: } Domestic currency completing the FX pair
\item {\tt CalibrationType: } Choose between {\em Bootstrap} and {\em BestFit} as in the IR section
\item {\tt Sigma/Calibrate: } Flag to enable/disable calibration of this particular parameter
\item {\tt Sigma/ParamType: } Choose between {\em Constant} and {\em Piecewise}
\item {\tt Sigma/TimeGrid: } Initial time grid for this parameter, can be left empty if ParamType is Constant
\item {\tt Sigma/InitialValue: } Vector of initial values, matching number of entries in time, or single value if the time grid is empty
\item {\tt CalibrationOptions: } Choice of calibration instruments by expiry and strike, strikes can be empty (implying the default, ATMF options), or explicitly specified (in terms of FX rates as absolute strike values, in delta notation such as $\pm 25D$, $ATMF$ for at the money)
\end{itemize}

\medskip
Finally, the instantaneous correlation structure is specified as follows.

{\footnotesize
\begin{lstlisting}[caption={Simulation model FX configuration}, label=lst:simulation_model_fx_configuration]
	<CrossAssetModel>	
		...
		<InstantaneousCorrelations>
			<Correlation factor1="IR:EUR" factor2="IR:USD">0.3</Correlation>
			<Correlation factor1="IR:EUR" factor2="IR:GBP">0.3</Correlation>
			<Correlation factor1="IR:USD" factor2="IR:GBP">0.3</Correlation>
			<Correlation factor1="IR:EUR" factor2="FX:USDEUR">0</Correlation>
			<Correlation factor1="IR:EUR" factor2="FX:GBPEUR">0</Correlation>
			<Correlation factor1="IR:GBP" factor2="FX:USDEUR">0</Correlation>
			<Correlation factor1="IR:GBP" factor2="FX:GBPEUR">0</Correlation>
			<Correlation factor1="IR:USD" factor2="FX:USDEUR">0</Correlation>
			<Correlation factor1="IR:USD" factor2="FX:GBPEUR">0</Correlation>
			<Correlation factor1="FX:USDEUR" factor2="FX:GBPEUR">0</Correlation>
			<!-- ... -->	
		</InstantaneousCorrelations>
	</CrossAssetModel>
\end{lstlisting}
} 

Any risk factor pair not specified explicitly here will be assumed to have zero correlation.

\subsubsection{Market}\label{sec:sim_market}

The last part of the simulation configuration file covers the specification of the simulated market.
Note that the simulation model will yield the evolution of risk factors such as short rates which need to be translated into entire yield curves which can be 'understood' by the instruments which we want to price under scenarios.
Moreover we need to specify how volatility structures evolve even if we do not explicitly simulate volatility. This translation is the role of the {\em simulation market} object, which is configured in the section within the enclosing tags {\tt <Market>} and {\tt </Market}, as follows.

{\footnotesize
\begin{lstlisting}[caption={Simulation model FX configuration}, label=lst:simulation_model_fx_configuration]
	<Market>
		<BaseCurrency>EUR</BaseCurrency>
			<Currencies>
				<Currency>EUR</Currency>
				<Currency>USD</Currency>
				...
			</Currencies>
      
		<YieldCurves>
			<Configuration>
				<Tenors>3M,6M,1Y,2Y,3Y,4Y,5Y,7Y,10Y,12Y,15Y,20Y</Tenors>
				<Interpolation>LogLinear</Interpolation>
				<Extrapolation>Y</Extrapolation>
			</Configuration>
		</YieldCurves>
      
      	<Indices>
			<Index>EUR-EURIBOR-6M</Index>
			<Index>EUR-EURIBOR-3M</Index>
			<Index>EUR-EONIA</Index>
			<Index>USD-LIBOR-3M</Index>
			...
		</Indices>
      
		<SwapIndices>
			<SwapIndex>
				<Name>EUR-CMS-1Y</Name>
				<ForwardingIndex>EUR-EURIBOR-6M</ForwardingIndex>
				<DiscountingIndex>EUR-EONIA</DiscountingIndex>
			</SwapIndex>
			<SwapIndex>...</SwapIndex>
			...
		</SwapIndices>
		
		<SwaptionVolatilities>
			<ReactionToTimeDecay>ForwardVariance</ReactionToTimeDecay>
			<Currencies>
	  			<Currency>EUR</Currency>
	  			<Currency>USD</Currency>
	  			...
			</Currencies>
			<Expiries>6M,1Y,2Y,3Y,5Y,10Y,12Y,15Y,20Y</Expiries>
			<Terms>1Y,2Y,3Y,4Y,5Y,7Y,10Y,15Y,20Y,30Y</Terms>
			<Strikes/>
		</SwaptionVolatilities>
      
		<FxVolatilities>
			<ReactionToTimeDecay>ForwardVariance</ReactionToTimeDecay>
			<CurrencyPairs>
				<CurrencyPair>EURUSD</CurrencyPair>
				<CurrencyPair>EURGBP</CurrencyPair>
				...
			</CurrencyPairs>
			<Expiries>6M,1Y,2Y,3Y,4Y,5Y,7Y,10Y</Expiries>
			<Strikes/>
		</FxVolatilities>
      
		<AdditionalScenarioDataCurrencies>
			<Currency>EUR</Currency>
			<Currency>USD</Currency>
			...
		</AdditionalScenarioDataCurrencies>
      
		<AdditionalScenarioDataIndices>
			<Index>EUR-EURIBOR-3M</Index>
			<Index>EUR-EONIA</Index>
			<Index>USD-LIBOR-3M</Index>
			...
		</AdditionalScenarioDataIndices>
      
    </Market>
    
  </Simulation>

\end{lstlisting}
} 

\todo[inline]{Remove DefaultCurves from SimMarket constructor and configuration section}

%--------------------------------------------------------
\subsection{Curves: {\tt curveconfig.xml}}\label{sec:curveconfig}
%--------------------------------------------------------

The configuration of various term structures required to price a portfolio is covered in a single configuration file which we will label {\tt curveconfig.xml} in the following though the file name can be chosen by the user. This configuration determines the composition of yield curves, default curves, Swaption and FX Option volatility structures.
 
\subsubsection{Yield Curves}

\subsubsection{Yield Curves}

The top level XML elements for each \lstinline!YieldCurve! node are shown in Listing \ref{lst:top_level_yc}.

\begin{listing}[H]
%\hrule\medskip
\begin{minted}[fontsize=\footnotesize]{xml}
<YieldCurve>
  <CurveId> </CurveId>
  <CurveDescription> </CurveDescription>
  <Currency> </Currency>
  <DiscountCurve> </DiscountCurve>
  <Segments> </Segments>
  <InterpolationVariable> </InterpolationVariable>
  <InterpolationMethod> </InterpolationMethod>
  <MixedInterpolationCutoff> </MixedInterpolationCutoff>
  <YieldCurveDayCounter> </YieldCurveDayCounter>
  <Tolerance> </Tolerance>
  <Extrapolation> </Extrapolation>
  <BootstrapConfig>
    ...
  </BootstrapConfig>
</YieldCurve>
\end{minted}
\caption{Top level yield curve node}
\label{lst:top_level_yc}
\end{listing}

The meaning of each of the top level elements in Listing \ref{lst:top_level_yc} is given below. If an element is labelled 
as 'Optional', then it may be excluded or included and left blank.
\begin{itemize}
\item CurveId: Unique identifier for the yield curve.
\item CurveDescription: A description of the yield curve. This field may be left blank.
\item Currency: The yield curve currency.
\item DiscountCurve: If the yield curve is being bootstrapped from market instruments, this gives the CurveId of the
yield curve used to discount cash flows during the bootstrap procedure. If this field is left blank or set equal to the
current CurveId, then this yield curve itself is used to discount cash flows during the bootstrap procedure.
\item Segments: This element contains child elements and is described in the following subsection.
\item InterpolationVariable [Optional]: The variable on which the interpolation is performed. The allowable values are
given in Table \ref{tab:allow_interp_variables}. If the element is omitted or left blank, then it defaults to
\emph{Discount}.
\item InterpolationMethod [Optional]: The interpolation method to use. The allowable values are given in Table
\ref{tab:allow_interp_methods}. If the element is omitted or left blank, then it defaults to \emph{LogLinear}.
\item MixedInterpolationCutoff [Optional]: If a mixed interpolation method is used, the number of segments to which the
  first interpolation method is applied. Defaults to 1.
\item YieldCurveDayCounter [Optional]: The day count basis used internally by the yield curve to calculate the time between
dates. In particular, if the curve is queried for a zero rate without specifying the day count basis, the zero rate that
is returned has this basis. If the element is omitted or left blank, then it defaults to \emph{A365}.

\item \lstinline!Tolerance! [Optional]: The tolerance used by the root finding procedure in the bootstrapping algorithm. If the
element is omitted or left blank, then it defaults to \num[scientific-notation=true]{1.0e-12}. It is preferable to use the 
\lstinline!Accuracy! node in the \lstinline!BootstrapConfig! node below for specifying this value. However, if this node is 
explicitly supplied, it takes precedence for backwards compatibility purposes.

\item Extrapolation [Optional]: Set to \emph{True} or \emph{False} to enable or disable extrapolation respectively. If
the element is omitted or left blank, then it defaults to \emph{True}.

\item \lstinline!BootstrapConfig! [Optional]: this node holds configuration details for the iterative bootstrap 
that are described in section \ref{sec:bootstrap_config}. If omitted, this node's default values described 
in section \ref{sec:bootstrap_config} are used.

\end{itemize}

\begin{table}[h]
\centering
  \begin{tabular}{|l|l|} 
    \hline
    {\bfseries Variable} & {\bfseries Description} \\
    \hline
    Zero & The continuously compounded zero rate \\ \hline
    Discount & The discount factor \\ \hline
    Forward & The instantaneous forward rate \\ \hline
  \end{tabular}
  \caption{Allowable interpolation variables.}
  \label{tab:allow_interp_variables}
\end{table}

\begin{table}[h]
\centering
\begin{tabular}{|l|p{11cm}|} 
    \hline
    {\bfseries Method} & {\bfseries Description} \\
    \hline
    Linear & Linear interpolation \\ \hline
    LogLinear & Linear interpolation on the natural log of the interpolation variable \\ \hline
    NaturalCubic & Monotonic Kruger cubic interpolation with zero second derivative at left and right \\ \hline
    FinancialCubic & Monotonic Kruger cubic interpolation with zero second derivative at left and 
                     zero first derivative at right \\ \hline
    ConvexMonotone & Convex Monotone Interpolation (Hagan, West) \\ \hline
    Quadratic & Quadratic interpolation \\ \hline
    LogQuadratic & Quadratic interpolation on the natural log of the interpolation variable \\ \hline
    LogNaturalCubic & Monotonic Kruger cubic interpolation with zero second derivative at left and right \\hline
    LogFinancialCubic & Monotonic Kruger cubic interpolation with zero second derivative at left and 
                     zero first derivative at right \\hline
    LogCubicSpline & Non-monotonic cubic spline interpolation with zero second derivative at left and right \\hline
    Hermite & Hermite cubic spline interpolation \\ \hline
    CubicSpline & Non-monotonic cubic spline interpolation with zero second derivative at left and right \\ \hline
    DefaultLogMixedLinearCubic & Mixed interpolation, first linear, then monotonic Kruger cubic spline \\ \hline
    MonotonicLogMixedLinearCubic & Mixed interpolation, first linear, then monotonic natural cubic spline \\ \hline
    KrugerLogMixedLinearCubic & Mixed interpolation, first linear, then non-monotonic Kruger cubic spline \\ \hline
    LogMixedLinearCubicNaturalSpline & Mixed interpolation, first linear, then non-monotonic natural cubic spline \\ \hline
    ExponentialSplines & Exponential Spline curve fitting, for Fitted Bond Curves only \\ \hline
    NelsonSiegel & Nelson-Siegel curve fitting, for Fitted Bond Curves only \\ \hline
    Svensson & Svensson curve fitting, for Fitted Bond Curves only \\ \hline
  \end{tabular}
  \caption{Allowable interpolation methods.}
  \label{tab:allow_interp_methods}
\end{table}
%- - - - - - - - - - - - - - - - - - - - - - - - - - - - - - - - - - - - - - - -
\subsubsection*{Segments Node} \label{ss:segments_node}
%- - - - - - - - - - - - - - - - - - - - - - - - - - - - - - - - - - - - - - - -
The \lstinline!Segments! node gives the zero rates, discount factors and instruments that comprise the yield curve. This
node consists of a number of child nodes where the node name depends on the segment being described. Each node has a
\lstinline!Type! that determines its structure. The following sections describe the type of child nodes that are
available. Note that for all segment types below, with the exception of \lstinline!DiscountRatio! and \lstinline!AverageOIS!, the 
\lstinline!Quote! elements within the \lstinline!Quotes! node may have an \lstinline!optional! attribute indicating whether or
not the quote is optional. Example:
%\hrule\medskip
\begin{minted}[fontsize=\footnotesize]{xml}
<Quotes>
  <Quote optional="true"></Quote>
</Quotes>
\end{minted}
%\hrule

\subsubsection*{Direct Segment}
When the node name is \lstinline!Direct!, the \lstinline!Type! node has the value \emph{Zero} or \emph{Discount} and the
node has the structure shown in Listing \ref{lst:direct_segment}. We refer to this segment here as a direct segment
because the discount factors, or equivalently the zero rates, are given explicitly and do not need to be
bootstrapped. The \lstinline!Quotes! node contains a list of \lstinline!Quote! elements. Each \lstinline!Quote! element
contains an ID pointing to a line in the {\tt market.txt} file, i.e.\ in this case, pointing to a particular zero rate
or discount factor. The \lstinline!Conventions! node contains the ID of a node in the {\tt conventions.xml} file
described in section \ref{sec:conventions}. The \lstinline!Conventions! node associates conventions with the quotes.

For \emph{Discount} type segments, the quotes can be given using a wildcard. Any valid and matching quotes will then be loaded from the provided market data. An example wildcard is:
\begin{itemize}
  \item {DISCOUNT/RATE/EUR/EUR3M/*}
\end{itemize}

\begin{listing}[H]
%\hrule\medskip
\begin{minted}[fontsize=\footnotesize]{xml}
<Direct>
  <Type> </Type>
  <Quotes>
    <Quote> </Quote>
    <Quote> </Quote>
     <!--...-->
  </Quotes>
  <Conventions> </Conventions>
</Direct>
\end{minted}
\caption{Direct yield curve segment}
\label{lst:direct_segment}
\end{listing}


\subsubsection*{Simple Segment}
When the node name is \lstinline!Simple!, the \lstinline!Type! node has the value \emph{Deposit}, \emph{FRA},
\emph{Future}, \emph{OIS}, \emph{Swap} or \emph{BMA Basis Swap} and the node has the structure shown in Listing
\ref{lst:simple_segment}. This segment holds quotes for a set of deposit, FRA, Future, OIS or swap instruments
corresponding to the value in the \lstinline!Type! node. These quotes will be used by the bootstrap algorithm to imply a
discount factor, or equivalently a zero rate, curve. The only difference between this segment and the direct segment is
that there is a \lstinline!ProjectionCurve! node. This node allows us to specify the CurveId of another curve to project
floating rates on the instruments underlying the quotes listed in the \lstinline!Quote! nodes during the bootstrap
procedure. This is an optional node. If it is left blank or omitted, then the projection curve is assumed to equal the
curve being bootstrapped i.e.\ the current CurveId. The \lstinline!PillarChoice! node determines the bootstrap pillars
that are used (MaturityDate, LastRelevantDate, if not given 'LastRelevantDate' is the default value).

The \lstinline!Priority! node determines the priority of the segment, this has to be a non-negative integer. A lower
number means a higher priority (more ``important'') segment. If two adjacent segments overlap w.r.t. the pillar dates of
their instruments, instruments from the segment with lower priority are removed until the overlap is resolved. In
addition, a minimum distance (measured in calendar days) between the segments is preserved. This distance is given in
the \lstinline!MinDistance! node for the instruments of the current and following segment. If not given, the priority of
a segment defaults to 0 (highest possible priority), the minimum distance defaults to $1$. Consider the example given in
\ref{lst:priorities_min_distances}. In this case:
\begin{itemize}
\item instruments from the start of the second segment with pillar date strictly earlier than $d_1 + 5$, where $d_1$ is
  the maximum pillar date of instruments in the first segment, will be removed
\item instruments from the end of the second segment with pillar date strictly later than $d_3 - 10$, where $d_3$ is the
  mininum pillar date of instruments in the third segment, will be removed
\end{itemize}


\begin{listing}[H]
%\hrule\medskip
\begin{minted}[fontsize=\footnotesize]{xml}
<Simple>
  <Type> </Type>
  <Quotes>
    <Quote> </Quote>
    <Quote> </Quote>
    <!--...-->
  </Quotes>
  <Conventions> </Conventions>
  <PillarChoice> </PillarChoice>
  <Priority> </Priority>
  <MinDistance> </MinDistance>
  <ProjectionCurve> </ProjectionCurve>
</Simple>
\end{minted}
\caption{Simple yield curve segment}
\label{lst:simple_segment}
\end{listing}

\begin{listing}[H]
%\hrule\medskip
\begin{minted}[fontsize=\footnotesize]{xml}
<Simple>
  ...
  <Priority>0</Priority>
  <MinDistance>5</MinDistance>
</Simple>
<Simple>
  ...
  <Priority>2</Priority>
  <MinDistance>10</MinDistance>
</Simple>
<Simple>
  ...
  <Priority>1</Priority>
</Simple>
\end{minted}
\caption{Example for priorities and min distances}
\label{lst:priorities_min_distances}
\end{listing}

\subsubsection*{Average OIS Segment}
When the node name is \lstinline!AverageOIS!, the \lstinline!Type! node has the value \emph{Average OIS} and the node
has the structure shown in Listing \ref{lst:average_ois_segment}. This segment is used to hold quotes for Average OIS
swap instruments. The \lstinline!Quotes! node has the structure shown in Listing \ref{lst:average_ois_quotes}. Each
quote for an Average OIS instrument (a typical example in a USD Overnight Index Swap) consists of two quotes, a vanilla
IRS quote and an OIS-LIBOR basis swap spread quote.  The IDs of these two quotes are stored in the
\lstinline!CompositeQuote! node. The \lstinline!RateQuote! node holds the ID of the vanilla IRS quote and the
\lstinline!SpreadQuote! node holds the ID of the OIS-LIBOR basis swap spread quote. The \lstinline!PillarChoice! node
determines the bootstrap pillars that are used (MaturityDate, LastRelevantDate, if not given 'LastRelevantDate' is the
default value).

For the \lstinline!Priority! and \lstinline!MinDistance! nodes see the explanation under ``Simple Segment''.

\begin{listing}[H]
%\hrule\medskip
\begin{minted}[fontsize=\footnotesize]{xml}
<AverageOIS>
  <Type> </Type>
  <Quotes>
    <CompositeQuote> </CompositeQuote>
    <CompositeQuote> </CompositeQuote>
    <!--...-->
  </Quotes>
  <Conventions> </Conventions>
  <PillarChoice> </PillarChoice>
  <Priority> </Priority>
  <MinDistance> </MinDistance>
  <ProjectionCurve> </ProjectionCurve>
</AverageOIS>
\end{minted}
\caption{Average OIS yield curve segment}
\label{lst:average_ois_segment}
\end{listing}

\begin{listing}[H]
%\hrule\medskip
\begin{minted}[fontsize=\footnotesize]{xml}
<Quotes>
  <CompositeQuote>
    <SpreadQuote> </SpreadQuote>
    <RateQuote> </RateQuote>
  </CompositeQuote>
  <!--...-->
</Quotes>
\end{minted}
\caption{Average OIS segment's quotes section}
\label{lst:average_ois_quotes}
\end{listing}

\subsubsection*{Tenor Basis Segment}
When the node name is \lstinline!TenorBasis!, the \lstinline!Type! node has the value \emph{Tenor Basis Swap} or
\emph{Tenor Basis Two Swaps} and the node has the structure shown in Listing \ref{lst:tenor_basis_segment}. This segment
is used to hold quotes for tenor basis swap instruments. The quotes may be for a conventional tenor basis swap where
Ibor of one tenor is swapped for Ibor of another tenor plus a spread. In this case, the \lstinline!Type! node has the
value \emph{Tenor Basis Swap}. The quotes may also be for the difference in fixed rates on two fair swaps where one swap
is against Ibor of one tenor and the other swap is against Ibor of another tenor. In this case, the \lstinline!Type!
node has the value \emph{Tenor Basis Two Swaps}. Again, the structure is similar to the simple segment in Listing
\ref{lst:simple_segment} except that there are two projection curve nodes. There is a \lstinline!ProjectionCurveReceive!
node for the index with the shorter tenor. This node holds the CurveId of a curve for projecting the floating rates on
the receiving side. Similarly, there is a \lstinline!ProjectionCurvePay! node for the index of the pay side. The deprecated 
values are short for receive, and long for pay. These are optional nodes. If they are left blank or omitted, then the projection 
curve is assumed to equal the curve being bootstrapped i.e.\ the current CurveId. However, at least one of the nodes 
needs to be populated to allow the bootstrap to proceed. The \lstinline!PillarChoice! node determines the bootstrap pillars
that are used (MaturityDate, LastRelevantDate, if not given 'LastRelevantDate' is the default value).

For the \lstinline!Priority! and \lstinline!MinDistance! nodes see the explanation under ``Simple Segment''.

\begin{listing}[H]
%\hrule\medskip
\begin{minted}[fontsize=\footnotesize]{xml}
<TenorBasis>
  <Type> </Type>
  <Quotes>
    <Quote> </Quote>
    <Quote> </Quote>
    <!--...-->
  </Quotes>
  <Conventions> </Conventions>
  <PillarChoice> </PillarChoice>
  <Priority> </Priority>
  <MinDistance> </MinDistance>
  <ProjectionCurvePay> </ProjectionCurvePay>
  <ProjectionCurveReceive> </ProjectionCurveReceive>
</TenorBasis>
\end{minted}
\caption{Tenor basis yield curve segment}
\label{lst:tenor_basis_segment}
\end{listing}

\subsubsection*{Cross Currency Segment}
When the node name is \lstinline!CrossCurrency!, the \lstinline!Type! node has the value \emph{FX Forward}, \emph{Cross
Currency Basis Swap} or \emph{Cross Currency Fix Float Swap}. When the \lstinline!Type! node has the value \emph{FX
Forward}, the node has the structure shown in Listing \ref{lst:fx_forward_segment}. This segment is used to hold quotes
for FX forward instruments. The \lstinline!DiscountCurve! node holds the CurveId of a curve used to discount cash flows
in the other currency i.e.\ the currency in the currency pair that is not equal to the currency in Listing
\ref{lst:top_level_yc}. The \lstinline!SpotRate! node holds the ID of a spot FX quote for the currency pair that is
looked up in the {\tt market.txt} file. The \lstinline!PillarChoice! node determines the bootstrap pillars that are used
(MaturityDate, LastRelevantDate, if not given 'LastRelevantDate' is the default value).

\begin{listing}[H]
%\hrule\medskip
\begin{minted}[fontsize=\footnotesize]{xml}
<CrossCurrency>
  <Type> </Type>
  <Quotes>
    <Quote> </Quote>
    <Quote> </Quote>
          ...
  </Quotes>
  <Conventions> </Conventions>
  <PillarChoice> </PillarChoice>
  <Priority> </Priority>
  <MinDistance> </MinDistance>
  <DiscountCurve> </DiscountCurve>
  <SpotRate> </SpotRate>
</CrossCurrency>
\end{minted}
\caption{FX forward yield curve segment}
\label{lst:fx_forward_segment}
\end{listing}

When the \lstinline!Type! node has the value \emph{Cross Currency Basis Swap} then the node has the structure shown in
Listing \ref{lst:xccy_basis_segment}. This segment is used to hold quotes for cross currency basis swap instruments. The
\lstinline!DiscountCurve! node holds the CurveId of a curve used to discount cash flows in the other currency i.e.\ the
currency in the currency pair that is not equal to the currency in Listing \ref{lst:top_level_yc}. The
\lstinline!SpotRate! node holds the ID of a spot FX quote for the currency pair that is looked up in the {\tt
  market.txt} file. The \lstinline!ProjectionCurveDomestic! node holds the CurveId of a curve for projecting the
floating rates on the index in this currency i.e.\ the currency in the currency pair that is equal to the currency in
Listing \ref{lst:top_level_yc}. It is an optional node and if it is left blank or omitted, then the projection curve is
assumed to equal the curve being bootstrapped i.e.\ the current CurveId. Similarly, the
\lstinline!ProjectionCurveForeign! node holds the CurveId of a curve for projecting the floating rates on the index in
the other currency. If it is left blank or omitted, then it is assumed to equal the CurveId provided in the
\lstinline!DiscountCurve! node in this segment.

For the \lstinline!Priority! and \lstinline!MinDistance! nodes see the explanation under ``Simple Segment''.

\begin{listing}[H]
%\hrule\medskip
\begin{minted}[fontsize=\footnotesize]{xml}
<CrossCurrency>
  <Type> </Type>
  <Quotes>
    <Quote> </Quote>
    <Quote> </Quote>
          ...
  </Quotes>
  <Conventions> </Conventions>
  <PillarChoice> </PillarChoice>
  <Priority> </Priority>
  <MinDistance> </MinDistance>
  <DiscountCurve> </DiscountCurve>
  <SpotRate> </SpotRate>
  <ProjectionCurveDomestic> </ProjectionCurveDomestic>
  <ProjectionCurveForeign> </ProjectionCurveForeign>
</CrossCurrency>
\end{minted}
\caption{Cross currency basis yield curve segment}
\label{lst:xccy_basis_segment}
\end{listing}

\subsubsection*{Zero Spread Segment}

When the node name is \lstinline!ZeroSpread!, the \lstinline!Type!
node has the only allowable value \emph{Zero Spread},  and the node has the structure shown in 
Listing \ref{lst:zero_spread_segment}. This segment is used to build yield
curves which are expressed as a spread over some reference yield curve.

\begin{listing}[H]
%\hrule\medskip
\begin{minted}[fontsize=\footnotesize]{xml}
    <ZeroSpread>
          <Type>Zero Spread</Type>
          <Quotes>
            <Quote>ZERO/YIELD_SPREAD/EUR/BANK_EUR_LEND/A365/2Y</Quote>
            <Quote>ZERO/YIELD_SPREAD/EUR/BANK_EUR_LEND/A365/5Y</Quote>
            <Quote>ZERO/YIELD_SPREAD/EUR/BANK_EUR_LEND/A365/10Y</Quote>
            <Quote>ZERO/YIELD_SPREAD/EUR/BANK_EUR_LEND/A365/20Y</Quote>
          </Quotes>
          <Conventions>EUR-ZERO-CONVENTIONS-TENOR-BASED</Conventions>
          <ReferenceCurve>EUR1D</ReferenceCurve>
    </ZeroSpread>
\end{minted}
\caption{Zero spread yield curve segment}
\label{lst:zero_spread_segment}
\end{listing}


\subsubsection*{Fitted Bond Segment}
\label{sec:fitted_bond_segment}

When the node name is \lstinline!FittedBond!, the \lstinline!Type! node has the only allowable value \emph{FittedBond},
and the node has the structure shown in Listing \ref{lst:fitted_bond_segment}. This segment is used to build yield
curves which are fitted to liquid bond prices. The segment has the following elements:

\begin{itemize}
\item Quotes: a list of bond price quotes, for each security in the list, reference data must be available
\item IborIndexCurves: for each Ibor index that is required by one of the bonds to which the curve is fitted, a mapping
  to an estimation curve for that index must be provided
\item ExtrapolateFlat: if true, the parametric curve is extrapolated flat in the instantaneous forward rate before the
  first and after the last maturity of the bonds in the calibration basket. This avoids unrealistic rates at the short
  end or for long maturities in the resulting curve.
\end{itemize}

The \lstinline!BootstrapConfig! has the following interpretation for a fitted bond curve:

\begin{itemize}
\item Accuracy [Optional, defaults to 1E-12]: the desired accuracy expressed as a weighted rmse in the implied quote,
  where 0.01 = 1 bp. Once this accuracy is reached in a calibration trial, the fit is accepted, no further calibration
  trials re run. In general, this parameter should be set to a higher than the default value for fitted bond curves.
\item GlobalAccuracy [Optional]: the acceptable accuracy. If the Accuracy is not reached in any calibration trial, but
  the GlobalAccuracy is met, the best fit among the calibration trials is selected as a result of the calibration. If
  not given, the best calibration trial is compared to the Accuracy parameter instead.
\item DontThrow [Optional, defaults to false]: If true, the best calibration is always accepted as a result, i.e. no
  error is thrown even if the GlobalAccuracy is breached.
\item MaxAttempts [Optional, defaults to 5]: The maximum number of calibration trials. Each calibration trial is run with a random calibration
  seed. Random calibration seeds are currently only supported for the NelsonSiegel interpolation method.
\end{itemize}

\begin{listing}[H]
%\hrule\medskip
\begin{minted}[fontsize=\footnotesize]{xml}
    <YieldCurve>
      ...
      <Segments>
        <FittedBond>
          <Type>FittedBond</Type>
          <Quotes>
            <Quote>BOND/PRICE/SECURITY_1</Quote>
            <Quote>BOND/PRICE/SECURITY_2</Quote>
            <Quote>BOND/PRICE/SECURITY_3</Quote>
            <Quote>BOND/PRICE/SECURITY_4</Quote>
            <Quote>BOND/PRICE/SECURITY_5</Quote>
          </Quotes>
          <!-- mapping of Ibor curves used in the bonds from which the curve is built -->
          <IborIndexCurves>
            <IborIndexCurve iborIndex="EUR-EURIBOR-6M">EUR6M</IborIndexCurve>
          </IborIndexCurves>
          <!-- flat extrapolation before first and after last bond maturity -->
          <ExtrapolateFlat>true</ExtrapolateFlat>
        </FittedBond>
      </Segments>
      <!-- NelsonSiegel, Svensson, ExponentialSplines -->
      <InterpolationMethod>NelsonSiegel</InterpolationMethod>
      <YieldCurveDayCounter>A365</YieldCurveDayCounter>
      <Extrapolation>true</Extrapolation>
      <BootstrapConfig>
        <!-- desired accuracy (in implied quote) -->
        <Accuracy>0.1</Accuracy>
        <!-- tolerable accuracy -->
        <GlobalAccuracy>0.5</GlobalAccuracy>
        <!-- do not throw even if tolerable accuracy is breached -->
        <DontThrow>false</DontThrow>
        <!-- max calibration trials to reach desired accuracy -->
        <MaxAttempts>20</MaxAttempts>
      </BootstrapConfig>
    </YieldCurve>
\end{minted}
\caption{Fitted bond yield curve segment}
\label{lst:fitted_bond_segment}
\end{listing}

\subsubsection*{Bond Yield Shifted}
\label{sec:bond_yield_shifted}

When the node name is \lstinline!BondYieldShifted!, the \lstinline!Type! node has the only allowable value \emph{Bond Yield Shifted},
and the node has the structure shown in Listing \ref{lst:bond_yield_shifted}. This segment is used to build yield
curves which are adjusted by liquid bond yields. The adjustment is derived as an average of the spreads between the bond's
yields-to-maturity and the reference curve level at the tenor points corresponding the bond durations.

Compared to the fitted bond segment the shifted curve can be built with only one liquid bond. This approach is useful in
cases of limited number of liquid comparable bonds and hence unstable fitting of Nelson Siegel. The average spread
at the average duration point may be considered as a sensitivity point of a corresponding zero coupon bond.

The segment has the following elements:

\begin{itemize}
  \item Quotes: a list of bond price quotes, for each security in the list, reference data must be available
  \item ReferenceCurve: the curve which will be used to calculate the bond spread. This curve will also be shifted by the resulting spread
  \item IborIndexCurves: for each Ibor index that is required by one of the bonds to which the curve is fitted, a mapping
    to an estimation curve for that index must be provided
  \item ExtrapolateFlat: if true, the parametric curve is extrapolated flat in the instantaneous forward rate before the
    first and after the last maturity of the bonds in the calibration basket. This avoids unrealistic rates at the short
    end or for long maturities in the resulting curve.
  \end{itemize}

  \begin{listing}[H]
    %\hrule\medskip
    \begin{minted}[fontsize=\footnotesize]{xml}
        <YieldCurve>
        <CurveId>USD.Benchmark.Curve_Shifted</CurveId>
        <CurveDescription>Curve shifted with a bond's spreads at the bond duration tenors</CurveDescription>
        <Currency>USD</Currency>
        <DiscountCurve/>
        <Segments>
          <BondYieldShifted>
            <Type>Bond Yield Shifted</Type>
            <ReferenceCurve>USD1D</ReferenceCurve>
            <Quotes>
              <Quote>BOND/PRICE/EJ7706660</Quote>
              <Quote>BOND/PRICE/ZR5330686</Quote>
              <Quote>BOND/PRICE/AS0644417</Quote>
            </Quotes>
            <Conventions>BOND_CONVENTIONS</Conventions>
            <ExtrapolateFlat>true</ExtrapolateFlat>
            <IborIndexCurves>
              <IborIndexCurve iborIndex="USD-LIBOR-3M">USD3M</IborIndexCurve>
            </IborIndexCurves>
          </BondYieldShifted>
        </Segments>
        <InterpolationVariable>Discount</InterpolationVariable>
        <InterpolationMethod>Linear</InterpolationMethod>
        <YieldCurveDayCounter>A365</YieldCurveDayCounter>
        <Tolerance> </Tolerance>
        <Extrapolation>true</Extrapolation>
        <BootstrapConfig> </BootstrapConfig>
    </YieldCurve>
    \end{minted}
    \caption{Bond Yield Shifted curve segment}
    \label{lst:bond_yield_shifted}
    \end{listing}


\subsubsection*{Yield plus Default Segment}
\label{sec:yield_plus_default}

When the node name is \lstinline!YieldPlusDefault!, the \lstinline!Type! node has the only allowable value \emph{Yield
 Plus Default}, and the node has the structure shown in Listing \ref{lst:yield_plus_default_segment}. This segment is
used to build all-in discounting yield curves from a benchmark curve and (a weighted sum of) default curves. The
construction is in some sense inverse to the benchmark default curve construction, see \ref{ss:benchmark_default_curve}.

\begin{itemize}
\item ReferenceCurve: the benchmark yield curve serving as the basis of the resulting yield curve
\item DefaultCurves: a list of default curves whose weighted sum is added to the benchmark yield curve
\item Weights: a list of weights for the default curves, the number of weights must match the number of default curves
\end{itemize}

Notice that it is explicitly allowed to use default curves in different currencies than the benchmark yield curve. In
the construction, the hazard rate is reinterpreted as an instantaneous forward rate, and the sum of the curves is being
built in the instantaneous forward rate.

The definition takes into account the recovery rates associated to each default curve. The resulting discount factor is
computed as

\begin{equation}
P(0,t) = \prod_i  S_i(t)^{(1-R)w_i}
\end{equation}

where $S_i$ and $R_i$ are the survival probabilities and recovery rates of the source default curves, and $w_i$ are the
weights.

\begin{listing}[H]
%\hrule\medskip
\begin{minted}[fontsize=\footnotesize]{xml}
  <YieldCurve>
    <CurveId>BenchmarkPlusDefault</CurveId>
    <CurveDescription>USD Libor 3M + 0.5 x CDX.NA.HY + 0.5 x EUR.10BP</CurveDescription>
    <Currency>USD</Currency>
    <DiscountCurve/>
    <Segments>
      <YieldPlusDefault>
        <Type>Yield Plus Default</Type>
        <ReferenceCurve>USD3M</ReferenceCurve>
        <DefaultCurves>
          <DefaultCurve>Default/USD/CDX.NA.HY</DefaultCurve>
          <DefaultCurve>Default/EUR/EUR.10BP</DefaultCurve>
        </DefaultCurves>
        <Weights>
          <Weight>0.5</Weight>
          <Weight>0.5</Weight>
        </Weights>
      </YieldPlusDefault>
    </Segments>
  </YieldCurve>
</YieldCurves>
\end{minted}
\caption{Yield plus default curve segment}
\label{lst:yield_plus_default_segment}
\end{listing}

\subsubsection*{Weighted Average Segment}
\label{sec:weigthed_average}

When the node name is \lstinline!WeightedAverage!, the \lstinline!Type! node has the only allowable value
\emph{Weighted Average}, and the node has the structure shown in Listing \ref{lst:weighted_average_segment}. This segment
is used to build a curve with instantaneous forward rates that are the weighted sum of instantaneous forward rates of
reference curves. This way a projection curve for non-standard Ibor curves can be build, e.g. to project a Euribor2M
index using the curves for 1M and 3M.

\begin{itemize}
\item ReferenceCurve1: the first source curve
\item ReferenceCurve2: the second source curve
\item Weight1: the weight of the first curve
\item Weights: the weight of the second curve
\end{itemize}

If $P_1(0,t)$ and $P_2(0,t)$ denote the discount factors of the two reference curves, the discount factor $P(0,t)$ of
the resulting curve is defined as

\begin{equation}
P(0,t) = P_1(0,t)^{w_1}P_2(0,t)^{w_2}
\end{equation}

\begin{listing}[H]
%\hrule\medskip
\begin{minted}[fontsize=\footnotesize]{xml}
<YieldCurve>
  <CurveId>EUR2M</CurveId>
  <CurveDescription>Euribor2M forwarding curve, interpolated from 1M and 3M</CurveDescription>
  <Currency>EUR</Currency>
  <DiscountCurve>EUR1D</DiscountCurve>
  <Segments>
    <WeightedAverage>
      <Type>Weighted Average</Type>
      <ReferenceCurve1>EUR1M</ReferenceCurve1>
      <ReferenceCurve2>EUR3M</ReferenceCurve2>
      <Weight1>0.5</Weight1>
      <Weight2>0.5</Weight2>
    </WeightedAverage>
  </Segments>
</YieldCurve>
\end{minted}
\caption{Weighted Average yield curve segment}
\label{lst:weighted_average_segment}
\end{listing}

\subsubsection*{Ibor Fallback Segment}
\label{sec:ibor_fallback_curve_segment}

When the node name is \lstinline!IborFallback!, the \lstinline!Type! node has the only allowable value \emph{Ibor
  Fallback}, and the node has the structure shown in Listing \ref{lst:ibor_fallback_segment}. This segment is used to
build a projection curve for an Ibor index based on a risk free rate and a spread.

\begin{listing}[H]
%\hrule\medskip
\begin{minted}[fontsize=\footnotesize]{xml}
<YieldCurve>
  <CurveId>USD-LIBOR-3M</CurveId>
  <CurveDescription>USD-Libor-3M built from USD-SOFR plus spread</CurveDescription>
  <Currency>USD</Currency>
  <DiscountCurve/>
  <Segments>
    <IborFallback>
      <Type>Ibor Fallback</Type>
      <IborIndex>USD-LIBOR-3M</IborIndex>
      <RfrCurve>Yield/USD/USD-SOFR</RfrCurve>
      <!-- optional, if not given the rfr index and spread are read from the ibor
           fallback configuration -->
      <RfrIndex>USD-SOFR</RfrIndex>
      <Spread>0.0026161</Spread>
    </IborFallback>
  </Segments>
</YieldCurve>
\end{minted}
\caption{Ibor fallback segment}
\label{lst:ibor_fallback_segment}
\end{listing}

\subsubsection*{Discount Ratio Segment}
\label{sec:dicount_ratio_segment}

When the node name is \lstinline!DiscountRatio!, the \lstinline!Type! node has the only allowable value \emph{Dicount
Ratio} and the node has the structure shown in Listing \ref{lst:discount_ratio_segment}. This segment is used to build a
curve with discount factors $P(0,t)$ from three input curves with discount factors $P_b(0,t)$, $P_n(0,t)$ and $P_d(0,t)$
(``base'', ``numerator'', ``denominator'' curves) following the equation

\begin{equation}\label{discount_ratio_df}
  P(0,t) = P_b(0,t) \frac{P_n(0,t)}{P_d(0,t)}
\end{equation}

The main use case of this segment is to build a discount curve ``CCY1-IN-CCY2'' for cashflows in CCY1 collateralized in
CCY2 when curves ``CCY1-IN-BASE'' and ``CCY2-IN-BASE'' are known for a common base currency BASE:

For a maturity $t$ denote the zero rate on a curve ``X'' by $r_X(t)$ and the correpsonding discount factor by
$P_X(0,t)$. Furthermore, write ``CCY'' as shorthand for ``CCY-IN-CCY'', i.e. for the discount curve for cashflows in the
same currency as the collateral currency ``CCY''. We write the desired zero rate as

\begin{equation}\label{discount_ratio_rates}
  \begin{split}
    r_{\text{CCY1-IN-CCY2}} = r_{\text{CCY2}} + & ( r_{\text{BASE-IN-CCY2}} - r_{\text{CCY2}} ) + \\
                               & (r_{\text{CCY1-IN-CCY2}} - r_{\text{BASE-IN-CCY2}})
  \end{split}
\end{equation}

We now assume that these two rate differentials stay the same when we switch from collateral currency ``CCY2'' to
``BASE'', i.e.

\begin{eqnarray}
r_{\text{BASE-IN-CCY2}} - r_{\text{CCY2}} &\approx& r_{\text{BASE}} - r_{\text{CCY2-IN-BASE}} \label{discount_ratio_rate1} \\
r_{\text{CCY1-IN-CCY2}} - r_{\text{BASE-IN-CCY2}} &\approx& r_{\text{CCY1-IN-BASE}} - r_{\text{BASE}}  \label{discount_ratio_rate2}
\end{eqnarray}

In less technical terms we assume that FX Forward Quotes CCY2 / BASE and CCY1 / BASE stay constant when the collateral
currency changes, which seems reasonable, if no further market information is available.

The discount factors associated to the RHS of \ref{discount_ratio_rate1} and \ref{discount_ratio_rate2} can be written

\begin{eqnarray}
  P_{\text{BASE}}(0,t) / P_{\text{CCY2-IN-BASE}}(0,t) \\
  P_{\text{CCY1-IN-BASE}}(0,t) / P_{\text{BASE}}(0,t)
\end{eqnarray}

and so \ref{discount_ratio_df} can be written

\begin{equation}\label{discount_ratio_df2}
  P_{\text{CCY1-IN-CCY2}}(0,t) = \frac{P_{\text{CCY2}}(0,t) P_{\text{CCY1-IN-BASE}}(0,t)}{P_{\text{CCY2-IN-BASE}}(0,t)}
\end{equation}

so the following choice of curves will result in the desired ``CCY1-IN-CCY2'' curve:

\begin{itemize}
\item base curve = ``CCY2-IN-CCY2''
\item numerator curve = ``CCY1-IN-BASE''
\item denominator curve = ``CCY2-IN-BASE''
\end{itemize}

\begin{listing}[H]
%\hrule\medskip
\begin{minted}[fontsize=\footnotesize]{xml}
<YieldCurve>
  <CurveId>GBP-IN-EUR</CurveId>
  <CurveDescription>GBP collateralized in EUR discount curve</CurveDescription>
  <Currency>GBP</Currency>
  <DiscountCurve/>
  <Segments>
    <DiscountRatio>
      <Type>Discount Ratio</Type>
      <BaseCurve currency="EUR">EUR1D</BaseCurve>
      <NumeratorCurve currency="GBP">GBP-IN-USD</NumeratorCurve>
      <DenominatorCurve currency="EUR">EUR-IN-USD</DenominatorCurve>
    </DiscountRatio>
  </Segments>
</YieldCurve>
\end{minted}
\caption{Discount Ratio segment}
\label{lst:discount_ratio_segment}
\end{listing}



\subsubsection{Default Curves}

{\footnotesize
\begin{lstlisting}[caption={Default curve configuration}, 	label=lst:defaultcurve_configuration]
<DefaultCurves>
	<DefaultCurve>
		<CurveId>BANK_SR_USD</CurveId>
		<CurveDescription>BANK SR CDS USD</CurveDescription>
		<Currency>USD</Currency>
		<Type>SpreadCDS</Type>
		<DiscountCurve>Yield/USD/USD3M</DiscountCurve>
		<DayCounter>A365</DayCounter>
		<RecoveryRate>RECOVERY_RATE/RATE/BANK/SR/USD</RecoveryRate>
		<Quotes>
			<Quote>CDS/CREDIT_SPREAD/BANK/SR/USD/1Y</Quote>
			<Quote>CDS/CREDIT_SPREAD/BANK/SR/USD/2Y</Quote>
			<Quote>CDS/CREDIT_SPREAD/BANK/SR/USD/3Y</Quote>
			<Quote>CDS/CREDIT_SPREAD/BANK/SR/USD/4Y</Quote>
			<Quote>CDS/CREDIT_SPREAD/BANK/SR/USD/5Y</Quote>
			<Quote>CDS/CREDIT_SPREAD/BANK/SR/USD/7Y</Quote>
			<Quote>CDS/CREDIT_SPREAD/BANK/SR/USD/10Y</Quote>
		</Quotes>
		<Conventions>CDS-STANDARD-CONVENTIONS</Conventions>
	</DefaultCurve>
	<DefaultCurve>
		<!-- ... -->
	</DefaultCurve>
</DefaultCurves>

\end{lstlisting}
}

\subsubsection{Swaption Volatility Structures}

{\footnotesize
\begin{lstlisting}[caption={Swaption volatility configuration}, 	label=lst:swaptionvol_configuration]
<SwaptionVolatilities>    
	<SwaptionVolatility>
		<CurveId>EUR_SW_N</CurveId>
		<CurveDescription>EUR normal swaption volatilities</CurveDescription>
		<!-- ATM (Smile not yet supported) -->
		<Dimension>ATM</Dimension>
		<!-- Normal or Lognormal or ShiftedLognormal -->
		<VolatilityType>Normal</VolatilityType>
		<!-- Flat or Linear -->
		<Extrapolation>Flat</Extrapolation>
		<!-- Day counter for date to time conversion -->
		<DayCounter>Actual/365 (Fixed)</DayCounter>
		<!--Calendar and Business day convention for option tenor to date conversion -->
		<Calendar>TARGET</Calendar>
		<BusinessDayConvention>Following</BusinessDayConvention>
		<OptionTenors>
			1M,3M,6M,1Y,2Y,3Y,4Y,5Y,7Y,10Y,15Y,20Y,25Y,30Y
		</OptionTenors>
		<SwapTenors>
			1Y,2Y,3Y,4Y,5Y,7Y,10Y,15Y,20Y,25Y,30Y
		</SwapTenors>
		<ShortSwapIndexBase>EUR-CMS-1Y</ShortSwapIndexBase>
		<SwapIndexBase>EUR-CMS-30Y</SwapIndexBase>
	</SwaptionVolatility>
    <SwaptionVolatility>
    	<!-- ... -->
    </SwaptionVolatility>
</SwaptionVolatilities>
\end{lstlisting}
}

\subsubsection{FX Volatility Structures}

{\footnotesize
\begin{lstlisting}[caption={FX option volatility configuration}, 	label=lst:fxoptionvol_configuration]
<FXVolatilities>
	<FXVolatility>
		<CurveId>EURUSD</CurveId>
		<CurveDescription />
		<Dimension>ATM</Dimension>
		<Expiries>
			1M,3M,6M,1Y,2Y,3Y,10Y
		</Expiries>
	</FXVolatility>
	<FXVolatility>
		<!-- ... -->
	</FXVolatility>
</FXVolatilities>
\end{lstlisting}
}

%--------------------------------------------------------
\subsection{Conventions: {\tt conventions.xml}}
%--------------------------------------------------------

%--------------------------------------------------------
\subsection{Conventions: {\tt conventions.xml}}
\label{sec:conventions}
%--------------------------------------------------------

The conventions to associate with a set market quotes in the construction of termstructures are specified in another xml
file which we will refer to as {\tt conventions.xml} in the following though the file name can be chosen by the user.
Each separate set of conventions is stored in an XML node. The type of conventions that a node holds is determined by
the node name. Every node has an \lstinline!Id! node that gives a unique identifier for the convention set. The
following sections describe the type of conventions that can be created and the allowed values.

%- - - - - - - - - - - - - - - - - - - - - - - - - - - - - - - - - - - - - - - -
\subsubsection{Zero Conventions}
%- - - - - - - - - - - - - - - - - - - - - - - - - - - - - - - - - - - - - - - -
A node with name \emph{Zero} is used to store conventions for direct zero rate quotes. Direct zero rate quotes can be
given with an explicit maturity date or with a tenor and a set of conventions from which the maturity date is
deduced. The node for a zero rate quote with an explicit maturity date is shown in Listing
\ref{lst:zero_conventions_date}. The node for a tenor based zero rate is shown in Listing
\ref{lst:zero_conventions_tenor}.

\begin{listing}[H]
%\hrule\medskip
\begin{minted}[fontsize=\footnotesize]{xml}
<Zero>
  <Id> </Id>
  <TenorBased>False</TenorBased>
  <DayCounter> </DayCounter>
  <CompoundingFrequency> </CompoundingFrequency>
  <Compounding> </Compounding>
</Zero>
\end{minted}
\caption{Zero conventions}
\label{lst:zero_conventions_date}
\end{listing}

\begin{listing}[H]
%\hrule\medskip
\begin{minted}[fontsize=\footnotesize]{xml}
<Zero>
  <Id> </Id>
  <TenorBased>True</TenorBased>
  <DayCounter> </DayCounter>
  <CompoundingFrequency> </CompoundingFrequency>
  <Compounding> </Compounding>
  <TenorCalendar> </TenorCalendar>
  <SpotLag> </SpotLag>
  <SpotCalendar> </SpotCalendar>
  <RollConvention> </RollConvention>
  <EOM> </EOM>
</Zero>
\end{minted}
\caption{Zero conventions, tenor based}
\label{lst:zero_conventions_tenor}
\end{listing}

The meanings of the various elements in this node are as follows:
\begin{itemize}
\item TenorBased: True if the conventions are for a tenor based zero quote and False if they are
for a zero quote with an explicit maturity date.
\item DayCounter: The day count basis associated with the zero rate quote (for choices see section
\ref{sec:allowable_values})
\item CompoundingFrequency: The frequency of compounding (Choices are {\em Once, Annual, Semiannual, Quarterly,
Bimonthly, Monthly, Weekly, Daily}).
\item Compounding: The type of compounding for the zero rate (Choices are {\em Simple, Compounded, Continuous,
SimpleThenCompounded}).
\item TenorCalendar: The calendar used to advance from the spot date to the maturity date by the zero rate tenor (for
choices see section \ref{sec:allowable_values}).
\item SpotLag [Optional]: The number of business days to advance from the valuation date before applying the zero rate
tenor. If not provided, this defaults to 0.
\item SpotCalendar [Optional]: The calendar to use for business days when applying the \lstinline!SpotLag!. If not
provided, it defaults to a calendar with no holidays.
\item RollConvention [Optional]: The roll convention to use when applying the zero rate tenor. If not provided, it
defaults to Following (Choices are {\em Backward, Forward, Zero, ThirdWednesday, Twentieth, TwentiethIMM, CDS, ThirdThursday, ThirdFriday, MondayAfterThirdFriday, TuesdayAfterThirdFriday, LastWednesday}).
\item EOM [Optional]: Whether or not to use the end of month convention when applying the zero rate tenor. If not
provided, it defaults to false.
\end{itemize}

%- - - - - - - - - - - - - - - - - - - - - - - - - - - - - - - - - - - - - - - -
\subsubsection{Deposit Conventions}
%- - - - - - - - - - - - - - - - - - - - - - - - - - - - - - - - - - - - - - - -

A node with name \emph{Deposit} is used to store conventions for deposit or index fixing quotes. The conventions can be
index based, in which case all necessary conventions are deduced from a given index family. The structure of the index
based node is shown in Listing \ref{lst:deposit_conventions_index}. Alternatively, all the necessary conventions can be
given explicitly without reference to an index family. The structure of this node is shown in Listing
\ref{lst:deposit_conventions_explicit}.

\begin{listing}[H]
%\hrule\medskip
\begin{minted}[fontsize=\footnotesize]{xml}
<Deposit>
  <Id> </Id>
  <IndexBased>True</IndexBased>
  <Index> </Index>
</Deposit>
\end{minted}
\caption{Deposit conventions}
\label{lst:deposit_conventions_index}
\end{listing}

\begin{listing}[H]
%\hrule\medskip
\begin{minted}[fontsize=\footnotesize]{xml}
<Deposit>
  <Id> </Id>
  <IndexBased>False</IndexBased>
  <Calendar> </Calendar>
  <Convention> </Convention>
  <EOM> </EOM>
  <DayCounter> </DayCounter>
</Deposit>
\end{minted}
\caption{Deposit conventions}
\label{lst:deposit_conventions_explicit}
\end{listing}


The meanings of the various elements in this node are as follows:
\begin{itemize}
\item IndexBased: \emph{True} if the deposit conventions are index based and \emph{False} if the conventions are given
explicitly.
\item Index: The index family from which to imply the conventions for the deposit quote. For example, this could be
EUR-EURIBOR, USD-LIBOR etc.
\item Calendar: The business day calendar for the deposit quote.
\item Convention: The roll convention for the deposit quote.
\item EOM: \emph{True} if the end of month roll convention is to be used for the deposit quote and \emph{False} if not.
\item DayCounter: The day count basis associated with the deposit quote.
\end{itemize}

%- - - - - - - - - - - - - - - - - - - - - - - - - - - - - - - - - - - - - - - -
\subsubsection{Future Conventions}\label{ss:conventions_future}
%- - - - - - - - - - - - - - - - - - - - - - - - - - - - - - - - - - - - - - - -

A node with name \emph{Future} is used to store conventions for money market (MM) or overnight index (OI) interest rate
future quotes, for example futures on Euribor 3M or SOFR 3M underlyings. The structure of this node is shown in Listing
\ref{lst:future_conventions}. The fields have the following meaning:

\begin{itemize}
\item Id: The name of the convention.
\item Index: The underlying index of the futures, this is either a MM (i.e. Ibor) index like e.g. EUR-EURIBOR-3M or an
  overnight index like e.g. USD-SOFR.
\item DateGenerationRule [Optional]: This should be set to 'IMM' when the start and end dates of the future are
  following the IMM date logic or 'FirstDayOfMonth' when the start and end date are the first day of a month. If not
  given this field defaults to 'IMM'.
  \begin{itemize}
  \item For MM futures only 'IMM' is allowed and the expiry date is determined as the next 3rd Wednesday of the expiry
    month of a future.
  \item For an overnight index future 'IMM' means that the end date of the future is set to the 3rd Wednesday of the
    expiry month and the start date is set to the 3rd Wednesday of the expiry month minus the future tenor. The setting
    'IMM' applies to SOFR-3M futures for example. 'FirstDayOfMonth' on the other hand means that the end date of the
    future is set to the first day in the month following the future's expiry month and the start date is set to the
    first day of the month lying $n$ months before the end date's month where $n$ is the number of months of the
    future's underlying tenor. The setting 'FirstDayOfMonth' applies to SOFR-1M futures for example. This tenor is
    derived from the market quote, see \ref{ss:market_data_oi_index_future_prices}.
  \end{itemize}
\item OvernightIndexFutureNettingType [Optional]: Only relevant for OI futures. Can be 'Compounding' (which is also the
  default value if no value is given) or 'Averaging'. For example, SOFR 3M futures are compounding while SOFR 1M futures
  are averaging the daily overnight fixings over the calculation period of the future.
\end{itemize}

Listings \ref{lst:future_conventions_euribor_3m}, \ref{lst:future_conventions_sofr_3m},
\ref{lst:future_conventions_sofr_1m} show examples for Euribor-3M, SOFR-3M and SOFR-1M future conventions.

\begin{listing}[H]
%\hrule\medskip
\begin{minted}[fontsize=\footnotesize]{xml}
<Future>
  <Id> </Id>
  <Index> </Index>
  <DateGenerationRule> </DateGenerationRule>
  <OvernightIndexFutureNettingType> </OvernightIndexFutureNettingType>
</Future>
\end{minted}
\caption{Future conventions}
\label{lst:future_conventions}
\end{listing}

\begin{listing}[H]
%\hrule\medskip
\begin{minted}[fontsize=\footnotesize]{xml}
<Future>
  <Id>EURIBOR-3M-FUTURES</Id>
  <Index>EUR-EURIBOR-3M</Index>
</Future>
\end{minted}
\caption{Euribor 3M MM Future conventions}
\label{lst:future_conventions_euribor_3m}
\end{listing}

\begin{listing}[H]
%\hrule\medskip
\begin{minted}[fontsize=\footnotesize]{xml}
  <Future>
    <Id>USD-SOFR-3M-FUTURES</Id>
    <Index>USD-SOFR</Index>
    <DateGenerationRule>IMM</DateGenerationRule>
    <OvernightIndexFutureNettingType>Compounding</OvernightIndexFutureNettingType>
  </Future>
\end{minted}
\caption{USD SOFR 3M OI Future conventions}
\label{lst:future_conventions_sofr_3m}
\end{listing}

\begin{listing}[H]
%\hrule\medskip
\begin{minted}[fontsize=\footnotesize]{xml}
  <Future>
    <Id>USD-SOFR-1M-FUTURES</Id>
    <Index>USD-SOFR</Index>
    <DateGenerationRule>FirstDayOfMonth</DateGenerationRule>
    <OvernightIndexFutureNettingType>Averaging</OvernightIndexFutureNettingType>
  </Future>
\end{minted}
\caption{USD SOFR 1M OI Future conventions}
\label{lst:future_conventions_sofr_1m}
\end{listing}


%- - - - - - - - - - - - - - - - - - - - - - - - - - - - - - - - - - - - - - - -
\subsubsection{FRA Conventions}
%- - - - - - - - - - - - - - - - - - - - - - - - - - - - - - - - - - - - - - - -
A node with name \emph{FRA} is used to store conventions for FRA quotes. The structure of this node is shown in Listing 
\ref{lst:fra_conventions}. The only piece of information needed is the underlying index name and this is given in the 
\lstinline!Index! node. For example, this could be EUR-EURIBOR-6M, CHF-LIBOR-6M etc.

\begin{listing}[H]
%\hrule\medskip
\begin{minted}[fontsize=\footnotesize]{xml}
<FRA>
  <Id> </Id>
  <Index> </Index>
</FRA>
\end{minted}
\caption{FRA conventions}
\label{lst:fra_conventions}
\end{listing}

%- - - - - - - - - - - - - - - - - - - - - - - - - - - - - - - - - - - - - - - -
\subsubsection{OIS Conventions}
%- - - - - - - - - - - - - - - - - - - - - - - - - - - - - - - - - - - - - - - -

A node with name \emph{OIS} is used to store conventions for Overnight Indexed Swap (OIS) quotes. The structure of this
node is shown in Listing \ref{lst:ois_conventions}.

\begin{listing}[H]
%\hrule\medskip
\begin{minted}[fontsize=\footnotesize]{xml}
<OIS>
  <Id> </Id>
  <SpotLag> </SpotLag>
  <Index> </Index>
  <FixedDayCounter> </FixedDayCounter>
  <FixedCalendar> </FixedCalendar>
  <PaymentLag> </PaymentLag>
  <EOM> </EOM>
  <FixedFrequency> </FixedFrequency>
  <FixedConvention> </FixedConvention>
  <FixedPaymentConvention> </FixedPaymentConvention>
  <Rule> </Rule>
  <PaymentCalendar> </PaymentCalendar>
  <RateCutoff> </RateCutoff>
</OIS>
\end{minted}
\caption{OIS conventions}
\label{lst:ois_conventions}
\end{listing}

The meanings of the various elements in this node are as follows:
\begin{itemize}
\item SpotLag: The number of business days until the start of the OIS.
\item Index: The name of the overnight index. For example, this could be EUR-EONIA, USD-FedFunds etc.
\item FixedDayCounter: The day count basis on the fixed leg of the OIS.
\item FixedCalendar [Optional]: The business day calendar on the fixed leg. Optional to retain backwards compatibility
  with older versions, if not given defaults to index fixing calendar.
\item PaymentLag [Optional]: The payment lag, as a number of business days, on both legs. If not provided, this defaults
to 0.
\item EOM [Optional]: \emph{True} if the end of month roll convention is to be used when generating the OIS schedule and
\emph{False} if not. If not provided, this defaults to \emph{False}.
\item FixedFrequency [Optional]: The frequency of payments on the fixed leg. If not provided, this defaults to
\emph{Annual}.
\item FixedConvention [Optional]: The roll convention for accruals on the fixed leg. If not provided, this defaults to
\emph{Following}.
\item FixedPaymentConvention [Optional]: The roll convention for payments on the fixed leg. If not provided, this
defaults to \emph{Following}.
\item Rule [Optional]: The rule used for generating the OIS dates schedule i.e.\ \emph{Backward} or \emph{Forward}. If
not provided, this defaults to \emph{Backward}.
\item PaymentCalendar [Optional]: The business day calendar used for determining coupon payment dates.
If not specified, this defaults to the fixing calendar defined on the overnight index.
\item RateCutoff: The rate cut-off on the overnight leg. Generally, the overnight fixing
is only observed up to a certain number of days before the end of the interest period date.
The last observed rate is applied for the remaining days in the period.
This rate cut-off gives the number of days e.g.\ 1 for ESTR or SOFR.
If not specified, this defaults to 0 days.
\end{itemize}

%- - - - - - - - - - - - - - - - - - - - - - - - - - - - - - - - - - - - - - - -
\subsubsection{Swap Conventions}
%- - - - - - - - - - - - - - - - - - - - - - - - - - - - - - - - - - - - - - - -
A node with name \emph{Swap} is used to store conventions for vanilla interest rate swap (IRS) quotes. The structure of
this node is shown in Listing \ref{lst:swap_conventions}.

\begin{listing}[H]
%\hrule\medskip
\begin{minted}[fontsize=\footnotesize]{xml}
<Swap>
  <Id> </Id>
  <FixedCalendar> </FixedCalendar>
  <FixedFrequency> </FixedFrequency>
  <FixedConvention> </FixedConvention>
  <FixedDayCounter> </FixedDayCounter>
  <Index> </Index>
  <FloatFrequency> </FloatFrequency>
  <SubPeriodsCouponType> </SubPeriodsCouponType>
</Swap>
\end{minted}
\caption{Swap conventions}
\label{lst:swap_conventions}
\end{listing}

The meanings of the various elements in this node are as follows:
\begin{itemize}
\item FixedCalendar: The business day calendar on the fixed leg.
\item FixedFrequency: The frequency of payments on the fixed leg.
\item FixedConvention: The roll convention on the fixed leg.
\item FixedDayCounter: The day count basis on the fixed leg.
\item Index: The Ibor index on the floating leg.
\item FloatFrequency [Optional]: The frequency of payments on the floating leg, to be used if the frequency is different to the tenor of the index (e.g. CAD swaps for BA-3M have a 6M or 1Y payment frequency with a Compounding coupon)
\item SubPeriodsCouponType [Optional]: Defines how coupon rates should be calculated when the float frequency is different to that of the index. Possible values are "Compounding" and "Averaging".
\end{itemize}

%- - - - - - - - - - - - - - - - - - - - - - - - - - - - - - - - - - - - - - - -
\subsubsection{Average OIS Conventions}
%- - - - - - - - - - - - - - - - - - - - - - - - - - - - - - - - - - - - - - - -
A node with name \emph{AverageOIS} is used to store conventions for average OIS quotes. An average OIS is a swap where a
fixed rate is swapped against a daily averaged overnight index plus a spread. The structure of this node is shown in
Listing \ref{lst:average_ois_conventions}.

\begin{listing}[H]
%\hrule\medskip
\begin{minted}[fontsize=\footnotesize]{xml}
<AverageOIS>
  <Id> </Id>
  <SpotLag> </SpotLag>
  <FixedTenor> </FixedTenor>
  <FixedDayCounter> </FixedDayCounter>
  <FixedCalendar> </FixedCalendar>
  <FixedConvention> </FixedConvention>
  <FixedPaymentConvention> </FixedPaymentConvention>
  <FixedFrequency> </FixedFrequency>
  <Index> </Index>
  <OnTenor> </OnTenor>
  <RateCutoff> </RateCutoff>
</AverageOIS>
\end{minted}
\caption{Average OIS conventions}
\label{lst:average_ois_conventions}
\end{listing}


The meanings of the various elements in this node are as follows:
\begin{itemize}
\item SpotLag: Number of business days until the start of the average OIS.
\item FixedTenor: The frequency of payments on the fixed leg.
\item FixedDayCounter: The day count basis on the fixed leg.
\item FixedCalendar: The business day calendar on the fixed leg.
\item FixedFrequency: The frequency of payments on the fixed leg.
\item FixedConvention: The roll convention for accruals on the fixed leg.
\item FixedPaymentConvention: The roll convention for payments on the fixed leg.
\item FixedFrequency [Optional]: The frequency of payments on the fixed leg. If not provided, this defaults to \emph{Annual}.
\item Index: The name of the overnight index.
\item OnTenor: The frequency of payments on the overnight leg.
\item RateCutoff: The rate cut-off on the overnight leg. Generally, the overnight fixing is only observed up to a
certain number of days before the payment date and the last observed rate is applied for the remaining days in the
period. This rate cut-off gives the number of days e.g.\ 2 for Fed Funds average OIS.
\end{itemize}

%- - - - - - - - - - - - - - - - - - - - - - - - - - - - - - - - - - - - - - - -
\subsubsection{Tenor Basis Swap Conventions}
%- - - - - - - - - - - - - - - - - - - - - - - - - - - - - - - - - - - - - - - -
A node with name \emph{TenorBasisSwap} is used to store conventions for tenor basis swap quotes. The structure of this 
node is shown in Listing \ref{lst:tenor_basis_conventions}.

\begin{listing}[H]
%\hrule\medskip
\begin{minted}[fontsize=\footnotesize]{xml}
<TenorBasisSwap>
  <Id> </Id>
  <PayIndex> </PayIndex>
  <PayFrequency> </PayFrequency>
  <ReceiveIndex> </ReceiveIndex>
  <ReceiveFrequency> </ReceiveFrequency>
  <SpreadOnRec> </SpreadOnRec>
  <IncludeSpread> </IncludeSpread>
  <SubPeriodsCouponType> </SubPeriodsCouponType>
</TenorBasisSwap>
\end{minted}
\caption{Tenor basis swap conventions}
\label{lst:tenor_basis_conventions}
\end{listing}


The meanings of the various elements in this node are as follows:
\begin{itemize}
\item PayIndex: The name of Ibor/Overnight Index of the pay leg.
\item PayFrequency [Optional]: The frequency of payments on the PayIndex leg. This is usually the same as the PayIndex's tenor. 
However, it can also be longer, e.g. overnight indexed vs overnight indexed basis swaps that may be quarterly on both legs. 
If not provided, this defaults to the PayIndex's tenor.
\item ReceiveIndex: The name of Ibor/Overnight Index of the receive leg.
\item ReceiveFrequency [Optional]: The frequency of payments on the ReceiveIndex leg. This is usually the same as the ReceiveIndex's tenor. 
However, it can also be longer, e.g. overnight indexed vs overnight indexed basis swaps that may be quarterly on both legs. 
If not provided, this defaults to the ReceiveIndex's tenor.
\item SpreadOnRec [Optional]: \emph{True}  if the tenor basis swap quote has the spread on the pay index leg and \emph{False} if not. 
If not provided, this defaults to \emph{True}.
\item IncludeSpread [Optional]: \emph{True} if the tenor basis swap spread is to be included when compounding is performed on the spread leg and \emph{False} if not. 
If not provided, this defaults to \emph{False}.
\item SubPeriodsCouponType [Optional]: This field can have the value \emph{Compounding} or \emph{Averaging}. It applies to Ibor vs OI and Ibor vs Ibor basis swaps when the frequency of payments on the spread leg does not equal the spread leg index’s tenor. 
If \emph{Compounding} is specified, then the spread tenor Ibor index is compounded and paid on the frequency specified in the corresponding node. 
If \emph{Averaging} is specified, then the short tenor Ibor index is averaged and paid on the frequency specified in the corresponding node.
\end{itemize}

%- - - - - - - - - - - - - - - - - - - - - - - - - - - - - - - - - - - - - - - -
\subsubsection{Tenor Basis Two Swap Conventions}
%- - - - - - - - - - - - - - - - - - - - - - - - - - - - - - - - - - - - - - - -
A node with name \emph{TenorBasisTwoSwap} is used to store conventions for tenor basis swap quotes where the quote is
the spread between the fair fixed rate on two swaps against Ibor indices of different tenors. We call the swap against
the Ibor index of longer tenor the long swap and the remaining swap the short swap. The structure of the tenor basis two
swap conventions node is shown in Listing \ref{lst:tenor_basis_two_conventions}.

\begin{listing}[H]
%\hrule\medskip
\begin{minted}[fontsize=\footnotesize]{xml}
<TenorBasisTwoSwap>
  <Id> </Id>
  <Calendar> </Calendar>
  <LongFixedFrequency> </LongFixedFrequency>
  <LongFixedConvention> </LongFixedConvention>
  <LongFixedDayCounter> </LongFixedDayCounter>
  <LongIndex> </LongIndex>
  <ShortFixedFrequency> </ShortFixedFrequency>
  <ShortFixedConvention> </ShortFixedConvention>
  <ShortFixedDayCounter> </ShortFixedDayCounter>
  <ShortIndex> </ShortIndex>
  <LongMinusShort> </LongMinusShort>
</TenorBasisTwoSwap>
\end{minted}
\caption{Tenor basis two swap conventions}
\label{lst:tenor_basis_two_conventions}
\end{listing}

The meanings of the various elements in this node are as follows:
\begin{itemize}
\item Calendar: The business day calendar on both swaps.
\item LongFixedFrequency: The frequency of payments on the fixed leg of the long swap.
\item LongFixedConvention: The roll convention on the fixed leg of the long swap.
\item LongFixedDayCounter: The day count basis on the fixed leg of the long swap.
\item LongIndex: The Ibor index on the floating leg of the long swap.
\item ShortFixedFrequency: The frequency of payments on the fixed leg of the short swap.
\item ShortFixedConvention: The roll convention on the fixed leg of the short swap.
\item ShortFixedDayCounter: The day count basis on the fixed leg of the short swap.
\item ShortIndex: The Ibor index on the floating leg of the short swap.
\item LongMinusShort [Optional]: \emph{True} if the basis swap spread is to be interpreted as the fair rate on the long
swap minus the fair rate on the short swap and \emph{False} if the basis swap spread is to be interpreted as the fair
rate on the short swap minus the fair rate on the long swap. If not provided, it defaults to \emph{True}.
\end{itemize}

%- - - - - - - - - - - - - - - - - - - - - - - - - - - - - - - - - - - - - - - -
\subsubsection{FX Conventions}\label{sss:fx_convention}
%- - - - - - - - - - - - - - - - - - - - - - - - - - - - - - - - - - - - - - - -
A node with name \emph{FX} is used to store conventions for FX spot and forward quotes for a given currency pair. The
structure of this node is shown in Listing \ref{lst:fx_conventions}.

\begin{listing}[H]
%\hrule\medskip
\begin{minted}[fontsize=\footnotesize]{xml}
<FX>
  <Id> </Id>
  <SpotDays> </SpotDays>
  <SourceCurrency> </SourceCurrency>
  <TargetCurrency> </TargetCurrency>
  <PointsFactor> </PointsFactor>
  <AdvanceCalendar> </AdvanceCalendar>
  <SpotRelative> </SpotRelative>
  <EOM> </EOM>
  <Convention> </Convention>
</FX>
\end{minted}
\caption{FX conventions}
\label{lst:fx_conventions}
\end{listing}


The meanings of the various elements in this node are as follows:
\begin{itemize}
\item SpotDays: The number of business days to spot for the currency pair.
\item SourceCurrency: The source currency of the currency pair. The FX quote is assumed to give the number of units of
target currency per unit of source currency.
\item TargetCurrency: The target currency of the currency pair.
\item PointsFactor: The number by which a points quote for the currency pair should be divided before adding it to the
spot quote to obtain the forward rate.
\item AdvanceCalendar [Optional]: The business day calendar(s) used for advancing dates for both spot and forwards. If
not provided, it defaults to a calendar with no holidays.
\item SpotRelative [Optional]: \emph{True} if the forward tenor is to be interpreted as being relative to the spot date.
\emph{False} if the forward tenor is to be interpreted as being relative to the valuation date. If not provided, it
defaults to \emph{True}.
\item EOM [Optional]: A flag indicating whether the end of month roll convention is to be used for FX forward quotes. If not provided, it defaults to \emph{False}.
\item Convention [Optional]: The business day convention used when advancing dates. If not provided, it defaults to \emph{Following}.
\end{itemize}

%- - - - - - - - - - - - - - - - - - - - - - - - - - - - - - - - - - - - - - - -
\subsubsection{Cross Currency Basis Swap Conventions}
%- - - - - - - - - - - - - - - - - - - - - - - - - - - - - - - - - - - - - - - -
A node with name \emph{CrossCurrencyBasis} is used to store conventions for cross currency basis swap quotes. The
structure of this node is shown in Listing \ref{lst:xccy_basis_conventions}.

\begin{listing}[H]
%\hrule\medskip
\begin{minted}[fontsize=\footnotesize]{xml}
<CrossCurrencyBasis>
  <Id> </Id>
  <SettlementDays> </SettlementDays>
  <SettlementCalendar> </SettlementCalendar>
  <RollConvention> </RollConvention>
  <FlatIndex> </FlatIndex>
  <SpreadIndex> </SpreadIndex>
  <EOM> </EOM>
  <IsResettable> </IsResettable>
  <FlatIndexIsResettable> </FlatIndexIsResettable>>
  <PaymentLag> </PaymentLag>
  <FlatPaymentLag> </FlatPaymentLag>
  <!-- for OIS only -->
  <IncludeSpread> </IncludeSpread>
  <Lookback> </Lookback>
  <FixingDays> </FixingDays>
  <RateCutoff> </RateCutoff>
  <IsAveraged> </IsAveraged>
  <FlatIncludeSpread> </FlatIncludeSpread>
  <FlatLookback> </FlatLookback>
  <FlatFixingDays> </FlatFixingDays>
  <FlatRateCutoff> </FlatRateCutoff>
  <FlatIsAveraged> </FlatIsAveraged>
</CrossCurrencyBasis>
\end{minted}
\caption{Cross currency basis swap conventions}
\label{lst:xccy_basis_conventions}
\end{listing}


The meanings of the various elements in this node are as follows:
\begin{itemize}
\item SettlementDays: The number of business days to the start of the cross currency basis swap.
\item SettlementCalendar: The business day calendar(s) for both legs and to arrive at the settlement date using the
SettlementDays above.
\item RollConvention: The roll convention for both legs.
\item FlatIndex: The name of the index on the leg that does not have the cross currency basis spread.
\item SpreadIndex: The name of the index on the leg that has the cross currency basis spread.
\item EOM [Optional]: \emph{True} if the end of month convention is to be used when generating the schedule on both legs, and \emph{False} if not. If not provided, it defaults to \emph{False}.
\item IsResettable [Optional]: \emph{True} if the swap is mark-to-market resetting, and \emph{False} otherwise. If not provided, it defaults to \emph{False}.
\item FlatIndexIsResettable [Optional]: \emph{True} if it is the notional on the leg paying the flat index that resets, and \emph{False} otherwise. If not provided, it defaults to \emph{True}.
\item FlatTenor [Optional]: the flat leg period length (typical value is 3M), defaults to index tenor except for ON indices for which it defaults to 3M
\item SpreadTenor [Optional]: the spread leg period length (typical value is 3M), defaults to index tenor except for ON indices for which it defaults to 3M
\item SpreadPaymentLag [Optional]: the payment lag for the spread leg, allowable values are 0, 1, 2, ..., defaults to 0 if not given
\item FlatPaymentLag [Optional]: the payment lag for the flat leg, allowable values are 0, 1, 2, ..., defaults to 0 if nove given
\item SpreadIncludeSpread [Optional]: Only relevant if spread leg is OIS, allowable values are true, false, defaults to false if not given
\item SpreadLookback [Optional]: Only relevant if spread leg is OIS, allowable values are 0D, 1D, ..., defaults to 0D if not given
\item SpreadFixingDays [Optional]: Only relevant if spread leg is OIS, allowable values are 0, 1, 2, ..., defaults to 0 if not given
\item SpreadRateCutoff [Optional]: Only relevant if spread leg is OIS, allowable values are 0, 1, 2, ..., defaults to 0 if not given
\item SpreadIsAveraged [Optional]: Only relevant if spread leg is OIS, allowable values are true, false, defaults to false if not given
\item FlatIncludeSpread [Optional]: Only relevant if spread leg is OIS, allowable values are true, false, defaults to false if not given
\item FlatLookback [Optional]: Only relevant if spread leg is OIS, allowable values are 0D, 1D, ..., defaults to 0D if not given
\item FlatFixingDays [Optional]: Only relevant if spread leg is OIS, allowable values are 0, 1, 2, ..., defaults to 0 if not given
\item FlatRateCutoff [Optional]: Only relevant if spread leg is OIS, allowable values are 0, 1, 2, ..., defaults to 0 if not given
\item FlatIsAveraged [Optional]: Only relevant if spread leg is OIS, allowable values are true, false, defaults to false if not given
\end{itemize}

\subsubsection{Inflation Swap Conventions}
A node with name \lstinline!InflationSwap! is used to store conventions for zero or year on year inflation swap quotes. The structure of this node is shown in Listing \ref{lst:inflation_conventions}

\begin{listing}[H]
%\hrule\medskip
\begin{minted}[fontsize=\footnotesize]{xml}
<InflationSwap>
  <Id>EUHICPXT_INFLATIONSWAP</Id>
  <FixCalendar>TARGET</FixCalendar>
  <FixConvention>MF</FixConvention>
  <DayCounter>30/360</DayCounter>
  <Index>EUHICPXT</Index>
  <Interpolated>false</Interpolated>
  <ObservationLag>3M</ObservationLag>
  <AdjustInflationObservationDates>false</AdjustInflationObservationDates>
  <InflationCalendar>TARGET</InflationCalendar>
  <InflationConvention>MF</InflationConvention>
</InflationSwap>
\end{minted}
\caption{Inflation swap conventions}
\label{lst:inflation_conventions}
\end{listing}

The meaning of the elements is as follows:

\begin{itemize}
\item \lstinline!FixCalendar!: The calendar for the fixed rate leg of the swap.
\item \lstinline!FixConvention!: The rolling convention for the fixed rate leg of the swap.
\item \lstinline!DayCounter!: The payoff or coupon day counter, applied to both legs.
\item \lstinline!Index!: The underlying inflation index.
\item \lstinline!Interpolated!: Flag indicating interpolation of the index in the swap's payoff calculation.
\item \lstinline!ObservationLag!: The index observation lag to be applied.
\item \lstinline!AdjustInflationObservationDates!: Flag indicating whether index observation dates should be adjusted or not.
\item \lstinline!InflationCalendar!: The calendar for the inflation leg of the swap.
\item \lstinline!InflationConvention!: The rolling convention for the inflation leg of the swap.

\item \lstinline!PublicationRoll!:
This is an optional node taking the values \lstinline!None!, \lstinline!OnPublicationDate! or \lstinline!AfterPublicationDate!. If omitted, the value \lstinline!None! is used. Currently, our only known use case for a value other than \lstinline!None! is for Australian zero coupon inflation indexed swaps (ZCIIS). Here, the index is published quarterly on the last Wednesday of the month following the end of the reference quarter. The start date and maturity date of the market quoted ZCIIS roll to the next quarterly date after the publication date of the index. For example, the AU CPI value for Q3 2020, i.e.\ 1 Jul 2020 to 30 Sep 2020 was released on 28 Oct 2020. On 27 Oct 2020, before the index publication date, the market 5Y ZCIIS would start on 15 Sep 2020 and end on 15 Sep 2025 and reference the Q2 inflation index value. On 29 Oct 2020, after the index publication date, the market 5Y ZCIIS would start on 15 Dec 2020 and end on 15 Dec 2025 and reference the Q3 inflation index value. On the release date, i.e. 28 Oct 2020, the market ZCIIS that is set up is determined by whether the \lstinline!PublicationRoll! value is \lstinline!OnPublicationDate! or \lstinline!AfterPublicationDate!. If it is set to \lstinline!OnPublicationDate!, the swap rolls on this date and hence the market 5Y ZCIIS would start on 15 Dec 2020 and end on 15 Dec 2025 and reference the Q3 inflation index value. If it is set to \lstinline!AfterPublicationDate!, the swap does not roll on the publication date and instead rolls on the next day, and hence the market 5Y ZCIIS would start on 15 Sep 2020 and end on 15 Sep 2025 and reference the Q2 inflation index value. The publication schedule for the index must be provided in the \lstinline!PublicationSchedule! node if \lstinline!PublicationRoll! is not \lstinline!None!. An example of the AU CPI conventions set up is given in Listing \ref{lst:aucpi_inflation_conventions}.

\item \lstinline!PublicationSchedule!:
This is an optional node and is not used if \lstinline!PublicationRoll! is \lstinline!None!. If \lstinline!PublicationRoll! is not \lstinline!None!, it must be provided and gives the publication dates for the inflation index. The node fields are the same fields that are described in the Section \ref{ss:schedule_data}, i.e.\ they are \lstinline!ScheduleData! elements. An example of the AU CPI conventions set up is given in Listing \ref{lst:aucpi_inflation_conventions}. The \lstinline!PublicationSchedule! must cover the dates on which you intend to perform valuations, i.e. the first publication schedule date must be less than the smallest valuation date that you intend to use and the last publication schedule date must be greater than the largest valuation date that you intend to use.

\end{itemize}

\begin{listing}[H]
\begin{minted}[fontsize=\footnotesize]{xml}
<InflationSwap>
  <Id>AUCPI_INFLATIONSWAP</Id>
  <FixCalendar>AUD</FixCalendar>
  <FixConvention>F</FixConvention>
  <DayCounter>30/360</DayCounter>
  <Index>AUCPI</Index>
  <Interpolated>false</Interpolated>
  <ObservationLag>3M</ObservationLag>
  <AdjustInflationObservationDates>false</AdjustInflationObservationDates>
  <InflationCalendar>AUD</InflationCalendar>
  <InflationConvention>F</InflationConvention>
  <PublicationRoll>AfterPublicationDate</PublicationRoll>
  <PublicationSchedule>
    <Rules>
      <StartDate>2001-01-24</StartDate>
      <EndDate>2030-01-30</EndDate>
      <Tenor>3M</Tenor>
      <Calendar>AUD</Calendar>
      <Convention>Preceding</Convention>
      <TermConvention>Unadjusted</TermConvention>
      <Rule>LastWednesday</Rule>
    </Rules>
  </PublicationSchedule>
</InflationSwap>
\end{minted}
\caption{AU CPI inflation swap conventions}
\label{lst:aucpi_inflation_conventions}
\end{listing}

%- - - - - - - - - - - - - - - - - - - - - - - - - - - - - - - - - - - - - - - -
\subsubsection{CMS Spread Option Conventions}
%- - - - - - - - - - - - - - - - - - - - - - - - - - - - - - - - - - - - - - - -

A node with name \emph{CmsSpreadOption} is used to store the conventions.

\begin{listing}[H]
%\hrule\medskip
\begin{minted}[fontsize=\footnotesize]{xml}
  <CmsSpreadOption>
    <Id>EUR-CMS-10Y-2Y-CONVENTION</Id>
    <ForwardStart>0M</ForwardStart>
    <SpotDays>2D</SpotDays>
    <SwapTenor>3M</SwapTenor>
    <FixingDays>2</FixingDays>
    <Calendar>TARGET</Calendar>
    <DayCounter>A360</DayCounter>
    <RollConvention>MF</RollConvention>
  </CmsSpreadOption>
\end{minted}
\caption{Inflation swap conventions}
\label{lst:cms_spread_option_conventions}
\end{listing}

The meaning of the elements is as follows:

\begin{itemize}
\item ForwardStart: The calendar for the fixed rate leg of the swap.
\item SpotDays: The number of business days to spot for the CMS Spread Index.
\item SwapTenor: The frequency of payments on the CMS Spread leg.
\item FixingDays: The number of fixing days.
\item Calendar: The calendar for the CMS Spread leg.
\item DayCounter: The day counter for the CMS Spread leg.
\item RollConvention: The rolling convention for the CMS Spread Leg.
\end{itemize}

%- - - - - - - - - - - - - - - - - - - - - - - - - - - - - - - - - - - - - - - -
\subsubsection{Ibor Index Conventions}
%- - - - - - - - - - - - - - - - - - - - - - - - - - - - - - - - - - - - - - - -

A node with name \emph{IborIndex} is used to store conventions for Ibor indices. This can be used to define new Ibor
indices without the need of adding them to the C++ code, or also to override the conventions of existing Ibor indices.

\begin{listing}[H]
%\hrule\medskip
\begin{minted}[fontsize=\footnotesize]{xml}
  <IborIndex>
    <Id>EUR-EURIBOR_ACT365-3M</Id>
    <FixingCalendar>TARGET</FixingCalendar>
    <DayCounter>A365F</DayCounter>
    <SettlementDays>2</SettlementDays>
    <BusinessDayConvention>MF</BusinessDayConvention>
    <EndOfMonth>true</EndOfMonth>
  </IborIndex>
\end{minted}
\caption{Ibor index convention}
\label{lst:ibor_index_conventions}
\end{listing}

The meaning of the elements is as follows:

\begin{itemize}
\item Id: The index name. This must be of the form ``CCY-NAME-TENOR'' with a currency ``CCY'', an index name ``NAME''
  and a string ``TENOR'' representing a period. The name should not be ``GENERIC'', since this is reserved.
\item FixingCalendar: The fixing calendar of the index.
\item DayCounter: The day count convention used by the index.
\item SettlementDays: The settlement days for the index. This must be a non-negative whole number.
\item BusinessDayConvention: The business day convention used by the index.
\item EndOfMonth: A flag indicating whether the index employs the end of month convention.
\end{itemize}

Notice that if another convention depends on an Ibor index convention (because it contains the Ibor index name defined
in the latter convention), the Ibor index convention must appear before the convention that depends on it in the
convention input file.

Also notice that customised indices can not be used in cap / floor volatility surface configurations.

%- - - - - - - - - - - - - - - - - - - - - - - - - - - - - - - - - - - - - - - -
\subsubsection{Overnight Index Conventions}
%- - - - - - - - - - - - - - - - - - - - - - - - - - - - - - - - - - - - - - - -

A node with name \emph{OvernightIndex} is used to store conventions for Overnight indices. This can be used to define
new Overnight indices without the need of adding them to the C++ code, or also to override the conventions of existing
Overnight indices.

\begin{listing}[H]
%\hrule\medskip
\begin{minted}[fontsize=\footnotesize]{xml}
  <OvernightIndex>
    <Id>EUR-ESTER</Id>
    <FixingCalendar>TARGET</FixingCalendar>
    <DayCounter>A360</DayCounter>
    <SettlementDays>0</SettlementDays>
  </OvernightIndex>
\end{minted}
\caption{Overnight index convention}
\label{lst:overnight_index_conventions}
\end{listing}

The meaning of the elements is as follows:

\begin{itemize}
\item Id: The index name. This must be of the form ``CCY-NAME'' with a currency ``CCY'' and an index name ``NAME''. The
  name should not be ``GENERIC'', since this is reserved.
\item FixingCalendar: The fixing calendar of the index.
\item DayCounter: The day count convention used by the index.
\item SettlementDays: The settlement days for the index. This must be a non-negative whole number.
\end{itemize}

Notice that if another convention depends on an Overnight index convention (because it contains the Overnight index name
defined in the latter convention), the Overnight index convention must appear before the convention that depends on it
in the convention input file.

Also notice that customised indices can not be used in cap / floor volatility surface configurations.

\subsubsection{Inflation Index Conventions}
A node with the name \lstinline!ZeroInflationIndex! is used to store data for the creation of a new inflation index. This avoids having to add the index definition to the C++ code and recompile. Note that the \lstinline!ZeroInflationIndex! node should be placed before its use in any other convention, e.g.\ in an \lstinline!InflationSwap! convention, to avoid an error due to the new index itself not being created. If the \lstinline!Id! node matches an existing inflation index, the newly created index will take precedence and its defintion will be used in the code for the given \lstinline!Id!.

\begin{listing}[H]
\begin{minted}[fontsize=\footnotesize]{xml}
<ZeroInflationIndex>
  <Id>...</Id>
  <RegionName>...</RegionName>
  <RegionCode>...</RegionCode>
  <Revised>...</Revised>
  <Frequency>...</Frequency>
  <AvailabilityLag>...</AvailabilityLag>
  <Currency>...</Currency>
</ZeroInflationIndex>
\end{minted}
\caption{\emph{ZeroInflationIndex} node}
\label{lst:zero_inflation_index_conventions}
\end{listing}

The meaning of each element is as follows:
\begin{itemize}
\item \lstinline!Id!: The new inflation index name.
\item \lstinline!RegionName!: The name of the region with which the inflation index is associated.
\item \lstinline!RegionCode!: A code for the region with which the inflation index is associated.
\item \lstinline!Revised!: A boolean flag indicating whether the index is a revised index or not. This is generally set to \lstinline!false! but is left as an option to align with the C++ \lstinline!InflationIndex! class definition.
\item \lstinline!Frequency!: A valid frequency indicating the publication frequency of the inflation index, generally \lstinline!Monthly!, \lstinline!Quarterly! or \lstinline!Annual!.
\item \lstinline!AvailabilityLag!: A valid period indicating the lag between the inflation index publication for a given period and the period itself. For example, if March's inflation index value is published in April, the \lstinline!AvailabilityLag! would be \lstinline!1M!.
\item \lstinline!Currency!: The ISO currency code of the currency associated with the inflation index, generally the currency of the region.
\end{itemize}

%- - - - - - - - - - - - - - - - - - - - - - - - - - - - - - - - - - - - - - - -
\subsubsection{Swap Index Conventions}
%- - - - - - - - - - - - - - - - - - - - - - - - - - - - - - - - - - - - - - - -

A node with name \emph{SwapIndex} is used to store conventions for Swap indices (also known as ``CMS'' indices).

\begin{listing}[H]
%\hrule\medskip
\begin{minted}[fontsize=\footnotesize]{xml}
  <SwapIndex>
    <Id>EUR-CMS-2Y</Id>
    <Conventions>EUR-EURIBOR-6M-SWAP</Conventions>
    <FixingCalendar>TARGET</FixingCalendar>
  </SwapIndex>
\end{minted}
\caption{Swap index convention}
\label{lst:swap_index_conventions}
\end{listing}

The meaning of the elements is as follows:

\begin{itemize}
\item Id: The index name. This must be of the form ``CCY-CMS-TENOR'' with a currency ``CCY'' and a string ``TENOR''
  representing a period. The index name can contain an optional tag ``CCY-CMS-TAG-TENOR'' which is an arbitrary label
  that allows to define more than one swap index per currency.
\item Conventions: A swap convention defining the index conventions.
\item FixingCalendar [Optional]: The fixing calendar for the swap index fixings publication. If not given, the fixed leg
  calendar from the swap conventions will be used as a fall back.
\end{itemize}

%- - - - - - - - - - - - - - - - - - - - - - - - - - - - - - - - - - - - - - - - 
\subsubsection{FX Option Conventions}\label{sss:fx_option_conv}
%- - - - - - - - - - - - - - - - - - - - - - - - - - - - - - - - - - - - - - - - 
A node with name \emph{FxOption} is used to store conventions for FX option quotes for a given currency pair. The 
structure of this node is shown in Listing \ref{lst:fx_option_conventions}. 
 
\begin{listing}[H] 
%\hrule\medskip 
\begin{minted}[fontsize=\footnotesize]{xml}
<FxOption>
  <Id>EUR-USD-FXOPTION</Id>
  <FXConventionID>EUR-USD-FX</FXConventionID>
  <AtmType>AtmDeltaNeutral</AtmType>
  <DeltaType>Spot</DeltaType>
  <SwitchTenor>2Y</SwitchTenor>
  <LongTermAtmType>AtmDeltaNeutral</LongTermAtmType>
  <LongTermDeltaType>Fwd</LongTermDeltaType>
  <RiskReversalInFavorOf>Call</RiskReversalInFavorOf>
  <ButterflyStyle>Broker</ButterflyStyle>
</FxOption>
\end{minted} 
\caption{FX option conventions} 
\label{lst:fx_option_conventions} 
\end{listing} 
 
 
The meanings of the various elements in this node are as follows: 
\begin{itemize}
\item FXConventionID: The FX convention for the currency pair (see \ref{sss:fx_convention}). Optional, if not given, the
  FX spot days default to $2$ and the advance calendar defaults to source ccy + target ccy default calendars.
\item AtmType: Convention of ATM option quote (Choices are {\em AtmNull, AtmSpot, AtmFwd, 
AtmDeltaNeutral, AtmVegaMax, AtmGammaMax, AtmPutCall50}). 
\item DeltaType: Convention of Delta option quote (Choices are {\em Spot, Fwd, PaSpot, 
    PaFwd}).
\item SwitchTenor [Optional]: If given, different ATM and Delta conventions will be used if the option tenor is greater
  or equal the switch tenor (``long term'' atm and delta type)
\item LongTermAtmType [Mandatory if and only if SwitchTenor is given]: ATM type to use for options with tenor > switch
  point, if SwitchTenor is given
\item LongTermDeltaType [Mandatory if and only if SwitchTenor is given]: Delta type to use for options with tenor >
  switch point, if SwitchTenor is given
\item RiskReversalInFavorOf [Optional]: Call (default), Put. Only relevant for BF, RR market data input.
\item ButterflyStyle [Optional]: Broker (default), Smile. Only relevant for BF, RR market data input.
\end{itemize} 

%- - - - - - - - - - - - - - - - - - - - - - - - - - - - - - - - - - - - - - - -
\subsubsection{Commodity Forward Conventions}
%- - - - - - - - - - - - - - - - - - - - - - - - - - - - - - - - - - - - - - - -
A node with name \lstinline!CommodityForward! is used to store conventions for commodity forward price quotes. The
structure of this node is shown in Listing \ref{lst:commodity_forward_conventions}.

\begin{listing}[H]
\begin{minted}[fontsize=\footnotesize]{xml}
<CommodityForward>
  <Id>...</Id>
  <SpotDays>...</SpotDays>
  <PointsFactor>...</PointsFactor>
  <AdvanceCalendar>...</AdvanceCalendar>
  <SpotRelative>...</SpotRelative>
  <BusinessDayConvention>...</BusinessDayConvention>
  <Outright>...</Outright>
</CommodityForward>
\end{minted}
\caption{Commodity forward conventions}
\label{lst:commodity_forward_conventions}
\end{listing}

The meanings of the various elements in this node are as follows:
\begin{itemize}
\item \lstinline!Id!: The identifier for the commodity forward convention. The identifier here should match the \lstinline!Name! that would be provided for the commodity in the trade XML as described in Table \ref{tab:commodity_data}.
\item \lstinline!SpotDays! [Optional]: The number of business days to spot for the commodity. Any non-negative integer is allowed here. If omitted, this takes a default value of 2.
\item \lstinline!PointsFactor! [Optional]: This is only used if \lstinline!Outright! is \lstinline!false!. Any positive real number is allowed here. When \lstinline!Outright! is \lstinline!false!, the commodity forward quotes are provided as points i.e. a number that should be added to the commodity spot to give the outright commodity forward rate. The \lstinline!PointsFactor! is the number by which the points quote should be divided before adding it to the spot quote to obtain the forward price. If omitted, this takes a default value of 1.
\item \lstinline!AdvanceCalendar! [Optional]: The business day calendar(s) used for advancing dates for both spot and forwards. The allowable values are given in Table \ref{tab:calendar}. If omitted, it defaults to \lstinline!NullCalendar! i.e. a calendar where all days are considered good business days.
\item \lstinline!SpotRelative! [Optional]: The allowable values are \lstinline!true! and \lstinline!false!. If \lstinline!true!, the forward tenor is interpreted as being relative to the spot date. If \lstinline!false!, the forward tenor is interpreted as being relative to the valuation date. If omitted, it defaults to \lstinline!True!.
\item \lstinline!BusinessDayConvention! [Optional]: The business day roll convention used to adjust dates when getting from the valuation date to the spot date and the forward maturity date. The allowable values are given in Table \ref{tab:allow_stand_data}. If omitted, it defaults to \lstinline!Following!.
\item \lstinline!Outright! [Optional]: The allowable values are \lstinline!true! and \lstinline!false!. If \lstinline!true!, the forward quotes are interpreted as outright forward prices. If \lstinline!false!, the forward quotes are interpreted as points i.e. as a number that must be added to the spot price to get the outright forward price. If omitted, it defaults to \lstinline!true!.
\end{itemize}

\subsubsection{Commodity Future Conventions}
\label{sec:commodity_future_conventions}
A node with name \lstinline!CommodityFuture! is used to store conventions for commodity future contracts and options on them. These conventions are used in commodity derivative trades and commodity curve construction to calculate contract expiry dates. The structure of this node is shown in Listing \ref{lst:commodity_future_conventions}.

\begin{listing}[h!]
\begin{minted}[fontsize=\footnotesize,breaklines]{xml}
<CommodityFuture>
  <Id>...</Id>
  <AnchorDay>
    ...
  </AnchorDay>
  <ContractFrequency>...</ContractFrequency>
  <Calendar>...</Calendar>
  <ExpiryCalendar>...</ExpiryCalendar>
  <ExpiryMonthLag>...</ExpiryMonthLag>
  <OneContractMonth>...</OneContractMonth>
  <OffsetDays>...</OffsetDays>
  <BusinessDayConvention>...</BusinessDayConvention>
  <AdjustBeforeOffset>...</AdjustBeforeOffset>
  <IsAveraging>...</IsAveraging>
  <OptionExpiryOffset>...</OptionExpiryOffset>
  <ProhibitedExpiries>
    <Dates>
      <Date forFuture="true" convention="Preceding" forOption="true" optionConvention="Preceding">...</Date>
        ...
    </Dates>
  </ProhibitedExpiries>
  <OptionExpiryMonthLag>...</OptionExpiryMonthLag>
  <OptionExpiryDay>...</OptionExpiryDay>
  <OptionContractFrequency>...</OptionContractFrequency>
  <OptionNthWeekday>
    <Nth>...</Nth>
    <Weekday>...</Weekday>
  </OptionNthWeekday>
  <OptionExpiryLastWeekdayOfMonth>...</OptionExpiryLastWeekdayOfMonth>
  <OptionExpiryWeeklyDayOfTheWeek>...</OptionExpiryWeeklyDayOfTheWeek>
  <OptionBusinessDayConvention>...</OptionBusinessDayConvention>
  <FutureContinuationMappings>
    <ContinuationMapping>
      <From>...</From>
      <To>...</To>
    </ContinuationMapping>
    ...
  </FutureContinuationMappings>
  <OptionContinuationMappings>
    <ContinuationMapping>
      <From>...</From>
      <To>...</To>
    </ContinuationMapping>
    ...
  </OptionContinuationMappings>
  <AveragingData>
    ...
  </AveragingData>
  <HoursPerDay>...</HoursPerDay>
  <SavingsTime>...<SavingsTime>
  <ValidContractMonths>
  	<Month>...</Month>
  </ValidContractMonths>
  <OptionUnderlyingFutureConvention>...</OptionUnderlyingFutureConvention>
</CommodityFuture>
\end{minted}
\caption{Commodity future conventions}
\label{lst:commodity_future_conventions}
\end{listing}

The meanings of the various elements in this node are as follows:
\begin{itemize}
\item \lstinline!Id!: The identifier for the commodity future convention. The identifier here should match the \lstinline!Name! that would be provided for the commodity in the trade XML as described in Table \ref{tab:commodity_data}.
\item \lstinline!AnchorDay! [Optional]: This node is not applicable for daily future contracts and hence is optional. It is necessary for future contracts with a monthly cycle or greater or if the option contracts cycle is monthly or greater.  This node is used to give a date in the future contract month to use as a base date for calculating the expiry date. It can contain a \lstinline!DayOfMonth! node, a \lstinline!CalendarDaysBefore! node or an \lstinline!NthWeekday! node:
    \begin{itemize}
    \item The \lstinline!DayOfMonth! This node can contain any integer in the range $1,\ldots,31$ indicating the day of the month. A value of 31 will guarantee that the last day in the month is used a base date.
    \item The \lstinline!CalendarDaysBefore! This node can contain any non-negative integer. The contract expiry date is this number of calendar days before the first calendar day of the contract month.
    \item The \lstinline!NthWeekday! This node has the elements shown in Listing \ref{lst:nth_weekday_node}. This node is used to indicate a date in a given month in the form of the n-th named weekday of that month e.g. 3rd Wednesday. The allowable values for \lstinline!Nth! are ${1,2,3,4}$. The \lstinline!Weekday! node takes a weekday in the form of the first three characters of the weekday with the first character capitalised.
    \item The \lstinline!LastWeekday! [Optional]: This node is used to indicate a date in a given month in the form of the last named weekday of that month e.g. last Wednesday. The node takes a weekday in the form of the first three characters of the weekday with the first character capitalised.
    \item The \lstinline!BusinessDaysAfter! This node can contain any integer. If the number is positive the contract expiry is the n-th business day of the contract month. If the number is negative the contract expiry date is this number of business days before the first calendar day of the contract month.
    \item The \lstinline!WeeklyDayOfTheWeek! [Optional]: This node is used to indicate a date in a given week in the form of the named weekday, e.g. Wednesday. This node is mandatory for weekly contract frequencies and is not allowed with any other frequency.  The node takes a weekday in the form of the first three characters of the weekday with the first character capitalised.
    \end{itemize}
\item \lstinline!ContractFrequency!: This node indicates the frequency of the commodity future contracts. The value here is usually \lstinline!Monthly! or \lstinline!Quarterly!, but allowed values are \lstinline!Daily!, \lstinline!Weekly!, \lstinline!Monthly!, \lstinline!Quaterly! and \lstinline!Annual!.
\item \lstinline!Calendar!: The business day trading calendar(s) applicable for the commodity future contract.
\item \lstinline!ExpiryCalendar! [Optional]: The business day expiry calendar(s) applicable for the commodity future contract. This calendar is used when deriving expiry dates. If omitted, this defaults to the trading day calendar specified in the \lstinline!Calendar! node.
\item \lstinline!ExpiryMonthLag! [Optional]: The allowable values are any integer. This value indicates the number of months from the month containing the expiry date to the contract month. If 0, the commodity future contract expiry date is in the contract month. If the value of \lstinline!ExpiryMonthLag! is $n > 0$, the commodity future contract expires in the $n$-th month prior to the contract month. If the value of \lstinline!ExpiryMonthLag! is $n < 0$, the commodity future contract expires in the $n$-th month after the contract month. The value of \lstinline!ExpiryMonthLag! is generally 0, 1 or 2. For example, \lstinline!NYMEX:CL! has an \lstinline!ExpiryMonthLag! of 1 and \lstinline!ICE:B! has an \lstinline!ExpiryMonthLag! of 2. If omitted, it defaults to 0.
\item \lstinline!OneContractMonth! [Optional]: This node takes a calendar month in the form of the first three characters of the month with the first character capitalised. The month provided should be an arbitrary valid future contract month. It is used in cases where the \lstinline!ContractFrequency! is not \lstinline!Monthly! in order to determine the valid contract months. If omitted, it defaults to January.
\item \lstinline!OffsetDays! [Optional]: The number of business days that the expiry date is before the base date where the base date is implied by the \lstinline!AnchorDay! node above. Any non-negative integer is allowed here. If omitted, this takes a default value of zero.
\item \lstinline!BusinessDayConvention! [Optional]: The business day roll convention used to adjust the expiry date. The allowable values are given in Table \ref{tab:allow_stand_data}. If omitted, it defaults to \lstinline!Preceding!.
\item \lstinline!AdjustBeforeOffset! [Optional]: The allowable values are \lstinline!true! and \lstinline!false!. If \lstinline!true!, if the base date implied by the \lstinline!AnchorDay! node above is not a good business day according to the calendar provided in the \lstinline!Calendar! node, this date is adjusted before the offset specified in the \lstinline!OffsetDays! is applied. If \lstinline!false!, this adjustment does not happen. If omitted, it defaults to \lstinline!true!. 
\item \lstinline!IsAveraging! [Optional]: The allowable values are \lstinline!true! and \lstinline!false!. This node indicates if the future contract is based on the average commodity price of the contract period. If omitted, it defaults to \lstinline!false!.
\item \lstinline!OptionExpiryOffset! [Optional]: The number of business days that the option expiry date is before the future expiry date. Any non-negative integer is allowed here. If omitted, this takes a default value of zero and the expiry date of an option on the future contract is assumed to equal the expiry date of the future contract.
\item \lstinline!ProhibitedExpiries! [Optional]: This node can be used to specify explicit dates which are not allowed as future contract expiry dates or as option expiry dates. A useful example of this is the ICE Brent contract which has the following constraint on expiry dates: \emph{If the day on which trading is due to cease would be either: (i) the Business Day preceding Christmas Day, or (ii) the Business Day preceding New Year’s Day, then trading shall cease on the next preceding Business Day}. Each \lstinline!Date! node can take optional attributes. The default values of these attributes is shown in Listing \ref{lst:commodity_future_conventions}. The \lstinline!convention! attribute accepts a valid business day convention in the list \lstinline!Preceding!, \lstinline!ModifiedPreceding!, \lstinline!Following! and \lstinline!ModifiedFollowing!. This \lstinline!convention! indicates how the future expiry date should be adjusted if it lands on the prohibited expiry \lstinline!Date!. If omitted, the default is \lstinline!Preceding!. Both \lstinline!Preceding! and \lstinline!ModifiedPreceding! indicate that the next available business day before the date is tested. \lstinline!Following! and \lstinline!ModifiedFollowing! indicate that the next available business day after the date is tested. The \lstinline!optionConvention! attribute allows the same values and behaves in the same way to determine how the option expiry date should be adjusted if it lands on the prohibited expiry \lstinline!Date!. The \lstinline!forFuture! and \lstinline!forOption! boolean attributes enable the prohibited expiry to apply only for the future expiry date or the option expiry date respectively by setting the value to \lstinline!false!.
\item \lstinline!OptionExpiryMonthLag! [Optional]: The allowable values are any integer. This value indicates the number of months from the month containing the option expiry date to the month containing the expiry date. If 0, the commodity future option contract expiry date is anchored in the same month as the commodity future contract expiry date. If the value of \lstinline!OptionExpiryMonthLag! is $n > 0$, the commodity option future contract expires in the $n$-th month prior to the commodity future contract expiry month. If the value of \lstinline!OptionExpiryMonthLag! is $n < 0$, the commodity option future contract expires in the $n$-th month after the commodity future contract expiry month. The value of \lstinline!OptionExpiryMonthLag! should be equal to \lstinline!ExpiryMonthLag! when \lstinline!OptionExpiryOffset! is used. The \lstinline!OptionExpiryMonthLag! is rarely used. An example is the Crude Palm Oil contract \lstinline!XKLS:FCPO! where the future contract expiry is in the delivery month and the option expiry is in the month that is 2 months prior to this. In this case, \lstinline!OptionExpiryMonthLag! is 2. If omitted, \lstinline!OptionExpiryMonthLag! defaults to 0.
\item \lstinline!OptionExpiryDay! [Optional]: This node can contain any integer in the range $1,\ldots,31$ indicating the day of the month on which an option expiry date is anchored. A value of 31 will guarantee that the last day in the month is used a base date. If omitted, this is not used. Setting this field takes precedence over \lstinline!OptionExpiryOffset!.\item \lstinline!OptionBusinessDayConvention! [Optional]: The business day convention used to adjust the option expiry date to a good business day if \lstinline!OptionExpiryDay! is used.
\item \lstinline!OptionContractFrequency! [Optional]: This node indicates the frequency of the commodity future options if it differs from the frequency of the underlying future contract. The value here is usually \lstinline!Monthly!
\item \lstinline!OptionNthWeekday! [Optional]: This node has the elements shown in Listing \ref{lst:nth_weekday_node}. This node is used to indicate a date in a given month in the form of the n-th named weekday of that month e.g. 3rd Wednesday. The allowable values for \lstinline!Nth! are ${1,2,3,4}$. The \lstinline!Weekday! node takes a weekday in the form of the first three characters of the weekday with the first character capitalised.
\item \lstinline!OptionBusinessDayConvention! [Optional]: The business day convention used to adjust the option expiry date to a good business day if \lstinline!OptionExpiryDay! is used.
\item \lstinline!OptionExpiryLastWeekdayOfMonth! [Optional]: This node is used to indicate a date in a given month in the form of the last named weekday of that month e.g. last Wednesday. The node takes a weekday in the form of the first three characters of the weekday with the first character capitalised.
\item \lstinline!OptionExpiryWeeklyDayOfTheWeek! [Optional]: This node is used to indicate a date in a given week in the form of the named weekday, e.g. Wednesday. The node takes a weekday in the form of the first three characters of the weekday with the first character capitalised. This node is mandatory for weekly expiring options. The node is not allowed to use with any other option contract frequency.
\item \lstinline!OptionUnderlyingFutureConvention! [Optional]: Sometimes the next contract expiry, as specified in the convention, is not the correct option underlying. For example the base metals options expiries on the 1st Wednesday of the contract month, and during the first 3 months there are daily future contracts available. The option underlying is not the future contract which matures on the option expiry but the one which matures on the 3rd Wednesday of the month. This field is referencing to an commodity future convention which specifies the correct expiry date for the underlying contract.
\item \lstinline!FutureContinuationMappings! [Optional]: When building future curves, we may use market data that has a continuation expiry, i.e. \lstinline!c1!, \lstinline!c2!, etc. , as opposed to an explicit expiry date or tenor. In some cases, the continuation expiries coming from the market data provider may skip serial months and therefore we use the mapping here to map from the market data provider index to the relevant serial month.
\item \lstinline!OptionContinuationMappings! [Optional]: When building option volatility structures, we may use market data that has a continuation expiry, i.e. \lstinline!c1!, \lstinline!c2!, etc. , as opposed to an explicit expiry date or tenor. In some cases, the continuation expiries coming from the market data provider may skip serial months and therefore we use the mapping here to map from the market data provider index to the relevant serial month. For example, for the Crude Palm Oil contract \lstinline!XKLS:FCPO!, the option expiry months are serial up to the 9th month and then alternate months. So, we would add a mapping from 10 to 11, 11 to 13 and so on so that the correct option expiry is arrived at when reading the market data quotes and constructing the option volatility structure.
\item \lstinline!AveragingData! [Optional]: This node is needed for future contracts that are used in a piecewise commodity curve \lstinline!PriceSegment! and whose underlying is the average of other future prices or spot prices over a given period. An example is the ICE PMI power contract with contract specifications outlined \href{https://www.theice.com/products/6590369/PJM-Western-Hub-Real-Time-Peak-1-MW-Fixed-Price-Future}{here}. It is described in detail below.
\item \lstinline!HoursPerDay! [Optional]: For power derivatives, quantities are sometimes given as a quantity per hour. To deduce the quantity for the day which is multiplied by that day's future price, one needs to know the number of hours in the day associated with the future price. For example ICE PDQ is the daily PJM Western Hub Real Time Peak future contract. The price each day for this contract is the average of the locational marginal prices (LMPs) for all hours ending 08:00 to 23:00 Eastern Pacific Time. In other words, there are 16 hours in the day that feed in to the average yielding this settlement price. For this contract, \lstinline!HoursPerDay! would be \lstinline!16!. This field is only needed if a trade XML references this commodity contract, has \lstinline!CommodityQuantityFrequency! set to \lstinline!PerHour! and has no \lstinline!HoursPerDay! value set directly in the XML.
\item \lstinline!SavingsTime! [Optional]: For some derivatives, quantities are given as quantity per calendar day and hour. The monthly quantity is then scaled by the number of calendar days times hours per day (see above) plus or minus a daylight savings correction. To compute the daylight savings correction a convention is needed that describes the dates on which dates one hour is gained resp. lost. Currently supported conventions are US, Null. Default is US if no convention is given.
\item \lstinline!ValidContractMonths! [Optional]: For some commodities the contract frequency is almost monthly but for some calendar months there are no contracts listed. For example Corn Futures are only listed for the expiry months March, May, July, September and December. For those contracts the \emph{ContractFrequency} need to be set to \emph{Monthly} and the valid months have to be added to this node. This node is ignored for all other frequencies and if its omitted all calendar months are valid.
\end{itemize}

\begin{listing}[h!]
\begin{minted}[fontsize=\footnotesize]{xml}
<NthWeekday>
  <Nth>...</Nth>
  <Weekday>...</Weekday>
</NthWeekday>
\end{minted}
\caption{\textnormal{\lstinline!NthWeekday!} node outline}
\label{lst:nth_weekday_node}
\end{listing}

An example \lstinline!CommodityFuture! node for the NYMEX WTI future contract, specified \href{https://www.cmegroup.com/trading/energy/crude-oil/light-sweet-crude_contract_specifications.html}{here}, is provided in Listing \ref{lst:ex_wti_comm_future_convention}.

\begin{listing}[h!]
\begin{minted}[fontsize=\footnotesize]{xml}
<CommodityFuture>
  <Id>NYMEX:CL</Id>
  <AnchorDay>
    <DayOfMonth>25</DayOfMonth>
  </AnchorDay>
  <ContractFrequency>Monthly</ContractFrequency>
  <Calendar>US-NYSE</Calendar>
  <ExpiryMonthLag>1</ExpiryMonthLag>
  <OffsetDays>3</OffsetDays>
  <BusinessDayConvention>Preceding</BusinessDayConvention>
  <IsAveraging>false</IsAveraging>
</CommodityFuture>
\end{minted}
\caption{NYMEX WTI \textnormal{\lstinline!CommodityFuture!} node}
\label{lst:ex_wti_comm_future_convention}
\end{listing}

The \lstinline!AveragingData! node referenced above has the structure shown in Listing \ref{lst:ave_data_comm_future_convention}. The meaning of each of the fields is as follows:

\begin{itemize}
\item \lstinline!CommodityName!: The name of the commodity being averaged.
\item \lstinline!Conventions!: The identifier for the conventions associated with the commodity being averaged.
\item \lstinline!Period!: This indicates the averaging period relative to the future expiry date. The allowable values are:
    \begin{itemize}
    \item \lstinline!PreviousMonth!: The calendar month prior to the month in which the (top level) future contract's expiry date falls is used as the averaging period.
    \item \lstinline!ExpiryToExpiry!: Given a (top level) future contract's expiry date, the averaging period is from and excluding the previous expiry date to and including the expiry date.
    \end{itemize}
\item \lstinline!PricingCalendar!: The pricing calendar(s) used to determine the pricing dates in the averaging period.
\item \lstinline!UseBusinessDays! [Optional]: A boolean flag that defaults to \lstinline!true! if omitted. When set to \lstinline!true!, the pricing dates in the averaging period are the set of \lstinline!PricingCalendar! good business days. When set to \lstinline!false!, the pricing dates in the averaging period are the complement of the set of \lstinline!PricingCalendar! good business days. This may be useful in certain situations. For example, the contract ICE PW2 with specifications \href{https://www.theice.com/products/71090520/PJM-Western-Hub-Real-Time-Peak-2x16-Fixed-Price-Future}{here} averages the PJM Western Hub locational marginal prices over each day in the averaging period that is a Saturday, Sunday or NERC holiday. So, in this case, \lstinline!UseBusinessDays! would be \lstinline!false! and \lstinline!PricingCalendar! would be \lstinline!US-NERC!.
\item \lstinline!DeliveryRollDays! [Optional]: This node allows any non-negative integer value. When averaging a commodity future contract price over the averaging period, the averaging period may include an underlying future contract expiry date. This node's value indicates when we should begin using the next future contract's price in the averaging. If the value is zero, we should include the future contract prices up to and including the contract expiry. If the value is one, we should include the contract prices up to and including the day that is one business day before the contract expiry and then switch to using the next future contract's price thereafter. Similarly for other non-negative integer values. If this node is omitted, it is set to zero.
\item \lstinline!FutureMonthOffset! [Optional]: This node allows any non-negative integer value. If this node is omitted, it is set to zero. This node indicates which future contract is being referenced on each \textit{Pricing Date} in the averaging period by acting as an offset from the next available expiry date. If \lstinline!FutureMonthOffset! is zero, the settlement price of the next available monthly contract that has not expired with respect to the \textit{Pricing Date} is used as the price on that \textit{Pricing Date}. If \lstinline!FutureMonthOffset! is one, the settlement price of the second available monthly contract that has not expired with respect to the \textit{Pricing Date} is used as the price on that \textit{Pricing Date}. Similarly for other positive values of \lstinline!FutureMonthOffset!.
\item \lstinline!DailyExpiryOffset! [Optional]: This node allows any non-negative integer value. It should only be used where the \lstinline!CommodityName! being averaged has a daily contract frequency. If this node is omitted, it is set to zero. This node indicates which future contract is being referenced on each \textit{Pricing Date} in the averaging period by acting as a business day offset, using the \lstinline!CommodityName!'s expiry calendar, from the \textit{Pricing Date}. It is useful in the base metals market where the future contract being averaged on each \textit{Pricing Date} is the cash contract on that \textit{Pricing Date} i.e.\ the contract with expiry date two business days after the \textit{Pricing Date}.
\end{itemize}

\begin{listing}[h!]
\begin{minted}[fontsize=\footnotesize]{xml}
<AveragingData>
  <CommodityName>...</CommodityName>
  <Conventions>...</Conventions>
  <Period>...</Period>
  <PricingCalendar>...</PricingCalendar>
  <UseBusinessDays>...</UseBusinessDays>
  <DeliveryRollDays>...</DeliveryRollDays>
  <FutureMonthOffset>...</FutureMonthOffset>
  <DailyExpiryOffset>...</DailyExpiryOffset>
</AveragingData>
\end{minted}
\caption{\lstinline!AveragingData! node structure}
\label{lst:ave_data_comm_future_convention}
\end{listing}

\subsubsection{Credit Default Swap Conventions}
\label{sss:cds_conventions}
A node with name \lstinline!CDS! is used to store conventions for credit default swaps. The structure of this node is shown in Listing \ref{lst:cds_conventions}.

\begin{listing}[H]
\begin{minted}[fontsize=\footnotesize]{xml}
<CDS>
  <Id>...</Id>
  <SettlementDays>...</SettlementDays>
  <Calendar>...</Calendar>
  <Frequency>...</Frequency>
  <PaymentConvention>...</PaymentConvention>
  <Rule>...</Rule>
  <DayCounter>...</DayCounter>
  <SettlesAccrual>...</SettlesAccrual>
  <PaysAtDefaultTime>...</PaysAtDefaultTime>
</CDS>
\end{minted}
\caption{CDS conventions}
\label{lst:cds_conventions}
\end{listing}

The meanings of the various elements in this node are as follows:
\begin{itemize}

\item \lstinline!Id!:
The identifier for the CDS convention.

\item \lstinline!SettlementDays!:
The number of days after the CDS trade date when protection starts i.e.\ the \textit{Protection effective date} or \textit{step-in date}. Any non-negative integer is allowed here. For standard CDS after, this is generally set to 1.

\item \lstinline!Calendar!:
The calendar associated with the CDS. For non-JPY currencies, this is generally \lstinline!WeekendsOnly! to agree with the ISDA standard. For JPY CDS, the ISDA standard calendar is \lstinline!TYO! documented at \url{https://www.cdsmodel.com/cdsmodel}. This could be set up as an additional calendar or \lstinline!JPN! could be used as a proxy. Allowable calendar values are given in Table \ref{tab:calendar}.

\item \lstinline!Frequency!:
The frequency of fee leg payments for the CDS. The ISDA standard is \lstinline!Quarterly! but any valid frequency is allowed.

\item \lstinline!PaymentConvention!:
The business day convention for payments on the CDS. The ISDA standard is \lstinline!Following! but any valid business day convention from Table \ref{tab:allow_stand_data} is allowed.

\item \lstinline!Rule!:
The date generation rule for the fee leg on the CDS. The ISDA standard is \lstinline!CDS2015! but any valid date generation rule is allowed.

\item \lstinline!DayCounter!:
The day counter for fee leg payments on the CDS. The ISDA standard is \lstinline!A360! but any valid day counter from Table \ref{tab:daycount} is allowed.

\item \lstinline!SettlesAccrual!:
A boolean value indicating if an accrued fee is due on the occurrence of a credit event. Allowable boolean values are given in the Table \ref{tab:boolean_allowable}. In general, this is set \lstinline!true!.

\item \lstinline!PaysAtDefaultTime!:
A boolean value indicating if the accrued fee, on the occurrence of a credit event, is payable at the credit event date or the end of the fee period. A value of \lstinline!true! indicates that the accrued is payable at the credit event date and a value of \lstinline!false! indicates that it is payable at the end of the fee period. In general, this is set \lstinline!true!.

\end{itemize}



%- - - - - - - - - - - - - - - - - - - - - - - - - - - - - - - - - - - - - - - -
\subsubsection{Bond Yield Conventions}
%- - - - - - - - - - - - - - - - - - - - - - - - - - - - - - - - - - - - - - - -
A node with name \lstinline!BondYield! is used to store conventions for the conversion
of bond prices into bond yields.
The structure of this node is shown in Listing \ref{lst:bondyield_conventions}.

\begin{listing}[H]
\begin{minted}[fontsize=\footnotesize]{xml}
<BondYield>
  <Id>CMB-DE-BUND-10Y</Id>
  <Compounding>Compounded</Compounding>
  <Frequency>Annual</Frequency>
  <PriceType>Clean</PriceType>
  <Accuracy>1.0e-8</Accuracy>
  <MaxEvaluations>100</MaxEvaluations>
  <Guess>0.05</Guess>
</BondYield>
\end{minted}
\caption{Bond yield conventions}
\label{lst:bondyield_conventions}
\end{listing}

The meaning of the elements is as follows:

\begin{itemize}
\item Id: The constant maturity index index name. This must be of the form ``CMB-FAMILY-TENOR'' where FAMILY can consist of any number of tags separated by ``-''
\item Compounding: Compounding of the yield - Simple, Compounded, Continuous, SimpleThenCompounded
\item Frequency: Frequency of the cash flows - Annual, Semiannual, Quarterly, Monthly etc.
\item PriceType: Dirty or Clean
\item Accuracy/MaxEvaluations/Guess: QuantLib parameters that control the convergence of the numerical price to yield conversion. 
\end{itemize}





%========================================================
\section{Trade Data}\label{sec:portfolio_data}
%========================================================

%========================================================
\section{Trade Data}\label{sec:portfolio_data}
%========================================================

The trades that make up the portfolio are specified in an XML file where the portfolio data is specified in a hierarchy
of nodes and sub-nodes.  The nodes containing individual trade data are referred to as elements or XML elements. These
are generally the lowest level nodes.

\vspace{1em}

The top level portfolio node is delimited by an opening {\tt <Portfolio>} and a closing {\tt </Portfolio>} tag. Within
the portfolio node, each trade is defined by a starting {\tt <Trade id="[Tradeid]">} and a closing {\tt </Trade>} tag.
Further, the trade type is set by the TradeType XML element. Each trade has an Envelope node that includes the same XML
elements for all trade types (Id, Type, Counterparty, Rating, NettingSetId) plus the Additional fields node, and after
that, a node containing trade specific data.

\vspace{1em}
An example of a {\tt portfolio.xml} file with one Swap trade including the full envelope node is shown in Listing \ref{lst:portfolio}.

\begin{listing}[H]
%\hrule\medskip
\begin{minted}[fontsize=\footnotesize]{xml}
<Portfolio>
  <Trade id="Swap#1">
    <TradeType> Swap </TradeType>
    <Envelope>
      <CounterParty> Counterparty#1 </CounterParty>
      <NettingSetId> NettingSet#2 </NettingSetId>
      <AdditionalFields>
        <Sector> SectorA </Sector>
        <Book> BookB </Book>
        <Rating> A1 </Rating>
      </AdditionalFields>
    </Envelope>
    <SwapData>
        ...
        [Trade specific data for a Swap]
        ...
    </SwapData>
  </Trade>
</Portfolio>
\end{minted}
\caption{Portfolio}
\label{lst:portfolio}
\end{listing}

A description of all portfolio data, i.e. of each node and XML element in the portfolio file, with examples and
allowable values follows below. There are two XML elements directly under the top level {\tt Portfolio} node:

\begin{itemize}
\item {\tt Trade id}: The first element of each trade is the {\tt Trade id} and it is used to identify trades within a
  portfolio. Trade ids should be unique within a portfolio.  The {\tt Trade id} element is entered twice in the
  instrument file, firstly with an attribute to the XML element {\tt <Trade>}, such as {\tt <Trade id="ExampleTrade">}
  in the beginning closed by {\tt </Trade>} at the end of the trade data, and secondly with a {\tt Id} element in the
  envelope node.  Both Trade id entries should be identical.

Allowable values:  Any alphanumeric string. The underscore (\_) sign may be used as well. 


\item {\tt TradeType}: %The Trade type is set with the {\tt TradeType} element, as well as with the {\tt Type} element in  the envelope. The two Trade type entries should be identical. 
ORE currently supports 11 trade types.

%Allowable values:  \emph{Swap, CapFloor, Swaption, FxForward, FxOption, Bond, CreditDefaultSwap}
Allowable values: \emph{ForwardRateAgreement, Swap, CapFloor, Swaption, FxForward, FxSwap, FxOption,
EquityForward, EquityOption, CreditDefaultSwap, Bond}

\end{itemize}

%- - - - - - - - - - - - - - - - - - - - - - - - - - - - - - - - - - - - - - - -
\subsection{Envelope}\label{ss:envelope}
%- - - - - - - - - - - - - - - - - - - - - - - - - - - - - - - - - - - - - - - -
The envelope node contains basic identifying details of a trade ({\tt
  Id}, {\tt Type}, {\tt Counterparty}, {\tt Rating}, {\tt
  NettingSetId}), plus an {\tt AdditionalFields} node where custom
elements can be added for informational purposes such as {\tt Book} or
{\tt Sector}. Beside the custom elements within the {\tt
  AdditionalFields} node, the envelope contains the same elements for
all Trade types.  The {\tt Id}, {\tt Type}, {\tt Counterparty} and
{\tt NettingSetId} elements must have non-blank entries for ORE to
run, whereas ORE will run without a {\tt Rating} element but fail to
produce a CVA. \\

The meanings and allowable values of the various elements in the \lstinline!Envelope!  node follow below.

\begin{itemize}
\item {\tt Id}: The {\tt Id} element in the envelope is used to identify trades within a portfolio. It should be set to
  identical values as the {\tt Trade id=" "} element.

  Allowable values: Any alphanumeric string. The underscore (\_) sign may be used as well.

%\item {\tt Type}: The Trade Type is in addition to being set in the {\tt ClassName} element, also set by the {\tt Type}
%  element in the envelope. Both elements should have the same entry.
%
%Allowable values: \emph{ForwardRateAgreement, Swap, CapFloor, Swaption, FxForward, FxSwap, FxOption,
%EquityForward, EquityOption, CreditDefaultSwap, Bond}

\item {\tt Counterparty}: Specifies the name of the counterparty of the trade.  It is used to show exposure analytics by
  counterparty.

Allowable values: Any alphanumeric string. Underscores (\_) and blank spaces may be used as well. 

\item {\tt Rating} [Optional]: The {\tt Rating} element specifies the default curve that will be used in the CVA
  calculations.  No CVA will be calculated if omitted or left blank.

Allowable values: An alphanumeric string that matches default curve names in the market configuration file.  

\item {\tt NettingSetId} [Optional]: The {\tt NettingSetId} element specifies the identifier for a netting set. If a
  \lstinline!NettingSetId! is specified, the trade is eligible for close-out netting under the terms of an associated
  ISDA agreement. The specified {\tt NettingSetId} must be defined within the netting set definitions file
  (see section \ref{sec:nettingsetinput}). If left blank or omitted the trade will not belong to any netting set, and thus not be
  eligible for netting.

Allowable values: Any alphanumeric string. Underscores (\_) and blank spaces may be used as well. 

% -- specified netting --Negative transaction values offset exposures from positive transaction values within specified
% netting groups. Total netting
% group %-- exposures and exposures from transactions not belonging to a netting group are then combined without offsetting.

\item \lstinline!AdditionalFields! [Optional]: The AdditionalFields node allows the insertion of additional trade
  information using custom XML elements.  For example, elements such as Sector, Desk or Folder can be used. The elements
  within the \lstinline!AdditionalFields! node are used for informational purposes only, and do not affect pricing or
  exposure calculations and analytics.

Allowable values: Any custom element.

\end{itemize}

%- - - - - - - - - - - - - - - - - - - - - - - - - - - - - - - - - - - - - - - -
\subsection{Trade Specific Data}
%- - - - - - - - - - - - - - - - - - - - - - - - - - - - - - - - - - - - - - - -

After the envelope node, trade-specific data for each trade type supported by
ORE is included. 
Each trade type has its own trade data container which is defined by an XML node containing a trade-specific
configuration of individual XML tags - called elements - and trade components. The trade components are defined by XML
sub-nodes that can be used within multiple trade data containers, i.e.  by multiple trade types.

\vspace{1em}

Details of  trade-specific data for all trade types follow below.

\subsubsection{Swap}

The \lstinline!SwapData! node is the trade data container for the Swap trade type. A Swap must have at least one leg,
and can have an unlimited number of legs. Each leg is represented by a \lstinline!LegData! trade component sub-node,
described in section \ref{ss:leg_data}. An example structure of a two-legged \lstinline!SwapData!
node is shown in Listing \ref{lst:swap_data}.

\begin{listing}[H]
%\hrule\medskip
\begin{minted}[fontsize=\footnotesize]{xml}
<SwapData>
  <LegData>
    ...
  </LegData>
  <LegData>
    ...
  </LegData>
</SwapData>
\end{minted}
\caption{Swap data}
\label{lst:swap_data}
\end{listing}

\subsubsection{Cap/Floor}

The \lstinline!CapFloorData! node is the trade data container for trade type CapFloors.  It's a cap, floor or collar
(i.e. a portfolio of a long cap and a short floor for a long position in the collar) on a series of Ibor rates. The
\lstinline!CapFloorData! node contains a \lstinline!LongShort! sub-node which indicates whether the cap (floor, collar)
is long or short, and a \lstinline!LegData!  sub-node where the LegType element must be set to \emph{Floating}, plus
elements for the Cap and Floor rates. An example structure with Cap rates is shown in in Listing
\ref{lst:capfloor_data}. A \lstinline!CapFloorData! node must have either \lstinline!Caps! or \lstinline!Floors!
elements, or both.

\begin{listing}[H]
%\hrule\medskip
\begin{minted}[fontsize=\footnotesize]{xml}
  <CapFloorData>
  <LongShort>Long</LongShort>
  <LegData>
    <Payer>false</Payer>
    <LegType>Floating</LegType>
     ...
  </LegData>
  <Caps>
    <Rate>0.05</Rate>
  </Caps>
</CapFloorData>
\end{minted}
\caption{Cap/Floor data}
\label{lst:capfloor_data}
\end{listing}

The meanings and allowable values of the elements in the \lstinline!CapFloorData!  node follow below.

\begin{itemize}

\item LongShort: This node defines the position in the cap (floor, collar) and can take values \lstinline!Long! or
  \lstinline!Short!

\item LegData: This is a trade component sub-node outlined in section \ref{ss:leg_data}. Exactly
  one \lstinline!LegData! node is allowed and the LegType element must be set to \emph{floating}.

\item Caps: This node has child elements of type \lstinline!Rate!
  capping the floating leg. The first rate value corresponds to the
  first coupon, the second rate value corresponds to the second
  coupon, etc. If the number of coupons exceeds the number of rate
  values, the rate will be kept flat at the value of last entered rate
  for the remaining coupons. For a fixed cap rate over all coupons,
  one single rate value is sufficient. The number of entered rate
  values cannot exceed the number of coupons. 

  Allowable values for each \lstinline!Rate! element: Any real number. The rate is expressed in decimal form, eg 0.05 is
  a rate of 5\%

\item Floors: This node has child elements of type
  \lstinline!Rate! flooring the floating leg.  The first rate value
  corresponds to the first coupon, the second rate value corresponds
  to the second coupon, etc. If the number of coupons exceeds the
  number of rate values, the rate will be kept flat at the value of
  last entered rate for the remaining coupons. For a fixed floor rate
  over all coupons, one single rate value is sufficient. The number of
  entered rate values cannot exceed the number of coupons.

  Allowable values for each \lstinline!Rate! element: Any real number. The rate is expressed in decimal form, eg 0.05 is
  a rate of 5\%

\end{itemize}

\subsubsection{Swaption}

The \lstinline!SwaptionData!  node is the trade data container for the Swaption trade type. The \lstinline!SwaptionData!
node has one and exactly one \lstinline!OptionData! trade component sub-node, and at least one \lstinline!LegData! trade
component sub-node.  These trade components are outlined in section \ref{ss:option_data} and section
\ref{ss:leg_data}.\\

Supported swaption exercise styles are European and Bermudan. European swaptions can have unlimited number of legs, with
each leg represented by a \lstinline!LegData! sub-node. Bermudan swaptions must have two legs, i.e. two
\lstinline!LegData! sub-nodes. See Table \ref{tab:bermudan_requirements} for further details on requirements for
Bermudan swaptions. Cross currency swaptions are not supported for either exercise style, i.e. the Currency element must
have the same value for all \lstinline!LegData! sub-nodes of a swaption.\\

The structure of an example \lstinline!SwaptionData!  node of a two-legged European swaption is shown in Listing
\ref{lst:swaption_data}.

\begin{listing}[H]
%\hrule\medskip
\begin{minted}[fontsize=\footnotesize]{xml}
<SwaptionData>
    <OptionData>
        <Style>European</Style>
        ...
    </OptionData>
    <LegData>
        <Currency>GBP</Currency>
        ...
    </LegData>
    <LegData>
        <Currency>GBP</Currency>
        ...
    </LegData>
</SwaptionData>
\end{minted}
\caption{Swaption data}
\label{lst:swaption_data}
\end{listing}

\begin{table}[H]
\centering
\begin{tabu} to 0.9\linewidth {| X[-1.5,l,m] | X[-5,l,m] |}
    \hline
        & \bfseries{A Bermudan Swaption requires:} \\  \hline
    \lstinline!OptionData! & One \lstinline!OptionData! sub-node  \\  \hline
   \lstinline!Style! &  \emph{Bermudan}\\ \hline
    \lstinline!ExerciseDates! & At least two \lstinline!ExerciseDate! child elements.\\ \hline
    \lstinline!LegData! &  Two \lstinline!LegData! sub-nodes \\ \hline
    \lstinline!LegType! & \emph{Fixed} on one node and \emph{Floating} on the other.\\ \hline    
    \lstinline!Currency! & The same currency for both nodes.\\ \hline 
    \lstinline!Notionals! & No accretion or amortisation, just a constant notional. Exactly one \lstinline!Notional! child element for each node.\\ \hline
    \lstinline!Rates! & A constant rate. The fixed rate node should have exactly one \lstinline!Rate! child element.\\ \hline
    \lstinline!Spreads! &  A constant spread. The floating rate node should have exactly one \lstinline!Spread! child element.\\ \hline
%    \bfseries{ScheduleData} &   \\ \hline   
%    \lstinline!IsRuleBased! & Must be \emph{true} for both nodes. TBC: Fixed by Henning?\\ \hline      
  \end{tabu}
  \caption{Requirements for Bermudan Swaptions}
  \label{tab:bermudan_requirements}
\end{table}

\subsubsection{FX Forward}

The \lstinline!FXForwardData!  node is the trade data container for the FX Forward trade type.  The structure -
including example values - of the \lstinline!FXForwardData!  node is shown in Listing \ref{lst:fxforward_data}. It
contains no sub-nodes.

\begin{listing}[H]
%\hrule\medskip
\begin{minted}[fontsize=\footnotesize]{xml}
        <FxForwardData>
            <ValueDate>2023-04-09</ValueDate>
            <BoughtCurrency>EUR</BoughtCurrency>
            <BoughtAmount>1000000</BoughtAmount>
            <SoldCurrency>USD</SoldCurrency>
            <SoldAmount>1500000</SoldAmount>
        </FxForwardData>
\end{minted}
\caption{FX Forward data}
\label{lst:fxforward_data}
\end{listing}

The meanings and allowable values of the various elements in the \lstinline!FXForwardData!  node follow below.  All elements are required.

\begin{itemize}
\item ValueDate: The value date of the FX Forward. \\ Allowable values:  See \lstinline!Date! in Table \ref{tab:allow_stand_data}.
\item BoughtCurrency: The currency to be bought on value date.  \\ Allowable values:  See \lstinline!Currency! in Table \ref{tab:allow_stand_data}.
\item BoughtAmount: The amount to be sold on value date.  \\ Allowable values:  Any positive real number.
\item SoldCurrency: The currency to be sold on value date.  \\ Allowable values:  See \lstinline!Currency! in Table \ref{tab:allow_stand_data}.
\item SoldAmount: The amount to be sold on value date.  \\ Allowable values:  Any positive real number.

\end{itemize}


\subsubsection{FX Option}

The \lstinline!FXOptionData!  node is the trade data container for the FX Option trade type.  Vanilla FX options are
supported, the exercise style must be \emph{European}. The strike rate of an FX option is: SoldAmount / BoughtAmount. The
\lstinline!FXOptionData!  node includes one and only one \lstinline!OptionData! trade component sub-node plus elements
specific to the FX Option. The structure of an example \lstinline!FXOptionData! node for a FX Option is shown in Listing
\ref{lst:fxoption_data}.

\begin{listing}[H]
%\hrule\medskip
\begin{minted}[fontsize=\footnotesize]{xml}
        <FxOptionData>
            <OptionData>
             ...
            </OptionData>
            <BoughtCurrency>EUR</BoughtCurrency>
            <BoughtAmount>1000000</BoughtAmount>
            <SoldCurrency>USD</SoldCurrency>
            <SoldAmount>1350000</SoldAmount>
        </FxOptionData>
\end{minted}
\caption{FX Option data}
\label{lst:fxoption_data}
\end{listing}

The meanings and allowable values of the elements in the \lstinline!FXOptionData!  node follow below.

\begin{itemize}
\item OptionData: This is a trade component sub-node outlined in section \ref{ss:option_data}. Note that the
  FX option type allows for \emph{European} option style only.

\item BoughtCurrency: The bought currency of the FX option.  

Allowable values:  See Currency in Table \ref{tab:allow_stand_data}.

\item BoughtAmount: The amount in the BoughtCurrency.  

Allowable values:  Any positive real number.

\item SoldCurrency: The sold currency of the FX option.  

Allowable values:  See Currency in Table \ref{tab:allow_stand_data}.

\item SoldAmount: The amount in the SoldCurrency.  

Allowable values:  Any positive real number.

\end{itemize}

\subsubsection{Equity Option}

The \lstinline!EquityOptionData!  node is the trade data container for the equity option trade type.  Vanilla equity 
options are supported, the exercise style must be \emph{European}. The \lstinline!EquityOptionData!  node includes one and 
only one \lstinline!OptionData! trade component sub-node plus elements specific to the equity option. The structure of 
an example \lstinline!EquityOptionData! node for an equity option is shown in Listing
\ref{lst:eqoption_data}.

\begin{listing}[H]
%\hrule\medskip
\begin{minted}[fontsize=\footnotesize]{xml}
<EquityOptionData>
    <OptionData>
      ...
    </OptionData>
    <Name>SP5</Name>
    <Currency>USD</Currency>
    <Strike>2147.56</Strike>
    <Quantity>17000</Quantity>
</EquityOptionData>
\end{minted}
\caption{FX Option data}
\label{lst:eqoption_data}
\end{listing}

The meanings and allowable values of the elements in the \lstinline!EquityOptionData!  node follow below.

\begin{itemize}
	\item OptionData: This is a trade component sub-node outlined in section \ref{ss:option_data} Option Data. Note 
	that the FX option type allows for \emph{European} option style only.	
	\item Name: The name of the underlying equity. \\
	Allowable values:  Any string (provided it is the ID of an equity in the market configuration).
	\item Currency: The bought currency of the equity option. \\
	Allowable values:  See Currency in Table \ref{tab:allow_stand_data}.	
	\item Strike: The option strike price.\\
	Allowable values:  Any positive real number.	
	\item Quantity: The number of units of the underlying covered by the transaction. \\
	Allowable values:  Any positive real number.
\end{itemize}

\subsubsection{Equity Forward}

The \lstinline!EquityForwardData!  node is the trade data container for the equity forward trade type.  Vanilla equity 
forwards are supported. The structure of an example \lstinline!EquityForwardData! node for an equity option is shown in 
Listing \ref{lst:eqfwd_data}.

\begin{listing}[H]
%\hrule\medskip
\begin{minted}[fontsize=\footnotesize]{xml}
<EquityForwardData>
  <LongShort>Long</LongShort>
  <Maturity>2018-06-30</Maturity>
  <Name>SP5</Name>
  <Currency>USD</Currency>
  <Strike>2147.56</Strike>
  <Quantity>17000</Quantity>
</EquityForwardData>
\end{minted}
\caption{Equity Forward data}
\label{lst:eqfwd_data}
\end{listing}

The meanings and allowable values of the elements in the \lstinline!EquityForwardData!  node follow below.

\begin{itemize}
	\item LongShort: Defines whether the trade is long or short the underlying equity
	Allowable values: Long, Short.
	\item Maturity: The maturity date of the forward contract
	Allowable values: Any date string.
	\item Name: The name of the underlying equity 	
	Allowable values:  Any string (provided it is the ID of an equity in the market configuration).
	\item Currency: The bought currency of the equity option.  	
	Allowable values:  See Currency in Table \ref{tab:allow_stand_data}.	
	\item Strike: The option strike price.  
	Allowable values:  Any positive real number.	
	\item Quantity: The number of units of the underlying covered by the transaction  
	Allowable values:  Any positive real number.
\end{itemize}

\subsubsection{CPI Swap}

A CPI swap is set up as a swap, with one leg of type {\tt CPI}. Listing \ref{lst:cpiswap} shows an example. The
CPI leg contains an additional {\tt CPILegData} block. See \ref{ss:cpilegdata} for details on the CPI leg specification.

\begin{listing}[H]
%\hrule\medskip
\begin{minted}[fontsize=\footnotesize]{xml}
    <SwapData>
      <LegData>
        <LegType>Floating</LegType>
        <Payer>true</Payer>
        ...
      </LegData>
      <LegData>
        <LegType>CPI</LegType>
        <Payer>false</Payer>
        ...
        <CPILegData>
        ...
        </CPILegData>
      </LegData>
    </SwapData>
\end{minted}
\caption{CPI Swap Data}
\label{lst:cpiswap}
\end{listing}

\subsubsection{Year on Year Inflation Swap}

A Year on Year inflation swap is set up as a swap, with one leg of type {\tt YY}. Listing \ref{lst:yyswap} shows an
example. The YY leg contains an additional {\tt YYLegData} block. See \ref{ss:yylegdata} for details on the YY leg
specification.

\begin{listing}[H]
%\hrule\medskip
\begin{minted}[fontsize=\footnotesize]{xml}
    <SwapData>
      <LegData>
        <LegType>Floating</LegType>
        <Payer>true</Payer>
        ...
      </LegData>
      <LegData>
        <LegType>YY</LegType>
        <Payer>false</Payer>
        ...
        <YYLegData>
        ...
        </YYLegData>
      </LegData>
    </SwapData>
\end{minted}
\caption{Year on Year Swap Data}
\label{lst:yyswap}
\end{listing}

\subsubsection{Bond}

A vanilla Bond is set up using a {\tt BondData} block as shown in listing \ref{lst:bonddata}. The bond specific elements
are

\begin{itemize}
\item IssuerId: A unique identifier for the issuer of the bond
\item CreditCurveId: The entity defining the default curve used for pricing, via the default curves block in {\tt
    todaysmarket.xml}
% \item LGD (optional): If given, this LGD is used for pricing, overriding the default LGD of the default curve
\item SecurityId: A unique identifier for the security, this defines the security specific spread to be used for
  pricing, see section \ref{sssec:securityspreads}
\item ReferenceCurveId: The benchmark curve to be used for pricing, this must match a name of a curve in the yield
  curves block in {\tt todaysmarket.xml}
\item SettlementDays: The settlement delay applicable to the security
\item Calendar: The calendar associated to the settlement lag
\item IssueDate: The issue date of the security.
\end{itemize}

A LegData block then defines the cashflow structure of the bond, this can be of type fixed, floating etc.

\begin{listing}[H]
%\hrule\medskip
\begin{minted}[fontsize=\footnotesize]{xml}
    <BondData>
        <IssuerId>CPTY_C</IssuerId>
        <CreditCurveId>CPTY_C</CreditCurveId>
        <SecurityId>SECURITY_1</SecurityId>
        <ReferenceCurveId>EUR-EURIBOR-6M</ReferenceCurveId>
        <SettlementDays>2</SettlementDays>
        <Calendar>TARGET</Calendar>
        <IssueDate>20160203</IssueDate>
        <LegData>
            <LegType>Fixed</LegType>
            <Payer>false</Payer>
            ...
        </LegData>
    </BondData>
\end{minted}
\caption{Bond Data}
\label{lst:bonddata}
\end{listing}

The bond pricing requires a recovery rate that can be specified in ORE per SecurityId, see sections \ref{sssec:securityrecoveryrates} and \ref{md:sec_rec_rates}. 

\subsubsection{Credit Default Swap}

A credit default swap is set up using a {\tt CreditDefaultSwapData} block as shown in listing \ref{lst:cdsdata}. The CDS specific elements
are

\begin{itemize}
\item IssuerId: A unique identifier for the issuer of the bond
\item CreditCurveId: The entity defining the default curve used for pricing, via the default curves block in {\tt
    todaysmarket.xml}
\item SettleAtAccrual: Whether or not the accrued coupon is due in the event of a default.
\item PaysAtDefaultTime: If set to true, any payments triggered by a default event are due at default time. If set to false, they are due at the end of the accrual period.
\item ProtectionStart: The first date where a default event will trigger the contract.
\item UpfrontDate: Settlement date for the upfront payment.
\item UpfrontFee: The upfront payment, expressed as a rate, to be multiplied by notional amount.
\end{itemize}

A LegData block then defines the cashflow structure of the credit default swap, this must be be of type fixed.

\begin{listing}[H]
%\hrule\medskip
\begin{minted}[fontsize=\footnotesize]{xml}
    <CreditDefaultSwapData>
      <IssuerId>CPTY_A</IssuerId>
      <CreditCurveId>BANK</CreditCurveId>
      <Qualifier>SECURITY_1</Qualifier>
      <SettlesAccrual>Y</SettlesAccrual>
      <PaysAtDefaultTime>Y</PaysAtDefaultTime>
      <ProtectionStart>20160206</ProtectionStart>
      <UpfrontDate>20160208</UpfrontDate>
      <UpfrontFee>0.0</UpfrontFee>
      <LegData>
            <LegType>Fixed</LegType>
            <Payer>false</Payer>
            ...
        </LegData>
    </CreditDefaultSwapData>
\end{minted}
\caption{CreditDefaultSwap Data}
\label{lst:cdsdata}
\end{listing}

\subsubsection{Forward Rate Agreement}

A forward rate agreement is set up using a {\tt ForwardRateAgreementData} block as shown in listing \ref{lst:ForwardRateAgreementdata}. The forward rate agreement specific elements
are

\begin{itemize}
\item StartDate: A FRA expires/settles on the startDate . \\ Allowable values:  See \lstinline!Date! in Table \ref{tab:allow_stand_data}.\item CreditCurveId: The entity defining the default curve used for pricing, via the default curves block in {\tt
    todaysmarket.xml}
\item EndDate: EndDate is the date when the forward loan or deposit ends. It follows that (EndDate - StartDate) is the tenor/term of the underlying loan or deposit
\item Currency: The currency of the FRA notional.  	
	Allowable values:  See Currency in Table \ref{tab:allow_stand_data}.	
\item Index: The name of the interest rate index the FRA is benchmarked against
\item LongShort: Specifies whether the option position is long  or
  short.
\item Strike: The agreed forward interest rate
\item Notional: No accretion or amortisation, just a constant notional.
\end{itemize}

A LegData block then defines the cashflow structure of the bond, this can be of type fixed, floating etc.

\begin{listing}[H]
%\hrule\medskip
\begin{minted}[fontsize=\footnotesize]{xml}
    <ForwardRateAgreementData>
        <StartDate>20161028</StartDate>
        <EndDate>20351028</EndDate>
        <Currency>EUR</Currency>
        <Index>EUR-EURIBOR-6M</Index>
        <LongShort>Long</LongShort>
        <Strike>0.00001</Strike>
        <Notional>1000000000</Notional>
    </ForwardRateAgreementData>
\end{minted}
\caption{Forward Rate Agreement Data}
\label{lst:ForwardRateAgreementdata}
\end{listing}

The bond pricing requires a recovery rate that can be specified in ORE per SecurityId, see sections \ref{sssec:securityrecoveryrates} and \ref{md:sec_rec_rates}. 

%- - - - - - - - - - - - - - - - - - - - - - - - - - - - - - - - - - - - - - - -
\subsection{Trade Components}
%- - - - - - - - - - - - - - - - - - - - - - - - - - - - - - - - - - - - - - - -

Trade components are XML sub-nodes used within the trade data containers to define sets of trade data that more than one
trade type can have in common, such as a leg or a schedule. A trade data container can include multiple trade components
such as a swap with multiple legs, and a trade component can itself contain further trade components in a nested way.

\vspace{1em}

An example of a \lstinline!SwapData! trade data container, including two \lstinline!LegData! trade components which in
turn include further trade components such as \lstinline!FixedLegData!, \lstinline!ScheduleData! and
\lstinline!FloatingLegData! is shown in Listing \ref{lst:trade_component}.

\begin{listing}[H]
%\hrule\medskip
\begin{minted}[fontsize=\footnotesize]{xml}
        <SwapData>
            <LegData>
                <Payer>true</Payer>
                <LegType>Fixed</LegType>
                <Currency>EUR</Currency>
                <PaymentConvention>Following</PaymentConvention>
                <DayCounter>30/360</DayCounter>
                <Notionals>
                    <Notional>1000000</Notional>
                </Notionals>
                <ScheduleData>
                ...
                </ScheduleData>
                <FixedLegData>
                    <Rates>
                        <Rate>0.035</Rate>
                    </Rates>
                </FixedLegData>
            </LegData>
            <LegData>
                ...
                <ScheduleData>
                    ...
                </ScheduleData>
                <FloatingLegData>
                    ...
                </FloatingLegData>
            </LegData>
        </SwapData>
\end{minted}
\caption{Trade Components Example}
\label{lst:trade_component}
\end{listing}

Descriptions of all trade components supported in ORE follow below:

\subsubsection{Option Data}
\label{ss:option_data} 
This trade component node is used within the \lstinline!SwaptionData! and \lstinline!FXOptionData! trade data
containers. It contains the \lstinline!ExerciseDates! sub-node which includes \lstinline!ExerciseDate! child
elements. An example structure of the \lstinline!OptionData! trade component node is shown in Listing
\ref{lst:option_data}.

\begin{listing}[H]
%\hrule\medskip
\begin{minted}[fontsize=\footnotesize]{xml}
            <OptionData>
                <LongShort>Long</LongShort>
                <OptionType>Call</OptionType>
                <Style>Bermudan</Style>
                <Settlement>Cash</Settlement> 
                <PayOffAtExpiry>true</PayOffAtExpiry>
                <ExerciseDates>
                    <ExerciseDate>2016-04-20</ExerciseDate>
                    <ExerciseDate>2017-04-20</ExerciseDate>
                </ExerciseDates>
            </OptionData>
\end{minted}
\caption{Option data}
\label{lst:option_data}
\end{listing}

The meanings and allowable values of the elements in the \lstinline!OptionData! node follow below.

\begin{itemize}
\item LongShort: Specifies whether the option position is long  or
  short.  

Allowable values: \emph{Long, LONG, long, L} or \emph{Short, SHORT,
  short, S}

\item OptionType: Specifies whether it is a call or a put option. 

Allowable values: \emph{Call} or \emph{Put} 

\item Style: The exercise style of the option. 

  Allowable values: \emph{European} or \emph{American} or \emph{Bermudan}. Note that trade type Swaption can have style
  \emph{European} or \emph{Bermudan}, but not \emph{American}. The FX Option trade type can have style \emph{European}
  or \emph{American}, but not \emph{Bermudan}.

\item Settlement: Derlivery type. 

Allowable values: \emph{Cash} or \emph{Physical} 

\item PayOffAtExpiry [Optional]: Relevant for options with early
  exercise, i.e. the exercise occurs before expiry; \emph{true}
  indicates payoff at expiry, whereas \emph{false}  indicates payoff
  at exercise. Defaults to \emph{false}  if left blank or omitted. 

Allowable values: \emph{true}, \emph{false}
%TBC: Do not see payoffatexpiry used in either fxoption or swaption build() functions, to be logged as a bug.       

\item ExerciseDates: This node contains child elements of type
  \lstinline!ExerciseDate!.  Options of style \emph{European} or
  \emph{American} require a single exercise date expressed by one
  single \lstinline!ExerciseDate! child element.  \emph{Bermudan}
  style options must have two or more \lstinline!ExerciseDate! child
  elements.

\end{itemize}



\subsubsection{Leg Data and Notionals}
\label{ss:leg_data}

The \lstinline!LegData! trade component node is used within the
\lstinline!CapFloorData!,  \lstinline!SwapData! and
\lstinline!SwaptionData! trade data containers. It contains a
\lstinline!ScheduleData! trade component sub-node, and depending on
the value of the \lstinline!LegType! element, one out of the following
sub-nodes:  \lstinline!FixedLegData!, \lstinline!FloatingLegData!. The
\lstinline!LegData! node also includes a \lstinline!Notionals!
sub-node  with \lstinline!Notional! child elements described below. An
example structure of a \lstinline!LegData! node of \lstinline!LegType!
floating is shown in Listing \ref{lst:leg_data}.

\begin{listing}[H]
%\hrule\medskip
\begin{minted}[fontsize=\footnotesize]{xml}
            <LegData>
                <Payer>false</Payer>
                <LegType>Floating</LegType>
                <Currency>EUR</Currency>
                <PaymentConvention>Following</PaymentConvention>
                <DayCounter>30/360</DayCounter>
                <Notionals>
                    <Notional>1000000</Notional>
                </Notionals>
                <ScheduleData>
                    ...
                </ScheduleData>
                <FloatingLegData>
                    ...
                </FloatingLegData>
            </LegData>
\end{minted}
\caption{Leg data}
\label{lst:leg_data}
\end{listing}

The meanings and allowable values of the elements in the \lstinline!LegData! node follow below.

\begin{itemize}
\item LegType:  Determines which of the available sub-nodes must be
  used. 

Allowable values:  \emph{Fixed, Floating, Cashflow, YY, CPI}

\item Payer:  The flows of the leg are paid to the counterparty if
  \emph{true}, and received if \emph{false}.  

Allowable values:  \emph{true, false} 

\item Currency: The currency of the leg. 

Allowable values:  See \lstinline!Currency! in Table \ref{tab:allow_stand_data}.

\item DayCounter: The day count convention of the leg coupons. 

Allowable values: See \lstinline!DayCount Convention! in Table \ref{tab:daycount}.

\item PaymentConvention: The payment convention of the leg coupons. 

Allowable values: See \lstinline!Roll Convention! in Table \ref{tab:allow_stand_data}.

\item Notionals: This node contains child elements of type
  \lstinline!Notional!. If the notional is fixed over the life of the
  leg only one notional value should be entered. If the notional is
  amortising or accreting, this is represented by entering multiple
  notional values, each represented by a \lstinline!Notional! child
  element. The first notional value corresponds to the first coupon,
  the second notional value corresponds to the second coupon, etc. If
  the number of coupons exceeds the number of notional values, the
  notional will be kept flat at the value of last entered notional for
  the remaining coupons.  The number of entered notional values cannot
  exceed the number of coupons.

Allowable values: Each child element can take any positive real number.

\vspace{1em}

An example of a \lstinline!Notionals! element for an amortising leg with four coupons is shown in Listing \ref{lst:notionals}.
\begin{listing}[H]
%\hrule\medskip
\begin{minted}[fontsize=\footnotesize]{xml}
                <Notionals>
                    <Notional>65000000</Notional>
                    <Notional>65000000</Notional>
                    <Notional>55000000</Notional>
                    <Notional>45000000</Notional>
                </Notionals>
\end{minted}
\caption{Notional list}
\label{lst:notionals}
\end{listing}

Another allowable specification of the notional schedule is shown in Listing \ref{lst:notionals_dates}. 
\begin{listing}[H]
%\hrule\medskip
\begin{minted}[fontsize=\footnotesize]{xml}
                <Notionals>
                    <Notional>65000000</Notional>
                    <Notional startDate='2016-01-02'>65000000</Notional>
                    <Notional startDate='2017-01-02'>55000000</Notional>
                    <Notional startDate='2021-01-02'>45000000</Notional>
                </Notionals>
\end{minted}
\caption{Notional list with dates}
\label{lst:notionals_dates}
\end{listing}
The first notional must not have a start date, it will be associated
with the schedule's start, The subsequent notionals can have a start
date specified from which date onwards the new notional is applied. This allows
specifying notionals only for dates where the notional changes. 

\vspace{1em} 

In case of exchange of currencies an initial exchange, a final exchange
and an amortising exchange can be specified using an \lstinline!Exchanges! child element with \break
\lstinline!NotionalInitialExchange!, \lstinline!NotionalFinalExchange! and \break
\lstinline!NotionalAmortizingExchange! as subelements, see listing
\ref{lst:notional_exchange}.

\begin{listing}[H]
%\hrule\medskip
\begin{minted}[fontsize=\footnotesize]{xml}
                <Notionals>
                    <Notional>65000000</Notional>
                    <Exchanges>
                      <NotionalInitialExchange>true</NotionalInitialExchange>
                      <NotionalFinalExchange>true</NotionalFinalExchange>
                      <NotionalAmortizingExchange>true</NotionalAmortizingExchange>
                    </Exchanges>
                </Notionals>
\end{minted}
\caption{Notional list with exchange}
\label{lst:notional_exchange}
\end{listing}

FX Resets can be specified using an \lstinline!FXReset! child element with subelements \break
\lstinline!ForeignCurrency! (currencyCode), \lstinline!ForeignAmount! (double), \lstinline!FXIndex! and \break
\lstinline!FixingDays! (integer) subelements, see listing
\ref{lst:notional_fxreset} for an example

 \begin{listing}[H]
%\hrule\medskip
\begin{minted}[fontsize=\footnotesize]{xml}
                <Currency>USD</Currency>
                <Notionals>
                    <Notional>65000000</Notional> <!-- in USD -->
                    <FXReset>
                      <ForeignCurrency> EUR </ForeignCurrency>
                      <ForeignAmount> 60000000 </ForeignAmount>
                      <FXIndex> FX-SOURCE-USD-EUR </FXIndex>
                      <FixingDays> 2 </FixingDays>
                    </FXReset>
                </Notionals>
\end{minted}
\caption{Notional list with exchange}
\label{lst:notional_fxreset}
\end{listing}


\item ScheduleData: This is a trade component sub-node outlined in section \ref{ss:schedule_data} Schedule Data and
Dates.
\item FixedLegData: This trade component sub-node is required if \lstinline!LegType! is set to \emph{fixed} It is
outlined in section \ref{ss:fixedleg_data} Fixed Leg Data and Rates.
\item FloatingLegData: This trade component sub-node is required if \lstinline!LegType! is set to \emph{floating} It is
outlined in section \ref{ss:floatingleg_data} Floating Leg Data and Spreads.
\item CPILegData: This trade component sub-node is required if \lstinline!LegType! is set to \emph{CPI}. It is
  outlined in section \ref{ss:cpilegdata} CPI Leg Data.
\item YYLegData: This trade component sub-node is required if \lstinline!LegType! is set to \emph{YY}. It is
  outlined in section \ref{ss:yylegdata} YY Leg Data.
\end{itemize}

\subsubsection{Schedule Data and Dates}\label{ss:schedule_data}

The \lstinline!ScheduleData! trade component node is used within the \lstinline!LegData! trade component. When
\lstinline!IsRulesBased! is set to \emph{false}, the \lstinline!ScheduleData! node includes a \lstinline!Dates! sub-node
where the schedule is determined directly by \lstinline!Date! child elements. The schedule can also be generated from a
set of rules based on the entries of the StartDate, EndDate, Tenor, Calendar, Convention, TermConvention, and Rule
elements.  Example structures of \lstinline!ScheduleData! nodes based on rules respectively dates are shown in Listing
\ref{lst:schedule_data_true} and Listing \ref{lst:schedule_data_false}, respectively.

\begin{listing}[H]
%\hrule\medskip
\begin{minted}[fontsize=\footnotesize]{xml}
              <ScheduleData>
                    <Rules>
                        <StartDate>2013-02-01</StartDate>
                        <EndDate>2030-02-01</EndDate>
                        <Tenor>1Y</Tenor>
                        <Calendar>UK</Calendar>
                        <Convention>MF</Convention>
                        <TermConvention>MF</TermConvention>
                        <Rule>Forward</Rule>
                    </Rules>
              </ScheduleData>
\end{minted}
\caption{Schedule data, rules based}
\label{lst:schedule_data_true}
\end{listing}

\begin{listing}[H]
%\hrule\medskip
\begin{minted}[fontsize=\footnotesize]{xml}
               <ScheduleData>
                    <Dates>
                        <Date>2012-01-06</Date>
                        <Date>2012-04-10</Date>
                        <Date>2012-07-06</Date>
                        <Date>2012-10-08</Date>
                        <Date>2013-01-07</Date>
                        <Date>2013-04-08</Date>
                    </Dates>
                </ScheduleData>
\end{minted}
\caption{Schedule data, date based}
\label{lst:schedule_data_false}
\end{listing}

The ScheduleData section can contain any number and combination of
{\tt <Dates>} and {\tt <Rules>} sections. The resulting schedule will
then be an ordered concatenation of individual schedules.
 
\medskip
The meanings and allowable values of the elements in the \lstinline!ScheduleData! node follow below.

\begin{itemize}
  % \item IsRulesBased: Determines whether the schedule is set by specifying dates directly, or by specifying rules that
  %   generate the schedule. If \emph{true}, the following entries are required: StartDate, EndDate, Tenor, Calendar,
  %   Convention, TermConvention, and Rule.  If false the Dates sub-node is required. \\ Allowable values: \emph{true,
  %   false}
\item StartDate:  The schedule start date.  

Allowable values:  See \lstinline!Date! in Table \ref{tab:allow_stand_data}.

\item EndDate: The schedule end date.  

Allowable values:  See \lstinline!Date! in Table \ref{tab:allow_stand_data}.

\item Tenor: The tenor used to generate schedule dates. 

Allowable values: A string where the last character must be D or W or
M or Y.  The characters before that must be a positive integer. \\D
$=$ Day, W $=$ Week, M $=$ Month, Y $=$ Year

\item Calendar: The calendar used to generate schedule  dates. 

Allowable values: See Table \ref{tab:calendar} Calendar.

\item Convention: Determines the adjustment of the schedule dates with
  regards to the selected calendar. 

Allowable values: See \lstinline!Roll Convention! in Table
\ref{tab:allow_stand_data}.

\item TermConvention: Determines the adjustment of the final schedule
  date with regards to the selected calendar. 

Allowable values: See \lstinline!Roll Convention! in Table \ref{tab:allow_stand_data}.

\item Rules: Rules for the generation of the schedule using given
  start and end dates, tenor, calendar and business day conventions. 

Allowable values and descriptions: See Table \ref{tab:rule} Rule.

\item Dates: This is a sub-node and contains child elements of type
  \lstinline!Date!. In this case the schedule dates are determined
  directly by the \lstinline!Date! child elements.  At least two
  \lstinline!Date! child elements must be provided.     

  Allowable values: Each \lstinline!Date!  child element can take the allowable values listed in \lstinline!Date! in
  Table \ref{tab:allow_stand_data}.

\end{itemize}



\begin{table}[H]
\centering
\begin{tabular}{|l|p{6cm}|}
\hline
\multicolumn{2}{|l|}{\lstinline!Rule!}                    \\ \hline
\textbf{Allowable Values}                   & \textbf{Effect}                       \\ \hline
\emph{Backward}   &   Backward from termination date to effective date.   \\ \hline
\emph{Forward}   &   Forward from effective date to termination date.  \\ \hline
\emph{Zero}   &   No intermediate dates between effective date and termination date.  \\ \hline
\emph{ThirdWednesday}   &   All dates but effective date and
                          termination date are taken to be on the
                          third Wednesday of their month (with forward calculation.) \\ \hline
\emph{Twentieth}   &   All dates but the effective date are taken to be the twentieth of their month (used for CDS schedules in emerging markets.)  The termination date is also modified. \\ \hline
\emph{TwentiethIMM}   &   All dates but the effective date are  taken to be the twentieth of an IMM month (used for CDS schedules.)  The termination date is also modified. \\ \hline
\emph{OldCDS}   &   Same as TwentiethIMM with unrestricted date ends and log/short stub coupon period (old CDS convention).\\ \hline
CDS   &   Credit derivatives standard rule since 'Big Bang' changes in 2009.\\ \hline
\end{tabular}
  \caption{Allowable Values for Rule}
  \label{tab:rule}
\end{table}

\subsubsection{Fixed Leg Data and Rates}
\label{ss:fixedleg_data}

The \lstinline!FixedLegData! trade component node is used within the \lstinline!LegData! trade component when the
\lstinline!LegType! element is set to \emph{Fixed}. The \lstinline!FixedLegData! node only includes the
\lstinline!Rates! sub-node which contains the rates of the fixed leg as child elements of type \lstinline!Rate!.

An example of a \lstinline!FixedLegData! node for a fixed leg with constant notional is shown in Listing \ref{lst:fixedleg_data}.
\begin{listing}[H]
%\hrule\medskip
\begin{minted}[fontsize=\footnotesize]{xml}
              <FixedLegData>
                    <Rates>
                        <Rate>0.05</Rate>
                    </Rates>
              </FixedLegData>
\end{minted}
\caption{Fixed leg data}
\label{lst:fixedleg_data}
\end{listing}

The meanings and allowable values of the elements in the \lstinline!FixedLegData! node follow below.

\begin{itemize}

\item Rates: This node contains child elements of type
  \lstinline!Rate!. If the rate is constant over the life of the fixed
  leg only one rate value should be entered. If two or more coupons
  have different rates, multiple rate values are required, each
  represented by a \lstinline!Rate! child element. The first rate
  value corresponds to the first coupon, the second rate value
  corresponds to the second coupon, etc. If the number of coupons
  exceeds the number of rate values, the rate will be kept flat at the
  value of last entered rate for the remaining coupons.  The number of
  entered rate values cannot exceed the number of coupons. 

  Allowable values: Each child element can take any  real number. The rate is
  expressed in decimal form, e.g. 0.05 is a rate of 5\%.

As in the case of notionals, the rate schedule can be specified with
dates as shown in Listing \ref{lst:fixedleg_data_dates}.
\begin{listing}[H]
%\hrule\medskip
\begin{minted}[fontsize=\footnotesize]{xml}
              <FixedLegData>
                    <Rates>
                        <Rate>0.05</Rate>
                        <Rate startDate='2016-02-04'>0.05</Rate>
                        <Rate startDate='2019-02-05'>0.05</Rate>
                    </Rates>
              </FixedLegData>
\end{minted}
\caption{Fixed leg data with 'dated' rates}
\label{lst:fixedleg_data_dates}
\end{listing}

\end{itemize}

\subsubsection{Floating Leg Data, Spreads, Gearings, Caps and Floors}
\label{ss:floatingleg_data}

The \lstinline!FloatingLegData! trade component node is used within the \lstinline!LegData! trade component when the
\lstinline!LegType! element is set to \emph{Floating}. It is also used directly within the \lstinline!CapFloor! trade
data container.  The \lstinline!FloatingLegData! node includes elements specific to a floating leg as well as the
\lstinline!Spreads! sub-node which contains the spreads of the floating leg as child elements of type
\lstinline!Spread!.

An example of a \lstinline!FloatingLegData! node is shown in Listing \ref{lst:floatingleg_data}.
\begin{listing}[H]
%\hrule\medskip
\begin{minted}[fontsize=\footnotesize]{xml}
                <FloatingLegData>
                    <Index>USD-LIBOR-3M</Index>
                    <IsInArrears>false</IsInArrears>
                    <FixingDays>2</FixingDays>
                    <Spreads>
                        <Spread>0.005</Spread>
                    </Spreads>
                    <Gearings>
                        <Gearing>2.0</Gearing>
                    </Gearings>
                    <Caps>
                        <Cap>0.05</Cap>
                    </Caps>
                    <Floors>
                        <Floor>0.01</Floor>
                    </Floors>
                </FloatingLegData>
                <NakedOption>N</NakedOption>
\end{minted}
\caption{Floating leg data}
\label{lst:floatingleg_data}
\end{listing}

The meanings and allowable values of the elements in the \lstinline!FloatingLegData! node follow below.

\begin{itemize}
\item Index:  The combination of currency, index and term that
  identifies the relevant fixings and yield curve of the floating leg.  

  Allowable values: An alphanumeric string on the form CCY-INDEX-TERM, matching available Ibor indices in the {\tt
    simulation.xml} file. CCY, INDEX and TERM must be separated by dashes (-). TERM must be an integer followed by D, W,
  M or Y. See Table \ref{tab:indices}.

\item IsInArrears:  \emph{true} indicates that  fixing is in arrears,
  i.e. the fixing gap is calculated in relation to the current period
  end date.\\ \emph{false} indicates that  fixing is in advance,
  i.e. the fixing gap is calculated in relation to the previous period
  end date.  

Allowable values:  \emph{true, false}

\item FixingDays: This is the fixing gap, i.e. the number of days
  before the period end date an index fixing is taken.   

Allowable values:  Positive integers.  

\item Spreads: This node contains child elements of type
  \lstinline!Spread!. If the spread is constant over the life of the
  floating leg only one spread value should be entered. If two or more
  coupons have different spreads, multiple spread values are required,
  each represented by a \lstinline!Spread! child element. The first
  spread value corresponds to the first coupon, the second spread
  value corresponds to the second coupon, etc. If the number of
  coupons exceeds the number of spread values, the spread will be kept
  flat at the value of last entered spread for the remaining coupons.
  The number of entered spread values cannot exceed the number of
  coupons. 

  Allowable values: Each child element can take any real number. The spread is expressed in decimal form, e.g. 0.005 is
  a spread of 0.5\% or 50 bp.

For the {\tt <Spreads>} section, the same applies as for notionals and
rates - a list of changing spreads can be specified without or with individual starte dates as shown
in Listing \ref{lst:spreads_dates}.
\begin{listing}[H]
%\hrule\medskip
\begin{minted}[fontsize=\footnotesize]{xml}
                    <Spreads>
                        <Spread>0.005</Spread>
                        <Spread startDate='2017-03-05'>0.007</Spread>
                        <Spread startDate='2019-03-05'>0.009</Spread>
                    </Spreads>
\end{minted}
\caption{'Dated' spreads}
\label{lst:spreads_dates}
\end{listing}

\item Gearings: This node contains child elements of type \lstinline!Gearing! indicating that the coupon rate is
  multiplied by the given factors. The mode of specification is analogous to spreads, see above.

\item Caps: This node contains child elements of type \lstinline!Cap! indicating that the coupon rate is capped at the
  given rate (after applying gearing and spread, if any). The mode of specification is analogous to spreads, see above.

\item Floors: This node contains child elements of type \lstinline!Floor! indicating that the coupon rate is floored at
  the given rate(after applying gearing and spread, if any). The mode of specification is analogous to spreads, see
  above.

\item NakedOption: Optional node (defaults to N), if Y the leg represents only the embedded floor, cap or collar. 
By convention these embedded options are considered long if the leg is a receiver leg, otherwise short. 

\end{itemize}

\subsubsection{Leg Data with Amortisation Structures}
\label{ss:amortisationdata}

Amortisation structures can (optionally) be added to a leg as
indicated in the following listing \ref{lst:amortisations}, within a
block of information enclosed by {\tt <Amortizations>} and {\tt
  </Amortizations>} tags.

\begin{listing}[H]
%\hrule\medskip
\begin{minted}[fontsize=\footnotesize]{xml}
      <LegData>
        <LegType> ... </LegType>
        <Payer> ... </Payer>
        <Currency> ... </Currency>
        <Notionals>
          <Notional>10000000</Notional>
        </Notionals>
        <Amortizations>
          <AmortizationData>
            <Type>FixedAmount</Type>
            <Value>1000000</Value>
            <StartDate>20170203</StartDate>
            <Frequency>1Y</Frequency>
            <Underflow>false</Underflow>
          </AmortizationData>
          <AmortizationData>
            ...
          </AmortizationData>
        </Amortizations>
        ...
      </LegData>
\end{minted}
\caption{Amortisation data}
\label{lst:amortisations}
\end{listing}

The user can specify a sequence of {\tt AmortizationData} items in
order to switch from one kind of amortisation to another etc.  
Within each {\tt AmortisationData} block the meaning of elements is

\begin{itemize}
\item Type: Amortisation type with allowable values {\em FixedAmount,
  RelativeToInitialNotional, RelativeToPreviousNotional, Annuity.}
\item Value: Interpreted depending on {\tt Type}, see below
\item StartDate: Amortisation starts on first schedule date on or
  beyond StartDate
\item Frequency, entered as a period: Frequency of amortisations
\item Underflow:  Allow amortisation below zero notional if {\tt true},
  otherwise amortisation stops at zero notional 
\end{itemize}

The amortisation data block's {\tt Value} element  is interpreted
depending on the chosen {\tt Type}:
\begin{itemize}
\item FixedAmount: The value is interpreted as a notional amount to be
  subtracted from the current notional on each amortisation date
\item RelativeToInitialNotional: The value is interpreted as a
  fraction of the {\bf initial} notional to be subtraced from the current
  notional on each amortisation date
\item RelativeToPreviousNotional: The value is interpreted as a
  fraction of the {\bf previous} notional to be subtraced from the current
  notional on each amortisation date
\item Annuity: The value is interpreted as annuity amount (redemption
  plus coupon)
\end{itemize}

Annuity type amortisation is supported for fixed rate legs as well as
floating (ibor) legs. 

Note:
\begin{itemize}
\item Floating annuities require at least one previous vanilla coupon
  in order to work out the first amortisation amount. 
\item Floating legs with annuity amortisation currently do not allow
  switching the amortisation type, i.e. only a  single block of {\tt
    AmortizationData}.
\end{itemize}

\subsubsection{CMS Leg Data}
\label{ss:cmslegdata}

Listing \ref{lst:cmslegdata} shows an example for a leg of type CMS. 

\begin{listing}[H]
%\hrule\medskip
\begin{minted}[fontsize=\footnotesize]{xml}
      <LegData>
        <LegType>CMS</LegType>
        <Payer>false</Payer>
        <Currency>GBP</Currency>
        <Notionals>
          <Notional>10000000</Notional>
        </Notionals>
        <DayCounter>ACT/ACT</DayCounter>
        <PaymentConvention>Following</PaymentConvention>
        <ScheduleData>
          ...
        </ScheduleData>
        <CMSLegData>
          <Index>EUR-CMS-10Y</Index>
          <Spreads>
            <Spread>0.0010</Spread>
          </Spreads>
          <Gearings>
            <Gearing>2.0</Gearing>
          </Gearings>
          <Caps>
            <Cap>0.05</Cap>
          </Caps>
          <Floors>
            <Floor>0.01</Floor>
          </Floors>
        </CMSLegData>
        <NakedOption>N</NakedOption>
      </LegData>
\end{minted}
\caption{CMS leg data}
\label{lst:cmslegdata}
\end{listing}
 
The CMSLegData block contains the following elements:

\begin{itemize}
\item Index: The underlying CMS index
\item Spreads: The spreads applied to index fixings. As usual, this can be a single value, a vector of values or a dated vector of
  values.
\item IsInArrears:  \emph{true} indicates that  fixing is in arrears,
  i.e. the fixing gap is calculated in relation to the current period
  end date.\\ \emph{false} indicates that  fixing is in advance,
  i.e. the fixing gap is calculated in relation to the previous period
  end date.  
\item FixingDays: This is the fixing gap, i.e. the number of days
  before the period end date an index fixing is taken.   
\item Gearings: This node contains child elements of type \lstinline!Gearing! indicating that the coupon rate is
  multiplied by the given factors. The mode of specification is analogous to spreads, see above.
\item Caps: This node contains child elements of type \lstinline!Cap! indicating that the coupon rate is capped at the
  given rate (after applying gearing and spread, if any). The mode of specification is analogous to spreads, see above.
\item Floors: This node contains child elements of type \lstinline!Floor! indicating that the coupon rate is floored at
  the given rate(after applying gearing and spread, if any). The mode of specification is analogous to spreads, see
  above.
\item NakedOption: Optional node (defaults to N), if Y the leg represents only the embedded floor, cap or collar. 
By convention these embedded options are considered long if the leg is a receiver leg, otherwise short. 
\end{itemize}

\subsubsection{CPI Leg Data}
\label{ss:cpilegdata}

Listing \ref{lst:cpilegdata} shows an example for a leg of type CPI. The CPILegData block contains the following
elements:

\begin{itemize}
\item Index: The underlying zero inflation index
\item Rates: The fixed rate(s) of the leg. As usual, this can be a single value, a vector of values or a dated vector of
  values.
\item BaseCPI: The base CPI used to determine the lifting factor for the fixed coupons
\item ObservationLag: The observation lag to be applied.
\item Interpolated: A flag indicating whether interpolation should be applied to inflation fixings.
\end{itemize}

\begin{listing}[H]
%\hrule\medskip
\begin{minted}[fontsize=\footnotesize]{xml}
      <LegData>
        <LegType>CPI</LegType>
        <Payer>false</Payer>
        <Currency>GBP</Currency>
        <Notionals>
          <Notional>10000000</Notional>
        </Notionals>
        <DayCounter>ACT/ACT</DayCounter>
        <PaymentConvention>Following</PaymentConvention>
        <ScheduleData>
          <Rules>
            <StartDate>2016-07-18</StartDate>
            <EndDate>2021-07-18</EndDate>
            <Tenor>1Y</Tenor>
            <Calendar>UK</Calendar>
            <Convention>ModifiedFollowing</Convention>
            <TermConvention>ModifiedFollowing</TermConvention>
            <Rule>Forward</Rule>
            <EndOfMonth/>
            <FirstDate/>
            <LastDate/>
          </Rules>
        </ScheduleData>
        <CPILegData>
          <Index>UKRPI</Index>
          <Rates>
            <Rate>0.02</Rate>
          </Rates>
          <BaseCPI>210</BaseCPI>
          <ObservationLag>2M</ObservationLag>
          <Interpolated>false</Interpolated>
        </CPILegData>
      </LegData>
\end{minted}
\caption{CPI leg data}
\label{lst:cpilegdata}
\end{listing}

\subsubsection{YY Leg Data}
\label{ss:yylegdata}

Listing \ref{lst:yylegdata} shows an example for a leg of type YY. The YYLegData block contains the following
elements:

\begin{itemize}
\item Index: The underlying zero inflation index
\item FixingDays: The number of fixing days
\item ObservationLag: The observation lag to be applied.
\item Interpolated: A flag indicating whether interpolation should be applied to inflation fixings.
\end{itemize}

\begin{listing}[H]
%\hrule\medskip
\begin{minted}[fontsize=\footnotesize]{xml}
      <LegData>
        <LegType>YY</LegType>
        <Payer>false</Payer>
        <Currency>EUR</Currency>
        <Notionals>
          <Notional>10000000</Notional>
        </Notionals>
        <DayCounter>ACT/ACT</DayCounter>
        <PaymentConvention>Following</PaymentConvention>
        <ScheduleData>
          <Rules>
            <StartDate>2016-07-18</StartDate>
            <EndDate>2021-07-18</EndDate>
            <Tenor>1Y</Tenor>
            <Calendar>UK</Calendar>
            <Convention>ModifiedFollowing</Convention>
            <TermConvention>ModifiedFollowing</TermConvention>
            <Rule>Forward</Rule>
            <EndOfMonth/>
            <FirstDate/>
            <LastDate/>
          </Rules>
        </ScheduleData>
        <YYLegData>
          <Index>EUHICPXT</Index>
          <FixingDays>2</FixingDays>
          <ObservationLag>2M</ObservationLag>
          <Interpolated>true</Interpolated>
        </YYLegData>
      </LegData>
\end{minted}
\caption{YY leg data}
\label{lst:yylegdata}
\end{listing}


%- - - - - - - - - - - - - - - - - - - - - - - - - - - - - - - - - - - - - - - -
\subsection{Allowable Values for Standard Trade Data}
\label{sec:allowable_values}
%- - - - - - - - - - - - - - - - - - - - - - - - - - - - - - - - - - - - - - - -

\begin{table}[H]
\centering
  \begin{tabu} to 0.9\linewidth {| X[-1.5,l,m] | X[-5,l,m] |}
    \hline
    \bfseries{Trade Data} & \bfseries{Allowable Values} \\
    \hline
    \lstinline!Date! & \begin{tabular}[l]{@{}l@{}} The following date formats are supported: \\  \emph{yyyymmdd} \\ \emph{yyyy-mm-dd} \\ \emph{yyyy/mm/dd} \\ \emph{yyyy.mm.dd} \\ \emph{dd-mm-yy} \\  \emph{dd/mm/yy} \\  \emph{dd.mm.yy} \\  \emph{dd-mm-yyyy} \\  \emph{dd/mm/yyyy} \\  \emph{dd.mm.yyyy} \\ and \\ Dates as  serial numbers, comparable to Microsoft Excel \\dates, with a minimum of 367 for Jan 1, 1901,\\ and a maximum of 109574 for Dec 31, 2199.  \end{tabular}  \\ \hline
    \lstinline!Currency! & \emph{ATS, AUD, BEF, BRL, CAD, CHF, CNY,
      CZK, DEM, DKK, EUR, ESP, FIM, FRF, GBP, GRD, HKD, HUF, IEP, ITL,
      INR, ISK, JPY, KRW, LUF, NLG, NOK, NZD, PLN, PTE, RON, SEK, SGD,
      THB, TRY, TWD, USD, ZAR, ARS, CLP, COP, IDR, ILS, KWD, PEN, MXN,
    SAR, RUB, TND, MYR, UAH, KZT, QAR, MXV, CLF, EGP, BHD, OMR, VND,
    AED, PHP, NGN, MAD},  Note: Currency codes must also match available currencies in the {\tt simulation.xml} file.  \\ \hline
    %\lstinline!DayCount!  \lstinline!Convention! & \begin{tabular}[l]{@{}l@{}}\indent Actual 360 can be expressed by:\\ \emph{A360, Actual/360, ACT/360}\\ \indent Actual 365 Fixed can be expressed by: \\ \emph{A365, A365F, Actual/365, Actual/365 (fixed)} \\ \indent Thirty 360 (US) can be expressed by: \\ \emph{T360, 30/360, 30/360 (Bond Basis), ACT/nACT} \\ \indent Thirty 360 (European) can be expressed by: \\ \emph{30E/360, 30E/360 (Eurobond Basis)}\\ \indent Thirty 360 (Italian) is expressed by: \\ \emph{30/360 (Italian)}  \\ \indent Actual Actual (ISDA) can be expressed by: \\ \emph{ActActISDA, ActualActual (ISDA), ACT/ACT, ACT} \\ \indent Actual Actual (ISMA) can be expressed by: \\ \emph{ActActISMA, ActualActual (ISMA)} \\ \indent Actual Actual (AFB) can be expressed by:\\ \emph{ActActAFB, Actual/Actual (AFB)} \end{tabular}  \\ \hline
    \lstinline!Roll Convention! & \begin{tabular}[l]{@{}l@{}} 
\emph{F,  Following, FOLLOWING}\\ 
\emph{MF, ModifiedFollowing, Modified Following, MODIFIEDF}\\ 
\emph{P, Preceding, PRECEDING}\\ 
\emph{MP, ModifiedPreceding, Modified Preceding, MODIFIEDP}\\ 
\emph{U, Unadjusted, INDIFF }\end{tabular}  \\ \hline
  \end{tabu}
  \caption{Allowable values for standard trade data.}
  \label{tab:allow_stand_data}
\end{table}

\begin{table}[H]
  \centering
  \begin{tabu} to 0.9\linewidth {| X[-1.5,l,m] | X[-5,l,m] |}
    \hline
%    \multicolumn{2}{|l|} {\lstinline{Calendar} } \\ \hline
    \multicolumn{2}{|l|} {\tt Calendar}  \\ \hline
    \bfseries{Allowable Values} & \bfseries{Resulting Calendar} \\
    \hline
    \emph{TARGET, TGT, EUR} & Target Calendar  \\ \hline
    \emph{CA,TRB, CAD} & Canada Calendar \\ \hline
    \emph{TKB, JP, JPY} & Japan Calendar \\ \hline
    \emph{ZUB, CHF} & Switzerland Calendar \\ \hline
    \emph{GB, LNB, UK} & UK Calendar \\ \hline
    \emph{US, NYB, USD} & US Calendar \\ \hline
    \emph{US-SET} & US Settlement Calendar \\ \hline
    \emph{US-GOV} & US Government Bond Calendar \\ \hline    
    \emph{US-NYSE} & US NYSE Calendar \\ \hline  
    \emph{US-NERC} & US NERC Calendar \\ \hline  
    \emph{AU, AUD} & Australia Calendar \\ \hline
    \emph{SA, ZAR} & South Africa Calendar \\ \hline
    \emph{SS, SEK} & Sweden Calendar \\ \hline
    \emph{ARS} & Argentina Calendar \\ \hline
    \emph{BRL} & Brazil Calendar \\ \hline
    \emph{CNY} & China Calendar \\ \hline
    \emph{CZK} & Czech Republic Calendar \\ \hline
    \emph{DEN, DKK} & Denmark Calendar \\ \hline
    \emph{FIN} & Finland Calendar \\ \hline
    \emph{HKD} & HongKong Calendar \\ \hline
    \emph{ISK} & Iceland Calendar \\ \hline
    \emph{INR} & India Calendar \\ \hline
    \emph{IDR} & Indonesia Calendar \\ \hline
    \emph{MXN} & Mexico Calendar \\ \hline
    \emph{NZD} & New Zealand Calendar\\ \hline
    \emph{NOK} & Norway Calendar \\ \hline
    \emph{PLN} & Poland Calendar \\ \hline
    \emph{RUB} & Russia Calendar \\ \hline
    \emph{SAR} & Saudi Arabia \\ \hline
    \emph{SGD} & Singapore Calendar \\ \hline
    \emph{KRW} & South Korea Calendar \\ \hline
    \emph{TWD} & Taiwan Calendar \\ \hline
    \emph{TRY} & Turkey Calendar \\ \hline
    \emph{UAH} & Ukraine Calendar \\ \hline
    \emph{WeekendsOnly} & Weekends Only Calendar \\ \hline
    % \emph{US+TARGET, NYB\_TGT, TGT\_NYB} & US and Target Calendar \\ \hline  
    % \emph{NYB\_LNB, LNB\_NYB} & US and UK Calendar \\ \hline    
    % \emph{LNB\_ZUB, ZUB\_LNB} & Switzerland and UK Calendar \\ \hline   
    % \emph{TGT\_ZUB, ZUB\_TGT} & Switzerland and Target Calendar \\ \hline
    % \emph{NYB\_SYB} & US and Australia Calendar \\ \hline 
    % \emph{TGT\_BDP, BDP\_TGT} & Hungary and Target Calendar \\ \hline         
    % \emph{LNB\_NYB\_TGT} & UK, US and Target Calendar \\ \hline
    % \emph{TKB\_TGT\_LNB} & Japan, Target and UK Calendar \\ \hline         
    % \emph{LNB\_NYB\_ZUB} & UK, US and Switzerland Calendar \\ \hline
    % \emph{LNB\_NYB\_TRB} & UK, US and Canada Calendar \\ \hline 
    % \emph{LNB\_NYB\_TKB} & UK, US and Japan Calendar \\ \hline   
    % \emph{NullCalendar} & Null Calendar, i.e. all days are business days \\ \hline                 
  \end{tabu}
  \caption{Allowable Values for Calendar. Combinations of up to four
    calendars can be provided using comma separated calendar names.}
  \label{tab:calendar}
\end{table}

\begin{table}[H]
\centering
  \begin{tabu} to 0.9\linewidth {| X[-1.5,l,m] | X[-5,l,m] |}
    \hline
    %\multicolumn{2}{|l|}{\lstinline{DayCount Convention} }                             \\ \hline
    \multicolumn{2}{|l|}{\tt DayCount Convention}                          \\ \hline
    \bfseries{Allowable Values} & \bfseries{Resulting DayCount Convention} \\
    \hline
    \emph{A360, Actual/360, ACT/360}& Actual 360  \\ \hline
    \emph{A365, A365F, Actual/365, Actual/365 (fixed)} & Actual 365 Fixed \\ \hline
    \emph{T360, 30/360, 30/360 (Bond Basis), ACT/nACT} & Thirty 360 (US) \\ \hline
    \emph{30E/360, 30E/360 (Eurobond Basis)} & Thirty 360 (European) \\ \hline
    \emph{30/360 (Italian)} & Thirty 360 (Italian) \\ \hline
    \emph{ActActISDA, ActualActual (ISDA), ACT/ACT, ACT} & Actual Actual (ISDA) \\ \hline
    \emph{ActActISMA, ActualActual (ISMA)} & Actual Actual (ISMA) \\ \hline
    \emph{ActActAFB, Actual/Actual (AFB)} & Actual Actual (AFB) \\ \hline           
  \end{tabu}
  \caption{Allowable Values for DayCount Convention}
  \label{tab:daycount}
\end{table}

\begin{table}[H]
\centering
\begin{tabular}{|l|l|}
\hline
%\multicolumn{2}{|l|}{\lstinline!Index!}   \\ \hline
\multicolumn{2}{|l|}{\tt Index}   \\ \hline
\multicolumn{2}{|l|}{On form CCY-INDEX-TENOR, and matching available  }   \\ 
\multicolumn{2}{|l|}{ indices in the {\tt simulation.xml} file. }   \\ \hline
\textbf{Index Component} & \textbf{Allowable Values}                                                                                                                                                                                                                                                           \\ \hline
CCY-INDEX                &
                           \textit{\begin{tabular}[c]{@{}l@{}}
EUR-EONIA\\ EUR-EURIBOR\\ EUR-LIBOR\\ 
USD-FedFunds\\ USD-LIBOR\\ 
GBP-SONIA\\ GBP-LIBOR\\ 
JPY-LIBOR\\ JPY-TIBOR \\
CHF-LIBOR\\ 
AUD-LIBOR\\ AUD-BBSW\\ 
CAD-CDOR\\ CAD-BA\\ 
SEK-STIBOR\\ SEK-LIBOR\\ 
DKK-LIBOR\\ DKK-CIBOR \\
SGD-SIBOR\\ SGD-SOR \\
HKD-HIBOR \\
NOK-NIBOR \\
HUF-BUBOR \\
IDR-IDRFIX \\
INR-MIFOR \\
MXN-TIIE \\
PLN-WIBOR \\
SKK-BRIBOR \\
NZD-BKBM \\
\end{tabular}} \\ \hline
TENOR                    & An integer followed by \emph{D, W, M or Y}                                                                                                                                                                                                                                                 \\ \hline
\end{tabular}
  \caption{Allowable values for Index.}
  \label{tab:indices}
\end{table}


%========================================================
\section{Netting Set Definitions}\label{sec:nettingsetinput}
%========================================================

%========================================================
\section{Netting Set Definitions}\label{sec:nettingsetinput}
%========================================================

The netting set definitions file - {\tt netting.xml} - 
contains a list of
definitions for various ISDA netting agreements. The file is written
in XML format. 

\vspace{1em}

Each netting set is defined within its own \lstinline!NettingSet!
node. All of these \lstinline!NettingSet! nodes are contained as
children of a \lstinline!NettingSetDefinitions! node.

\vspace{1em}

There are two distinct cases to consider:

\begin{itemize}
\item An ISDA agreement which does not contain a \emph{Credit Support
    Annex} (CSA).
\item An ISDA agreement which does contain a CSA.
\end{itemize}
%- - - - - - - - - - - - - - - - - - - - - - - - - - - - - - - - - - - - - - - -
\subsection{Uncollateralised Netting Set}
%- - - - - - - - - - - - - - - - - - - - - - - - - - - - - - - - - - - - - - - -
If an ISDA agreement does not contain a Credit Support Annex, the
portfolio exposures are not eligible for collateralisation. In such a
case the netting set can be defined within the following XML template:

\begin{listing}[H]
%\hrule\medskip
\begin{minted}[fontsize=\footnotesize]{xml}
    <NettingSet>
        <NettingSetId> </NettingSetId>
        <Counterparty> </Counterparty>
        <ActiveCSAFlag> </ActiveCSAFlag>
        <CSADetails></CSADetails>
    </NettingSet>
\end{minted}
\caption{Uncollateralised netting set definition}
\label{lst:nettingSetUncollat}
\end{listing}

The meanings of the various elements are as follows:
\begin{itemize}
\item NettingSetId: The unique identifier for the ISDA netting set.
\item Counterparty: The identifier for the counterparty to the ISDA agreement.
\item ActiveCSAFlag: Boolean indicating whether the netting set is
  covered by a Credit Support Annex. For uncollateralised netting sets
  this flag should be \emph{False}.
\item CSADetails: Node containing as children details of the governing
  Credit Support Annex. For uncollateralised netting sets there is no
  need to store any information within this node.
\end{itemize}
%- - - - - - - - - - - - - - - - - - - - - - - - - - - - - - - - - - - - - - - -
\subsection{Collateralised Netting Set}
%- - - - - - - - - - - - - - - - - - - - - - - - - - - - - - - - - - - - - - - -
If an ISDA agreement contains a Credit Support Annex, the
portfolio exposures are eligible for collateralisation. In such a
case the netting set can be defined within the following XML template:

\begin{listing}[H]
%\hrule\medskip
\begin{minted}[fontsize=\footnotesize]{xml}
    <NettingSet>
        <NettingSetId> </NettingSetId>
        <Counterparty> </Counterparty>
        <ActiveCSAFlag> </ActiveCSAFlag>
        <CSADetails>
            <Bilateral> </Bilateral>
            <CSACurrency> </CSACurrency>
            <Index> </Index>
            <ThresholdPay> </ThresholdPay>
            <ThresholdReceive> </ThresholdReceive>
            <MinimumTransferAmountPay> </MinimumTransferAmountPay>
            <MinimumTransferAmountReceive> </MinimumTransferAmountReceive>
            <IndependentAmount>
                <IndependentAmountHeld> </IndependentAmountHeld>
                <IndependentAmountType> </IndependentAmountType>
            </IndependentAmount>
            <MarginingFrequency>
                <CallFrequency> </CallFrequency>
                <PostFrequency> </PostFrequency>
            </MarginingFrequency>
            <MarginPeriodOfRisk> </MarginPeriodOfRisk>
            <CollateralCompoundingSpreadReceive> 
            </CollateralCompoundingSpreadReceive>
            <CollateralCompoundingSpreadPay> </CollateralCompoundingSpreadPay>
            <EligibleCollaterals>
                <Currencies>
                    <Currency>USD</Currency>
                    <Currency>EUR</Currency>
                    <Currency>CHF</Currency>
                    <Currency>GBP</Currency>
                    <Currency>JPY</Currency>
                    <Currency>AUD</Currency>
                </Currencies>
            </EligibleCollaterals>
            <ApplyInitialMargin>Y</ApplyInitialMargin>
            <InitialMarginType>Bilateral</InitialMarginType>
        </CSADetails>
    </NettingSet>
\end{minted}
\caption{Collateralised netting set definition}
\label{lst:nettingSetCollat}
\end{listing}

The first few nodes are shared with the template for uncollateralised
netting sets:
\begin{itemize}
\item NettingSetId: The unique identifier for the ISDA netting set.
\item Counterparty: The identifier for the counterparty to the ISDA agreement.
\item ActiveCSAFlag: Boolean indicating whether the netting set is
  covered by a Credit Support Annex. For collateralised netting sets
  this flag should be \emph{True}.
\item CSADetails: Node containing as children details of the governing
  Credit Support Annex. 
\end{itemize}

\subsubsection*{CSADetails}

The \lstinline!CSADetails! node contains details of the Credit Support
Annex which are relevant for the purposes of exposure calculation. The
meanings of the various elements are as follows:

\paragraph*{Bilateral} There are three possible values here:
\begin{itemize}
\item Bilateral: Both parties to the CSA are legally entitled to
  request collateral to cover their counterparty credit risk exposure
  on the underlying portfolio.
\item CallOnly: Only we are entitled to hold collateral; the
  counterparty has no such entitlement.
\item PostOnly: Only the counterparty is entitled to hold collateral;
  we have no such entitlement.
\end{itemize}

\paragraph*{CSACurrency} A three-letter ISO code specifying the master
currency of the CSA. All monetary values specified within the CSA are
assumed to be denominated in this currency.

\paragraph*{Index} The index is used to derive the fixing which is used
for compounding cash collateral in the master currency of the CSA.

\paragraph*{ThresholdPay} A threshold amount above which the
counterparty is entitled to request collateral to cover excess
exposure.

\paragraph*{ThresholdReceive} A threshold amount above which we are
entitled to request collateral from the counterparty to cover excess
exposure.

\paragraph*{MinimumTransferAmountPay} Any margin calls issued by the
counterparty must exceed this minimum transfer amount. If the
collateral shortfall is less than this amount, the counterparty is not
entitled to request margin.

\paragraph*{MinimumTransferAmountReceive} Any margin calls issued by us
to the counterparty must exceed this minimum transfer amount. If the
collateral shortfall is less than this amount, we are  not
entitled to request margin.

\paragraph*{IndependentAmount} This element contains two child nodes:
\begin{itemize}
\item IndependentAmountHeld: The netted sum of all independent amounts
  covered by this ISDA agreement/CSA. A negative number implies that
  the counterparty holds the independent amount.
\item IndependentAmountType: The nature of the independent amount as
  defined within the Credit Support Annex. The only supported value
  here is \emph{FIXED}. 
\end{itemize}
This covers only the case where only one party has to post an
independent amount. In a future release this will be extended to the
situation prescribed by the Basel/IOSCO regulation (initial margin to
be posted by both parties without netting).

\paragraph*{MarginingFrequency} This element contains two child nodes:
\begin{itemize}
\item CallFrequency: The frequency with which we are entitled to
  request additional margin from the counterparty (e.g. \emph{1D},
  \emph{2W}, \emph{1M}).
\item PostFrequency: The frequency with which the counterparty is entitled to
  request additional margin from us (e.g. \emph{1D},
  \emph{2W}, \emph{1M}).
\end{itemize}

\paragraph*{MarginPeriodOfRisk} The length of time assumed necessary
for closing out the portfolio position after a default event  (e.g. \emph{1D},
  \emph{2W}, \emph{1M}).

\paragraph*{CollateralCompoundingSpreadReceive} The spread over the O/N
interest accrual rate taken by the clearing house, when holding
collateral.

\paragraph*{CollateralCompoundingSpreadPay} The spread over the O/N
interest accrual rate taken by the clearing house, when collateral is
held by the counterparty.

\paragraph*{EligibleCollaterals} For now the only supported type of
collateral is cash. If the CSA specifies a set of currencies which
are eligible as collateral, these can be listed using
\lstinline!Currency! nodes.

\paragraph*{ApplyInitialMargin} Apply (dynamic) initial Margin in
addition to variation margin

\paragraph*{InitialMarginType} There are three possible values here:
\begin{itemize}
\item Bilateral: Both parties to the CSA are legally entitled to
  request collateral to cover their MPOR risk exposure
  on the underlying portfolio.
\item CallOnly: Only we are entitled to hold collateral; the
  counterparty has no such entitlement.
\item PostOnly: Only the counterparty is entitled to hold collateral;
  we have no such entitlement.
\end{itemize}

%========================================================
\section{Market Data}\label{sec:market_data}
%========================================================

%========================================================
\section{Market Data}\label{sec:market_data}
%========================================================

Market data in the {\tt market.txt} file is given in three columns;  Date,  Quote and  Quote value. 

\begin{itemize}
\item {\bf Date}: The as of date of the market quote value.  \\ Allowable values:  See \lstinline!Date! in Table \ref{tab:allow_stand_data}.

\item {\bf Quote}: A generic description that contains Instrument Type
  and Quote Type, followed by instrument specific descriptions (See \S
  \ref{ss:zero_rate} ff.). The base of a quote consists of InstType/QuoteType followed by instrument specific information separated by slashes "/".  \\Allowable values for Instrument Types and Quote Types are given in Table \ref{tab:allow_market_data}.

\item {\bf Quote Value}: The market quote value in decimal form for the given quote on the given as of date. Quote values are assumed to be mid-market. \\Allowable values: Any real number.
\end{itemize}

\begin{table}[H]
\centering
  \begin{tabu} to 0.9\linewidth {| X[-1.5,l,m] | X[-5,l,m] |}
    \hline
    \bfseries{Market Data Parameter} & \bfseries{Allowable Values} \\
    \hline
 Instrument Type & \emph{ZERO, DISCOUNT, MM, MM\_FUTURE, FRA,
  IR\_SWAP,  BASIS\_SWAP, CC\_BASIS\_SWAP, CDS, FX\_SPOT, FX\_FWD,
   SWAPTION, CAPFLOOR, FX\_OPTION, HAZARD\_RATE, RECOVERY\_RATE }     \\ \hline
    Quote Type & \emph{BASIS\_SPREAD, CREDIT\_SPREAD, YIELD\_SPREAD,
      RATE, RATIO, PRICE, RATE\_LNVOL, RATE\_NVOL,
      RATE\_SLNVOL, SHIFT }    \\ \hline
  \end{tabu}
  \caption{Allowable values for Instrument and Quote type market data.}
  \label{tab:allow_market_data}
\end{table}

An excerpt from a typical {\tt market.txt} file is shown in Listing \ref{lst:market_txt}.

\begin{listing}[H]
%\hrule\medskip
\begin{minted}[fontsize=\footnotesize]{xml}
2011-01-31 MM/RATE/EUR/0D/1D 0.013750
2011-01-31 MM/RATE/EUR/1D/1D 0.010500
2011-01-31 MM/RATE/EUR/2D/1D 0.010500
2011-01-31 MM/RATE/EUR/2D/1W 0.009500
2011-01-31 MM/RATE/EUR/2D/1M 0.008700
2011-01-31 MM/RATE/EUR/2D/2M 0.009100
2011-01-31 MM/RATE/EUR/2D/3M 0.010200
2011-01-31 MM/RATE/EUR/2D/4M 0.011000

2011-01-31 FRA/RATE/EUR/3M/3M 0.013080
2011-01-31 FRA/RATE/EUR/4M/3M 0.013890
2011-01-31 FRA/RATE/EUR/5M/3M 0.014630
2011-01-31 FRA/RATE/EUR/6M/3M 0.015230

2011-01-31 IR_SWAP/RATE/EUR/2D/3M/1Y 0.014400
2011-01-31 IR_SWAP/RATE/EUR/2D/3M/1Y3M 0.015400
2011-01-31 IR_SWAP/RATE/EUR/2D/3M/1Y6M 0.016500
2011-01-31 IR_SWAP/RATE/EUR/2D/3M/2Y 0.018675
2011-01-31 IR_SWAP/RATE/EUR/2D/3M/3Y 0.022030
2011-01-31 IR_SWAP/RATE/EUR/2D/3M/4Y 0.024670
2011-01-31 IR_SWAP/RATE/EUR/2D/3M/5Y 0.026870
2011-01-31 IR_SWAP/RATE/EUR/2D/3M/6Y 0.028700
2011-01-31 IR_SWAP/RATE/EUR/2D/3M/7Y 0.030125
2011-01-31 IR_SWAP/RATE/EUR/2D/3M/8Y 0.031340
2011-01-31 IR_SWAP/RATE/EUR/2D/3M/9Y 0.032450
\end{minted}
\caption{Excerpt of a market data file}
\label{lst:market_txt}
\end{listing}


%- - - - - - - - - - - - - - - - - - - - - - - - - - - - - - - - - - - - - - - -
\subsection{Zero Rate}\label{ss:zero_rate}
%- - - - - - - - - - - - - - - - - - - - - - - - - - - - - - - - - - - - - - - -

The instrument specific information to be captured for quotes representing Zero Rates is shown in Table \ref{tab:zero_quote}.

\begin{table}[H]
\centering
\begin{tabular}{|p{3.3cm}|p{5cm}|p{7cm}|}
\hline
{\bf Property} & {\bf Allowable values} & {\bf Description} \\
\hline
Currency & See \lstinline!Currency! in Table \ref{tab:allow_stand_data} & Currency of the Zero rate\\ \hline
CurveId& A CCY concatenated with a Tenor. Should match CurveIds in the {\tt yield-curves.xml} file & Unique identifier for the yield curve associated with the zero quote\\ \hline
DayCounter & See \lstinline!DayCount Convention! in Table \ref{tab:daycount} & The day count basis associated with the zero quote \\ \hline
Tenor or ZeroDate & Tenor: An integer followed by D, W, M or Y, ZeroDate: See \lstinline!Date! in Table \ref{tab:allow_stand_data} & Either a Tenor for tenor based zero quotes, or an explicit maturity date (ZeroDate)\\
\hline
\end{tabular}
  \caption{Zero Rate}
  \label{tab:zero_quote}
\end{table}



\medskip
Examples with a Tenor and with a ZeroDate: 
\begin{itemize}
\item {ZERO/RATE/USD/USD6M/A365F/6M}
\item {ZERO/RATE/USD/USD6M/A365F/12-05-2018}
\end{itemize}


%- - - - - - - - - - - - - - - - - - - - - - - - - - - - - - - - - - - - - - - -
\subsection{Discount Factor}\label{ss:discount_rate}
%- - - - - - - - - - - - - - - - - - - - - - - - - - - - - - - - - - - - - - - -

The instrument specific information to be captured for quotes representing Discount Factors is shown in Table \ref{tab:discount_quote}.

\begin{table}[H]
\centering
\begin{tabular}{|p{3.3cm}|p{5cm}|p{7cm}|}
\hline
{\bf Property} & {\bf Allowable values} & {\bf Description} \\
\hline
Currency & See \lstinline!Currency! in Table \ref{tab:allow_stand_data} & Currency of the Discount rate\\ \hline
CurveId& A CCY concatenated with a Tenor. Should match CurveIds in the {\tt yield-curves.xml} file & Unique identifier for the yield curve associated with the discount quote\\ \hline
Term or DiscountDate & Term: An integer followed by D, W, M or Y, DiscountDate: See \lstinline!Date! in Table \ref{tab:allow_stand_data} & Either a Term is used to determine the maturity date, or an explicit maturity date (Discount Date) is given.\\
\hline
\end{tabular}
  \caption{Discount Rate}
  \label{tab:discount_quote}
\end{table}

If a Term is given in the last element of the quote, it is converted to a maturity date using a weekend only calendar.

\medskip
Examples with a Term and with a DiscountDate: 
\begin{itemize}
\item {DISCOUNT/RATE/EUR/EUR3M/3Y}
\item {DISCOUNT/RATE/EUR/EUR3M/A365F/12-05-2018}
\end{itemize}

%- - - - - - - - - - - - - - - - - - - - - - - - - - - - - - - - - - - - - - - -
\subsection{FX Spot Rate}
%- - - - - - - - - - - - - - - - - - - - - - - - - - - - - - - - - - - - - - - -
\label{ss:fx_spot_rate}

\begin{table}[H]
\centering
\begin{tabular}{|p{3cm}|p{3.5cm}|p{7cm}|}
\hline
{\bf Property} & {\bf Allowable values} & {\bf Description}\\
\hline
Unit currency & See \lstinline!Currency! in Table \ref{tab:allow_stand_data} & Unit/Source currency\\ \hline
Target currency & See \lstinline!Currency! in Table \ref{tab:allow_stand_data} & Target currency\\ \hline
\end{tabular}
  \caption{FX Spot Rate}
  \label{tab:fxspot_quote}
\end{table}

Example:
\begin{itemize}
\item {FX/RATE/EUR/USD}
\end{itemize}

%- - - - - - - - - - - - - - - - - - - - - - - - - - - - - - - - - - - - - - - -
\subsection{Deposit Rate}
%- - - - - - - - - - - - - - - - - - - - - - - - - - - - - - - - - - - - - - - -

\begin{table}[H]
\centering
\begin{tabular}{|p{3cm}|p{3.5cm}|p{7cm}|}
\hline
{\bf Property} & {\bf Allowable values} & {\bf Description} \\
\hline
Currency & See \lstinline!Currency! in Table \ref{tab:allow_stand_data} & Currency of the Deposit rate\\ \hline
Forward start & An integer followed by D, W, M or Y.  & Period from today to start \\ \hline
Term & An integer followed by D, W, M or Y. & Period from start to maturity\\ \hline
\end{tabular}
  \caption{Deposit Rate}
  \label{tab:deposit_quote}
\end{table}


Deposits are usually quoted as ON (Overnight), TN (Tomorrow Next), SN (Spot Next), SW (Spot Week), 3W (3 Weeks), 6M (6 Months), etc.
 
Forward start for ON is today (i.e. forward start = 0D), for TN tomorrow (forward start = 1D), for SN two days from today (forward start = 2D). For longer term Deposits, forward start is derived from conventions, see \S \ref{sec:conventions}, and is between 0D and 2D, i.e. "spot days" are between 0 and 2. 

\medskip
Example: 
\begin{itemize}
\item {MM/RATE/EUR/2D/3M}
\end{itemize}

%- - - - - - - - - - - - - - - - - - - - - - - - - - - - - - - - - - - - - - - -
\subsection{FRA Rate}
%- - - - - - - - - - - - - - - - - - - - - - - - - - - - - - - - - - - - - - - -

\begin{table}[H]
\centering
\begin{tabular}{|p{3cm}|p{3.5cm}|p{7cm}|}
\hline
{\bf Property} & {\bf Allowable values} & {\bf Description} \\
\hline
Currency & See \lstinline!Currency! in Table \ref{tab:allow_stand_data} & Currency of the FRA rate\\ \hline
Forward start & An integer followed by D, W, M or Y  & Period from today to start \\ \hline
Term & An integer followed by D, W, M or Y & Period from start to maturity\\ \hline
\end{tabular}
  \caption{FRA Rate}
  \label{tab:fra_quote}
\end{table}

FRAs are typically quoted as e.g. 6x9 which means forward start 6M from today, maturity 9M from today, with appropriate adjustment of dates.

\medskip
Example: 
\begin{itemize}
\item {FRA/RATE/EUR/9M/3M}
\end{itemize}

%- - - - - - - - - - - - - - - - - - - - - - - - - - - - - - - - - - - - - - - -
\subsection{Money Market Futures Price}
%- - - - - - - - - - - - - - - - - - - - - - - - - - - - - - - - - - - - - - - -

\begin{table}[H]
\centering
\begin{tabular}{|p{3cm}|p{4.5cm}|p{7cm}|}
\hline
{\bf Property} & {\bf Allowable values} & {\bf Description} \\
\hline
Currency & See \lstinline!Currency! in Table \ref{tab:allow_stand_data}& Currency of the MM Future price\\ \hline
Expiry & Alphanumeric string on the form MMMYY & Expiry month\\ \hline
Contract & String & Contract name\\
% date rule & {\tt string} &  \\
\hline
\end{tabular}
  \caption{Money Market Futures Price}
  \label{tab:mmfp_quote}
\end{table}

Expiry month is typically quoted as JUN08, SEP09, DEC10, etc. The exact expiry date follows from a date rule such as 3rd Wednesday of the specified month, adjusted to the following business day. The date rule is not quoted directly, but defined in the futures contract.

\medskip
Example: 
\begin{itemize}
\item {MM\_FUTURE/PRICE/EUR/JUN18/LIF3ME}
\end{itemize}

%%- - - - - - - - - - - - - - - - - - - - - - - - - - - - - - - - - - - - - - - -
%\subsection{Equity Futures Price}
%%- - - - - - - - - - - - - - - - - - - - - - - - - - - - - - - - - - - - - - - -
%
%\begin{table}[H]
%\centering
%\begin{tabular}{|p{3cm}|p{4.5cm}|p{7cm}|}
%\hline
%{\bf Property} & {\bf Allowable values} & {\bf Description} \\
%\hline
%Name & String & Name of the underlying equity \\ \hline
%Expiry & Alphanumeric string on the form MMMYY & Expiry month\\ \hline
%Currency & See \lstinline!Currency! in Table \ref{tab:allow_stand_data}& Currency of the Equity Future price\\ \hline
%Contract & String & Contract name\\
%\hline
%\end{tabular}
%  \caption{Equity Futures Price}
%  \label{tab:equityfp_quote}
%\end{table}
%
%Expiry month is typically quoted as JUN08, SEP09, DEC10, etc. The exact expiry date follows from a date rule such as 3rd Wednesday of the specified month, adjusted to the following business day. The date rule is not quoted directly, but defined in the futures contract.
%
%\medskip
%Example: 
%\begin{itemize}
%\item {EQUITY\_FUTURE/PRICE/GE/JUN17/USD/LIF3ME}
%\end{itemize}

%%- - - - - - - - - - - - - - - - - - - - - - - - - - - - - - - - - - - - - - - -
%\subsection{Equity Price}
%%- - - - - - - - - - - - - - - - - - - - - - - - - - - - - - - - - - - - - - - -
%
%\begin{table}[H]
%\centering
%\begin{tabular}{|p{3cm}|p{4.5cm}|p{7cm}|}
%\hline
%{\bf Property} & {\bf Allowable values} & {\bf Description} \\
%\hline
%Name & String & Name of the equity \\ \hline
%Currency & See \lstinline!Currency! in Table \ref{tab:allow_stand_data}& Currency of the Equity price\\
%\hline
%\end{tabular}
%  \caption{Equity  Price}
%  \label{tab:equity_quote}
%\end{table}
%
%
%\medskip
%Example: 
%\begin{itemize}
%\item {EQUITY/PRICE/GE/USD}
%\end{itemize}

%%- - - - - - - - - - - - - - - - - - - - - - - - - - - - - - - - - - - - - - - -
%\subsection{Equity Forward Price}
%%- - - - - - - - - - - - - - - - - - - - - - - - - - - - - - - - - - - - - - - -
%
%\begin{table}[H]
%\centering
%\begin{tabular}{|p{3cm}|p{4.5cm}|p{7cm}|}
%\hline
%{\bf Property} & {\bf Allowable values} & {\bf Description} \\
%\hline
%Name & String & Name of the equity \\ \hline
%Currency & See \lstinline!Currency! in Table \ref{tab:allow_stand_data}& Currency of the Equity price\\ \hline
%Term or ExpiryDate & Term: An integer followed by D, W, M or Y, ExpiryDate: See \lstinline!Date! in Table \ref{tab:allow_stand_data} & Either a Term is used to determine the maturity date, or an explicit maturity date (Expiry Date) is given.\\
%\hline
%\end{tabular}
%  \caption{Equity Forward Price}
%  \label{tab:equityf_quote}
%\end{table}
%
%If a Term is given in the last element of the quote, it is converted to a maturity date using a weekend only calendar.
%
%\medskip
%Example: 
%\begin{itemize}
%\item {EQUITY\_FORWARD/PRICE/GE/USD/05-01-2019}
%\end{itemize}
%
%%- - - - - - - - - - - - - - - - - - - - - - - - - - - - - - - - - - - - - - - -
%\subsection{Equity Dividend Rate}
%%- - - - - - - - - - - - - - - - - - - - - - - - - - - - - - - - - - - - - - - -
%
%\begin{table}[H]
%\centering
%\begin{tabular}{|p{3cm}|p{5cm}|p{7cm}|}
%\hline
%{\bf Property} & {\bf Allowable values} & {\bf Description} \\
%\hline
%Name & String & Name of the equity \\ \hline
%Currency & See \lstinline!Currency! in Table \ref{tab:allow_stand_data}& Currency of the Equity price\\ \hline
%Term & An integer followed by D, W, M or Y & The frequency of dividend payments\\ \hline
%DayCounter & See \lstinline!DayCount Convention! in Table \ref{tab:daycount} & The day count basis used to determine dividend period length  \\ 
%\hline
%\end{tabular}
%  \caption{Equity Forward Price}
%  \label{tab:equityf_quote}
%\end{table}
%
%
%\medskip
%Example: 
%\begin{itemize}
%\item {EQUITY\_DIVIDEND/RATE/GE/USD/3M/A360}
%\end{itemize}

%- - - - - - - - - - - - - - - - - - - - - - - - - - - - - - - - - - - - - - - -
%\subsection{Equity Option Price}
%- - - - - - - - - - - - - - - - - - - - - - - - - - - - - - - - - - - - - - - -

%\begin{table}[H]
%\centering
%\begin{tabular}{|p{3cm}|p{4.5cm}|p{7cm}|}
%\hline
%{\bf Property} & {\bf Allowable values} & {\bf Description} \\
%\hline
%Name & String & Name of the equity \\
%Currency & See \lstinline!Currency! in Table \ref{tab:allow_stand_data}& Currency of the Equity price\\
%Term or ExpiryDate & Term: An integer followed by D, W, M or Y, ExpiryDate: See \lstinline!Date! in Table \ref{tab:allow_stand_data} & Either a Term is used to determine the maturity date, or an explicit maturity date (Expiry Date) is given.\\
%Strike & A real number & The equity option strike price  \\ 
%\hline
%\end{tabular}
%  \caption{Equity Forward Price}
%  \label{tab:equityf_quote}
%\end{table}

%If a Term is given in the last element of the quote, it is converted to a maturity date using a weekend only calendar.


%\medskip
%Example: 
%\begin{itemize}
%\item {EQUITY\_OPTION/PRICE/GE/USD/10Y/300}
%\end{itemize}

%- - - - - - - - - - - - - - - - - - - - - - - - - - - - - - - - - - - - - - - -
\subsection{Swap Rate}
%- - - - - - - - - - - - - - - - - - - - - - - - - - - - - - - - - - - - - - - -

\begin{table}[H]
\centering
\begin{tabular}{|p{3cm}|p{3.5cm}|p{7cm}|}
\hline
{\bf Property} & {\bf Allowable values} & {\bf Description} \\
\hline
Currency & See \lstinline!Currency! in Table \ref{tab:allow_stand_data} & Currency of the Swap rate\\ \hline
Forward start & An integer followed by D, W, M or Y & Generic period from today to start\\ \hline
Tenor & An integer followed by D, W, M or Y & Underlying index period \\ \hline
Term & An integer followed by D, W, M or Y & Swap length from start to maturity\\
\hline
\end{tabular}
  \caption{Swap Rate}
  \label{tab:swaprate_quote}
\end{table}


Forward start is usually not quoted, but needs to be derived from conventions. 

\medskip
Example:
\begin{itemize}
\item {IR\_SWAP/RATE/EUR/2D/6M/10Y}
\end{itemize}

% %- - - - - - - - - - - - - - - - - - - - - - - - - - - - - - - - - - - - - - - -
% \subsection{BMA Swap Ratio}
% %- - - - - - - - - - - - - - - - - - - - - - - - - - - - - - - - - - - - - - - -

% The BMA index is a reference index for US municipal tax-exempt bonds (floaters). BMA is fixed weekly. A BMA Swap exchanges quarterly BMA index-linked payments (the coupon rate is the average of weekly BMA index fixings) for quarterly payments of a fraction of USD Libor. The BMA Swap quote is the fraction of Libor and for a given Swap term. 

% \begin{table}[H]
% \centering
% \begin{supertabular}{|p{3cm}|p{3.5cm}|p{7cm}|}
% \hline
% {\bf Property} & {\bf Allowable values} & {\bf Description} \\
% \hline
% Currency & See \lstinline!Currency! in Table \ref{tab:allow_stand_data}& Currency underlying the BMA Swap ratio \\ \hline
% Forward start & An integer followed by D, W, M or Y  & Generic period from today to start\\ \hline
% Term & An integer followed by D, W, M or Y  & Swap length from start to maturity\\
% \hline
% \end{supertabular}
%   \caption{BMA Swap Ratio}
%   \label{tab:bmasratio_quote}
% \end{table}
% Forward start is usually not quoted, but needs to be derived from conventions. 

% \medskip
% Example: 
% \begin{itemize}
% \item {BMA/RATIO/USD/2D/10Y}
% \end{itemize}

%- - - - - - - - - - - - - - - - - - - - - - - - - - - - - - - - - - - - - - - -
\subsection{Basis Swap Spread}
%- - - - - - - - - - - - - - - - - - - - - - - - - - - - - - - - - - - - - - - -

\begin{table}[H]
\centering
\begin{tabular}{|p{3cm}|p{3.5cm}|p{7cm}|}
\hline
{\bf Property} & {\bf Allowable values} & {\bf Description} \\
\hline
Flat tenor & An integer followed by D, W, M or Y & Zero spread leg's index tenor\\ \hline
Tenor & An integer followed by D, W, M or Y & Non-zero spread leg's index tenor\\ \hline
Term & An integer followed by D, W, M or Y & Swap length from start to maturity\\ 
\hline
\end{tabular}
  \caption{Basis Swap Spread}
  \label{tab:basisspread_quote}
\end{table}


\medskip
Example: 
\begin{itemize}
\item {BASIS\_SWAP/BASIS\_SPREAD/6M/3M/10Y}
\end{itemize}

%- - - - - - - - - - - - - - - - - - - - - - - - - - - - - - - - - - - - - - - -
\subsection{Cross Currency Basis Swap Spread}
%- - - - - - - - - - - - - - - - - - - - - - - - - - - - - - - - - - - - - - - -

\begin{table}[H]
\centering
\begin{tabular}{|p{3cm}|p{3.5cm}|p{7cm}|}
\hline
{\bf Property} & {\bf Allowable values} & {\bf Description} \\
\hline
Flat currency & See \lstinline!Currency! in Table \ref{tab:allow_stand_data} & Currency for zero spread leg\\  \hline
Flat tenor & An integer followed by D, W, M or Y & Zero spread leg's index tenor\\ \hline
Currency & See \lstinline!Currency! in Table \ref{tab:allow_stand_data}& Currency for non-zero spread leg\\ \hline
Tenor & An integer followed by D, W, M or Y & Non-zero spread leg's index tenor\\ \hline
Term & An integer followed by D, W, M or Y & Swap length from start to maturity\\
\hline
\end{tabular}
  \caption{Cross Currency Basis Swap Spread}
  \label{tab:ccbasisspread_quote}
\end{table}


\medskip
Example: 
\begin{itemize}
\item {CC\_BASIS\_SWAP/BASIS\_SPREAD/USD/3M/JPY/6M/10Y}
\end{itemize}

%- - - - - - - - - - - - - - - - - - - - - - - - - - - - - - - - - - - - - - - -
\subsection{CDS Spread}
%- - - - - - - - - - - - - - - - - - - - - - - - - - - - - - - - - - - - - - - -

\begin{table}[H]
\centering
\begin{tabular}{|p{3cm}|p{3.5cm}|p{7cm}|}
\hline
{\bf Property} & {\bf Allowable values} & {\bf Description} \\
\hline
Issuer & String &  Issuer name \\ \hline
Seniority & String &  Seniority status \\ \hline
Currency & See \lstinline!Currency! in Table \ref{tab:allow_stand_data} & CDS Spread currency\\ \hline
Term & An integer followed by D, W, M or Y & Generic period from start to maturity\\
\hline
\end{tabular}
  \caption{CDS Spread}
  \label{tab:cdsspread_quote}
\end{table}
 
Example: 
\begin{itemize}
\item {CDS/CREDIT\_SPREAD/GE/SeniorUnsec/EUR/5Y}
\end{itemize}

%- - - - - - - - - - - - - - - - - - - - - - - - - - - - - - - - - - - - - - - -
\subsection{CDS Recovery Rate}
%- - - - - - - - - - - - - - - - - - - - - - - - - - - - - - - - - - - - - - - -

\begin{table}[H]
\centering
\begin{tabular}{|p{3cm}|p{3.5cm}|p{7cm}|}
\hline
{\bf Property} & {\bf Allowable values} & {\bf Description} \\
\hline
Issuer & String &  Issuer name \\ \hline
Seniority & String &  Seniority status \\ \hline
Currency & See \lstinline!Currency! in Table \ref{tab:allow_stand_data} & CDS Spread currency\\
\hline
\end{tabular}
  \caption{CDS Recovery Rate}
  \label{tab:cdsrecovery_quote}
\end{table}
 
Example: 
\begin{itemize}
\item {CDS/RECOVERY\_RATE/GE/SeniorUnsec/EUR}
\end{itemize}


% %- - - - - - - - - - - - - - - - - - - - - - - - - - - - - - - - - - - - - - - -
% \subsection{Asset Swap Spread}
% %- - - - - - - - - - - - - - - - - - - - - - - - - - - - - - - - - - - - - - - -

% Credit spread information expressed on an asset swap basis for a  given issuer/seniority and term rather than for a specific asset.
 
% \begin{table}[H]
% \centering
% \begin{tabular}{|p{3cm}|p{3.5cm}|p{7cm}|}
% \hline
% {\bf Property} & {\bf Allowable values} & {\bf Description} \\
% \hline
% Issuer & String &  Issuer name \\ \hline
% Seniority & String &  Seniority status \\ \hline
% Currency & See \lstinline!Currency! in Table \ref{tab:allow_stand_data} & Asset Swap Spread currency\\ \hline
% Term & An integer followed by D, W, M or Y & Generic period from start to maturity\\ \hline
% \end{tabular}
%   \caption{Asset Swap Spread}
%   \label{tab:aswapspread_quote}
% \end{table}

% Example: 
% \begin{itemize}
% \item {ASW/CREDIT\_SPREAD/GE/SeniorUnsec/EUR/10Y}
% \end{itemize}


%%- - - - - - - - - - - - - - - - - - - - - - - - - - - - - - - - - - - - - - - -
%\subsection{Zero Spread}
%%- - - - - - - - - - - - - - - - - - - - - - - - - - - - - - - - - - - - - - - -
%
%
%\begin{table}[H]
%\centering
%\begin{tabular}{|p{3cm}|p{3.5cm}|p{7cm}|}
%\hline
%{\bf Property} & {\bf Allowable values} & {\bf Description} \\
%\hline
%Issuer & String &  Issuer name \\ \hline
%Seniority & String &  Seniority status \\ \hline
%Currency & See \lstinline!Currency! in Table \ref{tab:allow_stand_data} & Asset Swap Spread currency\\ \hline
%Term & An integer followed by D, W, M or Y & Generic period from start to maturity\\ \hline
%\end{tabular}
%  \caption{Zero Spread}
%  \label{tab:zspread_quote}
%\end{table}
%
%The effect of Zero spread $s$ is increasing the zero rates of a given reference curve, i.e. modifying the discount factor $P(t)$ at time $t$ by factor $\exp(-s\cdot t)$. We assume Actual365Fixed day count for date/time conversion. 
%
%
%\medskip
%Example: 
%\begin{itemize}
%\item {ZERO/CREDIT\_SPREAD/GE/SeniorUnsec/EUR/5Y}
%\end{itemize}
%
%%- - - - - - - - - - - - - - - - - - - - - - - - - - - - - - - - - - - - - - - -
%\subsection{Par/Swap Spread}
%%- - - - - - - - - - - - - - - - - - - - - - - - - - - - - - - - - - - - - - - -
%
%
%\begin{table}[H]
%\centering
%\begin{supertabular}{|p{3cm}|p{3.5cm}|p{7cm}|}
%\hline
%{\bf Property} & {\bf Allowable values} & {\bf Description} \\
%\hline
%Issuer & String &  Issuer name \\ \hline
%Seniority & String &  Seniority status \\ \hline
%Currency & See \lstinline!Currency! in Table \ref{tab:allow_stand_data} & Par/Swap Spread currency\\ \hline
%Term & An integer followed by D, W, M or Y & Generic period from start to maturity\\ \hline
%\end{supertabular}
%  \caption{Par/Swap Spread}
%  \label{tab:parspread_quote}
%\end{table}
%
%The effect of a Par/Swap spread $s$ is increasing the input Swap rates of a given reference curve by $s$ and then bootstrapping a modified discount curve $P(t)$.
%
%
%\medskip
%Example: 
%\begin{itemize}
%\item {IR\_SWAP/CREDIT\_SPREAD/Issuer/SeniorUnsec/10Y}
%\end{itemize}

%- - - - - - - - - - - - - - - - - - - - - - - - - - - - - - - - - - - - - - - -
\subsection{Probability of Default}\label{ss:prob_default_quote}
%- - - - - - - - - - - - - - - - - - - - - - - - - - - - - - - - - - - - - - - -

This quote can represent a cumulative PD for the given rating and maturity, or a Loss Given Default for the given rating.

\begin{table}[H]
\centering
\begin{tabular}{|p{3cm}|p{3.5cm}|p{7cm}|}
\hline
{\bf Property} & {\bf Allowable values} & {\bf Description} \\
\hline
Rating & String & Internal or external rating code \\ \hline
Term & An integer followed by D, W, M or Y & Generic period from start to maturity\\ 
\hline
\end{tabular}
  \caption{Probability of Default}
  \label{tab:pd_quote}
\end{table}

Example:
\begin{itemize}
\item {NAME/PD/15/3Y}
\item{NAME/LGD/Baa3}
\end{itemize}



% %- - - - - - - - - - - - - - - - - - - - - - - - - - - - - - - - - - - - - - - -
% \subsection{Year-on-Year Inflation Swap Rate} \label{ss:yoy_inflation_swap_rate}
% %- - - - - - - - - - - - - - - - - - - - - - - - - - - - - - - - - - - - - - - -


% \begin{table}[H]
% \centering
% \begin{supertabular}{|p{3cm}|p{5cm}|p{7cm}|}
% \hline
% {\bf Property} & {\bf Allowable values} & {\bf Description} \\
% \hline
% Currency & See \lstinline!Currency! in Table \ref{tab:allow_stand_data} & Currency of inflation swap rate\\ \hline
% Region & A string that matches inflation Region codes in the {\tt config-scenario.xml} file & Region code for the inflation index \\ \hline
% Term & An integer followed by D, W, M or Y & Generic period from start to maturity\\
% \hline
% \end{supertabular}
%   \caption{Year-on-Year Inflation Swap Rate}
%   \label{tab:yoyinflation_quote}
% \end{table}

% Example:
% \begin{itemize}
% \item {YY\_INFLATIONSWAP/RATE/EUR/FRANCE/10Y}
% \end{itemize}

% %- - - - - - - - - - - - - - - - - - - - - - - - - - - - - - - - - - - - - - - -
% \subsection{Zero-Coupon Inflation Swap Rate}
% %- - - - - - - - - - - - - - - - - - - - - - - - - - - - - - - - - - - - - - - -


% \begin{table}[H]
% \centering
% \begin{tabular}{|p{3cm}|p{5cm}|p{7cm}|}
% \hline
% {\bf Property} & {\bf Allowable values} & {\bf Description} \\
% \hline
% Currency & See \lstinline!Currency! in Table \ref{tab:allow_stand_data} & Currency of inflation swap rate\\ \hline
% Region & A string that matches inflation Region codes in the {\tt config-scenario.xml} file & Region code for the inflation index \\ \hline
% Term & An integer followed by D, W, M or Y & Generic period from start to maturity\\
% \hline
% \end{tabular}
%   \caption{Zero Coupon Inflation Swap Rate}
%   \label{tab:zinflation_quote}
% \end{table}

% Example: 
% \begin{itemize}
% \item {ZC\_INFLATIONSWAP/RATE/GBP/UK/10Y}
% \end{itemize}

%- - - - - - - - - - - - - - - - - - - - - - - - - - - - - - - - - - - - - - - -
\subsection{FX Option Implied Volatility}
%- - - - - - - - - - - - - - - - - - - - - - - - - - - - - - - - - - - - - - - -

\begin{table}[H]
\centering
\begin{tabular}{|p{3cm}|p{3.5cm}|p{7cm}|}
\hline
{\bf Property} & {\bf Allowable values} & {\bf Description} \\
\hline
Unit currency & See \lstinline!Currency! in Table \ref{tab:allow_stand_data}& Unit/Source currency\\ \hline
Target currency & See \lstinline!Currency! in Table \ref{tab:allow_stand_data} & Target currency\\ \hline
Expiry & An integer followed by D, W, M or Y & Period from today to expiry \\ \hline
Strike & \emph{ATM, RR, BF} & ATM (Straddle), RR (Risk Reversal), BF (Butterfly) \\
%strike & {\tt Real} & strike forward exchange rate \\ 
\hline
\end{tabular}
  \caption{FX Option Implied Volatility}
  \label{tab:fximplvol_quote}
\end{table}

Volatilities are quoted in terms of strategies - at-the-money straddle, risk reversal and butterfly. 

\medskip
Example: 
\begin{itemize}
\item {FX\_OPTION/RATE\_LNVOL/EUR/USD/3M/ATM}
\end{itemize}

%- - - - - - - - - - - - - - - - - - - - - - - - - - - - - - - - - - - - - - - -
\subsection{Cap/Floor Implied Volatility}\label{ss:capfloor_impl_vol_quote}
%- - - - - - - - - - - - - - - - - - - - - - - - - - - - - - - - - - - - - - - -

\begin{table}[H]
\centering
\begin{tabular}{|p{3cm}|p{3.5cm}|p{7cm}|}
\hline
{\bf Property} & {\bf Allowable values} & {\bf Description} \\
\hline
Currency & See \lstinline!Currency! in Table \ref{tab:allow_stand_data}&  Currency of the Cap/Floor volatility\\ \hline
Term & An integer followed by D, W, M or Y & Period from start to expiry \\ \hline
IndexTenor & An integer followed by D, W, M or Y & Underlying index tenor \\ \hline
Atm & \emph{1, 0} & ATM volatility quote if true (1), otherwise (0) smile quote\\ \hline
Relative & \emph{1, 0} & Relative quote (to be added to atm vol) if true (1), otherwise (0) absolute quote\\ \hline
Strike & Real number & Strike rate\\ 
\hline
\end{tabular}
  \caption{FX Option Implied Volatility}
  \label{tab:fximplvol_quote}
\end{table}


\medskip
Examples: 
\begin{itemize}
\item {CAPFLOOR/RATE\_LNVOL/EUR/10Y/6M/0/0/0.0350} (smile, absolute, strike 3.5\%)
\item {CAPFLOOR/RATE\_LNVOL/EUR/10Y/6M/1/0/0.0000} (atm, absolute, irrelevant strike)
\end{itemize}

%- - - - - - - - - - - - - - - - - - - - - - - - - - - - - - - - - - - - - - - -
\subsection{Swaption Implied Volatility}
%- - - - - - - - - - - - - - - - - - - - - - - - - - - - - - - - - - - - - - - -

\begin{table}[H]
\centering
\begin{tabular}{|p{3cm}|p{3.5cm}|p{7cm}|}
\hline
{\bf Property} & {\bf Allowable values} & {\bf Description} \\
\hline
Currency & See \lstinline!Currency! in Table \ref{tab:allow_stand_data}&  Currency of the Swaption volatility\\ \hline
Term & An integer followed by D, W, M or Y & Period from start to expiry \\ \hline
%IndexTenor & An integer followed by D, W, M or Y & Underlying index tenor \\
Dimension & \emph{Smile, ATM}  & Whether volatility quote is a Smile or ATM \\ \hline
Strike & Real number & Strike rate - (not required for ATM), as deviation from the ATM strike\\ 
\hline
\end{tabular}
  \caption{Swaption Implied Volatility}
  \label{tab:swaptimplvol_quote}
\end{table}


\medskip
Note: The volatility quote is expected to be an absolute volatility, and not the deviation from the at-the-money volatility (the latter is e.g. the quotation convention used by BGC partners).

\medskip
Examples:
\begin{itemize}
\item { SWAPTION/RATE\_LNVOL/EUR/5Y/10Y/ATM} (absolute ATM vol quote) 
\item { SWAPTION/RATE\_LNVOL/EUR/5Y/10Y/SMILE/0.0050} (absolute vol quote for ATM strike plus 50bp)
\end{itemize}

% %- - - - - - - - - - - - - - - - - - - - - - - - - - - - - - - - - - - - - - - -
% \subsection{Year-on-Year Inflation Cap/Floor Volatility} \label{ss:yoy_inflation_vol}
% %- - - - - - - - - - - - - - - - - - - - - - - - - - - - - - - - - - - - - - - -

% \begin{table}[H]
% \centering
% \begin{tabular}{|p{3cm}|p{5cm}|p{7cm}|}
% \hline
% {\bf Property} & {\bf Allowable values} & {\bf Description} \\
% \hline
% Currency & See \lstinline!Currency! in Table \ref{tab:allow_stand_data}&  Currency of the Inflation Cap/Floor volatility\\ \hline
% Region & A string that matches inflation Region codes in the {\tt config-scenario.xml} file & Region code for the inflation index\\  \hline
% Expiry & An integer followed by D, W, M or Y & Period from today to expiry \\ \hline
% Strike & Real number & Absolute Strike rate\\ 
% \hline
% \end{tabular}
%   \caption{Year-on-Year Inflation Cap/Floor Volatility}
%   \label{tab:yoyinflvol_quote}
% \end{table}

% Examples: 
% \begin{itemize}
% \item YY\_INFLATION\_CAP/PRICE/GBP/UK/5Y/0.0450
% \item YY\_INFLATION\_FLOOR/PRICE/GBP/UK/5Y/0.0450
% \end{itemize}

% %- - - - - - - - - - - - - - - - - - - - - - - - - - - - - - - - - - - - - - - -
% \subsection{Zero Coupon Inflation Cap/Floor Volatility} \label{ss:zc_inflation_vol}
% %- - - - - - - - - - - - - - - - - - - - - - - - - - - - - - - - - - - - - - - -

% TODO


%========================================================
\section{Fixing History}
%========================================================

Historical fixings data in the {\tt fixings.txt} file is given in three columns;  Index Name,  Fixing Date and  Index value. Columns are separated by semicolons ";". Fixings are used in cases where the current coupon of a trade has fixed in the past, or other path dependent features.

\begin{itemize}

\item Index Name: The name of the Index. \\Allowable values are given in Tables \ref{tab:non-IR_indices} and \ref{tab:IR_indices}.

\item Fixing Date: The date of the fixing.  \\ Allowable values:  See \lstinline!Date! in Table \ref{tab:allow_stand_data}.

\item Index Value: The index value for the given fixing date. \\Allowable values: Any real number.
\end{itemize}

An excerpt of a fixings file is shown in Listing \ref{lst:fixings_file}. Note that alternative index name formats are used (Table \ref{tab:IR_indices}).

\begin{lstlisting}[label=lst:fixings_file,caption=Fixings File (fixings.txt)]

UK RPI;2011-10-02;238
UK RPI;2011-11-02;238.5
UK RPI;2012-10-02;245.6
UK RPI;2012-11-02;245.6

EU HICP;2012-10-02;113.91
EU HICP;2012-11-02;113.91
EU HICP;2012-12-02;113.91
EU HICP;2011-10-02;113.44
EU HICP;2011-11-02;113.53

EONIAON;2011-11-25;0.003240
EONIAON;2011-11-28;0.003180
EONIAON;2011-11-29;0.003320
EONIAON;2011-11-30;0.003340

EURLIBOR3M ACTUAL/360;2012-12-21;0.015000
EURLIBOR3M ACTUAL/360;2012-12-24;0.015000
EURLIBOR3M ACTUAL/360;2012-12-27;0.015100
EURLIBOR3M ACTUAL/360;2012-12-28;0.015200

   
\end{lstlisting}

\begin{table}[H]
\centering
  \begin{tabu} to 0.9\linewidth {| X[-1.5,l,m] | X[-5,l,m] |}
    \hline
    \bfseries{Index Type} & \bfseries{Allowable Values} \\
    \hline
    Inflation & \begin{tabular}[l]{@{}l@{}}  \\  \emph{UK RPI} \\ \emph{UK YYR\_RPI} \\ \emph{UK YY\_RPI} \\ \emph{EU HICP} \\ \emph{EU YY\_HICP} \\  \emph{EU YYR\_HICP} \\  \emph{EU HICPXT} \\  \emph{EU YY\_HICPXT} \\  \emph{EU YYR\_HICPXT} \\  \emph{FR HICPXT} \\  \emph{FR YY\_HICPXT} \\  \emph{FR YYR\_HICPXT} \\  \emph{US CPI} \\  \emph{US YY\_CPI} \\ \emph{US YYR\_CPI} \end{tabular}  \\ \hline
   Commodity & \begin{tabular}[l]{@{}l@{}}  \\  \emph{CMCI3M} \\ \emph{CMCI6M} \\ \emph{CMCI1Y} \\ \emph{CMCI2Y} \\ \emph{CMCI3Y} \end{tabular} \\ \hline
  \end{tabu}
  \caption{Allowable values for non-IR indices.}
  \label{tab:non-IR_indices}
\end{table}




\begin{table}[H]
\centering
\begin{tabular}{|l|l|}
\hline
\multicolumn{2}{|l|}{\textbf{IR Index on form CCY-INDEX-TENOR:}}                                                                                                                                                                                                                                               \\ \hline
\textbf{Index Component} & \textbf{Allowable Values}                                                                                                                                                                                                                                                           \\ \hline
CCY-INDEX                & \textit{\begin{tabular}[c]{@{}l@{}}EUR-EONIA\\ EUR-EURIBOR\\ EUR-LIBOR\\ USD-FedFunds\\ USD-LIBOR\\ GBP-SONIA\\ GBP-LIBOR\\ JPY-TONAR\\ JPY-LIBOR\\ CHF-LIBOR\\ AUD-LIBOR\\ AUD-BBSW\\ CAD-CDOR\\ CAD-BA\\ SEK-STIBOR\\ SEK-LIBOR\\ DKK-LIBOR\\ SGD-SIBOR\\ HKD-HIBOR\end{tabular}} \\ \hline
TENOR                    & An integer followed by \emph{D, W, M or Y}                                                                                                                                                                                                                                                 \\ \hline
\multicolumn{2}{|l|}{\textbf{IR Index on alternative IndexTenor form:}}                                                                                                                                                                                                                                        \\ \hline
\textbf{Index Component} & \textbf{Allowable Values}                                                                                                                                                                                                                                                           \\ \hline
Index                    & \textit{\begin{tabular}[c]{@{}l@{}}Eonia\\ Euribor\\ EURLibor\\ FedFunds\\ USDLibor\\ Sonia\\ GBPLibor\\ TONAR\\ JPYLibor\\ CHFLibor\\ AUDLibor\\ AUD-BBSW \\ CDOR\\ SEK-STIBOR\\ SEKLibor\\ DKKLibor\\ SGD-SIBOR\\ HKD-HIBOR\end{tabular}}                                                     \\ \hline
Tenor                    & \emph{ON, TN, SN} or an integer followed by \emph{D, W, M or Y}.                                                                                                                                                                                       \\ \hline
\end{tabular}
  \caption{Allowable values for IR indices.}
  \label{tab:IR_indices}
\end{table}



\begin{table}[H]
\centering
  \begin{tabu} to 0.9\linewidth {| X[-5,l,m] | X[-5,l,m] |}
    \hline
    \bfseries{Example IR indices on CCY-INDEX-TENOR form} & \bfseries{Corresponding indices on IndexTenor form} \\
    \hline
    EUR-EONIA-1D & EoniaON \\ \hline
    EUR-EURIBOR-3M & Euribor3M \\ \hline
    JPY-TONAR-1D & TONARON \\ \hline    
    HKD-HIBOR-6M & HKD-HIBOR6M \\ \hline    
  \end{tabu}
  \caption{Example IR indices.}
  \label{tab:example_IR_indices}
\end{table}




\begin{appendix}

%========================================================
\section{Methodology Summary}
%========================================================

\subsection{Risk Factor Evolution Model}

ORE applies the cross asset model described in detail in \cite{Lichters} to evolve  the market through time. So far the evolution model in ORE is limited to IR and FX risk factors for any number of currencies, extensions to further risk factor classes (Inflation, Credit, Equity, Commodity) will follow.
  
The Cross Asset Model is based on the Linear Gauss Markov model (LGM) for interest rates and lognormal FX processes. We identify a single {\em domestic} currency and its LGM process labelled $z_0$, and a set of $n$ foreign currencies with associated LGM processes labelled $z_i$, $i=1,\dots,n$.  If we consider $n$ foreign exchange rates for converting foreign currency amounts into the single domestic currency by multiplication, $x_i$, $i=1,\dots,n$, then the cross asset  model is given by the system of SDEs  
\begin{eqnarray*}
dz_0 &=& \alpha_0\,dW_0^z \\
dz_i &=& \gamma_i\,dt + \alpha_i\,dW_i^z,  \qquad i>0 \\
\frac{d x_i}{x_i} &=& \mu_i\, dt + \sigma_i\,dW_i^x, \qquad i > 0 \\ \\
\gamma_i &=& 
-\alpha_i^2\,H_i -\rho_{ii}^{zx}\,\sigma_i\,\alpha_i + \rho_{i0}^{zz}\,\alpha_i\,\alpha_0\,H_0\\ 
\mu_i &=& r_0 - r_i + \rho_{0i}^{zx}\,\alpha_0\,H_0\,\sigma_i\\
r_i &=& f_i(0,t) + z_i(t)\,H'_i(t) + \zeta_i(t)\,H_i(t)\,H'_i(t),
\quad \zeta_i(t) = \int_0^t \alpha_i^2(s)\,ds  \\ \\ 
  dW^a_i\,dW^b_j &=& \rho^{ab}_{ij}\,dt, \qquad a, b \in \{z, x\} 
%\zeta_i(t) &=& \int_0^t \alpha_i^2(s)\,ds, 
%\qquad H_i(t) = \int_0^t e^{-\beta_i(s)} \,ds \\
%\beta_i(t) &=& \int_0^t \lambda_i(s)\,ds, 
%\qquad \alpha_i(t) = \sigma_i^{HW}(t)\,e^{\beta(t)} \\
\end{eqnarray*}
where we have dropped time dependencies for readability, and $f_i(0,t)$ is the instantaneous forward curve in currency $i$. 

\medskip
Parameters $H_i(t)$ and $\alpha_i(t)$ (or alternatively $\zeta_i(t)$) are LGM model parameters which determine, together with the stochastic factor $z_i(t)$, the evolution of numeraire and zero bond prices in the LGM model:
\begin{align}
N(t) &= \frac{1}{P(0,t)}\exp\left\{H_t\, z_t + \frac{1}{2}H^2_t\,\zeta_t \right\}
\label{lgm1f_numeraire} \\
P(t,T,z_t)  
&= \frac{P(0,T)}{P(0,t)}\:\exp\left\{ -(H_T-H_t)\,z_t - \frac{1}{2} \left(H^2_T-H^2_t\right)\,\zeta_t\right\}. 
\label{lgm1f_zerobond}
\end{align}

Note that the LGM model is closely related to the Hull-White model in T-forward measure \cite{Lichters}.

\subsection{Exposures}\label{sec:app_exposure}

In ORE we use the following exposure definitions
\begin{align}
\EE(t) = \EPE(t) &= \E^N\left[ \frac{(NPV(t)-C(t))^+}{N(t)} \right] \label{EE}\\
\ENE(t) &= \E^N\left[ \frac{(-NPV(t)+C(t))^+}{N(t)} \right] \label{ENE}
\end{align}
where $\NPV(t)$ stands for the netting set NPV and $C$ is the posted collateral. Note that these exposures are expectations of values discounted with numeraire $N$ (in ORE the Linear Gauss Markov model's numeraire) to today, and expectations are taken in the measure associated with numeraire $N$. These are the exposures which enter into unilateral CVA and DVA calculation, respectively, see next section. Note that we sometimes label the expected exposure (\ref{EE}) EPE, not to be confused with the Basel III Expected Positive Exposure below.

\medskip
Basel III defines a number of exposures each of which is a 'derivative' of Basel's Expected Exposure:
\begin{align}
\intertext{Expected Exposure}
EE_B(t) &= \E[\max(NPV(t) - C(t), 0)] \label{basel_ee}\\
\intertext{Expected Positive Exposure}
EPE_B(T) &= \frac{1}{T} \sum_{t<T} EE_B(t)\cdot \Delta t  \label{basel_epe} \\
\intertext{Effective Expected Exposure}
EEE_B(t) &= \max(EEE(t-\Delta t), EE_B(t)) \label{basel_eee}\\
\intertext{Effective Expected Positive Exposure}
EEPE_B(T) &= \frac{1}{T} \sum_{t<T} EEE_B(t)\cdot \Delta t \label{basel_eepe}
\end{align}
The last definition, Effective EPE, is used in Basel documents since Basel II for Exposure At Default and capital calculation. The time average in the EEPE calculation is taken over {\em the first year} of the exposure evolution (or until maturity if all positions of the netting set mature before one year).

\medskip
To compute $EE_B(t)$ consistently in a risk-neutral setting, we compound (\ref{EE}) with the deterministic discount factor $P(t)$ up to horizon $t$:
$$
EE_B(t) = \frac{1}{P(t)} \:\EE(t)
$$

\subsection{CVA and DVA}\label{sec:app_cvadva}

Using the expected exposures in \ref{sec:app_exposure} unilateral discretised CVA and DVA are given by \cite{Lichters}
\begin{align}
\CVA &= \sum_{i} \PD(t_{i-1},t_i)\times\LGD\times \EPE(t_i) \label{CVA}\\
\DVA &= \sum_{i} \PD_{Bank}(t_{i-1},t_i)\times\LGD_{Bank}\times \ENE(t_i) \label{DVA}
\end{align}
where
\begin{align*}
\EPE(t) & \mbox{ expected exposure (\ref{EE})}\\
\ENE(t) & \mbox{ expected negative exposure (\ref{ENE})}\\
PD(t_i,t_j) & \mbox{ counterparty probability of default in } [t_i;t_j]\\
PD_{Bank}(t_i,t_j) & \mbox{ our probability of default in } [t_i;t_j]\\
LGD & \mbox{ counterparty loss given default}\\
LGD_{Bank} & \mbox{ our loss given default}\\
\end{align*}

Note that the choice $t_i$ in the arguments of $\EPE(t_i)$ and $\ENE(t_i)$ means we are choosing the {\em advanced} rather than the {\em postponed} discretization of the CVA/DVA integral \cite{BrigoMercurio}. This choice can be easily changed in the ORE source code or made configurable.

Moreover, formulas (\ref{CVA}, \ref{DVA}) assume independence of credit and other market risk factors, so that $\PD$ and $\LGD$ factors are outside the expectations. With the extension of ORE to credit asset classes and in particular for wrong-way-risk analysis, CVA/DVA formulas will be generalised. 

\subsection{FVA}

Any exposure (uncollateralised or residual after taking collateral into account) gives rise to funding cost or benefits depending on the sign of the residual position. This can be expressed as a Funding Value Adjustment (FVA). A simple definition of FVA can be given in a very similar fashion as the sum of unilateral CVA and DVA which we defined by (\ref{CVA},\ref{DVA}), namely as an expectation of exposures times funding spreads:
\begin{align}
\FVA &= \underbrace{\sum_{i=1}^n f_b(t_{i-1},t_i)\,\delta_i \, \E^N\left[S_C(t_{i-1})\, S_B(t_{i-1})\, (\NPV(t_i))^+\, D(t_i)\right]}_{\mbox{Funding Benefit Adjustment (FBA)}}\nonumber\\
& {} - \underbrace{\sum_{i=1}^n f_l(t_{i-1},t_i)\,\delta_i \, \E^N\left[S_C(t_{i-1})\, S_B(t_{i-1})\, (-\NPV(t_i))^+\, D(t_i)\right]}_{\mbox{Funding Cost Adjustment (FCA)}}\label{eq_simple_fva}
\end{align}
where
\begin{align*}
D(t_i) & \mbox{ stochastic discount factor, $1/N(t_i)$ in LGM}\\
\NPV(t_i) & \mbox{ portfolio value after potential collateralization}\\
S_C(t_j) & \mbox{ survival probability of the counterparty}\\
S_B(t_j) & \mbox{ survival probability of the bank}\\
f_b(t_j) & \mbox{ borrowing spread for the bank relative to the collateral compounding rate}\\
f_l(t_j) & \mbox{ lending spread for the bank relative to the collateral compounding rate}
\end{align*}
For details see e.g. Chapter 14 in Gregory \cite{Gregory12} and the discussion in \cite{Lichters}.

\subsection{COLVA}

When the CSA defines a collateral compounding rate that deviates from the overnight rate, this gives rise to another value adjustment labeled COLVA \cite{Lichters}. In the simplest case the deviation is just given by a constant spread $\Delta$:
\begin{align}
\COLVA &= \E^N\left[ \sum_i C(t_i)\cdot \Delta \cdot \delta_i \cdot D(t_{i+1}) \right]
\label{COLVA}
\end{align}
where $C(t)$ is the collateral posted and $D(t)$ is the stochastic discount factor $1/N(t)$ in LGM. Both $C(t)$ and $N(t)$ are computed in ORE's Monte Carlo framework, and the expectation yields the desired adjustment.
 
Replacing the constant spread by a time-dependent deterministic function in ORE is straight forward. 
 
\subsection{Collateral Floor Value}

A less trivial extension of the simple COLVA calculation above, also covered in ORE, is the case where the deviation between overnight rate and collateral rate is stochastic itself. A popular example is a CSA under which the collateral rate is   the overnight rate {\em floored at zero}. To work out the value of this CSA feature one can take the difference of discounted margin cash flows with and without the floor feature. It is shown in \cite{Lichters} that the following formula is a good approximation to the collateral floor value
\begin{align}
\Pi_{Floor} &= \E^N\left[ \sum_i C(t_i)\cdot (-r(t_i))^+\cdot\delta_i \cdot D(t_{i+1}) \right]
\label{CSA_floor_value_approx}
\end{align}
where $r$ is the stochastic overnight rate and $(-r)^+ = r^+ - r$ is the difference between floored and 'un-floored' compounding rate. 

Taking both collateral spread and floor into account, the value adjustment is 
\begin{align}
\Pi_{Floor,\Delta} &= \E^N\left[ \sum_i C(t_i)\cdot ((r(t_i)-\Delta)^+-r(t_i))\cdot\delta_i \cdot D(t_{i+1}) \right] 
\label{CSA_floor_value_approx_2}
\end{align}


\subsection{Collateral Model}

The collateral model implemented in ORE is based on the evolution of collateral account balances along each Monte Carlo path taking in to account
thresholds, minimum transfer amounts and independent amounts
defined in the CSA, as well as margin periods of risk.  

ORE computes the collateral requirement (aka \emph{Credit Support Amount}) through time along each Monte Carlo path
\begin{align}\label{eq:CSA}
CSA(t_m) &= 
\begin{cases}
\max(0, V_{set}(t_m) - I_A - T_{hold}),& V_{set}(t_m) - I_A \ge 0 \\
\min(0, V_{set}(t_m) - I_A + T_{hold}),& V_{set}(t_m) - I_A < 0
\end{cases}
\end{align}
where
\begin{itemize}
\item $V_{set}(t_m)$ is the value of the netting set as of
  time $t_m$
  \item $T_{hold}$ is the threshold exposure below which no collateral is
  required (possibly asymmetric)
%\item $MTA$ is the minimum transfer amount for collateral margin
%  flow requests (possibly asymmetric)
\item $I_A$ is the sum of all collateral independent amounts attached to
  the underlying portfolio of trades (positive amounts imply that the bank
  has received a net inflow of independent amounts from the
  counterparty), assumed here to be cash.
\end{itemize}

As the collateral account already has a value of $C(t_m)$ at time
$t_m$, the collateral shortfall is simply the difference between
$C(t_m)$ and $CSA(t_m)$. However, we also need to account for the
possibility that margin calls issued in the past have not yet been
settled (for instance, because of disputes). If $M(t_m)$ denotes the net value of all outstanding margin calls at $t_m$, and $\Delta(t)$ is the difference
$\Delta(t) = CSA(t_m) - C(t_m) - M(t_m)$ between the  
{\em Credit Support Amount} and the current and outstanding collateral, then the actual margin \emph{Delivery Amount} $D(t_m)$ is calculated as follows:
\begin{align}\label{eq:DA}
D(t_m) &= 
\begin{cases}
\Delta(t),& \left| \Delta(t) \right| \ge MTA \\
0,& \left| \Delta(t) \right| < MTA
\end{cases}
\end{align}
where $MTA$ is the minimum transfer amount.

\medskip
Finally, the {\em Delivery Amount }�is settled with a delay specified by the {\em Margin Period of Risk} (MPoR) which leads to residual exposure and XVA even for daily margining, zero thresholds and minimum transfer amounts.

\subsection{Exposure Allocation}

XVAs and exposures are typically computed at netting set level. For accounting purposes it is typically required to {\em allocate} XVAs from netting set to individual trade level such that the allocated XVAs add up to the netting set XVA. This distribution is not trivial, since due to netting and imperfect correlation single trade (stand-alone) XVAs hardly ever add up to the netting set XVA: XVA is sub-additive similar to VaR. ORE provides an allocation method (labeled {\em marginal allocation } in the following) which slightly generalises the one proposed in \cite{PykhtinRosen}. Allocation is done pathwise which first leads to allocated expected exposures and then to allocated CVA/DVA by inserting these exposures into equations (\ref{CVA},\ref{DVA}). The allocation algorithm in ORE is as follows:
\begin{itemize}
\item Consider the netting set's discounted $\NPV$ after taking collateral into account, on a given path at time $t$:
$$
	E(t)=D(0,t)\,(\NPV(t)-C(t))
$$ 
\item On each path, compute contributions $A_i$ of the latter to trade $i$ as
$$
A_{i} (t) = \left\{ \begin{array}{ll} 
E(t) \times \NPV_{i}(t) / \NPV(t), & |\NPV(t)| > \epsilon \\
E(t) / n, & |\NPV(t)| \le \epsilon
\end{array}
\right. 
$$
with number of trades $n$ in the netting set and trade $i$'s value $\NPV_i(t)$.
\item The $\EPE$ fraction allocated to trade $i$ at time $t$ by averaging over paths:
$$
\EPE_i(t) = \E\left[ A_i^+(t) \right]
$$
\end{itemize}
By construction, $\sum_i A_i(t) = E(t)$ and hence $\sum_i \EPE_i(t) = \EPE(t)$.

We introduced the {\em cutoff } parameter $\epsilon>0$ above in order to handle the case where the netting set value $\NPV(t)$ (almost) vanishes due to netting, while the netting set 'exposure' $E(t)$ does not. This is possible in a model with nonzero MTA and MPoR. Since a single scenario with vanishing $\NPV(t)$ suffices to invalidate the expected exposure at this time $t$, the cutoff is essential. Despite introducing this cutoff, it is obvious that the marginal allocation method can lead to spikes in the allocated exposures. And generally, the marginal allocation leads to both positive and negative $\EPE$ allocations.

\medskip
As a an example for a simple alternative to the marginal allocation of $\EPE$ we provide allocation based on today's single-trade CVAs
$$
w_i = \CVA_i / \sum_i \CVA_i.
$$
This yields allocated exposures proportional to the netting set exposure, avoids spikes and negative $\EPE$, but does not distinguish the 'direction' of each trade's contribution to $\EPE$ and $\CVA$.

\subsection{Monte Carlo VaR and Expected Shortfall}

\todo[inline]{Add MC VaR}

\subsection{ISDA Standard Initial Margin}
The ISDA Standard Initial Margin Model (SIMM\texttrademark) is a methodology for computing initial margin that is based solely on the sensitivities of trades. It is work in progress; the current documentation \cite{SIMM} is for version 3.15. It is accompanied by a document describing the standard which the risk data has to follow, \cite{SIMM_Data_Standards}, which currently has the version 1.22.

The SIMM has four {\em product classes}: IR/FX, Credit, Equity and Commodity, which are seen as more or less independent of each other; their initial margins are just added up. Furthermore, there are six {\em risk classes}: IR, FX, Credit qualifying, Credit non-qualifying, Equity and Commodity. The trades in each product class may have risk contributions from one or several risk classes. For example, a credit trade like a cross-currency total return swap has IR, FX and Credit risk. The risk in each risk class is a combination of three risk types: Delta (based on sensitivities to standard risk factors), Vega (based on volatility sensitivities) and Curvature (also based on volatility sensitivities). These three types are again seen as independent in that their initial margins are just summed up.

All sensitivities are multiplied by standard risk weights (specified in \cite{SIMM}) and further scaled by a {\em concentration risk factor}. The method for computing this factor is not yet published, we therefore take them to be 1 at the moment. Most risk classes come with buckets which then imply different risk weights. For IR, those are the tenors of the interest curve; for Credit and Equity, they are the sector of the underlying name; for commodities, they are the type of commodity. The weighted sensitivities are summed up using intra- and inter-bucket correlations which are defined in the methodology \cite{SIMM}.

\vspace{0.5cm}

{\bf Assumptions and Interpretations.} In our implementation, we use the following assumptions and interpretations.
\begin{itemize}
\item The Concentration Risk Factors are all set to 1;
\item It is not clear in E.2 of \cite{SIMM} what a `different source' for a given issuer/seniority combination is, so this is currently ignored;
\item In the formula for Curvature margin
$$CVR_{ik} = \sum_j SF(t_{kj})\,\sigma_{kj}\,\frac{dV_i}{d\sigma}$$
in  Section 12 (a) of \cite{SIMM}, we assume that the sum is taken over tenors/vertices and not over expiries, just as in 11 (c), the corresponding formula for Vega margin. This means that the scaling factors $SF(t_{kj})$ can be applied outside of the sum.
\item Qualifiers (such as bonds or equities) can only show up in a uniques bucket. If this is not the case, the report stops with an error message;
\item The Amount Currency as defined in the risk data standard \cite{SIMM_Data_Standards} is assumed to be the collateral currency. The reporting currency is removed from the FX sensitivities;
\item The USD Amount column defined in the risk data standard \cite{SIMM_Data_Standards} is not used (Why is it there at all?);
\item A portfolio is the same as a netting set, i.e. a set of trades under one ISDA CSA;
\item Qualifiers generally have to follow the risk data standard \cite{SIMM_Data_Standards}, e.g. 1W $\neq$ 1w, and 1W will cause an error. Some are overwritten with an interpreted correct version, e.g. Libor1y $\rightarrow$ Libor12m, Libor28d $\rightarrow$ Libor1m;
\item Wrong volatility buckets for currencies (standard, low, high) are overwritten with what is listed in \cite{SIMM};
\item Missing tenors for IR sensis from product classes other than IR/FX are set automatically to OIS (overnight).
\end{itemize}

\vspace{0.5cm}

\todo[inline]{Add ISDA SIMM}

%========================================================
\section{Design Overview}
%========================================================

\end{appendix}

%========================================================
%\section{References}
%========================================================

\begin{thebibliography}{*}

\bibitem{ORE} http://www.openrisk.org

\bibitem{QRM} http://www.quaternion.com

\bibitem{QL} http://www.quantlib.org
 
\bibitem{Lichters} 
Roland Lichters, Roland Stamm, Donal Gallagher, 
{\em Modern Derivatives Pricing and Credit Exposure Analysis, Theory and Practice of CSA and XVA Pricing, Exposure Simulation and Backtesting}, 
Palgrave Macmillan, 2015.

\bibitem{Gregory12}
Jon Gregory, {\em Counterparty Credit Risk and Credit Value Adjustment, 2nd Ed.}, Wiley Finance, 2013.

\bibitem{BrigoMercurio}
Damiano Brigo and Fabio Mercurio, {\em Interest Rate Models: Theory and Practice, 2nd Edition}, Springer, 2006.

\bibitem{PykhtinRosen}
Michael Pykhtin and Dan Rosen, {\em Pricing Counterparty Risk at the Trade Level and CVA Allocations}, Finance and Economics Discussion Series, Divisions of Research \& Statistics and Monetary Affairs,
Federal Reserve Board, Washington, D.C., 2010

\bibitem{LO} http://www.libreoffice.org

\bibitem{xlwings} http://www.xlwings.org

\bibitem{SIMM}{SIMM Methodology\\ \tiny http://www2.isda.org/attachment/ODM1Mw==/ISDA\%20SIMM\%20Methodology\_7\%20April\%202016\_v3.15\%20(PUBLIC).pdf}

\bibitem{SIMM_Data_Standards}{SIMM Risk Data Standards\\ \tiny https://www2.isda.org/attachment/ODQzMg==/Risk\%20Data\%20Standards\_24\%20May\%202016\_v1.22\%20(PUBLIC).pdf}

%\bibitem{OO} http://www.openoffice.org

\bibitem{Anaconda} https://docs.continuum.io/anaconda

\end{thebibliography}

\newpage
\addcontentsline{toc}{section}{Todo}
\listoftodos[Todo]%\todos

\end{document}
