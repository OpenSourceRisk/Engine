%- - - - - - - - - - - - - - - - - - - - - - - - - - - - - - - - - - - - - - - - 
\subsubsection{Overnight Index Swap}
\label{pricing:ir_ois}

An OIS Swap has a floating leg that compounds on a daily basis using an overnight 
index. The present value of a floating OIS leg can be calculated by summing the 
NPVs for each coupon payment including spread (if any). 
Flat compounding is implemented for OIS Swaps (i.e. the spread does not compound).
The  OIS leg's present value can be written
$$
\NPV_{\OIS}= N \, \sum_{i=1}^n (f_i + s_i)\,\delta_i \,P_{\OIS}(t^p_i)
$$
where
\begin{itemize}
\item $N$: notional, assumed constant
\item $f_i$: forward rate for future starting period $i$ computed as 
$$
f_i = \frac{1}{\delta_i} \left(\frac{P_{\OIS}(t^s_i)}{P_{\OIS}(t^e_i)} - 1\right)
$$ 
\item $t^{s,e,p}_i$: the value dates associated to the first and last fixing dates in period $i$, and the payment date of period $i$
\item $\delta_i$: year fraction of period $i$
\item $P_{\OIS}(t)$: the discount factor for time $t$ bootstrapped from the OIS curve
\item $s_i$: the spread applicable to coupon $i$
\end{itemize}

When the valuation date $t$ falls into the accrual period then we replace the forward $f_i$ as follows, to take the
compounded accrued interest from period start to today into account:
$$
f_i = \frac{1}{\delta_i} \left[\prod_{j=0}^{j=m-1} \left(1 + \ON(t^f_j)\,\delta(t^f_j, t^f_{j+1})\right) \frac{P_{\OIS}(t^v_m)}{P_{\OIS}(t^e_i)} - 1\right]
$$
where $\ON(t^f_j)$ is the overnight fixing for fixing date $t^f_j$ with $t^f_0$, ..., $t^f_m$ denoting the fixing dates
in period $i$ with associated value dates $t^v_j$. Here $t^f_m$ is either equal to today if today's fixing is known or to
the last fixing date before today if today's fixing is not known.

\medskip
The payment frequency on OIS legs with maturities up to a year is typically once 
at maturity, for maturities beyond a year it is annually.

\medskip
In an Overnight Index Swap one typically swaps an OIS leg for fixed leg.

\subsubsection*{US Overnight Index Swap}

In the US OIS market another flavour of Overnight Index Swaps is more common, where
an OIS leg is exchanged for a floating, USD-LIBOR-3M linked, leg. Moreover,
the OIS leg is not a compounded OIS leg as discussed above, but the OIS leg pays
a weighted arithmetic average of overnight (FedFunds) fixings quarterly, 
see e.g. \cite{Takada_2011}. Apart from conventions, this changes the accrued 
interest calculation for the case when the valuation date $t$ falls into coupon
period $i$ to
$$
f_i = \frac{1}{\delta_i} \left[\sum_{j=0}^{j=m-1} \ON(t^f_j)\,\delta(t^f_j, t^f_{j+1}) + \ln\left(\frac{P_{\OIS}(t^v_m)}{P_{\OIS}(t^e_i)}\right) \right]
$$
using Takada's approximation for efficient forecasting from today to the coupon period end, see \cite{Takada_2011},
formula 3. As in \ref{pricing:ir_ois} $t^f_m$ is either equal to today or the last fixing date before today depending on
whether today's fixing is known.
