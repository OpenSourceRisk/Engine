\subsubsection{Credit Default Swap}
\label{pricing:cr_cds}

CDS are priced by discounting premium payments adjusted for default probability, 
and the recovery rate. Mid-point approximation is used (i.e. it is assumed the 
default, if it happens, occurs in the middle of a premium period, in between 
two premium payments)

The net present value of a CDS from a protection-seller perspective can be 
expressed:
\begin{align*}
\NPV=& N \cdot\sum_{i=1}^n K\cdot S(t_i)\cdot\delta(t_{i-1},t_i)\cdot P(t_i)\\
&- N\cdot \sum_{i=1}^n \left(LGD - K \frac{\delta(t_{i-1}, t_i)}{2} \right)
\cdot (S(t_{i-1}) - S(t_i)) \cdot P\left(t_{i-1} + \frac{t_i-t_{i-1}}{2}\right)
\end{align*}
where:
\begin{itemize}
\item $K$: the CDS premium rate
\item $S(t_i)$: the survival probability  up to cash flow time $t_i$ as seen 
at valuation time 
\item $P\left(t_{i-1} + \frac{t_i-t_{i-1}}{2}\right)$: the discount factor at the mid-point 
between times $t_{i-1}$ and $t_i$
\item $\delta(t_{i-1},t_i)$: the daycount fraction in years for the period from 
$t_{i-1}$ to $t_i$
\item $LGD$: loss given default
\end{itemize}

The first line of the $\NPV$ expression reflects the premium payments weighted 
with the survival probability for each full period. The second line reflects that 
if default occurs, the protection seller pays the $LGD$ (times notional), but receives 
half of the premium (the accrued spread for the time up to default in the middle 
of the period); the cash flow is discounted back from the middle of the period 
where the default occurs, and each flow is weighted with the probability of default 
in this period
$$
\mbox{ProbDefault}(t_{i-1},t_i) = S(t_{i-1}) - S(t_i).
$$

In a standard CDS, the $LGD$ is equal to $1-R$ where the recovery rate $R$ is 
determined once a credit event has been deemed to have occured. The recovery 
rate for a given reference entity and debt tier is determined in a credit event auction.

In a fixed recovery CDS, the recovery rate $R$ is specified at trade inception in the 
contract terms.

Per ISDA Standard Model, the real recovery rate is used for conversion from ParSpread to Upfront. 
The CDS Real Recovery Rate (also known as the Composite Recovery Rate) is a consensus value that is calculated daily based on observable CDS market data submitted 
by sell-side bank contributors.

\subsubsection{Asset Backed Credit Default Swap}
\label{pricing:cr_abcds}

The pricing of an ABCDS follows the approach of section \ref{pricing:cr_cds} closely,
with the generalization that changing notional amounts are possible in each period.
The ABCDS pricing assumes a deterministic notional schedule that follows the expected
notional reductions according to the underlying security's realised and expected future prepayments.

\subsubsection{Index Credit Default Swap}
\label{pricing:cr_index_cds}

There are two pricing approaches for Index Credit Default Swaps. The first approach uses the credit curve of index itself and follow the approach of section \ref{pricing:cr_cds}. 
The second approach is a looking through / bottom up approach 
and calculate the protection leg as the notional weighted sum of the underlying basket constituents using each constituent's credit curve and recovery.

With the second approach we have a natural breakdown of the 
credit sensitivities to each constituent while on the other hand the pricing is may not consitent to the market (using the index curve).
In the index curve approach we can decompose the index sensitivities into the credit sensitivity for each constituent using a
weighting schema. 

Let $CR01_{Index,T}$ be the credit sensitivities of the index curve on the tenor $T$, the decomposed sensitivity for constituent $i=1,...,n$ is 
then given by

$$ CR01_{i, T} = w_i \, CR01_{Index,T}$$

with $\sum_{j=1}^n w_i = 1$.

There are three weighting schemas:

In the \emph{notional weighted} schema the weight is given derived from the notionals $N_i$ of the index constituents, $w_i = \frac{N_i}{\sum_{j=1}^n N_j}$. 

In the \emph{delta weighted} schema the weight is derived from a simplified credit delta. Assuming a zero bond with maturity of the index cds $t_m$ for each constituent,
the approximate credit delta is given by

$$ \delta_i = N_i \: t_m \: S_i(t_m) $$

with $S_i(t_m)$ being the survival probability up to maturity of constituent $i$. 
The approximated credit delta will be normalized to weights $w_i = \frac{\delta_i}{\sum_{j=1}^n \delta_j}$.

The third schema \emph{expected loss weighted} uses as similar approach as the \emph{delta weighted} schema, but instead of the normalized credit delta is uses the expected loss

$$EL_i = N_i \: (1 \, - \, S_i(t_m)) \: (1 \, - \, RR_i).$$



