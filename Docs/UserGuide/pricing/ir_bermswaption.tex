%- - - - - - - - - - - - - - - - - - - - - - - - - - - - - - - - - - - - - - - - 
\subsubsection{Bermudan/American Swaption and non-standard European Swaption}
\label{pricing:ir_bermswaption}

Bermudan/American Swaptions and non-standard European Swaptions (see \ref{pricing:ir_euroswaption} for the criteria to
distinguish standard from non-standard swaptions) are priced using a one-factor Linear Gauss Markov (LGM) model, which
is closely related to the Hull-White one-factor model, see section \ref{models:LGM} for details.

\underline{General Model Parametrization}

The reversion and volatility type is specified in the pricing engine configuration, as shown in listing
\ref{lst:ir_bermswaption_pricing_config}. The allowed values are ``HullWhite'' and ``Hagan'' for both parameters
``ReversionType'' and ``VolatilityType'', and all combinations except ReversionType = Hagan, VolatilityType = HullWhite
are allowed.

Listing \ref{lst:ir_bermswaption_pricing_config_fd} shows an alterative configuration using a finite-difference pricer
instead of numerical integration, see below in the section on rollback implementation for details.

Listing \ref{lst:ir_amswaption_pricing_config_fd} shows a configuration for American swaptions. The configuration has
the same parameters as the one for Bermudan swaptions except for the additional parameter ExerciseTimeStepsPerYear which
controls how many exercise dates are effective generated for pricing purposes. Usually this parameter is set to the same
value as the TimeStepsPerYear parameter that controls the FD Grid, i.e. on each finite-difference time step an exercise
is implemented (if it falls into the exercise period specified in the trade). Note: It is possible to use engine type
Grid for American swaptions, but this is known to be slow in comparison because of the many time steps to process in the
rollback.


\begin{listing}[h]
\begin{minted}[fontsize=\footnotesize]{xml}
  <Product type="BermudanSwaption">
    <Model>LGM</Model>
    <ModelParameters>
      <Parameter name="Calibration">Bootstrap</Parameter>
      <Parameter name="CalibrationStrategy">CoterminalDealStrike</Parameter>
      <Parameter name="ReferenceCalibrationGrid">400,3M</Parameter>
      <Parameter name="Reversion">0.0</Parameter>
      <Parameter name="ReversionType">HullWhite</Parameter>
      <Parameter name="Volatility">0.01</Parameter>
      <Parameter name="VolatilityType">Hagan</Parameter>
      <Parameter name="ShiftHorizon">0.5</Parameter>
      <Parameter name="Tolerance">0.10</Parameter>
    </ModelParameters>
    <Engine>Grid</Engine>
    <EngineParameters>
      <Parameter name="sy">5.0</Parameter>
      <Parameter name="ny">30</Parameter>
      <Parameter name="sx">5.0</Parameter>
      <Parameter name="nx">30</Parameter>
    </EngineParameters>
  </Product>
\end{minted}
\caption{Bermudan Swaption pricing configuration Grid Engine}
\label{lst:ir_bermswaption_pricing_config}
\end{listing}

\begin{listing}[h]
\begin{minted}[fontsize=\footnotesize]{xml}
  <Product type="BermudanSwaption">
    <Model>LGM</Model>
    <ModelParameters>
      <Parameter name="Calibration">Bootstrap</Parameter>
      <Parameter name="CalibrationStrategy">CoterminalDealStrike</Parameter>
      <Parameter name="ReferenceCalibrationGrid">400,3M</Parameter>
      <Parameter name="Reversion">0.0</Parameter>
      <Parameter name="ReversionType">HullWhite</Parameter>
      <Parameter name="Volatility">0.01</Parameter>
      <Parameter name="VolatilityType">Hagan</Parameter>
      <Parameter name="ShiftHorizon">0.5</Parameter>
      <Parameter name="Tolerance">0.10</Parameter>
    </ModelParameters>
    <Engine>FD</Engine>
    <EngineParameters>
      <Parameter name="Scheme">Douglas</Parameter>
      <Parameter name="StateGridPoints">64</Parameter>
      <Parameter name="TimeStepsPerYear">24</Parameter>
      <Parameter name="MesherEpsilon">1E-4</Parameter>
    </EngineParameters>
  </Product>
\end{minted}
\caption{Bermudan Swaption pricing configuration FD Engine}
\label{lst:ir_bermswaption_pricing_config_fd}
\end{listing}

\begin{listing}[h]
\begin{minted}[fontsize=\footnotesize]{xml}
  <Product type="AmericanSwaption">
    <Model>LGM</Model>
    <ModelParameters>
      <Parameter name="Calibration">Bootstrap</Parameter>
      <Parameter name="CalibrationStrategy">CoterminalDealStrike</Parameter>
      <Parameter name="ReferenceCalibrationGrid">400,3M</Parameter>
      <Parameter name="Reversion">0.0</Parameter>
      <Parameter name="ReversionType">HullWhite</Parameter>
      <Parameter name="Volatility">0.01</Parameter>
      <Parameter name="VolatilityType">Hagan</Parameter>
      <Parameter name="ShiftHorizon">0.5</Parameter>
      <Parameter name="Tolerance">0.10</Parameter>
      <Parameter name="ExerciseTimeStepsPerYear">24</Parameter>
    </ModelParameters>
    <Engine>FD</Engine>
    <EngineParameters>
      <Parameter name="Scheme">Douglas</Parameter>
      <Parameter name="StateGridPoints">64</Parameter>
      <Parameter name="TimeStepsPerYear">24</Parameter>
      <Parameter name="MesherEpsilon">1E-4</Parameter>
    </EngineParameters>
  </Product>
\end{minted}
\caption{American Swaption pricing configuration FD Engine}
\label{lst:ir_amswaption_pricing_config_fd}
\end{listing}

If ReversionType is set to HullWhite, the model parameter $H$ is derived from a piecewise constant mean reversion
function $\kappa(t)$ via

\begin{equation}\label{ir_bermswaption_hw_hw_H1}
  H(t) = A \int_0^t e^{-\int_0^s \kappa(u) du} ds + B
\end{equation}

see \cite{Hagan_LGM}, formula (4.10a). Here $A$ is a scaling factor and $B$ is a shift parameter. If ReversionType is set
to Hagan on the other hand, the model parameter $H$ is derived from a piecewise constant function $h>0$ via

\begin{equation}\label{ir_bermswaption_hw_hw_H2}
  H(t) = A \int_0^t h(s)\,ds + B
\end{equation}

which results in a piecewise linear $H$. If the VolatilityType is set to HullWhite, the model parameter $\alpha$ is
derived from a piecewise constant volatility function $\sigma(t)$ via

\begin{equation}\label{ir_bermswaption_hw_hw_alpha}
  \alpha(t) = \frac{\sigma(t)}{H'(t)}
\end{equation}

see \cite{Hagan_LGM}, formula (4.9b). The function $\sigma(t)$ is the volatility of the Hull White model in its classic
form, i.e. this variant exactly replicates a Hull White model with piecewise constant volatility. If the VolatilityType
is set to Hagan, the model parameter $\alpha$ is assumed to be piecewise constant.

We note that the formulas \ref{ir_bermswaption_hw_hw_H1}, \ref{ir_bermswaption_hw_hw_H2} and
\ref{ir_bermswaption_hw_hw_alpha} can be implemented using closed-form expressions thanks to the piecewise constant
nature of the input functions, i.e. no numerical integration is required.



\underline{Model Parametrization for Bermudan/American Swaptions}

For Bermudan/American swaption pricing the reversion parameter $\kappa(t)$ (ReversionType = HullWhite) resp. $h(t)$
(ReversionType = Hagan) is restricted to a constant value, which is assumed to be given as an external parameter
``Reversion'', see listing \ref{lst:ir_bermswaption_pricing_config}.

The parameter ``Calibration'' can be set to one of the values ``None'', ``Bootstrap'', ``BestFit''.

In case of ``None'' the model will not be calibrated before pricing, instead the volatility will be set to the value
specified as ``Volatility''. The volatility function can be made time dependent by specifying a comma separated list of
$n$ volatilities along with a list of $n-1$ times as an additional parameter ``VolatilityTimes''. This setting is useful
when an external model parameterization should be used.

In case of ``Bootstrap'' the model volatility function will be calibrated to a set of coterminal swaptions. The
(constant) value given as the paramter ``Volatility'' will be used as the initial guess for the calibration in this
case. ``VolatilityTimes'' will be ignored and instead overwritten by a time grid corresponding to the option expiries of
the coterminal swaption calibration instruments. See below how the coterminal swaptions are constructed.

In case of ``BestFit'' a constant model volatility will be calibrated to a set of coterminal swaptions that is
constructed in the same way as in the case of ``Bootstrap''.

The scaling parameter $A$ is set to $1$ and the shift parameter $B$ is set to $-H(T)$ with $T$ denoting $0.5$ times the
remaining time to maturity of the trade to price.



\underline{Construction of coterminal calibration swaptions for Bootstrap}

The set of calibration instruments is determined as follows:

The swaption expiries are chosen as the exercise dates of the trade to price. If a parameter
``ReferenceCalibrationGrid'' is given, at most one calibration instrument is kept per grid interval, in the example
given in listing \ref{lst:ir_bermswaption_pricing_config} this means that at most one swaption per 3M interval is kept. This can
be both useful to stabilize the calibrated model volatility function (avoid osciallations if calibration instrument
expiries lie too close to each other) and to speed up the calibration step.

For American swaptions the ``ReferenceCalibrationGrid'' parameter is mandatory and effectively one instrument per period
is generated if the expiry falls into the exercise range of the trade.

The underlying maturity of the calibration swaptions is chosen to be identical to the trade maturity.

The strikes of the calibration swaptions are chosen dependent on the parameter ``CalibrationStrategy'' (see listing
\ref{lst:ir_bermswaption_pricing_config}). This parameter can be set to ``None'' (if ``Calibration'' is set to ``None'',
too) or to ``CoterminalATM'' or to ``CoterminalDealStrike''.

In case of ``CoterminalATM'' the calibration swaption strikes are set to their fair forward swap level.

In case of ``CoterminalDealStrike'' the calibration swaption strikes are set to the trade strike. More precisely, the
strike $K$ of the calibration swaption corresponding to an exercise date $d$ of the trade to price is set to

\begin{equation}
K = r - s
\end{equation}

where $r$ is the rate of the first fixed rate coupon with accrual start date greater or equal to $d$ and $s$ is the
first floating rate coupon spread with accrual start date greater or equal to $d$.

The calibration swaptions are always constructed following the conventions of the relevant swaption surface / cube. In
particular, the index tenor of these swaptions is set to the market swaptions tenor.



\underline{Numerical implementation of calibration}

Both the bootstrap and best fit calibrations are implemented using a Levenberg-Marquardt optimizer. The target function
is defined as the RMSE of errors of calibration swaptions and the single swaption error is taken to be the relative
price error $e$

\begin{equation}
e = \frac{p-m}{m}
\end{equation}

where $p$ is the model price of the swaption and $m$ is the market price of the swaption.

If the market value of a calibration swaption is below $10^{-20}$ we replace the strike of the calibration swaption with
the forward ATM level of that swaption to stabilize the calibration. Furthermore, if the market value of the calibration
swaption (possibly after the strike amendment just mentioned) is below $10^{-8}$ we use an absolute price error

\begin{equation}
e = {p-m}
\end{equation}

in the target function for this swaption instead of a relative one. Again, this is done to stabilize the calibration.

In the case of a best fit calibration we optimize the model parameters simultaneously. In case of a bootstrap
calibration we perform a series of one dimensional optimizations for each calibration swaption in ascending order of
their option expiry times.

The calibration is executed in terms of transformed parameters always, so that restrictions on positivity of parameters
are ensured by the transformation, yet the optimization itself is run in terms of variables for which all real values
are admissable.

After the optimizer has finished the value of the target function is compared against the parameter ``Tolerance'' in
listing \ref{lst:ir_bermswaption_pricing_config} and either the pricing is aborted or continued with an alert issued to the log
depending on the global pricing parameter ``ContinueOnCalibrationError''.

We note that the discount curve used in the calibration step to price the calibrations swaption is the curve specified
by the ``IrLgmCalibration'' configuration, usually the in-ccy OIS curve since the swaptions are collateralized in their
trade currency.


\underline{Implementation of Roll Back}

To compute conditional expectation $E_t$ at time $t$

\begin{equation}
  E_t \left( \frac{V(T)}{N(T)} \right)
\end{equation}

of a deflated value $V(T) / N(T)$ with $T>t$, we use either numerical integration or a finite difference solver:

\begin{itemize}
\item For numerical integration we use the numerical integration scheme proposed in \cite{Hagan_Bermudan},
section 4.1. The numerical parameters for this scheme is read from the parameters ``sx'', ``sy'', ``nx'', ``ny'', see
listing \ref{lst:ir_bermswaption_pricing_config}.
\item For the finite difference solver we use ``StateGridPoints'' points covering states that are attained
with a probability of at least ``MesherEpsilon'' times $2$. We use ``TimeStepsPerYear'' equidistant steps per year for
the rollback in time direction. See
listing \ref{lst:ir_bermswaption_pricing_config_fd}, \ref{lst:ir_amswaption_pricing_config_fd}.
\end{itemize}

For each cashflow of the underlying of the swaption we associate the latest exercise date for which the cashflow is
relevant. Here, relevant means that the cashflow is part of the exercise-into right at the exercise date. We then
estimate an amount for the cashflow so that it is available on this latest relevant exercise date. The exact date on
which the coupon amount is estimated is determined as follows:

\begin{itemize}
  \item for FixedRate, Ibor, ON compounded, ON averaged, BMA coupons, the greater of the latest relevant exercise date
    and the evaluation date is taken
  \item for CappedFloored Ibor coupons the greater of the latest relevant exercise date, the evaluation date and the coupon fixing date is taken
  \item for CappedFloored ON compunded and ON averaged the greater of the latest relevant exercise date, the evaluatino
    date and the first of the relevant ON fixing dates of the coupon is taken
\end{itemize}

The coupon amount estimation is done using the LGM discount bond formula

\begin{equation}
P(t,T,x) = \frac{P(0,T)}{P(0,t)} e^{ -(H(T)-H(t))x - 0.5 ( H(T)^2 - H(t)^2 ) \zeta(t)}
\end{equation}

where $x$ is the LGM model state, see \cite{Hagan_LGM}, section 4.1, and $P(0,t)$ is the market zerxo bond with maturity
$t$ computed on the relevant forwarding curve of the coupon.

The amount of a coupon is estimated once during a rollback and then rolled back to prior exercise dates. The total set of dates used in the rollback comprises

\begin{itemize}
\item the simulation dates for all coupons as determined above
\item the exercise dates of the swaption to price
\item the valuation date itself
\end{itemize}

Since the computation of conditional expectations is exact up to numerical errors, no further intermediate time steps
need to be added as it would for example be the case for a finite difference solver.

The numeraire to deflate the values during the rollback is calculated using the ``Pricing'' configuration discount
curve, usually the cross-OIS curve associated to the VM colalteral currency of the trade to price.

