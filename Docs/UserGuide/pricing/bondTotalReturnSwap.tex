\subsubsection{Bond Total Return Swap}


A total return swap is a derivative contract entered at time $t$ in
which one counterparty pays out the total returns of an underlying
asset and receives a regular fixed or floating cash flow from the
other counterparty. In this section total return swaps with underlying
bond are described. The total return of the bond is comprised of
%
\begin{itemize}
\item coupon, redemption and amortization payments of the bond,
  including  recovery payments in case of default
\item compensation payments that reflect changes of the clean bond
  value along the TRS schedule.
\end{itemize}
Let $T$ denote the maturity of the underlying bond. $P(t_1,t_2)$,
$t_1\leq t_2$ is a discount factor from $t_2$ to $t_1$. To highlight
that $P$ belongs to a certain curve a subscript with the curve's name
will be added, e.g.$P_{BRY}$ is the discount factor corresponding to
the Bond Reference Yield curve. For the purpose of this description
notional is assumed to be $1$. Furthermore for the sake of simplicity
of this description we assume that the swap is not composite, i.e. the
return is paid in the bond currency.

Hence the value of the ordinary bond (with payment schedule
$\{t_i\}_i$, coupons $c_{t_i}$ and assuming notional-recovery) in Bond
currency is given by
\begin{equation}
V(t)=\sum_{t_i>t}c_{t_i}{\bar P}(t,t_i)+{\bar P}(t,T)+R(t,T)
\label{bondprice}
\end{equation}
where 
\begin{itemize}
\item $\bar{P}(t,t_i)=P_Y(t,t_i)\cdot S(t,t_i)$ is
  the combined discount factor that accounts for reference yield,
  bond specific liquidity and  potential default, with
  $P_Y(t,t_i)=P_{BRY}(t,t_i)\cdot P_L(t,t_i)$ 
\item $P_{BRY}(t,t_i)$ is the discount factor of the chosen bond reference
  yield curve 
\item $P_L(t,t_i)$ is the discount factor reflecting a
  bond-specific liquidity spread adjustment $z_i$ to the reference
  yield curve  above, i.e.~$P_L(t,t_i)=\exp(-z_i(t-t_i))$ where $z_i$
  may also be constant/flat
\item $S(t_1,t_2)$ is the survival
  probability between $t_1$ and $t_2$.
\item $R(t,T)$ reflects face-value recovery between $t$ and
  $T$: 
  $$R(t,T)=R\int_{t}^T\rho(s)\,P_{Y}(t,s) \,ds$$ 
  where $\rho$ is the time-density of default and  
  $P_Y(t,t_i)=P_{BRY}(t,t_i)\cdot P_L(t,t_i)$.
\end{itemize}
Note that the liquidity spread adjustment factor $P_L(t,t_i)$  above
is used to match quoted prices of liquid bonds, given the bond
reference yield curve $P_{BRY}(t,t_i)$ and the bond survival
probability curve $S(t,t_i)$. 

%
\medskip
In the pricing of the TRS we will also encounter the valuation of a
\emph{restricted bond}, i.e.~a bond whose cashflows are restricted to
occur after a certain time $T_f$.
Today's price of the restricted default-able bond is
$$
V_{T_f}(t)=\sum_{T_f<t_i<T} c_{t_i}
\bar P(t,t_i)+\bar P(t,T)+R(T_f,T).
$$
where $R(T_f,T)$ reflects face-value recovery between $T_f$ and
$T$: 
$$
R(T_f,T)=R\int_{T_f}^T\rho(s)\,P_{Y}(T_f,s)\,ds
$$
%
To compute the change of bond value between $t$ and $T_f$ we must
forecast the value of the bond at the future time $T_f$,  
this is similar to the valuation of forward bonds,
section~\ref{forwardBond}.

In the multicurve framework the following discount curves will be relevant:
\begin{itemize}
\item  $P_{Y}$: curve for discounting the bond cashflows, typically
  a Repo curve
\item $P_{D}$: the curve used for discounting derivative cash flows
  taking the CSA currency into account, see section \ref{sec:curves}
\end{itemize}

The discounted compensation payment for the time interval $[t_n,t_{n+1}]$
is given by
\begin{align*}
  C(t_n,t_{n+1})&=\left(\frac{V_{t_{n+1}}(t)}{P_{Y}(t,t_{n+1})} -AccruedAmount(t_{n+1})\right) \times P_{D}(t,t_{n+1}) \\
                &-\left(\frac{V_{t_{n}}(t)}{P_{Y}(t,t_{n})} -AccruedAmount(t_{n})\right) \times P_{D}(t,t_{n+1})
\end{align*}
This means we compute clean Bond prices as of both observation
dates at period start and end (this is achieved by compouding the
time-t prices to the respective period dates using the Bond yield
curve);  the cash flow at period end is then given by the clean price 
difference, and the cash flow is finally discounted to time $t$ using 
the derivative discount curve.

In case dirty instead of clean prices are usd to compute the compensation payments the above formula is amended in the
obvious way, in this case we do not subtract the accrued amount as of $t_{n+1}$ resp. $t_{n}$.

\medskip
The NPV of the long derivative contract is then given by
%
\begin{align*}
NPV(t)&=\sum_{n}C(t_n,t_{n+1})+V'(t)-NPV_{FundingLeg}(t).
\end{align*}
where 
\begin{itemize}
\item $\{t_n\}_n$ is the schedule of the compensation
payments;
\item $V'(t)$ is the value of the bond cash flows as in
\eqref{bondprice} but replacing $P_Y(t,t_i)\rightarrow
P_D(t,t_i)$ in the dicount factors $\bar{P}(t,t_i)$
 since we are valuing the bond cash flows
now in a derivative context, i.e. $\bar{P}(t,t_i) \rightarrow
P_D(t,t_i)\,S(t,t_i)$ still taking potential default into account; if the bond maturity is later than the swap maturity,
cashflows after the swap maturity are excluded from the valuation and the recovery value is only computed until the swap
maturity in this context.
\item $NPV_{FundingLeg}(t)$ is the value of the TRS pay leg which is
 assumed here to be  a vanilla fixed or floating leg. 
\end{itemize}


