%--------------------------------------------------------
\subsection{Calendar Adjustment: {\tt calendaradjustment.xml}}\label{sec:calendaradjustment}
%--------------------------------------------------------

\medskip
 This file {\tt calendaradjustment.xml} list out all additional holidays and business days that are added to a specified calendar in ORE.
 These dates would originally be missing from the calendar and has to be added.The general structure is shown in listing \ref{lst:calendar_adjustment}.
In this example, two additional dates had been added to the calendar "Japan", one additional holiday and one additional business day. If the user is not certain
wether the date is already included or not, adding it to the {\tt calendaradjustment.xml} to be safe won't raise any errors.
A sample {\tt calendaradjustment.xml} file can be found in the global example input directory. However, it is only used in Example\_1.

\begin{longlisting}
\begin{minted}[fontsize=\scriptsize]{xml}
<CalendarAdjustments>
  <Calendar name="Japan">
    <AdditionalHolidays>
      <Date>2020-01-01</Date>
    </AdditionalHolidays>
    <AdditionalBusinessDays>
      <Date>2020-01-02</Date>
    </AdditionalBusinessDays>
</CalendarAdjustments>
\end{minted}
\caption{Calendar Adjustment}\label{lst:calendar_adjustment}
\end{longlisting}

If the parameter \lstinline!BaseCalendar! is provided then a new calendar will be created using the specified calendar as a base, and adding any \lstinline!AdditionalHolidays! or \lstinline!AdditionalBusinessDays!. In the example below a new calendar \lstinline!CUSTOM_Japan! is being created, it will include any additional holidays or business days specified in the original \lstinline!Japan! calendar plus one additional date.

If a new calendar is added in this way and the schema is being used to validate XML input, the corresponding calendar name must be prefixed with `CUSTOM\_'.

\begin{longlisting}
\hrule\medskip
\begin{minted}[fontsize=\scriptsize]{xml}
<CalendarAdjustments>
  <Calendar name="CUSTOM_Japan">
    <BaseCalendar>Japan</BaseCalendar>
    <AdditionalHolidays>
      <Date>2020-04-06</Date>
    </AdditionalHolidays>
</CalendarAdjustments>
\end{minted}
\caption{Calendar Adjustment creating a new calendar}
\label{lst:calendar_adjustment_2}
\end{longlisting}